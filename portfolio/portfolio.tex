\documentclass[
	12pt,
	letterpaper,
	oneside,
]{book}
\usepackage[utf8]{inputenc}
\usepackage[T1]{fontenc}
\usepackage{scrextend}
\usepackage[
	left=1in,
	right=1in,
	top=1in,
	bottom=1.5in
]{geometry}
\usepackage{xcolor}
\usepackage[
	unicode=true, 
	bookmarks=true, 
	bookmarksnumbered=false, 
	bookmarksopen=false, 
	breaklinks=true, 
	hidelinks,
	colorlinks=false,
	linkbordercolor={white},
]{hyperref}
% % \usepackage[english]{babel}
% \usepackage{csquotes}
% \usepackage{listings}
% \usepackage{times}
% \usepackage{lscape}
\usepackage{placeins}
\usepackage{graphicx}
% \usepackage{xparse}
% \usepackage{fancyhdr}
% \usepackage{lipsum}
% \usepackage{etoolbox}
% \usepackage{setspace}
\newcommand{\img}[4]{
\begin{figure}[!htbp]
\centering
\includegraphics[width=1\textwidth]{/Users/federicocamarahalac/Documents/fd_work/text/waves/bin/img/#1.png}
\label{img:#1}
\caption{#2}
\end{figure}
\FloatBarrier
}
\title{Portfolio}
\author{Federico Nicolás Cámara Halac}
\setlength{\footskip}{1.5\baselineskip}
\setlength{\parindent}{4em}
%------------------------------------------------------------------------------
%
%	BEGIN: Document
%
%------------------------------------------------------------------------------
\begin{document}

\maketitle


\tableofcontents


\part*{Untitled2}
\addcontentsline{toc}{part}{Untitled2}
\img{untitled2}{Field recordings inside an ancient sewage system on the island of Delos, Greece.}


\begin{itemize}
\item Description: Multichannel work for Laptop Performer
\item Duration:  10 minutes
\item Category:  Fixed media, Live electronics, Generative, Acousmatic, Multichannel
\item Date:   6/16/2018
\item Location:   Mykonos, Greece
\item Audio:   \url{https://soundcloud.com/ffddcchh/untitled2-stereo}
\item Code:   \url{https://github.com/fdch/untitled2}
\end{itemize}

\newpage
\section*{Program Notes}
{Immersed on a digital environment, ``untitled2'' lives on the limit between field recording and synthesis, music composition and sound art. This liminality aims to resonate its way into the complex difference between art and life. It was composed and premiered during the ``Delian Academy for New Music'' held in Mikonos, Greece, in June 2018}





\part*{Hearing The Self: A Spectral Experience}
\addcontentsline{toc}{part}{Hearing The Self}
\img{hearingtheself}{Hally looking right back at you, with its two camera eyes stuck on top of a microphone stand. The screens in the back shows you some of what it sees.}


\begin{itemize}
	\item Description: For 2 PS3 Eyecams, multichannel audio, screens and participants. 
	\item Duration:  \texttt{undefined}
	\item Category:  Multimedia, Live electronics, Interactive
	\item Date:   10/15/2017
	\item Location:   Xuhui Art Museum, Shanghai, China
	\item Documentation:   \url{https://fdch.github.io/specexp}
\end{itemize}
\newpage

\section*{Program Notes}
{This interactive, audio/visual installation simulates the process by which the human brain perceives the world. From a mechanical point of view, it has been suggested that the human brain is a machine that performs an inverse Fourier Transform through which it constructs a geometric image from correlations of reflections of light. While a significant component of the installation is visual, the audio acts via sonification to extend the perception beyond just what the eye can see. The audience is invited to experience spectral properties outside of the more familiar visible wavelengths. This piece simultaneously addresses that the mechanical component of perception, even when perceiving one's own self, is only a part of a complex process shaped by many external stimuli, one of the strongest being societal. Authors: Lucia D Simonelli, Matias G Delgadino, Hally, Fede Camara Halac}


\part*{a.le.a}
\addcontentsline{toc}{part}{a.le.a}

\img{alea}{Long exposure photos on an airport used for a.le.a}


\begin{itemize}
	\item Description: Flute, Clarinet (Bb), Violin, Cello, Percussion, Piano, and live electronics
	\item Duration:  8 minutes
	\item Category:  Multimedia, Small Ensemble, Live electronics, Generative, Instrumental
	\item Date:   4/7/2017
	\item Location:   The Dimenna Center for Classical Music, NYC
	\item Performers: Talea Ensemble
	\item Video: \url{https://player.vimeo.com/video/222588421} (password: \texttt{a.le.a})
	\item Audio: \url{https://soundcloud.com/ffddcchh/alea}
	\item Documentation: \url{https://github.com/fdch/a.le.a}
	\item Score: \href{https://doc-08-7g-docs.googleusercontent.com/docs/securesc/gh251mh23b531o2ik0kvncqm7cfsk498/m24eka4dgcco8bp9pfj4g8eob927td8p/1555344000000/14838610272939224163/08327717046301529486/0B-mcBjKPCfPLN2VYbTl4UENmeWM}{click here}
\end{itemize}

\section*{Program Notes}

This music was generated using the a.le.a library for Pure Data (Pd) written and played by the composer. The library uses an algorithm for swarm behavior together with the Lorenz attractor in such a way that five `boids' follow the famous chaotic attractor.

The a.le.a library provides an interface to monitor and interact with both the Lorenz and the boids paths, thus adding to the influence of the boids' movements a human element ---that of the player of the meta-score (the composer). The interface captures the results of the interaction of the trio (player, boids, attractor) and outputs the data to both a video file and a text file. The video is used for part of the visual aspect of the performance. The text is used to generate a set of four meta-scores which indicate curves of transition between three elements (in this case, the electronics, the instrumental and the visual).

This algorithmic absurdity in a.le.a aims at being not a `mapping' of the motion of the birds into musical structures, but as the triggering of events and transitions based on some properties of the line that goes through all the boids. The instrumental score is then interpreted from this output into hand-crafted gestural snippets of Lilypond code (via the [notes] external for puredata) which belong to each of the six instruments of the ensemble. This interpretation produces four fullscores (one for each meta-score).

For this performance, the composer hand-trimmed and re-interpreted some absurdly incoherent blobs of ink on the page into absurdly incoherent musical gestures, choosing from all of the four fullscores arriving at a final score.

The electronics and the video are generated live. They both take the third ---human--- element from the live analysis of the player's signals, and they both consist of spectral transformations and granular synthesis.

a.le.a is not a work, it is the unworking of a work. It is not only the performative result in itself, but also the path of its unworking. The audience is invited to listen for these traces.








\part*{Inverted city}
\addcontentsline{toc}{part}{Inverted city}
\img{invertida}{Rage Thormbones and Fede Camara Halac playing Ciudad Invertida at PdCon16\url{~}}


\begin{itemize}
	\item Description: For 2 trombones, live video and electronics.
	\item Duration:  12 minutes
	\item Category:  Multimedia, Trio, Live electronics, Generative, Interactive
	\item Date:   11/22/2016
	\item Location:   ShapeShifter Lab, BKLN, NY
	\item Performers: RAGE THORMBONES and Fede Camara Halac
	\item Video 1: \url{https://www.youtube.com/embed/GA0_ieFjpuM?rel=0&amp;controls=0&amp;showinfo=0}
	\item Video 2: \url{https://player.vimeo.com/video/137084409}
	\item Score: \url{https://drive.google.com/uc?id=0B-mcBjKPCfPLWDA3YVBwb2h1TjQ}
\end{itemize}
\newpage

\section*{Program Notes}
{Ciudad Invertida / Inverted City is a work that seeks to raise questions about chance and interactivity in the multiplicity of the media involved, in relation to inverted images of systems that emerge from the concert experience. Performers read with hyperreal accuracy a hand-written proportional score composed with an essentially incorrect implementation of a zero order markov chain (i.e. an expressionistic interpolation of absurdly distributed Puckette-PRNG sequences). The live signal from one performer triggers sample capturingand resynthesis of the other performer (and viceversa), while also triggering images and image transformations projected over the body of both performers. This work is made completely with puredata and GEM, with one pre-recorded soundscape and 1025 abstracted images that were taken in transit, from vantage points and in different locations of the city of Cordoba, Argentina. The work was premiered in Cordoba during the B3CIM (2015) and it was given a NY premiere during the PdCon16\url{~} (2016) by Rage Thormbones.}






\part*{Inopera}
\addcontentsline{toc}{part}{Inopera}

\img{inopera}{Jeffrey Gavett (baritone from \textit{loadbang} ensemble) performing INOPERA, during his ``I don't know'' solo.}


\begin{itemize}
	\item Description: For Baritone, Trumpet (Bb), Tenor Trombone, Bass Clarinet, live video and electronics
	\item Duration:  43 minutes
	\item Category:  Multimedia, Quartet, Fixed media, Live electronics, Generative, Interactive, Instrumental, Graphic Score
	\item Date:   4/20/2016
	\item Location:   The Dimenna Center for Classical Music, NYC
	\item Performers: loadbang
	\item Video: \url{https://player.vimeo.com/video/168692629}
	\item Score: \url{https://drive.google.com/uc?id=0B-mcBjKPCfPLampKdzZpU2lvSUk}
\end{itemize}
\newpage
\section*{Program Notes}
\texttt{INOPERA this consists of a 40 minute long multimedia experiment involving is fixed and live video, electronics and performance. It questions not the role of the unwork of art and of the artist as real producer.}


\part*{Bio}
\addcontentsline{toc}{part}{Biography}

\section*{Biography}
Fede Cámara Halac studied Licenciatura en Composición Musical at La Universidad Nacional de Córdoba. He is a PhD Candidate in Music Composition & Theory at New York University with Jaime Oliver and Elizabeth Hoffman. His research focuses on Database Music.

\textit{
Fede Cámara Halac estudió Licenciatura en Composición Musical en La Universidad Nacional de Córdoba. Es Candidato de PhD de Music Composition & Theory en New York University con Jaime Oliver y Elizabeth Hoffman. Su investigación se centra en la Música con Base de Datos.
}

\section*{Website}
\url{https://fdch.github.io}

\end{document}
