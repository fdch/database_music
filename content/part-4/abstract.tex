% BIG PICTURE
In this dissertation, I embarked on an adventure throughout a sonic history of databases. Database music has been sounding in computer music, in sonification practices, and in \gls{mir}, in a similar way that has been shining in new media theory in the past decades. This similarity is not only interdisciplinary, and not only at the level of computers, algorithms, and data structures; it is a similarity of a different nature, one that we can call spectral, and that places us and our experience of art at an intersection. It is interdisciplinary because it exists along the edges of our practices, as a shared practice that we have inherited, one that continues to progress as we understand and reconfigure its performativity. It is a technological reflection of ourselves that changes us just as well as we reflect it back and change it. It is spectral because it reconfigures our own notions of what is human and what is nonhuman.

Since computers have changed communications and science in extreme ways in recent decades, our aesthetic experiences have also changed. Music made with databases cannot simply be considered music. If there is something we can take from new media theory is that human experience is mediated by technology. With this in mind, database music has been used as an experiment throughout this dissertation. On one hand, I tested how much of the database we can find in art, in computers, and in sound. On the other, I delved into mediation with terms such as listening, memory, and performance. Both of these experiments found a way to collide towards the end, with a musical approach that focused on how what once was cutting-edge now has become a common practice; not to disregard the latter for the former, but to point to what is in common and how it sounds. This project takes on a granularity that I have tried to hold back, only to find that the more I was finding, the more I needed to write. For this reason, I can suggest that the topic of database music is still largely unexplored. Nevertheless, I covered some of the central questions that I have found in the literature, and tried to pose some of my own. What is the role of the database in art? How has this role been contextualized? Much of the literature focuses on Internet art, digital art, and also on visual and virtual reality, topics of great interest that had to be left out of this text. Instead, I emphasized the structures underlying databases, their histories, and how these have reconfigured our own sense of corporeality in art. In this sense, I have explored how databases have been present and evolved in music practices, and how these practices have undergone significant changes in return. Namely, these changes appear at the level of music software development as well as in the resulting artworks and in the different approaches to music composition. 

Due to the variety of shapes that database music has taken over the years, I focused on what I found to be central questions with which to approach a framework for discussing aesthetics of database music. What is it about database music that we find so fascinating? I ask this question and try to answer it, but I have warned already about my condition of composer in the introduction. A `condition' that began as an obsession for musical acoustic instruments, and for digital instruments, but that has never separated from listening, imagining, and performing sound. In this way, I evaluated how these three terms relate to database music. Listening brought the discussion of databases to resonance, and to a resonance that redefines our notion of self, as well as our notion of community. Imagination brought the database and its relation to memory, which reconfigures how we dream database music, how we remember it, and how we document it. Performance brought the database to a stage, which reconfigures the surfaces upon which we transform at every moment, in every act. With this threefold approach, I suspend a framework for an aesthetic discussion of database music that I contextualize within new media theory as well as music composition; and that I, at once, begin to delineate, but leave incomplete and untethered to the limits of its constitution. Perhaps the reader might find new ways of approaching this discussion, and I hope that by then, the inexhaustibility of the topic would raise new questions about the aesthetics of database music. The final step of this adventure takes on the topic of work within database music, contextualizing musical work in light of this strange hybrid database music. How has the database changed music composition? and, How can we think of database music composition? I have begun answering these questions, but I have not arrived to any conclusions, so this text might perhaps be ``an attempt to incite\dots a provocation before the question'' but ---and I cannot stress this enough---: this is only a incitement in terms of acoustic laws, an aesthetic provocation, and a question that perhaps remains unasked.
