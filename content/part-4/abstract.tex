% BIG PICTURE
In this dissertation, I embarked on an adventure throughout a sonic history of databases. Database music has been sounding in computer music, in sonification practices, and in \gls{mir}, in a similar way that has been shining in new media theory in the past decades. This similarity is not only interdisciplinary, and not only at the level of computers, algorithms, and data structures; it is a similarity of a different nature, one that we can call spectral, and that places us and our experience of art at an intersection. It is interdisciplinary because it exists along the edges of our practices, as a shared practice that we have inherited, one that continues to progress as we understand and reconfigure its performativity. It is a technological reflection of ourselves that changes us just as well as we reflect it back and change it. It is spectral because it reconfigures our own notions of what is human and what is nonhuman.

Since computers have changed communications and science in extreme ways in recent decades, our aesthetic experiences have also changed. Music made with databases cannot simply be considered music. If there is something we can take from new media theory is that human experience is mediated by technology. With this in mind, database music has been used as an experiment throughout this dissertation. On one hand, I tested how much of the database we can find in art, in computers, and in sound. On the other, I delved into mediation with terms such as listening, memory, and performance. Both of these experiments found a way to collide towards the end, with a musical approach that focused on how what once was cutting-edge now has become a common practice; not to disregard the latter for the former, but to point to what is in common and how it sounds. This project takes on a granularity that I have tried to hold back, only to find that the more I was finding, the more I needed to write. For this reason, I can suggest that the topic of database music is still largely unexplored. Nevertheless, I covered some of the central questions that I have found in the litearture, and tried to pose some of my own. What is the role of the database in art? How has this role been contextualized? Much of the literature focuses on Internet art, digital art, and also on visual and virtual reality, topics of great interest that had to be left out of this text. Instead, I emphasized on the structures underlying databases, their history and their structures, and how these have reconfigured our own sense of corporeality in art. 


% very important in the conclusions and in the introduction explains how these ideas work for you as an artist and composer
% How has your composer workflow changed in the last 5 years?
% How to think about memory helps you to think about the idea in a musical way?

% I think what you have to do is draw a conclusion from each chapter
% even if it's a little summary and I'll help you edit it






























