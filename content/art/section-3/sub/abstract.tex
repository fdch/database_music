\img{mir_comp_sonif_interaction}{png}{1}{
	Database performance in relation to computer music practices and their interdisciplinary feedback. The arrows between databases (cylinders) and computers (squares) represent data flow. Left: the database is `visibly next' to the computer, as is the case with \gls{mir}; the two bottom arrows indicate the intervention of the human operator. Right: the database is `visibly below' the computer as is the case with Sonification; the database feeds the computer from an external source (right arrow). Middle: the database is `invisibly behind' the computer, within the softwares used for (and as) music works. The arrows in between the practices represent interdisciplinary feedback.
}

Having discussed the current state of new media theory and the theory of databases and data structures, in this section I theorize the use of databases in relation to sound. To a certain extent, ever since the first computers were used to make music the database has been an invisible partner in the music literature. I argue that by sheding some light to this inherent aspect of computers we can arrive at a clearer notion of how databases sound. Praticularly, by placing the database along a visibility continuum, we may find a reverse relation with audibility: the more invisible the database, the more present its sound. By this I do not argue in favor of neither loudness or quietness. I am only addressing the different possibilities that come from multiple access points to computers. Here I will use the words `database' and `computer' somewhat interchangeably. This decision comes from the fact, as I described in earlier sections, that computers cannot exist without databases. From this, we can further ask ourselves if all computer music is database music. As I hope to demonstrate, there are overt and covert uses of the database, but the database is ubiquitous in all computer practices \fsee{mir_comp_sonif_interaction}. The various disciplines at the intersection of music and computers take each a different approach to databases and, thus, to database performance. In this sense I describe and discuss the scope of actions that comprise database performance within three practices using computers and sound: \gls{mir}, sonification, and computer music.