% This is why I now focus on Database practices before entering the `age of information' (McLuhan 1964, Chion 1994, Manovich 2001, Simons 2002, Vesna 2007, Paul 2008, van Dijck 2017). I define two precursors of the database as `statistics' and `archival practices.' First, I contextualize briefly the practice of statistics, and describe the notions of `sampling' and `data collection,' in order to provide a glance on what I consider the origin of data-based practices. Then, I begin relating the concept of the Archive ---as presented by Derrida (Archive Fever, 1995)---, so as to deepen the understanding of the principles involved in archival practices. I describe the main technical concepts behind Database Navigation and provide use cases from both appendices A and B, the former relating to joint image and audio databases, and the latter to text databases. I then reflect on the quality of this navigation in relation to the type of navigation and results that they obtain.
