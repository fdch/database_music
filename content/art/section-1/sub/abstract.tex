% In this section I engage with the existing literature on data-driven art. In relation to new media, Manovich first considered the Database to be a new `symbolic form' of art \parencite{Man01:The}. The database became a term related to `internet' and `digital' art, and as such it was conceptualized in relation to interface design and interactivity. Hansen provided further insight to Database practices through theories of embodiment \parencite{Han04:New}. He claimed that since Manovich's theorizations did not take into account the Body as an active agent, human-computer interaction so far fell short of reaching its aesthetic potential. Vesna, Daniel and Lovejoy proposed a theorization of context within Database Art \parencite{Ves07:See,Dan07:The,Pau07:The,Kle07:Wai}. In their theorizations of interactivity in Database Art, they reassessed authorship in Digital Art. These authors later provided a panoramic view of database practices within the digital art world, and arrived at the broader term \textit{Database Aesthetics}. Since then, studies on database aesthetics have only been present in the existing literature in terms of data practices.