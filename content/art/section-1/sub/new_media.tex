\begin{quote}
	The world appears to us as an endless and unstructured collection of images, texts, and other data records, it is only appropriate that we will be moved to model it as a database ---but it is also appropriate that we would want to develop the poetics, aesthetics, and ethics of this database. \parencite[219]{Man01:The}
\end{quote}

To point to the origin of the database as it is known today is not an easy task. Certainly, databases are closely related to the history of computers, but they also relate to the history of lists. The common link between these two is the fact that they are written ---on a memory-card, on a page---, which would take its history to the origins of the written word\dots. However, there is a point where the history of storage takes an operational turn. At this point, the `word' becomes a type of data, and data begins to bloom exponentially, impulsing faster and more efficient storage and retrieval technologies. Database systems were modelled hand-in-hand with computer languages and architectures from the late 1950s until the present day, when they continue to be developed for almost all aspects of the business world.

In the artworld of the 1990s, the increasing availability of personal desktop computers ---with software suites, programming languages, and compilers--- resulted in the emergence of new media art. Lev Manovich \parencite{Man01:The} was the first media historian to argue that the database became the center of the creative process in the computer age. The database had become the content and the form of the artwork in \citetitle{Man01:The}. Furthermore, Manovich recognized that the artwork itself had become an interface to a database; an interface whose variability allowed the same content to appear in individualized narratives. Thus, he claimed that narrative and meaning in new media art had been reconfigured differently. Narrative became the trajectory through the database \parencite[227]{Man01:The}, and meaning became tethered to the internal arrangement of data.\footnote{Graham Weinbren writes that a database ``does not present data: it contains data. The data must always be in an arrangement\dots that gives the data its meaning'' \parencite[67-9]{Wei07:Oce}.} Therefore, for Manovich, the ``ontology of the world as seen by a computer'' \parencite[223]{Man01:The} was the symbiotic relationship between algorithms and data structures. As a consequence of the use of databases in art, the architecture of the computer was transferred to culture at large \parencite[235]{Man01:The}. Manovich's `database as symbolic form' thus became a technologically determined shadow that haunted much of new media.
