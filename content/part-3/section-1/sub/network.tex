\begin{quote}
	\dots there is an actor whose definition of the world outlines, traces, delineate, limn, describe, shadow forth, inscroll, file, list, record, mark, or tag a trajectory that is called a network. \textit{No net exists independently of the very act of tracing it, and no tracing is done by an actor exterior to the net. A network is not a thing but the recorded movement of a thing}. \im \parencite[14]{Lat90:On}
\end{quote} % 11/18/2017 10:23:54

Nancy's resonance points to instrumental music listening, specially because of his references to composers like Wagner or Berlioz. However, the case of listening to music with computers can also be understood together with Nancy's resonance. The same principles apply, namely because of the practice of sound recording and reproduction, and how it was developed to replicate wave conditions by way of transducers. In other words, considering modern transducers and digital technology only, sound waves can be reconstructed in such a way that the illusion of sources that are physically absent in one place, can be felt as if they were exciting the medium and body without much complications. This fact alone brings out a plethora of concepts that have been widely accounted for in the literature. However, in order to account for the presence of the database within listening, I focus here on one way of understanding Nancy's resonance that comes from Bruno Latour's actor-network theory \parencite{Lat90:On, Lat93:We}.

\paragraph{An Illusory Violin}
Consider, for example, an acousmatic concert in which one of the music works is made with pre-recorded violin samples. Thus, when this violin begins playing sounds, an illusion may very well begin to emerge, that is, as listeners, we can imagine a violin player. Furthermore, if the imaginary player continues to play sounds and move them in space, this illusion continues in the direction of physical but illusory motion \textit{in-space}, that is, a virtual perception of an actual violin, and an actual violin player. Therefore, this virtuality may very well project itself throughout the complete music work, thus grounding the music work on an imaginary force that is only alive because of the listener's virtuality. The ghostly qualities of this force will be addressed further down this text \see{spectrality}. Most presently is the fact that this magic show ---happening in front and because of the listener's body-sensing mind--- can be understood in terms of a resonating linkage between the human and the nonhuman: a network of interconnected objects that refer to each other. 

\paragraph{Virtuality}
As I described in earlier sections, virtuality is our brain's capacity to create images from the world. In listening, the virtuality of the human mind engages with the attack of the sonorous presence, and with the relationship in self. Therefore, the body listening to itself listening results not only in the self-image of the body, it also creates an image of the listened to the point of embodying a violin player altogether. That is to say, the illusory violin, however imaginary, can be listened to and, thus, it can enter into the infinite game of resonance. In this listening process, therefore, the listening subject exposes itself to itself and to the `virtual' self of the violin. The `virtual' in this context does not mean in opposition to the real. Quite the contrary, the virtual comes as the affective presence of the real, and it becomes the possibility condition for the reality of images. From this, it follows that the self is already an image of itself and, as such, it is a virtuality that comes out of the process of listening.

\paragraph{Performativity of Networks}
Latour understands `networks' as made of interconnected frames of reference. He refers to these frames as `semiotic actors' as well as their accounts of each other. In this sense, the network is comprised entirely of motion and activity. The performativity of semiotic actors is the network's very own movement reflected back onto itself. That is to say, the network itself is the set of links that are being established as nodes, and these nodes are in constant reference to each other. Within a universe of actors, or what Latour calls `ontological hybrids', or world-making \textit{etceteras}, only meaning and connectivity is present, and there is nothing that falls outside the network. The network encompasses its own actors and its own performance. Thus, performance is the defining gesture of networks. Latour's actor-network theory, therefore, is a way of thinking the complex ways in which things are connected between each other. It is an image of a world made out of ontological hybrids which constitute the bare nodes on a decentralized web of meaning. These actors/hybrids build their frames of reference by way of navigating from node to node, thus traversing a network. Within this network, the emphasis is placed on performativity.

\paragraph{A Resonant Movement of a Thing}
Considering this concept of the network as one grounded on movement, and, precisely, on the movement of meaning, there exists the possibility of thinking this movement under the scope of Nancy's ontology of sound. That is to say, given that sound is also composed of the oscillatory motion of particles in space, and, given that this structure of infinite referral and deferrals, then Latour's phrase can be reformulated as follows: \textit{the network is not a thing, but resonant movement of a thing}. Considering the concept of a resonating network of humans and nonhumans, the database not only enters as a node (semiotic actor) in the network, it enters as a resonating subject in its own right. If semiotic actors in Latour's network are in constant reference to each other, it can be argued that they are in infinite resonance with each other, in a permanent state of vibration, or simply, \textit{listening}. Therefore, these listening actors may be not only human, they are listening things that engage with resonance just as much as Nancy's resonant subject. This is the crucial leap that comes out of the idea of a resonant network: the moment the nonhuman in the network is comprehended as resonating, it is the moment that they engage with an approach to self. In other words, when the database of samples used by our virtual violin begins to sound, it is not only being listened to by the audience in the hall, it also begins to listen to itself listening. When considering the database as a resonant subject in itself, the database is granted with creative agency. In other words, the database engages within an emergent (hybrid) community of the human and the nonhuman. Our virtual violin, made of samples in \gls{ram}, makes the network resonate while resonating itself to itself. The hierarchical links between the human and the non begin to disappear in favor of a multiplicity of edges in a networked condition of resonance.

\begin{quote}
	The surface `in between' networks is either connected ---but then the network is expanding--- or non-existing. Literally, a network has no outside. \parencite[6]{Lat90:On}
\end{quote} % 11/19/2017 20:10:57

\paragraph{Positive Feedback}
Connection and expansion are the mode of being of what I am calling resonant networks. Networks exist by the constant growth of their nodes and edges. Therefore, given the self-propagating quality of networks, the notion of a network unlinking itself until extinction is utterly impossible. Although it would be rational to think this way, the case of the network would imply that once a connection is disconnected, another new connection must refer to the unlinked status of the affected nodes. This new connection would, in turn, unfold a new set of connections into newer and yet unexpected directions. Therefore, a sharp distinction with sound arises: while the network is expanding in redundancy and self-reference, sound can be understood as a system that, by its very same propagation attenuates towards an imperceptible threshold. Notwithstanding the infinitesimal motion of media which prevents conceptualizing sound as fading into nothingness, the embodied presence of sound does fade away into a perceptual nothingness. In other words, coupling `resonant' and `network' results in a sort of positive feedback, that unless intentionally built into an electronic circuit serving that purpose, simply falls out of the mechanical aspect of waves. Thus, sonic entropy plays against network expansion.

\paragraph{The Work of Actors}
The notion of network is to a certain extent mathematical. The spatial metaphors that are generally used, such as ``close and far, up and down, local and global, inside and outside,'' come to be replaced by ``associations and connections'' \parencite[6]{Lat90:On}. For example, consider the case of a person writing on a laptop computer. The two actors (user, computer) have to perform what Latour calls ``enormous supplementary work'' \parencite[6]{Lat90:On}. While the computer has to display characters correlating glyphs to keys, the user has to visualize letters out of pixel data on screen. Therefore, the conjunction of the keyboard-to-character database with the user's ability to create images out of a flickering lights results in an inverted proximity: letters on screen appear closer to the eyes than they are to the fingers. Distance and proximity are two concepts based on geography, a concept that within networks tends to give way to connectivity.


