


\paragraph{The Recorded Movement of a Thing}
\textcite{Lat90:On, Lat93:We} developed a theory of networks called \gls{ant}. \gls{ant} can be understood as a way of thinking the complex ways in which things ---understood as every singular \textit{thing} within the world--- are connected between and associated to each other. This theory can be understood as an image of the world made out of nodes on a decentralized web of meaning. As I have mentioned earlier with the network model \see{model:network}, navigating through networks implies traversing from node to node. However, for Latour, navigating through the network means building frames of references. He refers to these frames as both `semiotic actors' ---or `agents'--- and as the interconnected accounts that these semiotic actors have of each other. The motion of these interconnected accounts can thus be understood as a mode of navigation. This is when the (visual) two-dimentional metaphor of a `network' as a `surface' might limit the understanding of its activity. Nodes and edges collapse into one aother. Therefore, since `semiotic actors' are themselves frames of reference, the `surface' of the network folds over itself. In this sense, the network is comprised entirely of motion and activity: ``no net exists independently of the very act of tracing it, and no tracing is done by an actor exterior to the net'' \parencite[14]{Lat90:On}. Meaning and connectivity are enabled by the activity or work of actors. Furthermore, this work, which is the movement of associations and connections, is what best describes the web-like activity (\textit{net}-work) of the network: ``\textit{A network is not a thing but the recorded movement of a thing}'' \im \parencite[14]{Lat90:On}.
 
\paragraph{The Work of Actors}
Furthermore, nothing falls outside the network: ``the surface `in between' networks is either connected ---but then the network is expanding--- or non-existing. Literally, a network has no outside \parencite[6]{Lat90:On}.'' The network encompasses its own actors and its own motion: the set of links that are being established by nodes are in constant reference to each other. Connection and expansion are the mode of being of networks: they exist by the constant growth of their nodes and edges. Therefore, given the self-propagating quality of networks, the notion of a network unlinking itself until extinction is utterly impossible. Although it would be rational to think this way, the case of the network would imply that upon disconnection, another new connection must refer to the unlinked status of the affected nodes. This new connection would, in turn, unfold a new set of connections into newer and yet unexpected directions, and so on. This is where a (sonic) three-dimentional metaphor of the network as a mechanical wave might also differ from its activity. Coupling `resonant' and `network' results in a sort of positive feedback. While a network expands in redundancy and self-reference, a sound attenuates towards imperceptible thresholds. Therefore, in undertanding networks as resonant networks, it follows that resonance would increase at every step. 

\paragraph{Howling}
For example, consider Alvin Lucier's \textit{I am sitting in a room}. This sound art work is self explanatory (see quote above), and it is as a crossfade between speech and music (but also timbre and space) throught the circuitry of a closed and controlled feedback loop between a microphone, a speaker, and a room. The aeshtetic considerations of this work I will leave for some other time, and I will refer to \textcite{icmc/bbp2372.2012.006} for further considerations on feedback systems. What I would like to bring here is an experimental revision of the Larsen effect: ``In every sound reinforcement system, where the loudspeaker and the microphone are placed in the same acoustic environment, a certain amount of the signal radiated by the loudspeaker is fed back into the microphone'' \parencite[11]{Kro11:Aco}. When these systems become unstable, the Larsen effect appears (also refered to as `howling'), ``resulting in a positive feedback producing pitched tones from the iterated amplification of a signal'' \parencite[31]{icmc/bbp2372.2012.006}. Therefore, in Lucier's room, what occurs is the Larsen phenomenon ``stretched in time\doots The room itself acts like a filter'' \parencite[34]{icmc/bbp2372.2012.006}. Considering that there might be a mechanical contradiction in thinking resonant networks, I believe it necessary, then, to expand the `mechanical' side of the system. That is to say, Lucier's \textit{room} needs to be expanding as well. Therefore, the ``resonant frequencies'' of the expanding network cease to ``reinforce themselves.'' However, (and here is the experiment) this does not mean that they would cease to act. In this sense, we can ask ourselves how would Lucier's \textit{I sitting in a room\dots} sound like? \textit{Where} would he be actually \textit{sitting}?

\paragraph{I am sitting in a database\dots}
Considering that networks are comprised on the movement of meaning, it is possible to think about this movement with Nancy's ontology of sound. That is to say, given that we can understand sound as an experience within an interconnected web-like texture that I described earlier as a feedback delay network, with \gls{ant} this 
\dots
then Latour's phrase can be reformulated as follows: \textit{the network is not a thing, but resonant movement of a thing}. If Latour's semiotic actors are in constant reference to each other, it can be argued that they are in resonance with each other, in a permanent state of vibration, or simply, \textit{listening}. Therefore, these listening actors are resonant subjects in Nancy's sense described above. This is the crucial leap that comes out of the idea of a resonant network: the moment the nonhuman in the network is comprehended as resonating, it is the moment that they engage with an approach to self. Following this logical thread, the database can be considered as a semiotic actor as well as a resonating subject. 




