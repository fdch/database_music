\paragraph{The Recorded Movement of a Thing}
Philosopher Bruno \textcite{Lat90:On, Lat93:We} developed a theory of networks called \gls{ant}. \gls{ant} can be understood as a way of thinking the complex ways in which things ---understood as every singular \textit{thing} within the world--- are connected between and associated to each other. This theory can be understood as an image of the world made out of nodes on a decentralized web of meaning. As I have mentioned earlier with the network model \see{model:network}, navigating through networks implies traversing from node to node. However, for Latour, navigating through the network means building frames of references. He refers to these frames as both `semiotic actors' ---or `agents'--- and as the interconnected accounts that these semiotic actors have of each other. The motion of these interconnected accounts can thus be understood as a mode of navigation. This is when the (visual) two-dimentional metaphor of a `network' as a `surface' might limit the understanding of its activity. Nodes and edges collapse into one another. Therefore, since semiotic actors are themselves frames of reference, the `surface' of the network folds over itself. In this sense, the network is comprised entirely of motion and activity: ``no net exists independently of the very act of tracing it, and no tracing is done by an actor exterior to the net'' \parencite[14]{Lat90:On}. Thus, meaning and connectivity are enabled by the activity or work of actors. Furthermore, this work (\textit{net}-work), what Latour calls ``tracing,'' is the movement of associations and connections: ``\textit{A network is not a thing but the recorded movement of a thing}'' \im \parencite[14]{Lat90:On}.
 
\paragraph{Positive Feedback}
Furthermore, nothing falls outside the network: ``the surface `in between' networks is either connected ---but then the network is expanding--- or non-existing. Literally, a network has no outside \parencite[6]{Lat90:On}.'' The network encompasses its own actors and its own motion: the set of links that are being established by nodes are in constant reference to each other. Connection and expansion is the mode of being of networks: they exist by the constant growth of their nodes and edges. Therefore, given the expansive quality of networks, the notion of a network unlinking itself until extinction is utterly impossible. Although it would be rational to think this way, the case of the network would imply that upon disconnection, another new connection must refer to the unlinked status of the affected nodes. This new connection would, in turn, unfold a new set of connections into newer and yet unexpected directions, and so on. This is where a thinking networks in terms of a (sonic) three-dimentional metaphor (as a mechanical wave) might also differ from its activity. Coupling `resonant' and `network' results in a sort of positive feedback. While a network expands in redundancy and self-reference, a sound attenuates towards imperceptible thresholds. Therefore, in undertanding networks as resonant networks, it follows that resonance would increase at every step. 

\paragraph{Howling}
For example, consider \poscite{Luc70:Iam} \citetitle{Luc70:Iam}. This sound art piece is self explanatory \see{lucierlude}. It can be understood as a dual crossfade, a first crossfade between speech and music that gradually shifts into the second one, between timbre and space. Through the circuitry of a closed and controlled feedback loop between a microphone, a speaker, and a room. More considerations of this work I will leave for some other time, and I will refer to \textcite{icmc/bbp2372.2012.006} for further readings on feedback systems. What I would like to bring here is an experimental revision of what is known as the Larsen effect: ``in every sound reinforcement system, where the loudspeaker and the microphone are placed in the same acoustic environment, a certain amount of the signal radiated by the loudspeaker is fed back into the microphone'' \parencite[11]{Kro11:Aco}. When these systems become unstable, the Larsen effect appears (also refered to as `howling'), ``resulting in a positive feedback producing pitched tones from the iterated amplification of a signal'' \parencite[31]{icmc/bbp2372.2012.006}. Therefore, in Lucier's room, what occurs is quite literally the Larsen phenomenon, but ``stretched in time\dots The room itself acts like a filter'' \parencite[34]{icmc/bbp2372.2012.006}. Considering the mechanical contradiction in thinking resonant networks, I believe it necessary, then, to expand the `mechanical' side of the feedback system in question: Lucier's \textit{room} needs to be expanding as well. As a consequence, the ``resonant frequencies'' (nodes) of the expanding network would cease to ``reinforce themselves.'' Furthermore, Latour emphasizes, ``instead of thinking in terms of surfaces ---two dimension--- or spheres ---three dimension--- one is asked to think in terms of nodes that have as many dimensions as they have connections'' \parencite[3]{Lat90:On}. 

However, (and here is the experiment) this does not mean that these resonant nodes would cease to act. 

In this sense, we can ask ourselves how would Lucier's \textit{I sitting in a room\dots} sound like if the room were expanding? \textit{Where} would he be actually \textit{sitting}?

\paragraph{The Resonant Movement of a Thing}
Considering that networks are comprised on the movement of meaning, it is possible to think about this movement with Nancy's ontology of sound. The \gls{fdn} I described earlier would change all its delay lines in continuous expansion. As a result, such a feedback network would redefine the notion of a temporal delay into a \textit{spatial} delay. Instead of the Larsen effect being ``spread in time,'' in Lucier's work it would also spread in space. The room as a filter would resonate differently because it would be understood as a texture of referrals and deferrals, a networked resonance. Latour's phrase can be reformulated: \textit{the network is not a thing, but the resonant movement of a thing}. If Latour's semiotic actors are in constant reference to each other, it can be argued that they are in resonance with each other, in a permanent state of vibration, or simply, \textit{listening}.

\paragraph{I am sitting in a database\dots}
This is the crucial leap that comes out of the idea of a resonant network: the moment the nonhuman in the network is comprehended as resonating, it is the moment that they engage with an approach to self. Following this logical thread, a database can be considered as a semiotic actor as well as a resonating subject. On one hand, databases are not networks, they are agents: acting, tracing, and listening within a network. On the other hand, since databases are indeed listening, to what extent can we think of them as listening to themselves listening? Bringing back Lucier and his room, I would like to address this question with another aspect of the work. (And by `work' I begin to introduce an important aspect of this dissertation, a concept that embraces activity, productivity, but also product, and objects: operativity and opus.) There is indeed a third crossfade, between the `I' in the text, and the `I' in the voice that reads it. The simplest way to approach this is by asking ourselves, if after recording the first input signal Lucier remained seated \textit{in} the room or not. This is a difference that cannot be approached from the recording itself because it is inaudible. I will refer to this difference further down this text. For now I point to the fact that the moment Lucier recorded his voice, the `I' began residing in the loop. I believe this is the most crucial ``irregularity.'' In the interlude at the beginning of this chapter, I transcribe the text as Lucier reads it. I attempted to be as clear as possible, crossing out, replacing, extending all the consonants into what I thought was a more faithful score for the read fragment. Thinking as a composer, this score (with all its notated irregularities) would explain the first minutes so fiercefully that the mystery of the last minutes would be solved. But, the most crucial aspect of the piece cannot be rendered in symbolic transcription, because the `I' is somewhere in between the transcribed and the inscribed \see{spectrality}. This `I' is what is at stake when databases begin to resonate, that is, it is the approach to this notion of `self' what begins to redefine ourselves in general. That is to say, within the resonant network we face a `self' that changes our own notion of `self' in general. In this sense, a `self' \textit{sitting in a database} returns to us (resonates back) putting into question a relationship (a difference): what is the difference between the two `I's? Is is this same difference at stake between the human and the nonhumam? The implications of these question I will move forth in the remaining sections of this dissertation. However, the most present step is analyzing the conjunction that the two clauses of the question points to: the sharing of the `I,' an exposure of community.

