\paragraph{The Recorded Movement of a Thing}
Philosopher Bruno \textcite{Lat90:On, Lat93:We} developed a theory of networks called \gls{ant}. \gls{ant} can be understood as a way of connecting and associating entities to one another. It is a tool that builds an image of the world made of nodes along a decentralized web of meaning. Latour reformulates nodes and edges, with what he calls `semiotic actors,' `actants,' or `agents' (nodes) and of the interconnected accounts that these have of each other (edges). As I have mentioned earlier with the network model in databases \see{model:network}, navigating through networks is traversing from node to node. However, the (visual) two-dimentional metaphor of a `network' as a `surface' limits the understanding of its topology: ``instead of surfaces one gets filaments.'' \parencite[3]{Lat90:On} In this sense, he points to a misunderstanding that comes from giving \gls{ant} a technical definition such as the one described with the database model: ``nothing is more intensely connected, more distant, more compulsory and more strategically organized than a computer network'' \parencite[2]{Lat90:On}. \gls{ant} points towards a topological shift. Nevertheless, the navigational paradigm advanced by \textcite{Bachman:1973:PN:355611.362534} in relation to databases whose `keys' become n-dimensional space, does translate well to Latour's model, because nodes have ``as many dimensions as they have connections'' \parencite[3]{Lat90:On}. In any case, what this navigation points to is that \gls{ant} is comprised entirely of motion and activity: ``no net exists independently of the very act of tracing it, and no tracing is done by an actor exterior to the net'' \parencite[14]{Lat90:On}. Thus, meaning and connectivity are enabled by the activity or work of actors: ``In order to explain, to account, to observe, to prove, to argue, to dominate and to see, [an observer] has to move around and work, I should say it has to `network''' \parencite[13]{Lat90:On}. This work, \textit{net}-work, or `tracing,' is not only the movement of associations and connections, it is also the `recording' of this movement. In this sense, Latour claims that ``a network is not a thing but the recorded movement of a thing'' \parencite[14]{Lat90:On}. Furthermore, nothing falls outside the network: ``the surface `in between' networks is either connected ---but then the network is expanding--- or non-existing. Literally, a network has no outside \parencite[6]{Lat90:On}.'' The network encompasses its own actors and its own expansive motion. Most importantly, \gls{ant} is a tool aimed at describing the nature of society. However, in this description, \gls{ant} ``does not limit itself to human individual actors but extend[s] the word actor\dots to non-human, non individual entities'' \parencite[2]{Lat90:On}.

\paragraph{Howling}
Thinking networks in terms of a (sonic) three-dimentional metaphor (as a mechanical wave) is thus misunderstanding it. Coupling `resonant' and `network' results in a sort of (impossible) positive feedback. While a network expands in redundancy, overflow, accumulation, and self-reference, a sound attenuates towards imperceptible and infinitesimal thresholds. Lending an ear to the sound of \gls{ant} we would find ourselves listening to expanding filaments. However, as an acoustic experiment that would combine the circuitry of a feedback with the accumulative quality of networks, I propose to consider \poscite{Luc70:Iam} \citetitle{Luc70:Iam}. I have transcribed this sound art piece at the beginning of this chapter, as it is self explanatory \see{lucierlude}. It can be understood as a triple crossfade, first between speech and music, gradually crossfading into a second crossfade, between timbre and space. Through the circuitry of a closed and controlled feedback loop between a microphone, a speaker, and a room. More considerations of this work I will leave for some other time, and I will refer to \textcite{icmc/bbp2372.2012.006} for further readings on feedback systems. What I would like to bring here is an experimental revision of what is known as the Larsen effect: ``in every sound reinforcement system, where the loudspeaker and the microphone are placed in the same acoustic environment, a certain amount of the signal radiated by the loudspeaker is fed back into the microphone'' \parencite[11]{Kro11:Aco}. When these systems become unstable, the Larsen effect appears (also refered to as `howling'), ``resulting in a positive feedback producing pitched tones from the iterated amplification of a signal'' \parencite[31]{icmc/bbp2372.2012.006}. Therefore, in Lucier's room, what occurs is quite literally the Larsen phenomenon, but ``stretched in time,'' and thus the ``room itself acts like a filter'' \parencite[34]{icmc/bbp2372.2012.006}. Considering the mechanical contradiction in thinking resonant networks, I believe it necessary, then, to expand the `mechanical' side of the feedback system in question: Lucier's \textit{room} needs to be expanding as well. As a consequence, the ``resonant frequencies'' (nodes) of the expanding network would cease to ``reinforce themselves.'' However, (and here is the experiment) this does not mean that these nodes would cease to act, let alone resonate. In this sense, we can ask ourselves how would \textit{this} sound like? \textit{Where} would the `I' be actually \textit{sitting}?

\paragraph{The Resonant Movement of a Thing}
Such a feedback network would redefine the notion of a temporal delay into a \textit{spatial} delay. Instead of the Larsen effect being ``spread in time,'' in Lucier's work it would also spread through space. The room as a filter would resonate differently because it would be understood as a texture, a networked resonance. If Latour's semiotic actors are in constant reference to each other, it can be argued that they are in resonance with each other, in a permanent state of vibration, or simply, \textit{listening}. Thus, Latour's phrase can (perhaps) be reformulated: \textit{the network is not a thing, but the resonant movement of a thing}. 

\paragraph{I am sitting in a database\dots}
This is the crucial leap that comes out of the idea of a resonant network: the moment the nonhuman in the network is comprehended as resonant, it is the moment that they engage with an approach to self (in Nancy's terms). Following this logical thread, a database can be considered as a semiotic actor as well as a resonant subject. On one hand, databases are not just networks, they are actor-networks: acting, tracing, and listening. On the other hand, since databases are indeed listening, to what extent can we think of them as listening to themselves listening? Bringing back Lucier and his room, I would like to address this question with another aspect of this work. (And by `work' I begin to introduce an important aspect of this dissertation, a concept that embraces activity, productivity, but also product, and objects: operativity and opus.) There is indeed a fourth crossfade, between the `I' in the text, and the `I' in the voice that reads it. The simplest way to approach this is by asking ourselves, if after recording the first input signal Lucier remained seated \textit{in} the room or not. This is a difference that cannot be approached from the recording itself because it is inaudible. I will refer to this difference further down this text. For now I point to the fact that the moment Lucier recorded his voice, the `I' began residing in the loop. I believe this is one of the most crucial `irregularities' that can be found throughout the work. In the interlude at the beginning of this chapter, I transcribe the text as Lucier reads it. I attempted to be as clear as possible, crossing out, replacing, extending all the consonants into what I thought was a more faithful score for the read fragment. Thinking as a composer, this score (with all its notated irregularities) would explain the first minutes so fiercefully that the mystery of the last minutes would be solved. But, the most crucial aspect of the piece cannot be rendered in symbolic transcription, because the `I' is somewhere in between the transcribed and the inscribed \see{spectrality}. This `I' is what is at stake when databases begin to resonate, that is, it is the approach to this notion of `self' what begins to redefine ourselves in general. That is to say, within the resonant network we face a `self' that changes our own notion of `self' in general. In this sense, a `self' \textit{sitting in a database} returns to us (resonates back) putting into question a relationship (a difference): what is the difference between the two `I's? Is is this same difference at stake between the human and the nonhumam? The implications of these question I will move forth in the remaining sections of this dissertation. However, the most present step is analyzing the conjunction that the two clauses of the question points to: the sharing of the `I,' an exposure of community.

