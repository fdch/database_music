
\paragraph{The Recorded Movement of a Thing}
\textcite{Lat90:On, Lat93:We} developed a theory of networks called \gls{ant}. It can be understood as a way of thinking the complex ways in which things ---understood as every singular \textit{thing} within the world--- are connected between each other. In this sense, it is an image of a world made of nodes on a decentralized web of meaning. As I have mentioned earlier with the network model \see{model:network}, navigating through networks implies traversing from node to node. For Latour, navigating through the network means building frames of references. He refers to these frames as both `semiotic actors' ---or agents--- and the interconnected acounts that these semiotic actors have of each other. The motion of these interconnected accounts can thus be understood as a mode of navigation. However, since `semiotic actors' are themselves frames of reference, the `surface' of the network folds over itself. In this sense, the network is comprised entirely of motion and activity: ``no net exists independently of the very act of tracing it, and no tracing is done by an actor exterior to the net. \textit{A network is not a thing but the recorded movement of a thing}'' \im \parencite[14]{Lat90:On}. Within this movement of actors, only meaning and connectivity is present, and there is nothing that falls outside the network: ``the surface `in between' networks is either connected ---but then the network is expanding--- or non-existing. Literally, a network has no outside \parencite[6]{Lat90:On}.'' The network encompasses its own actors and its own motion: the set of links that are being established by nodes are in constant reference to each other.

% \paragraph{The Work of Actors}
% The notion of network is to a certain extent mathematical. The spatial metaphors that are generally used, such as ``close and far, up and down, local and global, inside and outside,'' come to be replaced in \gls{ant} by ``associations and connections'' \parencite[6]{Lat90:On}. For example, consider the case of a person writing on a laptop computer. The two actors (user, computer) have to perform what Latour calls ``enormous supplementary work'' \parencite[6]{Lat90:On}. While the computer has to display characters correlating glyphs to keys, the user has to visualize letters out of pixel data on screen. Therefore, the conjunction of the keyboard-to-character database with the user's ability to create images out of a flickering lights results in an inverted proximity: letters on screen appear closer to the eyes than they are to the fingers. Distance and proximity are two concepts based on geography, a concept that within networks tends to give way to connectivity.



\paragraph{The Resonant Movement of a Thing}
Considering that networks are comprised on movement, and, precisely, on the movement of meaning, there exists a possibility of thinking this movement under the scope of Nancy's ontology of sound. That is to say, given that we can understsand sound as an experience within an interconnected web-like texture of delays and distinctions, then Latour's phrase can be reformulated as follows: \textit{the network is not a thing, but resonant movement of a thing}. If semiotic actors in Latour's network are in constant reference to each other, it can be argued that they are in infinite resonance with each other, in a permanent state of vibration, or simply, \textit{listening}. Therefore, these listening actors may be not only human, they are resonant subjects in Nancy's sense described above. In this sense, semiotic actors within the network can be considered resonant agents. This is the crucial leap that comes out of the idea of a resonant network: the moment the nonhuman in the network is comprehended as resonating, it is the moment that they engage with an approach to self. The database can be considered, therefore, as a semiotic actor as well as a resonating subject. 





% From this, it follows that when the database of samples used by our (virtual) violin begins to sound, it is not only being listened to by the audience in the hall, it also begins to listen to itself. 


% the self is already an image of itself and, as such, it is a virtuality that comes out of the process of listening.







% In other words, 






% \paragraph{Positive Feedback}
% Connection and expansion are the mode of being of networks: they exist by the constant growth of their nodes and edges. Therefore, given the self-propagating quality of networks, the notion of a network unlinking itself until extinction is utterly impossible. Although it would be rational to think this way, the case of the network would imply that upon disconnection, another new connection must refer to the unlinked status of the affected nodes. This new connection would, in turn, unfold a new set of connections into newer and yet unexpected directions, and so on. Therefore, a sharp distinction with sound arises: coupling `resonant' and `network' results in a sort of positive feedback. While the network is expanding in redundancy and self-reference, sound can be understood as a system that, by its very same propagation attenuates towards an imperceptible threshold. Notwithstanding the infinitesimal motion of media which prevents conceptualizing sound as fading into nothingness, the embodied presence of sound does fade away into a perceptual nothingness. Therefore, in undertanding networks as resonant networks, it follows that resonance would increase at every step

















% Nancy's resonance points to instrumental music listening, specially because of his references to composers like Wagner or Berlioz. However, the case of listening to music with computers can also be understood together with Nancy's resonance. The same principles apply, namely because of the practice of sound recording and reproduction, and how it was developed to replicate wave conditions by way of transducers. In other words, considering modern transducers and digital technology only, sound waves can be reconstructed in such a way that the illusion of sources that are physically absent in one place, can be felt as if they were exciting the medium and body without much complications. This fact alone brings out a plethora of concepts that have been widely accounted for in the literature. However, in order to account for the presence of the database within listening, I focus here on one way of understanding Nancy's resonance that comes from Bruno Latour's actor-network theory \parencite{Lat90:On, Lat93:We}.

% or world-making \textit{etceteras}.


