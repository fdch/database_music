\begin{quote}
	Community necessarily takes place in what Blanchot has called `unworking,' referring to that which, before or beyond the work, withdraws from the work, and which, no longer having to do either with production or with completion, encounters interruption, fragmentation, suspension. Community is made of the interruption of singularities, or of the suspension that singular beings are. Community is not the work of singular beings, nor can it claim them as its works, just as communication is not a work or even an operation of singular beings, for community is simply their being ---their being suspended upon its limit. \textit{Communication is the unworking of work that is social, economic, technical, and institutional}. \im \parencite[31]{Nan91:The} 
\end{quote} % 4/3/2018 10:01:12

% 
% 
% EH:
% Why use the term ‘work’ in the first place? Explain a bit.
% 
% 
% 

\paragraph{Community as unwork}
Referring to Maurice Blanchot's concept of \textit{desoevrement}, \textcite{Nan91:The} proposed `unworking' or `inoperativity' as the grounds for a theory of `community.' He claims that community and the possibility of community to occur, needs to be liberated from the concept of `work.' Work within a community, understood by Nancy, comes to be a form of withdrawal. That is to say, in a community, withdrawal is a form of interruption and suspension of both production and product. Therefore, if work is understood as the means by which a community arrives at the production of any given product, then withdrawal comes to place this system of production into a halt. If the process of production was thus interrupted, the system of production leading to completion is truncated and never achieved. Therefore, within this logic of production, if community itself is thought of as product, that is, as the objective of the community itself (the process which composes itself as process), then the concept of community is dismantled and broken. In other words, when community is thought of as work, the work of community can never be achieved, and community as its product can never exist. The reasons for this I will show in a moment. Nancy's conclusion is that community can never result out of a work, but it is only something that unfolds as unworking. In this sense, the system of production is rendered unviable for a relation to exist between selves, and only an inoperative force can arrive at the common exposure of selves.

\paragraph{Resonant Inoperativity}
Given that Nancy thinks of community as an exposure of selves towards one another, we can understand this movement of unworking as the movement of resonant networks. The exposure that occurs in listening allows a comparison between resonance and inoperativity. In this sense, the distance between the resonant subject and the sonorous presence becomes a limit upon which community is suspended. Nancy's concept of resonance, as I described earlier, has two dimensions, one relating the body sensing itself and the other to the structure of infinite referrals and deferrals. Listeners are exposed to one another and in contact with themselves through these dimensions. Resonance exposes thus a liminality: the listening subject exposing itself at a limit.

\paragraph{Space of Community}
This limit is also an exposure to fragmentation. Fragmentation, in this sense, means the inability for a thing to complete itself. The opening of a wound that precludes the unity. As the fragile, liminal state of fracture that this fragmentation points to, however, the limit maintains itself in suspension. That is to say, the contact that exists between selves does not conclude. Therefore, if this touch, in the sense that touching permits being in common with the other, is the mark of community, it follows that it is impossible to arrive at it by means of a work. Since, the concept of work comes from the necessity to finish things, to close objects, to sever a beginning and an ending out of a temporal and spatial continuum. The activity of work is aimed at effectively arriving at a result, applying certain effort for the purpose of a given task. Labor itself is both related to the application of forces and to creation itself, to giving birth. Art traditionally comes in the form of work. Community, on the other hand, comes from the form of unwork.

\paragraph{At the Limit}
Nancy's concept of community can be recognized within `resonance.' Given the fact that Nancy's ontology of sound points to the distance or the interval between sense and signification, and thus, to the emergence of a resonant subject during the sonorous presence, this distance can be thought of as suspended at the limit. A limit, in the sense that it constitutes an edge between two objects (actors). Using Nancy's conceptualization of community, this limit exposes selves to themselves and to one another. Therefore, by understanding resonant networks in terms of community, the result is a resonant self-exposure of the human and nonhuman at the limit. Thus, the instance of inoperativity: because of this (resonant) suspension, there is no possibility for completion, only expansion. Within this quality of incompleteness, which relates to suspension, but also to fracture, fragility, instability, and unpredictability, is how the notion of community as product can never be realized.

\paragraph{Reticulated Skin}
Since there is nothing outside of community, and, since there is only exposure, propagation, repetition, and expansion, there is room for thinking of the resonant network as an inoperative agent of community itself. What this amounts to is that the human and the nonhuman resonate as community. The human and the nonhuman unfold their relations towards each other, suspending themselves in between one another. This in-between-ness is not to mean a gap between selves, but their connectedness and the same network of associations and referrals that exists between them. To put this differently, this liminality can be thought of as a skin, not in the sense of a layer that separates the interior form the exterior of a body, or, for that matter, as a surface under which or over which two selves can connect. This skin is not a surface, but a texture; it is not a layer, but an interweaving of minuscular threads that, in their own locality are fragile, but in their state of being reticulated, expand into a redundancy of fragilities that prevent concepts such as unity, concentration, or purity, to enter into the picture. This constitutes Latour's ``material resistance argument,'' for example, one that refers to the heterogeneous, disseminated, and careful ``plaiting of weak ties'' \parencite[3]{Lat90:On}.

\paragraph{Database Community}
The resistance of the skin represents, thus, the resistance of connectivity itself, that is, the resistance of a community. Therefore, by substituting, on one hand, the human with the database performer (databaser) and, on the other, the nonhuman with the database, the resonant network, as an instantiation of a process of unworking, enables the thought of a \textit{database community} to emerge. With this concept of database community, the notion of database music can be further understood as a hybridly social and communicative event. This is to say that, following Latour's hybridity of objects in his understanding of society \parencite[2]{Lat90:On}, database music is a social practice that comes as a result of databasers and databases in resonance with each other. Furthermore, database music is an instance of community, in the sense that it is an event of communication, understanding the communicative as ``the unworking of work that is social, economic, technical, and institutional'' \parencite[31]{Nan91:The}. Therefore, the database community emerges as ---but also, from, through, during, etc.--- the unworking of databasers and databases. Database community can be considered as, precisely, the skin upon which the human and the nonhuman resonate, which amounts to ---but never finalizes in--- a musical event.
