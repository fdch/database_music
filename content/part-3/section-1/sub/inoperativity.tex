\begin{quote}
	Communication is the unworking of work that is social, economic, technical, and institutional. \parencite[31]{Nan91:The} 
\end{quote} % 4/3/2018 10:01:12

\paragraph{Community as unwork}
Referring to Maurice Blanchot's concept of \textit{desoevrement}, \textcite{Nan91:The} proposed `unworking' or `inoperativity' as a way to understand `work' within an ontology of `community.' He claims that community and the possibility of community to occur, needs to be liberated from the concept of `work,' understood as the means by which a community arrives at the production of any given product. 

A `self,' for Nancy, is made of ``the singular occurrences of a state, a tension, or, precisely, a `sense''' \parencite{Nan07:Lis}. Since these occurrences are in a permanent state of occurring, that is, never fixated in a whole, he describes singular beings as `interrupted' or `suspended.' Extending this definition of the self to the definition of community, then, ``community is made of the interruption of singularities, or of the suspension that singular beings are'' \parencite[31][All subsequent quotes from this passage]{Nan91:The}. This is why community is ontological, and not teleological. In other words, if community is thought of as a product of the work of `selves' (teleologically), it follows that selves could also become the product of community. Nancy thus underscores that ``community is not the work of singular beings, nor can it claim them as its works.'' The symmetry of this teleology also relates to communication. Understood as the being in common of singularities, communication is also ``not a work, or even an operation of singular beings.'' Since community is ontological, Nancy understands it as the being of singular beings ``suspended upon its limit.'' Therefore, within this ontology of community, how is it possible for us to speak of the `work' of `selves' and of community if `work' is something that does not enter into its definition? Furhtermore, how is the concept of the work of art redefined or framed within this ontology? Nancy's conclusion is that community can never result out of `work,' but it is something that unfolds as `unworking.' `Unwork' is work withdrawing from itself: ``that which before or beyond the work, withdraws from the work.'' That is to say, Nancy points to a moment in which operativity separates from itself, and by that gesture, it distinguishes itself from both production and from completion: it becomes incomplete. He writes: ``no longer having to do either with production or with completion, encounters interruption, fragmentation, suspension.'' Is this not the resonance of a return? An ontology of unworking that finds itself upon withdrawing from itself?

% \paragraph{At the Limit}
% Nancy's concept of community can be recognized within his later and broader concept of `resonance.' Given the fact that Nancy's ontology of sound points to the distance or the interval between sense and signification, and thus, to the emergence of a resonant subject during the sonorous presence, this distance can be thought of as suspended at a limit. We can think of this limit as an edge between two objects (actors) in the resonant network. Using Nancy's conceptualization of community, this limit exposes selves to themselves and to one another. 
% % work on this sentences more
% Therefore, by understanding resonant networks in terms of community, the result is a resonant self-exposure of the human and nonhuman at the limit. Thus, the instance of inoperativity: because of this (resonant) suspension, there is no possibility for completion, only expansion. 


% % This is a good way to think about fracture. IT makes me think you don;t need a whole section on it. however, the structure of this sentence is weird: "within x, which relates to y, is how z is not"

% Within this quality of incompleteness, which relates to suspension, but also to fracture, fragility, instability, and unpredictability, is how the notion of community as product can never be realized.



% this liminality can be thought of as a skin, not in the sense of a layer that separates the interior form the exterior of a body, or, for that matter, as a surface under which or over which two selves can connect. This skin is not a surface, but a texture; it is not a layer, but an interweaving of minuscular threads that, in their own locality are fragile, but in their state of being reticulated, expand into a redundancy of fragilities that prevent concepts such as unity, concentration, or purity, to enter into the picture. 











% (the process which composes itself as process), 











% \paragraph{Resonant Inoperativity}
% Given that Nancy thinks of a community as an exposure of selves towards one another, we can understand this movement of unworking as the movement of resonant networks. 

% Listeners are exposed to one another and in contact with themselves through these dimensions. Resonance exposes thus a liminality: the listening subject exposing itself at a limit.

% \paragraph{Space of Community}
% This limit is also an exposure to fragmentation. Fragmentation, in this sense, means the inability for a thing to complete itself. The opening of a wound that precludes the unity. As the fragile, liminal state of fracture that this fragmentation points to, however, the limit maintains itself in suspension. That is to say, the contact that exists between selves does not conclude. Therefore, if this touch, in the sense that touching permits being in common with the other, is the mark of community, it follows that it is impossible to arrive at it by means of a work. Since, the concept of work comes from the necessity to finish things, to close objects, to sever a beginning and an ending out of a temporal and spatial continuum. The activity of work is aimed at effectively arriving at a result, applying certain effort for the purpose of a given task. Labor itself is both related to the application of forces and to creation itself, to giving birth. Art traditionally comes in the form of work. Community, on the other hand, comes from the form of unwork.


% % the [problem with the three paragraphs before, for me, are a couple. 
% % 
% % work seems to be used in 
% % both the sense of 
% % work of art, 
% % and of labor. 
% % 
% % This makes it confusing. 
% % 
% % 
% % Saying expressly what you mean is important, 
% % 
% % but also using the right articles. 
% % “a work” (an object, a thing) 
% % vs 
% % “doing work” or “working” (an action, an occupation).
% % 
% % 
% % Give an example! 
% % this might clear things more than just explaining theoretical concepts! 
% % 
% % the application of your model to the databaser and the database 
% % feels dissociated from the model you are proposing 
% % because it only shows up 
% % in the end of the section. 
% % 
% % Could you prepare this earlier? 
% % give more examples? 
% % 
% % 
% % 
% % what does it mean that a databaser and a database resonate with each other???









% \paragraph{Reticulated Skin}



% % unless you are going to explain it and use it later, why drop it?
% % 
% % This constitutes Latour's ``material resistance argument,'' for example, one that refers to the heterogeneous, disseminated, and careful ``plaiting of weak ties'' \parencite[3]{Lat90:On}.




% \paragraph{Database Community}

% % Why did we talk about a skin at all if you just leave the idea there? 
% % What is the purpose of the skin in your model? 
% % how does it connect to the idea of databasers below?
% % why does the community has a resistance?

% The resistance of the skin represents, thus, the resistance of connectivity itself, that is, the resistance of a community. 









% {
	

% 	In the resonant network model I am proposing here database performers (or databasers) are human actors in the network, while nonhuman actors include the database. 

% % because the preceding discussion is so abstract, that is, 
% % lacking examples, 
% % it seems hard to understand other things. 
% % 
% % Where do the things that databases are made of fall into this model? 
% % where are the sounds, images, algorithms placed here?


% 	The resonant network, as an instantiation of a process of unworking, enables the thought of a database community to emerge. This community consists of ……………...
% }

% Therefore, by substituting, on one hand, the human with the database performer (databaser) and, on the other, the nonhuman with the database, the resonant network, as an instantiation of a process of unworking, enables the thought of a \textit{database community} to emerge. 



% In a database community, database music can be understood as a hybridly social and communicative event. 

% This is to say that, following Latour's hybridity of objects in his understanding of society \parencite[2]{Lat90:On}, database music is a social practice that comes as a result of databasers and databases in resonance with each other. 

% % Can you give examples? 
% % how do they resonate in practice? 
% % what does it mean to resonate with each other??



% Furthermore, database music is an instance of community, 

% % this suggests that 
% % database music = database community. 
% % shouldn't database music be a community event instead per your earlier definition?


% in the sense that it is an event of communication, understanding the communicative as ``the unworking of work that is social, economic, technical, and institutional'' \parencite[31]{Nan91:The}. 

% % What this means is that …. explain in lay terms the “unworking of work”...


% Therefore, the database community emerges as the unworking of databasers and databases. 

% Database community can be considered as, precisely, the skin 

% % ok. 
% % I see the return of the idea of a skin. 
% % the problem I have with a lot of these ideas is that 
% % terms are used multiple times in multiple ways. 
% % 
% % if you have established the notion of 
% % a network that is made of human and non-human actors, 
% % and that these actors resonate through each other, 
% % the idea of a skin is confusing.


% upon which the human and the nonhuman resonate, which amounts to ---but never finalizes in--- a musical event.
