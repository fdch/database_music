\paragraph{Community as unwork}
As I have described above, for Nancy, a resonant self (the `self' from now on) is made of ``the singular occurrences of a state, a tension, or, precisely, a `sense''' \parencite[8]{Nan07:Lis}. Since these occurrences are in a permanent state of occurring, that is, never fixated in a whole (in `tension'), we can understand singular beings as being `interrupted' or `suspended.' These descriptions come from an earlier text \textcite{Nan91:The}, in which he extends this definition of the self to the definition of a community: ``community is made of the interruption of singularities, or of the suspension that singular beings are'' \parencite[31][All subsequent quotes from this passage.]{Nan91:The}. This is why we can understsand `community' as ontological (grounded on a nature of being), and not as teleological (grounded on a purpose or an objective). In other words, if community is thought of as a product of the work of `selves' (teleologically), it follows that selves could also become the product of community. Nancy thus underscores that ``community is not the work of singular beings, nor can it claim them as its works'' \fsee{work}. Since community is ontological, Nancy understands it as the being of singular beings ``suspended upon its limit.'' Therefore, within this ontology of community, how is it possible for us to speak of the `work' of `selves' and of community if `work' is something that does not enter into its definition? Furhtermore, how is the concept of the work of art redefined or framed within this ontology? Nancy's conclusion is that community can never result out of `work,' but it is something that unfolds as `unworking.' Borrowing from Maurice Blanchot's concept of \textit{desoevrement}, Nancy proposes `unworking' or `inoperativity' as a way to understand `work' within an ontology of community. `Unwork' is work withdrawing from itself: ``that which before or beyond the work, withdraws from the work.'' That is to say, Nancy points to a moment in which operativity separates from itself, and by that gesture, it distinguishes itself from both production and from a whole, or a finished product: ``no longer having to do either with production or with completion, encounters interruption, fragmentation, suspension.'' Is this not the resonance of a return? Work returning as the interrupted resonance of its own unworking?

\img{work}{0.5}{
	A graph displaying the teleology of the work (arrows) of selves (dots) in community (dots joined by arrows), in relation to the first two models with which databases were designed: hierarchical (right) and network (left).
}{Community as work}


\paragraph{At the Limit}
Nancy's concept of community can be recognized within his later and broader concept of `resonance.' Given the fact that Nancy's ontology of sound points to the distance or the interval between sense and signification, and thus, to the emergence of a resonant subject during the sonorous presence, this distance can be thought of as suspended at a limit. We can think of this limit as an edge in the resonant network that, in Nancy's terms, exposes selves to themselves and to one another. Further, Nancy provides us with an essential insight, suggesting that ``it is not obvious that the community of singularities is limited to `man' and excludes, for example, the `animal.''' \parencite[28]{Nan91:The} Therefore, by understanding resonant networks in terms of community we can speak of an exposure of selves, or a self-exposure, between humans and nonhumans in a liminality can be thought of as a skin. Not in the sense of a layer that separates the interior form the exterior of a body, or, for that matter, as a surface under which or over which two selves can connect. For this would would make us fall out of Latour's definition of the network. This skin is not a surface, but it can be thought of as a texture; not a layer, but an interweaving of minuscular threads that, in their own locality are fragile, but due to their reticulated structure expand into a redundancy of fragilities that prevents concepts such as unity, concentration, or purity, to enter into the picture. Within this quality of incompleteness, which relates to suspension, but also to fracture, instability, and unpredictability, is how the symmetry of this teleology of work also relates to communication. Understood as the being in common of singularities: ``communication is the unworking of work that is social, economic, technical, and institutional \parencite[31]{Nan91:The}. I am tempted here to suggest that it communication, as well as community, is also the unworking of work that is resonant.


\img{unwork}{0.2}{
	A graph displaying community as an ontology of unwork.
}{Community as unwork}


(A process which composes itself as process), 

% for Nancy ``there is no communion of singularities in a totality superior to them and immanent to their common being'' \parencite[28]{Nan91:The}. 


In the network model, the nodes within the network are connected and associated to each other, that is, the `work' of agents in Latour's terminology. As I describe earlier, Latour's network encompasses the human and the nonhuman: 

Agency, therefore, is what is in common across all nodes. However, a crucial distinction appears if we understand this agency as a form of `networking,' that is, as the way this web of associations and connections works: 
In other words, consider the graph that I include here \fsee{work}. 

``extend the word actor -or actant- to non-human, non individual entities'' \parencite[2]{Lat90:On}.

``The notion of network allows us to lift the tyranny of social theorists and to regain some margin of manoeuvers between the ingredients of society ---its vertical space, its hierarchy, its layering, its macro scale, its wholeness, its overarching character--- and how these features are achieved and which stuff they are made of'' \parencite[5]{Lat90:On}.

``how these features are achieved and which stuff they are made of.The notion of network, in its barest topological outline, allows us already to reshuffle spatial metaphors that have rendered the study of society-nature so difficult: close and far, up and down, local and global, inside and outside. They are replaced by associations and connections (which AT does not have to qualify as being either social or natural or technical as I will show below) . This is not to say that there is nothing like “macro” society, or “outside” nature as the AT is often accused to, but that in order to obtain the effects of distance, proximity, hierarchies, connectedness, outsiderness and surfaces, an enormous supplementary work has to be done. This work however is not captured by the topological notion of network no matter how sophisticated we wish to make it.'' \parencite[6]{Lat90:On}

``. An actor-network is an entity that does the tracing and the inscribing. It is an ontological definition and not a piece of inert matter in the hands of others, especially of human planners or designers. It is in order to point out this essential feature that the word “actor” was added to it. \parencite[7]{Lat90:On}


HYBRIDS: extending semiotics to things instead of limiting it to meaning

``What really matters is that it is an elevation instead of a reduction and that the new hybrid status give to all entities both the action, variety and circulating existence recognized in the study of textual characters and also the reality, solidity, externality that was recognized in things “out of” our representations. What is lost is the absolute distinction between representation and things -but such is exactly what AT wishes to redistribute through what I call a counter-copernican revolution'' \parencite[11]{Lat90:On}.









% 
% Give an example! 
% 
% the application of your model to the databaser and the database 
% feels dissociated from the model you are proposing 
% because it only shows up 
% in the end of the section. 
% 
% Could you prepare this earlier? 
% give more examples? 
% 
% 
% 
% what does it mean that a databaser and a database resonate with each other???


\paragraph{Database Community}

% Why did we talk about a skin at all if you just leave the idea there? 
% What is the purpose of the skin in your model? 
% how does it connect to the idea of databasers below?
% why does the community has a resistance?

The resistance of the skin represents, thus, the resistance of connectivity itself, that is, the resistance of a community. 

{
	

	In the resonant network model I am proposing here database performers (or databasers) are human actors in the network, while nonhuman actors include the database. 

% because the preceding discussion is so abstract, that is, 
% lacking examples, 
% it seems hard to understand other things. 
% 
% Where do the things that databases are made of fall into this model? 
% where are the sounds, images, algorithms placed here?


	The resonant network, as an instantiation of a process of unworking, enables the thought of a database community to emerge. This community consists of ……………...
}

Therefore, by substituting, on one hand, the human with the database performer (databaser) and, on the other, the nonhuman with the database, the resonant network, as an instantiation of a process of unworking, enables the thought of a \textit{database community} to emerge. 

In a database community, database music can be understood as a hybridly social and communicative event. 

This is to say that, following Latour's hybridity of objects in his understanding of society \parencite[2]{Lat90:On}, database music is a social practice that comes as a result of databasers and databases in resonance with each other. 

% Can you give examples? 
% how do they resonate in practice? 
% what does it mean to resonate with each other??

Furthermore, database music is an instance of community, 

% this suggests that 
% database music = database community. 
% shouldn't database music be a community event instead per your earlier definition?

in the sense that it is an event of communication, understanding the communicative as ``the unworking of work that is social, economic, technical, and institutional'' \parencite[31]{Nan91:The}. 

% What this means is that …. explain in lay terms the “unworking of work”...

Therefore, the database community emerges as the unworking of databasers and databases. 

Database community can be considered as, precisely, the skin 

% ok. 
% I see the return of the idea of a skin. 
% the problem I have with a lot of these ideas is that 
% terms are used multiple times in multiple ways. 
% 
% if you have established the notion of 
% a network that is made of human and non-human actors, 
% and that these actors resonate through each other, 
% the idea of a skin is confusing.

upon which the human and the nonhuman resonate, which amounts to ---but never finalizes in--- a musical event.
