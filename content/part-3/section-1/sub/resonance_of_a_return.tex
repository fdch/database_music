\paragraph{A Reverb}
Sound reaches, enters, and traverses bodies in media. Media here refers to the matter through which sound propagates, such as a space filled with gas, liquid, or solid particles of matter including human and nonhuman bodies. More generally, sound propagation is conditioned by the qualities of the medium. Waves change direction by way of reflection or refraction, and they fade out by way of attenuation. Furthermore, while the combination of density, pressure, temperature, and motion affect the speed of sound, a medium's viscosity affects the sound's attenuation rate. For instance, within hot and humid climates sound will move slower, or if there is wind blowing in the same direction of a sound, it will make the sound travel faster. This means that sound waves are affected in different ways by different media, some being more (concert halls) or less (anechoic chambers) resonant.

\paragraph{A Filter}
A listening body is part of the medium through which sound propagates: the body's sense perception is immersed within that medium. Being part of sound, bodies change sound even before listening. Sound, in its most basic and general form makes listeners vibrate as listeners become part of sound. On a mechanical level, the body is an a priori physical filter. Sound is filtered differently and uniquely within each body: my body changes the incoming sound for me, just as it does for others. In other words, a longitudinal wave passing through a body affects how it will arrive at other points in space. Therefore, bodies filter sounds for other bodies while affecting sound waves before they reach the listener. That is to say, since the listener's body itself refracts, reflects, and attenuates waves, the singular filter that is the body changes wave propagation not only for itself and its own listening experience, also for the listening experience of others. Empty concert halls are thus more reverberant than filled ones. 

\paragraph{A Loop}
The filtering qualities of the listening body reveal the extent to which listening is such a singular and personal experience. Furthermore, this singularity can be understood as emerging out of the plurality that is sound. Plurality, in this sense, refers to the infinitesimal activity of waves. The interaction between the singular and the plural, in this sense, can be approached with the structure of a loop. I listen to myself as resonant subject, while creating meaning from a certain quality of a sound. I do this in simultaneity with others, who also create themselves as resonant subjects, while giving meaning to other sound waves. In this resonance, the vibrating link in between ourselves is also simultaneously changing the way we are listening. Thus, every singular listening subject is in a state of being (mutually) (self) exposed to every other listening subject, that is, in resonance or in touch with one other. 

\paragraph{An Attack}
Philosopher Jean-Luc \textcite{Nan07:Lis} brings forth an ontology of sound that can be understood in terms of resonance. He speaks about a ``sonorous presence'' \parencite[143-144]{Gra15:The} that exposes listeners to themselves and to one another. The duration of this exposure is always an instant. All mechanical waves require an initial energy input and in the case of sound, particularly in musical contexts, this input is generally referred to as an attack. Instead, Nancy uses this term to describe the exact moment when a sound arrives and simultaneously leaves the body: the instantaneous appearance of sound within the body. An attack therefore instantiates the sonorous presence. Within this attack, that is, during the experience of this exposure, sound is understood as a sensing experience in itself as well as the experience of what a given sound might signify. As Brian Kane writes, to be listening in the sonorous presence constitutes ``a mode of listening that exposes itself to sense'' \parencite[143-144]{Gra15:The}. This means that in the sonorous presence, the body begins to listen to itself listen. As I described with Hansen's notion of virtuality \see{framing}, virtuality can be understood as our brain's capacity to create images from the world. In listening, the virtuality of the human mind engages with the attack of the sonorous presence. In this sonorous present the first `image' is the body itself. In this sense, the creative capacity of the body enables the body itself to be \textit{self}-in-formed during the sonorous present. Furthermore, the body listening to itself listening results not only in the self-image of the body, it also creates an image of the listened. 

\paragraph{An Sampler}
Consider, for example, an acousmatic concert in which one of the music works is made with pre-recorded violin samples. When this violin begins playing sounds, an illusion may very well begin to emerge: we can imagine a violin player. If the imaginary player continues to play sounds and move them in space, this illusion continues in the direction of physical but illusory motion in space, that is, we can perceive an actual violin and an actual violin player. Therefore, this virtuality may project itself throughout the complete music work, thus grounding the music work on an imaginary force that is only alive because of the listener's virtuality. The ghostly qualities of this force will be addressed further down this text \see{spectrality}. Most presently is the fact that this `magic' show ---happening in front and because of the listener's body-sensing mind--- can be understood in terms of a resonating linkage between the human and the nonhuman: a network of interconnected objects that refer to each other. In this listening process, therefore, the listening subject exposes itself to itself and to the virtual self of the violin. Virtual, in this context, does not mean in opposition to the real. The virtual comes as the affective presence of a reality, and thus it becomes the possibility condition for the reality of images. 

\paragraph{A Texture}
Since every `body' is immersed within sound, and since sound refers to the thing that makes it, for Nancy this immersion is within a web of references. Furthermore, this web of references moves like waves: in time and space, back and forth, delaying in every next moment. Therefore, instead of a loop, a more convoluted circuitry appears that can be understood as a \gls{fdn}. The trick here is that this delay network sounds without input or output: it is already playing and sounding as a web-like endless texture. Brian Kane refers to this structure as ``a structure of infinite referrals and deferrals'' \parencite[143]{Gra15:The}, where references are at once postponed or delayed, and distinguished from each other. Within this texture, Nancy approaches a notion of meaning: ``meaning is made of a totality of referrals: from sign to a thing, from a state of things to a quality, from a subject to another subject or to itself, all simultaneously'' \parencite[4-9]{Nan07:Lis}. Therefore, since sound  ``is also made of referrals: it spreads in space, where it resounds while still resounding `in me''' \parencite[4-9]{Nan07:Lis}, the result can be understood as a process that intertwines sense and signification. If this is the case, then sense refers to the body sensing itself sensing, and signification points to the referential quality of the texture. In both cases, what is at stake in the listening experience is this interconnected web-like texture of delays and distinctions. On the one hand, the points in this texture are distributed in time, delayed to further moments, separated temporally. On the other, these same points are marks that indicate the extent to which they differ or ressemble each other, thus they can be understood distributed spatially.

\paragraph{A Return}
For Nancy, to be listening is to enter into ``tension'' and to be attentive for a relation to self \parencite[12][All subsequent quotes from this passage.]{Nan07:Lis}. In this context, `self' refers neither to yourself ---``not \dots a relationship to `me' (the supposedly given subject)''---, nor to the self of another ---``the `self' or the other (the speaker, the musician, also supposedly given, with his subjectivity).'' The structure of resonance can be understood in terms of a ``relationship in self.'' That is to say, because of this relationship (in self) that appears in the play of the web-like texture of delays and references, to be listening is an ontological passage: ``passing over to the register of presence to self.'' The self appears, it becomes present, as something that emerges from a resonant plurality. However, `self' is not an expressive substance inherent to bodies, or already in the body, as if it were some originary essence that appears out of resonance ---``nothing available (substantial or subsistent) to which one can be `present.''' On the contrary, for Nancy, the self comes in the form of a return, the ``resonance of a return [\textit{renvoi}].'' 

The more general implications of this ontology would extend the limits of this dissertation. For a commentary on Nancy's work, see \textcite{Gra15:The}. Nevertheless, I would like to point to one particularity of this ontology of sound: Listening is an activity of sensing bodies through which the ontological condition of these bodies becomes available. In this sense, to what extent can we consider the database as a listening body? And if so, to what extent is there an ontology of the database? These are the questions that I address during the following sections.

