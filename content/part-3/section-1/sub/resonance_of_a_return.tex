\begin{quote}
	To be listening is thus to enter into tension and to be on the lookout for a relation to self: \textit{not} it should be emphasized, a relationship to `me' (the supposedly given subject), or to the `self' or the other (the speaker, the musician, also supposedly given, with his subjectivity), but to the \textit{relationship in self}, so to speak, as it forms a `self' or a `to itself' in general, and if something like that ever does reach the end of its formation. Consequently, listening is passing over to the register of presence to self, it being understood that the `self' is precisely nothing available (substantial or subsistent) to which one can be `present,' but precisely the resonance of a return [\textit{renvoi}]. \parencite[12]{Nan07:Lis}
\end{quote} % 11/22/2017 17:40:13

\paragraph{Sonorous presence}
Philosopher \textcite{Nan07:Lis} brings forth an ontology of sound that is based on a performativity of listening that links an embodied theory of listening with a phenomenology of the self. For \textcite{Gra15:The}, Nancy’s `sonorous presence’ is built on the premises that the listening body is part of the medium through which sound propagates and that the body's sense perception is immersed within that medium. Medium here refers to the matter through which sound propagates, such as a space filled with gas, liquid, or solid particles of matter including human and nonhuman bodies. Sound reaches, enters, and traverses bodies. This means that sound, in its most basic and general form makes listeners vibrate and therefore, the listening subject is always already part of the listened sound.

\paragraph{An a priori filter}
Being part of sound, bodies change sound even before listening. On a mechanical level, the body is an a priori physical filter. Sound is filtered differently and uniquely within each body: my body changes the incoming sound for me, just as it does for others. For instance, a longitudinal wave passing through a body affects how it will arrive at other points in space. Sound propagation is conditioned by the qualities of the medium. That is to say, while the combination of density, pressure, temperature, and motion affect sound speed, a medium’s viscosity affects its attenuation rate. This means that within hot and humid climates sound will move slower, and if there is wind blowing in the same direction of a sound it will travel faster, but it also means the listening body changes how sound moves both inside and outside of itself. Waves change direction by way of reflection or refraction, and they fade out by way of attenuation. This means that waves are affected in different ways by different media, some being more (concert halls) or less reflective (anechoic chambers). Therefore, bodies filter sounds for other bodies while affecting sound waves before they reach the tympani. That is to say, since the listener’s body itself refracts, reflects, and attenuates waves, the singular filter that is the body changes wave propagation not only for itself and its own listening experience, also for the listening experience of others. This explains why empty concert halls sound more reverberant than filled concert halls. Most importantly, the filtering qualities of the listening body reveal the extent to which listening as such is a singular experience.

\paragraph{Sonorous presence in an attack}
The sonorous presence exposes listeners to themselves and to one another. The duration of this exposure is always an instant. Nancy refers to this instant as an `attack.' All mechanical waves require an initial energy input and in the case of sound, particularly in musical contexts, this input is generally referred to as an attack. Instead, Nancy uses this term to describe the exact moment when a sound arrives and simultaneously leaves the body: the instantaneous appearance of sound within the body. An attack therefore instantiates the sonorous presence. The experience of this exposure points to the presence of sound as a sensing experience in itself and not to what a given sound might signify. As Kane writes, to be listening in the sonorous present constitutes ``a mode of listening that exposes itself to sense'' \parencite[143-144]{Gra15:The}. This means that in the sonorous present, the listening body emerges as a creation of itself as sense.


This means that in the sonorous present, the listening body emerges as a creation of itself as sense. 

Therefore, in the sonorous present the first image is that of the body. In this sense, the body is given form by itself: it is self-in-formed during the sonorous present. Furthermore, the duration of this sonorous presence is none other than an instant, or what Nancy refers to as an `attack.' All mechanical waves require an initial energy input. In the case of sound, particularly in musical contexts, this input is generally referred to as attack. Nancy, however, takes this notion of the attack to another plane. He refers to it as the exact moment when a sound arrives and simultaneously leaves the body: the instantaneous appearance of sound within the body. An attack therefore instantiates the sonorous presence.

\paragraph{Referrals and Deferrals}
Kane finds Nancy's ontology of sound as ``a structure of infinite referrals and deferrals'' \parencite[143]{Gra15:The}, that is, what Nancy refers to as `resonance.' This structure comprises, for Kane, Nancy's reading of Jacques Derrida's concept of \textit{différance} \see{human}.\footnote{For a commentary on this concept, see \parencite[71-72]{Gra15:The}; for Derrida's original essay on the matter, see \parencite{Der78:Wri,Der82:Mar}} Resonance, in this terms, can be understood as an activity that takes place in 

Inasmuch as the sensing body is aware of its own multiplicity of sense, this multiplicity is itself dislocated spatially and temporally. 

Thus, the distances in both time and space, between all points of the listening experience, of the singular listener and the plurality of listeners, create a web of references that is instantiated in the attack of the sonorous presence. By way of listening, Nancy approaches the notion of meaning. That is to say, for him, meaning ``is made of a totality of referrals: from sign to a thing, from a state of things to a quality, from a subject to another subject or to itself, all simultaneously'' \parencite[4-9]{Nan07:Lis}. Therefore, since sound  ``is also made of referrals: it spreads in space, where it resounds while still resounding `in me'' \parencite[4-9]{Nan07:Lis}, the result is an understanding of listening as a process that intertwines sense and signification. As a consequence of meaning within the listening experience, the self comes along resounding, emerging out of a resonating plurality.

\paragraph{A Loop}
The condition of repetition, oscillation, or circularity explains the `infinite' quality that Nancy gives to the structure of the listening experience. For Nancy, the constant oscillatory motion that is present throughout the listening experience has the structure of a feedback circuit, or a loop. For example, I listen to myself as resonant subject, while creating meaning from a certain quality of a sound. I do this in simultaneity with another subject, who is also creating itself while giving meaning to other sound waves. In listening, the vibrating link in between ourselves is also simultaneously changing the way we are listening. Thus, every singular listening subject is in a state of being exposed to every other listening subject, that is, in resonance or in touch with one other. 

\paragraph{An Approach to Self}

In this context, the self is not an expressive substance inherent to bodies, or already in the body, as if it were some originary essence that appears out of resonance. On the contrary, for Nancy, the self comes as a result of the resonance, in the form of a return. 



% In other words, the self is something that appears to itself, at the limit of itself, very much like sound does. 

% This is why, for Nancy, to be listening [\textit{être à l'écoute}] engages resonant subjects with an approach to self. This approach, however, is neither to the self, not to the self of another, but to the structure of resonance as understood in terms of a \textit{relationship in self} from itself. The quote at the beginning of this section (``if something like that ever does reach the end of its formation'') points precisely to the infinite resonant process with which Nancy builds his concept of listening, but specifically, it is how he sets forth an image of a self coming from this ontology of sound.


\footnote{In this sense, Nancy is one of the first philosophers of the self to propose such a theorization of the self as resonance, extending his speculations to, for example, considering if the philosophical truth could be something listened to, as opposed to something seen: ``\dots shouldn't truth `itself,' as transitivity and incessant transition of a continual coming and going, be listened to rather than seen?'' \parencite[4]{Nan07:Lis}}



 % you could probably benefit from an example to make things less abstract? perhaps a reference to the composers you mention below might be good?






