\paragraph{A Reverb}
Sound reaches, enters, and traverses bodies in media. Media here refers to the matter through which sound propagates, such as a space filled with gas, liquid, or solid particles of matter including human and nonhuman bodies. More generally, sound propagation is conditioned by the qualities of the medium. Waves change direction by way of reflection or refraction, and they fade out by way of attenuation. Furthermore, while the combination of density, pressure, temperature, and motion affect the speed of sound, a medium's viscosity affects the sound's attenuation rate. For instance, within hot and humid climates sound will move slower, or if there is wind blowing in the same direction of a sound, it will make the sound travel faster. This means that sound waves are affected in different ways by different media, some being more (concert halls) or less (anechoic chambers) reflective.

\paragraph{A Filter}
A listening body is part of the medium through which sound propagates: the body's sense perception is immersed within that medium. Being part of sound, bodies change sound even before listening. Sound, in its most basic and general form makes listeners vibrate as listeners become part of sound. On a mechanical level, the body is an a priori physical filter. Sound is filtered differently and uniquely within each body: my body changes the incoming sound for me, just as it does for others. In other words, a longitudinal wave passing through a body affects how it will arrive at other points in space. Therefore, bodies filter sounds for other bodies while affecting sound waves before they reach the tympani. That is to say, since the listener's body itself refracts, reflects, and attenuates waves, the singular filter that is the body changes wave propagation not only for itself and its own listening experience, also for the listening experience of others. Empty concert halls thus sound more reverberant than filled ones. Furthermore, the filtering qualities of the listening body reveal the extent to which listening is such a singular and personal experience, which occurs out of the plurality that is reverberant sound.

\paragraph{An Attack}
Philosopher Jean-Luc \textcite{Nan07:Lis} brings forth an ontology of sound that can be understood in terms of resonance. He speaks about a ``sonorous presence'' that exposes listeners to themselves and to one another. The duration of this exposure is always an instant, thus he refers to it as an `attack.' All mechanical waves require an initial energy input and in the case of sound, particularly in musical contexts, this input is generally referred to as an attack. Instead, Nancy uses this term to describe the exact moment when a sound arrives and simultaneously leaves the body: the instantaneous appearance of sound within the body. An attack therefore instantiates the sonorous presence. Within this attack, that is, during the experience of this exposure, sound is understood as a sensing experience in itself and not to what a given sound might signify. As Brian Kane writes, to be listening in the sonorous present constitutes ``a mode of listening that exposes itself to sense'' \parencite[143-144]{Gra15:The}. This means that in the sonorous present, the body begins to listen to itself listen. If I borrow here Hansen's terminology \see{framing} in the sonorous present the first `image' is the body itself. In this sense, the body is given form by itself: it is \textit{self}-in-formed during the sonorous present.

\paragraph{A Loop}
The condition of repetition, oscillation, or circularity refer to a quality that Nancy gives to the structure of what he calls resonance. For Nancy, the constant oscillatory motion that is present throughout the listening experience has the structure of a feedback circuit, or a loop. For example, I listen to myself as resonant subject, while creating meaning from a certain quality of a sound. I do this in simultaneity with others, who also create themselves while giving meaning to other sound waves. In resonance, the vibrating link in between ourselves is also simultaneously changing the way we are listening. Thus, every singular listening subject is in a state of being exposed to every other listening subject, that is, in resonance or in touch with one other.

\paragraph{A Texture}
Since every `body' in media is immersed within sound, and since sound refers to the thing that makes it, for Nancy this immersion is within a web of references. Furthermore, this web of references moves like waves: in time and space, back and forth, delaying in every next moment. Therefore, instead of a loop, a more convoluted structure appears that can be understood as a feedback delay network. The trick here is that this delay network sounds without input or output: it is already playing and sounding as a web-like endless texture. Brian Kane refers to this structure as ``a structure of infinite referrals and deferrals'' \parencite[143]{Gra15:The}, where references are at once postponed or delayed, and distinguished from each other. Within this texture, Nancy approaches a notion of meaning: ``meaning is made of a totality of referrals: from sign to a thing, from a state of things to a quality, from a subject to another subject or to itself, all simultaneously'' \parencite[4-9]{Nan07:Lis}. Therefore, since sound  ``is also made of referrals: it spreads in space, where it resounds while still resounding `in me''' \parencite[4-9]{Nan07:Lis}, the result can be understood as a process that intertwines sense and signification. If this is the case, then sense refers to the body sensing itself sensing, and signification points to the referencial quality of the texture. In both cases, what is at stake is this interconnected web-like texture of delays and distinctions. 

\paragraph{A Return}
For Nancy, to be listening is to enter into ``tension'' and to be attentive for a relation to self \parencite[12][All subsecuent quotes from this passage.]{Nan07:Lis}. Neither to yourself ---``not \dots a relationship to `me' (the supposedly given subject)---, nor to the self of another ---``the `self' or the other (the speaker, the musician, also supposedly given, with his subjectivity).'' The structure of resonance can be understood in terms of a ``relationship in self.'' That is to say, because of this relationship (in self) that appears in the play of the web-like texture of delays and references, to be listening is an ontological passage: ``passing over to the register of presence to self.'' The self appears, it becomes present, as something that emerges from a resonant plurality. However, `self' is not an expressive substance inherent to bodies, or already in the body, as if it were some originary essence that appears out of resonance ---``nothing available (substantial or subsistent) to which one can be `present.''' On the contrary, for Nancy, the self comes in the form of a return, the ``resonance of a return [\textit{renvoi}].'' 

The more general implications of this ontology would extend the limits of this dissertation. For a commentary on Nancy's work, see \textcite{Gra15:The}. I would like to point to, however, to one particularity of this ontology of sound. Listening is an activity of sensing bodies, and through listening their ontological condition can be accessed. In this sense, to what extent can we consider the database as a listening body? And if so, to what extent is there an ontology of the database? These are the questions that I address during the following sections.

