
% In this chapter, I analyze resonant networks in order to assess the extent to which databasing can be reconfigured by listening, and viceversa. The following questions will be revised: To what extent is the listening subject present within database music? How is the notion of listening subject reconfigured by way of the database? To what extent can the database be thought of as a listening subject, and, if so, to what extent does the agency of the database as listening subject resonate aesthetically?

% In section one, I delineate Jean-Luc Nancy's ontology of sound in order to present the database as a resonant subject in itself. 

% In section two and three, given the multiplicity of factors that are in play when databases enter into the process of music listening, I focus on the links that exist between Nancy's resonance and Bruno Latour's actor-network theory \parencite{Lat90:On, Lat93:We}, arriving at the concept of a resonant network. Then, a distinction between sound and networks is made, and the concept of the work of actors is introduced.

% Finally, I understand resonant networks in terms of community. Since the notion of inoperativity is closely related to that of community ---and the exposure of selves---, this relation of selves can also be understood as the resonating force that unfolds hand in hand with the performativity of the network. Thus, the expansion of the network, the propagation of sound, and the exposure of selves, can be connected to each other with a force of inoperativity.
