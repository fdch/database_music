A person that encounters a database feels a resonance, recognizes some of her or his bodily and mental functions in the artwork. In this chapter, I explore a way in which we can experience database art as a mirror or echo of ourselves, yet at the same time, we recognize the presence of something different from ourselves that guides and structures our experience of that work. This unclear presence, like a specter, often reveals through a combination of a variety of databases that can be found at the intersection of art an technology. This specter is what I consider the key to understanding how databases claim a certain aesthetic agency that is underway when computers are involved in art, particularly in the field of music composition. Therefore, by addressing a specter of the database, I take on the adventure of delineating what can be considered a database music.
