A person that encounters a database feels a resonance, recognizes some of her or his bodily and mental functions in the artwork. In this part, I explore a way in which we can experience database art as a mirror or echo of ourselves, yet at the same time, we recognize the presence of something different from ourselves that guides and structures our experience of that work. This unclear presence, like a specter, often reveals through a combination of a variety of databases that can be found at the intersection of art and technology. This specter is what I consider the key to understanding how databases claim a certain aesthetic agency that is underway when computers are involved in art, particularly in the field of music composition. Therefore, by addressing a specter of the database, I take on the adventure of outlining a framework to discuss an aeshtetics of database music. To this end, I propose an aesthetics that is not humanly derived but that it is still dependant upon human temporalitites. In other words, while databases operate below human thresholds of perception, we can still speak of an aesthetics of the database precisely because we can perceive the presence of the database. Therefore, we can understand certain aspects of what is perceived of database music works as comprehensive of a working definition of aesthetics that I will only suggest throughout this text. The reader might perhaps find these ideas about aesthetics as a starting point to further questions and developments about the deeper resonances database music proposes. 

\paragraph{A restless encounter at the limit}
Databases can be understood as certain restless creatures that exist and participate in the world in ubiquitous ways. This restlessness can be understood as the dynamic feeling we have of the database as not being settled in one place and of being, at once, ubiquitous and in constant motility. I begin to delineate this restlessness in the previous sections, by suggesting the kind of motion that both visibility and development of the database has been taking place over the years in and between sound practices with computers. In this part, I propose to deepen the understanding of this restlessness, suggesting it now as a starting point to understand an aesthetics of database music. To this end, we can consider this restlessness as being a result of the encounter with the uncanniness of a spectral force, at the limit, or in the moment of a certain `touch' of this liminality in between the human and the nonhuman. An aesthetics of database music can thus be approached from this restless encounter at the limit. The uncanny feeling that results from an encounter with the spectral speaks of an experience with that which is familiar and, at the same, not familiar to us; known to us as being close to us but, the moment we recognize this closeness, it slips away from our proximity into an ungraspable feeling that shakes our comprehension of reality. Thus ---and through its performance---, the presence of the database within a database music work makes its ghostly appearance, and reconfigures our understanding of the composer (but also the programmer, the performer, etc.) into a hybrid `databaser' comprising both human and nonhuman aspects. This specter is an illusory figure that is imprinted upon what I call the (infra) skin of the database, and it is the source of our encounter with the uncanny. As such, this source suggests a certain agency within the database music work that resonates differently in the production and reception of the work. On the one hand, the productive force of this agency can be summarized with the way in which small changes to a database have significant effects on not only the resulting music work, but also the operativity throughout that music work, and thus they lead to new and uncharted artistic territories. I understand these territories to be `worlds' that databases reveal: sets of possibilities, but also the creative potentials awakened in the producer. On the other hand, the social, ethical, and political resonances that database music works have upon reception, reveal the tensions that exist between control and ungovernability in databasing. These tensions can serve, in sum, as a metaphor for a certain contemporary subjectivity in which we are embedded by the ubiquity of the database in most aspects of our lives. Thus, the aesthetics of database music evidences a limit between the human and the nonhuman, as well as between the self and community; a limit that is in no way fixed, but that exists as a textural play of resistance and expansion. What the aesthetic agency of the database might suggest, then, is new approaches to our own mundane activities that are reconfigured by the awareness of the presence of the database that, as listeners, we obtain from database music works and the encounter with the spectral. 


