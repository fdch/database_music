\begin{quote}
	The structure of the archive is \textit{spectral}. It is spectral \textit{a priori}: neither present nor absent ``in the flesh,'' neither visible nor invisible, a trace always referring to another whose eyes can never be met\dots
\parencite[54]{Der95:Arc}
\end{quote} % 3/9/2018 9:24:10

\paragraph{Computer Memory and Writing}
Despite the multiplicity of elements that constitute software programming, such as compiler instructions, hardware stipulations, collaboration platforms, etc., the concepts of inscription and erasure (writing) are still in place. Because of this, the concept of the Derridean trace applies to programming. This fact can be linked further back to the conception of the computer itself, as it was previously discussed \see{databasing}. In his architecture, Von Neumann proposed that the storage unit of the computer would allow for data to be written and erased in different locations and times. In fact, he was following Turing's conceptualization of the \textit{a-machine} ---i.e., the \textit{Turing machine}---, which was a mathematical model for computation, that can be represented by a symbol scanner and an infinite tape, where the scanner gets, sets, or unsets a symbol on the tape, and the tape moves to the next slot accordingly. These setting and unsetting movements represent inscription and erasure, to the point that, as Kittler notes: ``the two most important directing signals which link the central processing unit of the computer to external memory are being called \texttt{READ} and \texttt{WRITE}'' \parencite[131]{Ern13:Dig}.

\paragraph{Memory Replacement}
However, an important distinction needs to be made here. While I am arguing for the similarities that exist between memory and database, Ernst proposes that databases ---i.e., what he refers to as digital an-archives--- enter as a form of replacement. This is to say that, given the convergence of all media into digital media, `reading' is rendered null, giving way to mathematical processes that come to interpret the data. In Ernst's words, ``signal processing replaces \textit{pure} reading'' \im \parencite[130]{Ern13:Dig}. This statement is only valid within the Kittlerian, disembodied worldview \see{convergence}, which would also allow to convey the following: signal processing replaces \textit{listening}. Thus, in such a world where data structures and algorithms perform our own capacities to create and, by doing so, remove our bodies from ourselves, there would be no need for neither reading nor writing. Since, if the database reads itself, it also writes itself, and, in the case of music, if it listens to itself, it also sounds itself, removing listening altogether. Therefore, in order to arrive at the point of resonance that enables both the nonhuman and the human to coexist, a reconceptualization of the anarchic in relation to databases needs to be made; specifically, one that is grounded not on substitution, but on difference, and one that explains how authority enters into the sphere of databases.

\paragraph{Anarchic Records}
In his conceptualization of the \textit{anarchoarchive}, Ernst \parencite{Ern13:Dig} opposes technical recording with symbolic transcription. Given the fact that a microphone captures the entire sonic environment, this involuntary memory ---of ``past acoustic, not intended for tradition: a noisy memory'' \parencite[174]{Ern13:Dig}--- comes to be a form of anarchic archive, or \textit{an-archive}. Therefore, for example, he claims that Bela Bartok's transcriptions to musical notation of Milman Parry's Serbian epic song recordings\footnote{\url{https://mpc.chs.harvard.edu/}} becomes an archivization process, that is, a process by which symbolic transcription leads to an ordered archive, i.e., a score.\footnote{Another example Ernst provides of the anarchive is the Internet itself: `` [The Internet] is a collection not just of unforeseen texts but of sound and images as well, an anarchive of sensory data for which no genuine archival culture has been developed so far in the occident'' \parencite[139]{Ern13:Dig}.} In Ernst's sense, a sound file storing samples of the recorded world is anarchic because of the unfiltered (`pure') way in which the world is inscribed by the recorder. He bases this idea on another media theorist, Sven Spieker, who claims that, following Derrida's conceptualization of the archive, a central feature of archives is not memory, but a need to ``discard, erase, eliminate'' that which is not intended for archivization \parencite[113]{Ern13:Dig}. In other words, the possibility condition of an archive is filtering, or better: framing. The problem is that, the moment Ernst deposits on the recording technology the disappearance of the body, he dislocates framing altogether, and thus is how the concept of `pure' recording comes in. Therefore, if for Ernst, Parry's audio recordings are anarchic, this is because they are constructed upon a Kittlerian idea of purity that leaves the human out of the equation.

\paragraph{Memory and Framing}
If we consider the body's capacity to create images from the world of images, then framing is also a possibility condition for human memory. In fact, we have seen with the Derridean trace that it is a resisting force what first enables the violent rupture of traces, and thus, this force can also be considered as an instance of framing itself. Resistance of the psyche as creativity taking action as the moment of framing. In the an-archic memory graph that I am conceptualizing here, framing can be thought of as the edge (the arrow) that thus connects the public to the private. Furthermore, framing can also be inversely considered as the arrow between the private and the public in the case of archives, since, for example, in the case of a contract, only a limited amount of signatures are needed, thus a frame needs to be an active part of the process of archivization.

\paragraph{Nonhuman Tympans}
Therefore, given that framing is in between the public and the private, in order to consider the case of audio recording as either memory or archive, these two categories, privacy and publicness need to be taken into account in relation to nonhumans. In other words, given that the world can be considered public, because it is in a constant state of availability at any time, then, a microphone can be considered an actor of privacy, since it deprives some sounds from being recorded ---e.g., because it has not been designed for certain frequency bands--- while recording other sounds. Therefore, if transducers have a filtering capacity, they engage with the passage from the public to the private ---and vice versa, since both speakers and microphones are transducers. Consider, for example, Jonathan Sterne's definition of the `tympanic' function of transducers, as quoted by Cathy Van Eyck in her PhD dissertation on microphones and loudspeakers: ``every apparatus of sound reproduction has a tympanic function at precisely the point where it turns sound into something else\dots and when it turns something else into sound'' \parencite[107]{Eck13:Bet}. Therefore, considering these nonhuman tympans as actors of privacy and publicness, audio reproduction technology can be compared to the structure of memory and archives. For example, audio recording can be conceived as a form of memory, only to the extent that its inscription is singular yet reproducible. This is because a magnetic tape or a hard drive cannot be considered in the same (plural) way as Derridean traces, since once data is stored, it exists only once in discrete, limited space, unless copied to another location, in which case it becomes a duplicate, an identical (in-different) clone of itself. Furthermore, the case of audio playback can be a form of archive, since it consists of the passage from the private ---the stored air pressure waves--- to the public ---the reproduced air pressure waves in space---, only to the extent that the latter is not considered space as such, but resonance within space. In this sense, Ernst' consideration of the Parry's recordings as anarchic needs to be reconfigured.

\paragraph{Spectrality of Archives}
Given that these nonhuman tympans constitute the limit between sonic ---but also tactile \parencite[223]{Eck13:Bet}--- world and the binary world of databases, their comparison to memory and archives renders a hybrid object: one that, on the one side, becomes a private and singular `trace', and, on the other, becomes a public, resonant `space.' Thus, databases represent neither a trace nor a space. At this point it is important to address the quote at the beginning of this section, that is, the structure of the archontic that Derrida assigned to archives: spectrality. Derrida claims that, addressing a phantom is a ``transaction of signs and values, but also of some familial domesticity'' \parencite[55]{Der95:Arc}, meaning, on the one hand, that in the uncanny encounter with a ghost, there is familiarity, that is, there emerge feelings of what is known to be close to us, but also that which composes the authority of that closeness. Therefore, embedded in this familiarity is the archontic, the oedipal, etc., and thus the expression of power that this apparition brings forth. On the other hand, the familiar is also related to an economy ---Derrida points to the greek root \textit{oikos}, meaning `house'---, that is, to the passing through (trans-action) of signs, but also de translation ---or better, the `transduction' of things. This is why Derrida considers any encounter with the spectral to be an instance of addressing, that is of transaction. Finally, what this uncanniness of the ghost entails is nothing other than the haunting itself, as Derrida writes: ``haunting implies places, a habitation, and always a haunted house'' \parencite[55]{Der95:Arc}. To bring back our database, the hybridity that the database projects when compared to memory or archives points to a certain uncanniness, that is, precisely to the hauntedness that comes from its spectrality. Furthermore, it can be argued that considering the database in such way is plain and simple delusion, that is, insanity at its best. Not surprisingly, this is the point exactly; within this delusion exists truth:

\begin{quote}
	\dots it resists and \textit{returns}, as such, as the spectral truth of delusion or of hauntedness. It \textit{returns}, it belongs, it comes down to spectral truth. Delusion or insanity, hauntedness is not only haunted by this or that ghost\dots but by the specter of the truth which has been thus repressed. The truth is spectral, and this is its part of truth which is irreducible by explanation\dots \parencite[54-56]{Der95:Arc}
\end{quote} % 3/9/2018 9:24:10

\paragraph{Spectrality of Databases}
Therefore, given that Derrida considers the structure of the archive to be indeed spectral, I can bring this spectrality to the database itself, and consider it just as spectral but in a different sense. The spectrality of the database comes not from its hauntedness, that is, it is not domiciled since, as humans, we have no access to data space (address space) directly with our bodies. As humans, we cannot engage in transaction with the specter directly: we need transducers. In the case of audio playback, we need loudspeakers because we need to create, with our bodies, the image that was recorded in the first place. In consequence, while embodying stored data, we embody its specter. Therefore, the hauntedness needs to be `transduced' into space, and with this transduction the agency of the database can be effectively felt. Likewise, the uncanniness of the case of audio recording can be felt as the ghost that will `remember us,' because of this computer `memory' that will keep (to its own privacy) the sonic environment that attests to our presence in space ---because we sing, we make sounds, etc. Hence, the haunting that resides in-memory, together with the stored data and pointers.

\paragraph{Agency of the Uncanny}
In recognizing this presence, which is the archontic specter of the database, comes the recognition of the agency of the database itself, and, furthermore, of the quality of the aesthetic experience that we encounter whenever there is a database in art. As Timothy Morton \parencite{Mor13:Hyp} writes in relation to what he calls hyperobjects:

\begin{quote}
	Recognition of the uncanny nonhuman must by definition first consist of a terrifying glimpse of ghosts, a glimpse that makes one's physicality resonate (suggesting the Latin \textit{horreo}, I bristle): as Adorno says, the primordial aesthetic experience is goose bumps. Yet this is precisely the aesthetic experience of the hyperobject, which can only be detected as a ghostly spectrality that comes in and out of phase with normalized human spacetime. \parencite[169]{Mor13:Hyp}
\end{quote}

To the point that databases are spectral, they can be considered hyperobjects in Morton's sense, and thus, agents translating through aesthetic networks. However, I will not dwell much longer on this definition in this text. If we consider the ``recorded movement of a thing'' with which Latour identified his concept of the network, databases are agents not only of recording, also of motility, and of thing-hood itself. This is to say that, in the constitution of networks, the spectrality of the database expands in multiple directions. Bringing back the illusory violin of the acousmatic concert I mentioned earlier \see{resonance_of_a_return}, it exemplified Hansen's concept of the creation of images our brain's capacity for virtuality has, that is, our imagination. The sound of a violin can be recorded ---but also synthesized--- into the privacy of a database. Then, it can be played back with loudspeakers located in such a way that they emulate the location of an actual violin player. As a result, the listener could very likely imagine a physically present violin in the room, that is, a ghost. This ghost comes in as the phantom of a human player; of the violin itself; of the histories and traditions that those two elements bring forth; of the presence of the nonhuman that the database implies; of the privacy that is not human but that it is still uncannily private; of the plurality that is embedded in the construction of databases; of the hauntedness of the archontic that the above sets forth; and so on. In this way, the spectrality of the database attests to its relation to memory and archives, and, thus, to its aesthetic resonance within our experience.

In what follows, a crucial aspect of the database will be addressed, one that defines the condition of possibility of this hauntedness to be indeed instantiated into appearance, not only in the form of authority, as I have shown in the case of its archontic presence, but also in the form of style, and most important, gender: the performativity of the database.
