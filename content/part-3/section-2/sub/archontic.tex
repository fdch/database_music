
\paragraph{Archives and Memory}
For Derrida, the archive cannot be reduced to memory, it is is neither a form of rememoration nor a ``conscious reserve,'' but simply a place where ``the origin speaks by itself'' \parencite[60]{Der95:Arc}. In this definition, the word `origin' refers to the greek root of the word `archive' [arkhē], whose meaning is twofold: on one hand, its definition is topological (space) on the other, it is nomological (law). In the topological sense, \textit{arkhē} is related to the greek word arkheion, which, in turn, refers to ``a house, a domicile, an address, the residence of the superior magistrates, the \textit{archons}, those who commanded'' \parencite[9]{Der95:Arc}. In the nomological sense, \textit{arkhē} refers to the ruling, to authority, to the command or commandment, i.e., the law. Thus, the magistrates [archons] are those who ``have the power to interpret the archives,'' and they indeed reside inside this place called \textit{archeion} \parencite[9]{Der95:Arc}. Therefore, considering the fact that these magistrates actually reside in place of the archive, Derrida refers to an ``institutional passage from the private to the public'' which is constitutive of the archive itself \parencite[9]{Der95:Arc}.

\paragraph{Hierarchies}
From this definition of the archive, what is alluded is the hierarchical structure of civilization itself, that is, of government and legislation. Going further into this aspect of the concept of the archive would extend the limits of this text. Derrida, with this conceptualization of the archive, focused on the exponential growth of archives that spawned during the 20th century, defining it as an impulse, or better, as a `fever' and a drive. Thus, he proceeded to perform a psychoanalysis of this symptomatic condition of mid-1990 society, beginning with the archivization of the house of the \textit{father} of psychoanalysis, Sigmund Freud. Therefore, I would like to focus on one principle that Derrida described as belonging to the concept of the archive, what he calls the \textit{archontic} principle. 

\paragraph{Archontic Principle}
The archontic principle is a type of authority that the archive projects, which can be understood as powerful dictating rule that is before anything else. Hence, its categorization as principle, which is also related to the origin (e.g., the latin root \textit{principium} which refers to the beginning) and to the figure of the ruler (e.g., principal, prince, etc.). For Derrida, this principle is constitutive of all words deriving from \textit{archē}, which is why he understands the archive, for example, as being patriarchic. Since the archontic principle is present in the etymology of the word `patriarchy' ---which refers, likewise, to `lineage,' `father,' and `I rule'---, therefore, he argues that the archontic is embedded first with a sense of filiation, that is, with fatherhood and the relationship between father and child. Thus, the archontic is ``paternal and patriarchic'' \parencite[60]{Der95:Arc}. 

\paragraph{Patriarchy}
What follows is that this patriarchy is grounded in a domicile (house) or an institution (family). Within this domiciliation or institutionalization is from where rules are prescribed, and where the ruling takes place. In the case of the patriarchic concept of family, the father is the ruler; in the case of civilization, the governor is the ruler. Therefore, this is what Derrida means when he performs a psychoanalysis of the archive: the topo-nomological aspect of the archive comes precisely from a ---strategically male-centered--- concept of fatherhood. In this sense, Derrida discovers that the archontic has the form of Freud's oedipus complex. An oedipus complex constitutes a desire of the child's unconscious hatred towards a parent. It is itself based in Sophocles' drama \textit{Oedipus Rex}, in which such desire results in eventual parricide. Derrida notes that this complex has the form of an infinite loop between father and child, which is bound to repeat itself, just as is the case with Freud's definition of the complex, which, according to the greek drama, is articulated by the figure of the parricide. In this sense, Derrida claims that Freud ``has shown how this archontic, that is, paternal and patriarchic, principle only posited itself to repeat itself and returned to re-posit itself only in the parricide'' \parencite[60]{Der95:Arc}.
% 3/9/2018 9:47:58
\paragraph{Institutional Passage}
What is the case when this patriarchic principle is found on archives? I mentioned above the passage from the private to the public that Derrida recognized in the concept of the archive. This fact resonates with the plurality that is condensed in the place of the archive, which is opposed to the singularity that is condensed in the plurality of memory. In other words, while archives are a singular place where documents ---made by a plurality of authors, that is, by different people, technologies, etc--- are singularly kept, in the case of human memory, the reverse occurs. While the psyche is singular, that is, while there is only `myself' remembering, the difference between each trace of memory is indeed a plurality. While I remember, the memory I rememorate is composed out the set of differences of traces that I have effracted in the moment of remembering. 

The following pseudocode may help in this distinction:

\begin{flushleft}
\small
\begin{lstlisting}
function make_an_archive() {
	public 	 	= PRIVATE
	singular 	= PLURAL
	place 	 	= COMMUNITY
	archive  	= place = singular = public
	return archive
	}

function make_a_memory() {
	private 	= PUBLIC
	plural 		= SINGULAR
	trace 		= SELF
	memory 		= trace = plural = private 
	return memory
}
\end{lstlisting}
\end{flushleft}

\paragraph{Authorities}
For example, the case of making an archive can be exemplified with the passing of a law or the signing of a contract. Derrida speaks of the act of consignation as constitutive of the archontic principle, since it is an act of ``gathering together signs'' \parencite[10]{Der95:Arc}, as is the case of the signatures on a bill or a contract. One's signature is private, therefore, the moment one signs a document one makes it public. Furthermore, in making a contract, multiple signatures need to be gathered into the same space: the plurality of the community needs to be condensed in the singularity of the document. Finally, the document itself comes to be a place, but it itself needs yet another place where it is stored, which comes to be the archive. What this act of consignation entails is, precisely, the archontic principle, since the contract itself now becomes an authoritative presence that emerged out of this function of making an archive. The structure of this presence will be discussed in the following section \see{spectrality}. For now, let us continue with the case of making a memory. The moment one perceives the world, one makes it private, inasmuch as one engages with remembering it. When one begin remembering, oneself begins the process of pathbreaking and effraction of traces ---and of forgetting, as I will continue to develop below--, which results in the plurality of the condition of traces, that is, on the difference between traces constitutive of memory.

So, given that if this constitution of archive and memory are indeed reflections of each other, that is, one being the reverse of the other, how is the archontic present in the making of a memory? What would a reversed archontic principle look like? How would it act? These questions deserve a few paragraphs because they will explain the relationship between database, memory, and archive. 

\paragraph{Anarchic Memory}
What is important to note here, before continuing, is that just as memory is in a state of fracture and rupture, an archive is also in such state of discontinuity, and thus it is this condition of being in the form of disconnected gaps that makes memories and archives so alike. Spieker notes is that of discontinuity and rupture: ``Like all kinds of data banks, [the archive] `forms relationships not on the basis of causes and effects, but through networks''' \parencite[113]{Ern13:Dig}. From these two qualities of archives (filtering and fracture), their resemblance to memory can be drawn. However, while the archive is indeed patriarchic, memory, on the other hand, in terms of the archontic principle, it can only be described as an-archic. While there is indeed oneself remembering, there is no possibility of ruling over the plurality of traces that one has in memory. There is no place to go to remember things, one may try to trick the body into feeling similar things as when one was remembering so as to make a memory resurface, but in the end, memories remember themselves. One has no control over neither remembering nor forgetting. Memories forget themselves.

\paragraph{Collective Memory}
In the fantastic case of Funes \see{funeslude}, for example, forgetting is a mysterious lack that cannot be reduced to a simple loss of information. It is also a necessary condition for memory to be considered as trace, given that the moment a trace begins is when its own erasure begins. Thus, every memory comes with its own forgetting mechanism, a mechanism that is triggered on its own, and, further, that it is still not fully understood \parencite[292]{Wes08:How}. However, in the case of an archive or, what some (perhaps wrongfully) refer to as `collective' memory, the opposite occurs: data loss consists of plain and simple erasure, that is, a destruction that cannot be considered an instance of forgetting: ``in collectivistic memory, where the database has a tangible form, it is more apparent that permanent loss is a possibility. In archives, ink may fade, paper may crumble and entire files may end up in the shredder. \parencite[292]{Wes08:How}''

\paragraph{Writing Code}
Since writing can function as a link between human and the nonhuman memory \see{human}, several instances of the metaphor of writing need to be explained in relation to databasing. For example, not only programming languages imply writing in terms of symbols (words and characters), also data structures appropriate the concepts of writing and erasing. Furthermore, in resonance to the Derridean trace, erasure is embedded in the structure of writing. This is to say, that C++ classes include within their own data structure a call to their \texttt{destructor}. This means that, whether explicitly or implicitly, all classes ---i.e., all data structures which correspond to instantiated objects--- have a way to self-erase, or self-destruct after the object is no longer needed. This self-destruction means, precisely, releasing the object's resources, that is, to free the physical memory space that it has occupied throughout its lifetime.\footnote{\url{https://en.cppreference.com/w/cpp/language/destructor}}

\paragraph{Anarchic Computer Memory}
This comparison between \texttt{destructor}s and forgetting serves as a starting point to determine the extent to which computer memory ---i.e., data structure handling in restricted, discrete space--- can be thought of as human memory itself, and, if so, the extent to which it also constitutes an instance of an-archic structure. (The first thing that comes to mind is an an-archic program written in C++ with self-destructing classes that fire at will, in complete unpredictability, rendering the software utterly anarchic, but utterly useless.) Therefore, there has to be a different way to think of an-archy in software. In what follows, I address the extent to which databases can be understood in terms of memory and anarchy, taking German media theorist Wolfgang Ernst's concept of the \textit{anarchoarchive} to a different dimension.

