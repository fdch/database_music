Derrida's broaded notion of `writing' relates to what he calls an `originary' trace. As I described earlier, the inscription of the trace contains from the start its own erasure. This means that, in a reciprocal motion, the origin of the trace is itself originary and non-originary: the origin of a disappearance and the disappearance of the origin. This apparent contradiction can be approached by the actual process of writing something on a page: where or when does the written word begin, when the pen touches the paper or when the paper lends itself for writing? The same question applies to the crossfade between the `I's in the case of Alvin Lucier's work \see{network}, where or when does the `I' begin `sitting' in the loop? Further, no matter how many metaphors we may give to this paradox, it simply falls out of empirical quests. Therefore, this paraodx of the trace is understood precisely by the concept of \textit{différance} that we outlined above. The repercussions of this word that ``is not a word and it not a concept'' have now a long history of what is known as deconstruction. I will not dwell on this history at all here, and simply jump thirty years in time, after computers entered the scene, and after the Internet became popularized by the World Wide Web. 

\paragraph{Archontic Principle}
In the mid-1990s, Derrida wrote another text in which he treats what he defined as an `archive fever' \parencite{Der95:Arc}. He focused on the exponential growth of archives that spawned during the 20th century, defining it as an impulse, or better, as a `fever' and a drive. He exemplified the intricacies that come out of the archivization of a private place with Sigmund Freud Museum, which is located in the same house where Freud lived.\footnote{\url{https://www.freud-museum.at/en/}} Derrida delineated an economy of archives that he calls the \textit{archontic} principle. `Economy' is related to the greek word \textit{oikos} (house), which in this case is the private place that became public the moment it became an archive. In this context, `archive' refers to the greek word \textit{arkhē}, whose meaning is twofold: on one hand, its definition is topological (space) on the other, it is nomological (law). In the topological sense, \textit{arkhē} is related to the greek word arkheion, which refers to ``a house, a domicile, an address, the residence of the superior magistrates, the \textit{archons}, those who commanded'' \parencite[9]{Der95:Arc}. In the nomological sense, \textit{arkhē} refers to the ruling, to authority, to the command or commandment, i.e., the law. Thus, the magistrates [archons] are those who ``have the power to interpret the archives,'' and they reside inside this place called \textit{archeion} \parencite[9]{Der95:Arc}. Therefore, considering the fact that these magistrates actually reside in the place of the archive, Derrida refers to an ``institutional passage from the private to the public'' \parencite[9]{Der95:Arc}. 

This passage resonates with the plurality that is condensed in the place of the archive, which is opposed to the singularity that is condensed in the plurality of memory. 

While the psyche is singular, that is, while there is only `myself' remembering, the difference between each trace of memory is indeed a plurality. 


In other words, while archives are a singular place where documents ---made by a plurality of authors, that is, by different people, technologies, etc--- are singularly kept, in the case of human memory, the reverse occurs. 


The archontic principle is a type of authority that the archive projects, which can be understood as a powerful dictating rule that is before anything else. Hence, its categorization as principle, which is also related to the origin (e.g., the latin root \textit{principium} which refers to the beginning) and to the figure of the ruler (e.g., principal, prince, etc.). For Derrida, this principle is constitutive of all words deriving from \textit{archē}, which is why he understands the archive, for example, as being patriarchic. Since the archontic principle is present in the etymology of the word `patriarchy' ---which refers, likewise, to `lineage,' `father,' and `I rule'---, therefore, he argues that the archontic is embedded first with a sense of filiation, that is, with fatherhood and the relationship between father and child. Thus, the archontic is ``paternal and patriarchic'' \parencite[60]{Der95:Arc}. 


For Derrida, the archive cannot be reduced to memory, it is neither a form of rememoration nor a ``conscious reserve,'' but simply a place where ``the origin speaks by itself'' \parencite[60]{Der95:Arc}. 







	\paragraph{Hierarchies}
	% why do we need thse ideas about the archive as government building and law? why not go direclty to the archontic principle?


	From this definition of the archive, what is alluded is the hierarchical structure of civilization itself, that is, of government and legislation. Going further into this aspect of the concept of the archive would extend the limits of this text. 

	Derrida, with this conceptualization of the archive,
	
	% why is this relevant again to database art? database music? You abandon the idea of archive as a law and of archive as patriarchy so why do they need sections of their own? could they be covered in a few sentences and then move on to the archive ?


	\paragraph{Patriarchy}
	What follows is that this patriarchy is grounded in a domicile (house) or an institution (family). Within this domiciliation or institutionalization is from where rules are prescribed, and where the ruling takes place. In the case of the patriarchic concept of family, the father is the ruler; in the case of civilization, the governor is the ruler. Therefore, this is what Derrida means when he performs a psychoanalysis of the archive: the topo-nomological aspect of the archive comes precisely from a ---strategically male-centered--- concept of fatherhood. In this sense, Derrida discovers that the archontic has the form of Freud's oedipus complex. An oedipus complex constitutes a desire of the child's unconscious hatred towards a parent. It is itself based in Sophocles' drama \textit{Oedipus Rex}, in which such desire results in eventual parricide. Derrida notes that this complex has the form of an infinite loop between father and child, which is bound to repeat itself, just as is the case with Freud's definition of the complex, which, according to the greek drama, is articulated by the figure of the parricide. In this sense, Derrida claims that Freud ``has shown how this archontic, that is, paternal and patriarchic, principle only posited itself to repeat itself and returned to re-posit itself only in the parricide'' \parencite[60]{Der95:Arc}.
	% 3/9/2018 9:47:58

	\paragraph{Institutional Passage}
	What is the case when this patriarchic principle is found on archives? 
	% you start with this question which has phrasing problems, but you don't answer it. patriarchy is in fact forgotten


	

	% I though effraction of the trace referred to the moment when you wrote the memory.


	% This is somewhat confusing actually

	The following pseudocode may help in this distinction:

	% \begin{flushleft}
	% \small
	% \begin{lstlisting}
	% function make_an_archive() {
	% 	public 	 	= PRIVATE
	% 	singular 	= PLURAL
	% 	place 	 	= COMMUNITY
	% 	archive  	= place = singular = public
	% 	return archive
	% 	}

	% function make_a_memory() {
	% 	private 	= PUBLIC
	% 	plural 		= SINGULAR
	% 	trace 		= SELF
	% 	memory 		= trace = plural = private 
	% 	return memory
	% }
	% \end{lstlisting}
	% \end{flushleft}

	\paragraph{Authorities}
	For example, the case of making an archive can be exemplified with the passing of a law or the signing of a contract. Derrida speaks of the act of consignation as constitutive of the archontic principle, since it is an act of ``gathering together signs'' \parencite[10]{Der95:Arc}, as is the case of the signatures on a bill or a contract. One's signature is private, therefore, the moment one signs a document one makes it public. Furthermore, in making a contract, multiple signatures need to be gathered into the same space: the plurality of the community needs to be condensed in the singularity of the document. Finally, the document itself comes to be a place, but it itself needs yet another place where it is stored, which comes to be the archive. What this act of consignation entails is, precisely, the archontic principle, since the contract itself now becomes an authoritative presence that emerged out of this function of making an archive. The structure of this presence will be discussed in the following section \see{spectrality}. For now, let us continue with the case of making a memory. The moment one perceives the world, one makes it private, inasmuch as it is inscribed in our memory. When one begins to remember, one begins the process of pathbreaking and effraction of traces ---and of forgetting, as I will continue to develop below--, which results in the plurality of the condition of traces, that is, on the difference between traces constitutive of memory.

	So, given that if this constitution of archive and memory are indeed reflections of each other, that is, one being the reverse of the other, how is the archontic present in the making of a memory? What would a reversed archontic principle look like? How would it act? These questions deserve a few paragraphs because they will explain the relationship between database, memory, and archive. 

	\paragraph{Anarchic Memory}
	What is important to note here, before continuing, is that just as memory is in a state of fracture and rupture, an archive is also in such state of discontinuity, and thus it is this condition of being in the form of disconnected gaps that makes memories and archives so alike. Spieker notes is that of discontinuity and rupture: ``Like all kinds of data banks, [the archive] `forms relationships not on the basis of causes and effects, but through networks''' \parencite[113]{Ern13:Dig}. From these two qualities of archives (filtering and fracture), their resemblance to memory can be drawn. However, while the archive is indeed patriarchic, memory, on the other hand, in terms of the archontic principle, it can only be described as an-archic. While there is indeed oneself remembering, there is no possibility of ruling over the plurality of traces that one has in memory. There is no place to go to remember things, one may try to trick the body into feeling similar things as when one was remembering so as to make a memory resurface, but in the end, memories remember themselves. One has no control over neither remembering nor forgetting. Memories forget themselves.

	\paragraph{Collective Memory}
	In the fantastic case of Funes \see{funeslude}, for example, forgetting is a mysterious lack that cannot be reduced to a simple loss of information. It is also a necessary condition for memory to be considered as trace, given that the moment a trace begins is when its own erasure begins. Thus, every memory comes with its own forgetting mechanism, a mechanism that is triggered on its own, and, further, that it is still not fully understood \parencite[292]{Wes08:How}. However, in the case of an archive or, what some (perhaps wrongfully) refer to as `collective' memory, the opposite occurs: data loss consists of plain and simple erasure, that is, a destruction that cannot be considered an instance of forgetting: ``in collectivistic memory, where the database has a tangible form, it is more apparent that permanent loss is a possibility. In archives, ink may fade, paper may crumble and entire files may end up in the shredder. \parencite[292]{Wes08:How}''






% I think you have a problem of form: In this last chapter in particular each section is a little world of its own and every time you start a new section 
% 
% the reader doesn’t know where you are going with the ideas presented. 
% 
% The reader is expecting to hear an argument, a new way of seeing things, but instead is getting new stuff then new stuff then new stuff and then sometimes a big reveal that is significantly shorter than the long sections that precede it. 
% 
% The problem of starting new sections from scratch every couple of pages is that 
% 
% it is tiring to try to figure out what this is about, so your reader (like me) is thinking : 
% 
% why are we talking about patriarchy in the archive right now? 
% Is this relevant to the dissertation and how? 
% how is the un-archic idea so small and yet it seems to be the only one that keeps being used after this? 
% 
% The fact that I am thinking about these things means that there is a problem in the structure. The problem is twofold, on one hand 
%
% these independent sections are too independent from each other and from the main argument, and on the other, 
% 
% the sections are too long, so I have to wait a couple pages to get an update on how you see these things related to “the aesthetic of the database” which is the title of this third chapter. 
% SO:::: [[and this is true for this whole chapter (nancy, latour, nancy again, derrida&freud)]]

% solutions are:

	% keep the connection to database aesthetics present all the time, not just every once in a while.

	% only introduce the things you really need for your argument and use them as soon as you can, don’t wait three pages to say, this is how it all fits together.

	% make these interludes shorter!! it just takes too long to get to the point where you are talking about database aesthetics!












% This was long overdue! It took so long to get here. It surprises me that there is barely any archive talk in this section here. So why is the archive section so long????

