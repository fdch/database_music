\begin{quote}
	Lo cierto es que vivimos postergando todo lo postergable; tal vez todos sabemos profundamente que somos inmortales y que tarde o temprano, todo hombre hará todas las cosas y sabrá todo.\footnote{Anthony Kerrigan translated this fragment as: ``The truth is that we all live by leaving behind; no doubt we all profoundly know that we are immortal and that sooner or later every man will do all things and know everything.'' In a more literal translation of the first sentence, it reads: ``What is certain is that we live deferring all that can be deferred.''} \parencite{Bor42:Fun}
\end{quote}
\begin{quote}
	All these differences in the production of the trace may be reinterpreted as moments of deferring\dots Is it not already death at the origin of life which can defend itself against death only through an \textit{economy} of death, through deferment, repetition, reserve? \parencite[202]{Der78:Wri}
\end{quote}

I would like to take a brief, but necessary, psychoanalytic detour, so as to set some context for the concept of memory. For this purpose, I take Derrida's reading of Freud's conceptualization of memory as a starting point towards distinguishing between human and nonhuman memory. 

\paragraph{Memory as Breaching}
According to Derrida \parencite{Der78:Wri}, Freud understood memory as the essence of the psyche: ``Memory, thus, is not a psychical property among others; it is the very essence of the psyche: resistance, and precisely, thereby, an opening to the effraction of the trace'' \parencite[201]{Der78:Wri}. In this sense, the perceived world enters as a force of effraction into the resisting unconscious, resulting in the inscription of a trace, that is, a memory. By definition, this is a violent process of impression (pressing \textit{in}), which is only possible by the notion of resistance. Thus, in order for something to leave a mark, there has to be some force acting against an other. In this case, one force is the world with its constantly changing images; another, the unconscious with its resistance to change. In this sense, Freud describes memory as breaching, or better, a state of being opened to this effraction. Memories are inscribed with incredible violence, one that can define the extent to which the inscription can be recalled later on, for instance, the case of repression, which is not forgetfulness but the deferred case of a force of containment. Furthermore, this breaching can be equally thought of as a fracture, hence the notion of effraction, which speaks of a the violent rupture. Despite, however, this violent image of path-breaking, Derrida writes of memory as an opening, in a paradoxical game between the resistance of the unconscious and the unavoidable (unclosing) state of being innocently ready for the next perception. 

\paragraph{Breaching and \textit{différance}}
Derrida points to one issue in Freud's opposing forces: the case where resisting forces meet equally strong forces, which would result in a paralysis of memory. Therefore, he proposes that it is ``the difference between breaches which is the true origin of memory, and thus of the psyche'' \parencite[201]{Der78:Wri}. In other words, Derrida reconfigures the psyche with the interplay of \textit{différance}, that is, a structure of infinite referrals and deferrals: ``trace as memory is not a pure breaching that might be reappropriated at any time as simple presence; it is rather the ungraspable and invisible difference between breaches'' \parencite[201]{Der78:Wri} Therefore, since the essence of the psyche (memory) is breaching (tracing) and the space and temporal dislocation of traces (\textit{différance}), a link between human and nonhuman can be established: writing. 

\paragraph{Hypomnesis and the Mystic Pad}
As Derrida points out, writing as \textit{hypomnesis} (as an externalization of memory) has been considered since Plato \parencite[221]{Der78:Wri}. However, Freud's metaphor for the perceptual system found yet another place: the Mystic Pad. There were many derivations of this device, but it consists in something like a writable surface board. When you write on it, a thin layer on top sticks to a sort of wax on the back, thus leaving a trace; when you lift the upper layer, the traces vanish completely. This special device came to represent, for Freud, the structure of the perceptual apparatus itself, or as Derrida says, it is where ``the psychical is caught up in an apparatus, and what is written will be more readily represented as a part extracted from the apparatus and `materialized''' \parencite[222]{Der78:Wri}. Therefore, this new (1925) hypomnesic device allowed Freud to shift from considering regular writable surfaces (paper), to a combination of resisting textures in a device which allowed for ``a perpetually available innocence and an infinite reserve of traces'' \parencite[223]{Der78:Wri}. Derrida, however, reconceptualized this differently. The crucial distinction he finds on this writing device is the notion of erasure, but most pressingly, that of repetition. In this sense, traces ``produce the space of their inscription only by acceding to the period of their erasure'' \parencite[226]{Der78:Wri}. This means that tracing constitutes a process whose only possibility is its own negation, that is, its own undoing of itself. Furthermore, these traces are, from the first moment, that is, from the beginning of the process of impression, ``constituted by the double force of repetition and erasure, legibility and illegibility'' \parencite[226]{Der78:Wri}. Hence, the spatio-temporal duality of the concept of \textit{différance}, that is, the interplay between spacing and deferments, which comes to be the ``work of memory'' itself \parencite[226]{Der78:Wri}.

\begin{quote}
	The `subject' of writing does not exist if we mean by that some sovereign solitude of the author. \textit{The subject of writing is a system of relations between strata}: the Mystic Pad, the psyche, society, the world. Within that scene, on that stage, the punctual simplicity of the classical subject is not to be found. In order to describe the structure, it is not enough to recall that one always writes for someone; and the oppositions sender-receiver, code-message, etc., remain extremely coarse instruments. We would search the `public' in vain for the first reader: i.e., the first author of a work. \im \parencite[227]{Der78:Wri}
\end{quote}

\paragraph{Nonhuman Authors}
Since the structure of memory, as I outlined above, can be comprehended as path breaking and \textit{différance}, the concepts of resistance, referrals, and deferrals play an important role in its definition. Furthermore, the externalization of memory comes as an instantiation of the very structure of not only our perceptual machine, but also of the structure of the psyche itself. This hypomnesis allows for a materialization of the psyche that is constituted as a tool, that is, as an apparatus to be handled. With this conceptualization of memory as writing, Derrida reconfigures the notion of the self of writing, that is, of an author. Therefore, when memory is thought of as writing, the classical notion of self begins to disappear, opening up the space for the nonhuman. In what sense can this disappearance of the self be accounted for if we substitute writing for databasing? How does the databasing self emerges in this non-origin of the originary moment of writing that Derrida describes above? How is this ``system of relations between strata'' to be understood in terms of the work of databasers? This is how a further step into the conceptualization of memory can be of aid, one that considers memory as resonance, and writing as databasing. I have already described above how resonance and memory constitute processes of \textit{différance}, that is, how the structure of sound, on the one hand, and traces, on the other, relate to spatiality and temporality. This is to say that, while the infinite situations of reference create a space of multiple situations, connections, associations, etc., which constitute an instance of signification, simultaneously, the ongoing process of deferral creates the oscillatory condition of perception, which constitutes the infinite return, the repetition, and the reverberation of a self. This self, however, is not an essential one, that is, it is not a substance, but it is understood as a resonant subject, \textit{the resonance of a return}, which, like Derrida's ``subject of writing'', does not exist in itself, and only appears as a result of a resonant ``system of relations between strata.'' 

\paragraph{Database as Agents}
Therefore, if this resonant self can only emerge out of this system ---or better, network--- of relations, then, every resonant point of the network must be considered as an agent in the constitution of a self. This means not only that the database, in this particular case of database music, becomes an agent of self-hood ---and, for that matter, an agent of authorship relating to the resulting work of database music. It also means that, as an instance of hypomnesis, that is, as technology that externalizes memory, the database appropriates the qualities relating to memory itself that were described above: breaching and \textit{différance}. Thus, not only can the nonhuman be reconceptualized within these qualities, also the human itself becomes reconfigured when faced upon the situation that memory, as breaching and \textit{différance} can also be found in the nonhuman. Therefore, given that the nonhuman and the human are engaged in this resonant network that cancels the classical notion of self, then, distinguishing between the human and the nonhuman cannot be carried out in classical terms, that is, in terms of substance, essence, etc., since they do not apply anymore. The only thing that remains is, basically, the body, the singular instance of a resonant skin, whether it be of a human or a nonhuman, of a databaser or a database; a skin, but also an open resistance to the forces of change.
