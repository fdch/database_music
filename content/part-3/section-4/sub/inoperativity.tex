\paragraph{Community as unwork}
As I have described above, for Nancy, a resonant self (self, from now on) is made of ``the singular occurrences of a state, a tension, or, precisely, a `sense''' \parencite[8]{Nan07:Lis}. Since these occurrences are in a permanent state of occurring, that is, never fixated in a whole (in `tension'), we can understand singular beings as being `interrupted' or `suspended.' These descriptions come from an earlier text \textcite{Nan91:The}, in which he extends this definition of the self to the definition of a community: ``community is made of the interruption of singularities, or of the suspension that singular beings are'' \parencite[31][All subsequent quotes from this passage.]{Nan91:The}. This is why we can understand `community' as ontological (grounded on a nature of being), and not as teleological (grounded on a purpose or an objective). In other words, if community is thought of as a product of the work of `selves' (teleologically), it follows that selves could also become the product of community. 

\img{work}{0.5}{
	A graph displaying the teleology of the work (arrows) of selves (dots) in community (dots joined by lines)
}{Community as work}

Nancy thus underscores that ``community is not the work of singular beings, nor can it claim them as its works'' \fsee{work}. Since community is ontological, Nancy understands it as the being of singular beings ``suspended upon its limit.'' Therefore, within this ontology of community, how is it possible for us to speak of the `work' of `selves' and of community if `work' is something that does not enter into its definition? Furthermore, how is the concept of the work of art redefined or framed within this ontology? Nancy's conclusion is that community can never result out of `work,' but it is something that unfolds as `unworking.' Borrowing from Maurice Blanchot's concept of \textit{desoevrement}, Nancy proposes `unworking' or `inoperativity' as a way to understand `work' within an ontology of community. `Unwork' is work withdrawing from itself: ``that which before or beyond the work, withdraws from the work.'' That is to say, Nancy points to a moment in which operativity separates from itself, and by that gesture, it distinguishes itself from both production and from a whole, or a finished product: ``no longer having to do either with production or with completion, encounters interruption, fragmentation, suspension'' \fsee{unwork}. Is this not the resonance of a return? Work returning as the interrupted resonance of its own unworking?

\paragraph{At the Limit}
Nancy's concept of community can be recognized within his later and broader concept of `resonance.' Given the fact that Nancy's ontology of sound points to the distance or the interval between sense and signification, and thus, to the emergence of a resonant subject during the sonorous presence, this distance can be thought of as suspended at a limit. We can think of this limit as an edge in the resonant network that, in Nancy's terms, exposes selves to themselves and to one another. 

\img{unwork}{0.2}{
	A graph displaying community as an ontology of unworking.
}{Community as unwork}

Further, Nancy provides us with an essential insight, suggesting that ``it is not obvious that the community of singularities is limited to `man' and excludes, for example, the `animal''' \parencite[28]{Nan91:The}. Therefore, by understanding resonant networks in terms of community we can speak of an exposure of selves, or a self-exposure, between humans and nonhumans in a liminality can be thought of as a skin: ``a singular being \textit{appears}, as finitud itself: at the end (or at the beginning), with the contact of the skin (or the heart) of another singular being\dots'' \parencite[28]{Nan91:The}. This skin (or heart) makes at least two references, one to the affective presence of the body in relation to touch, and another to the finitud that singular beings are, that is, what (for Nancy) is the being in common of beings: their life, but fundamentally, their death. First, with this latter reference, we can depart a bit from Nancy's `skin,' and stretch it away from this living/nonliving distinction altogether. Latour's ontological hybrid can thus enter into the considerations I am proposing here: ``an actor-network is an entity that does the tracing and the inscribing. It is an ontological definition and not a piece of inert matter in the hands of others, especially of human planners or designers. It is in order to point out this essential feature that the word `actor' was added to it'' \parencite[7]{Lat90:On}. Therefore, what is in common along the network is not finitud in terms of death, but in terms of the very condition of liminality itself: actor-network at the limit.

\paragraph{Infralanguage}
Latour considers that the descriptive project in \gls{at} does not compose a metalanguage with which to define overarching explanations. Instead, \gls{at} searches for explications by retaining ``only a very few terms ---its \textit{infra}language--- which are just enough to sail in between frames of reference'' \im \parencite[16]{Lat90:On}. On the one hand, with this infralanguage \gls{at} arrives at an ``empty frame for describing how any entity builds its world'' \parencite[16]{Lat90:On}. That is to say, it is a descriptive project focused on ``meaning-production,'' that is, creating associations and connections. On the other hand, with this infralanguage \gls{at} ``grants back to the actors themselves the ability to build precise accounts of one another by the very way they behave'' \parencite[16]{Lat90:On}. That is to say, precisely by disarticulating the search for `overarching' explanations by means of a \textit{meta}language (\textit{the} key to open \textit{the} door), this infralanguage is a keychain (of tiny keys) that opens many doors for the new ``ontological hybrid'' that is the actor-network, because it is what allows actors themselves to build their own account, that is, their own world: ``world making entities'' \parencite[16]{Lat90:On}. 

\img{expand}{0.4}{
	``We required the material for the head to be elastic, to have a contrasting dark color, and to resist deformation and breaking. We are currently using spandex'' \parencite[3]{DBLP:conf/icmc/OliverJ08}. Image taken from the ``Spandex'' entry of Wikipedia: ``Polyurethane Fibers Optical microscopy Polarized Light Magnification 100x (cross-polarized light illumination, magnification 100x). Created: 1 January 2015.''
}{Spandex fibers under an optical microscope}

\paragraph{Infraskin}
Unlike in Latour's network, affectivity appears in Nancy's `sense,' that is, on the body, its touch, and as before, in listening. \gls{at} quite literally forbids this, treating any ``homogenous morphism'' as ``exceptions which should be accounted for'' \parencite[16]{Lat90:On}. That is to say, if we may think of a skin upon which we (humans) resonate in self-exposure (community as unwork), then this skin must be an exception, because it is only human, or animal, or reduced to living entities with affective bodies. However, perhaps we can ``account for'' this skin as exceptional, and consider it ``x-morphic,'' that is, ``anthropo-morphic, but also zoo-morphic, phusi-morphic, logo-morphic, techno-morphic, ideo-morphic'' \parencite[16]{Lat90:On}. This x-morphic skin is what we (humans and nonhumans) have in common as singular beings in the network, what for Nancy is ``at the confines of the \textit{same} singularity'' \parencite[28]{Nan91:The} would be at the irreducible limit of an `infra' skin. Nancy is aware the this skin is not a total, complete, or superior skin:\footnote{For Nancy ``there is no communion of singularities in a totality superior to them and immanent to their common being'' \parencite[28]{Nan91:The}.} like selves and community, it is interrupted. Interruption, however, in Latour's terminology relates to an ethics of the network where `good' and `bad' are understood in terms of navigation, mediation, and reduction. More connectivity is good, less is bad: ``either an account leads you to all the other accounts\dots or it interrupts constantly the movement, letting frames of reference distant and foreign\dots'' \parencite[13-14]{Lat90:On}. Nevertheless, we can understand interruptedness as a general condition of irreducibility, which is, contradictorily enough ``the highest ethical standard for \gls{at} \parencite[14]{Lat90:On}. In being interrupted, the relation itself becomes ontological, since it is thus the unworking of the work of actors. The new `ontological hybrid' actor-network can be understood, then, as comprised of the \textit{unwork} of actors. This is the suspension at the limit: within the irreducibility of this interruptedness, this infraskin is simultaneously ``always \textit{other}, always shared, always exposed'' \parencite[28]{Nan91:The}. Like the skin of the Silent Drum Controller \see{applications:hybrid_queries}, suspended along the rims, stretching and compressing to touch, and making mechanical waves through the resonances of its silence, this infraskin would resonate at the limit. The Silent Drum tracks shapes made with the elastic fabric folding and unfolding against the human skin, therefore it ``acts as the limits, but also as an extension of the human body'' \parencite[2]{DBLP:conf/icmc/OliverJ08}. I would interject here, and propose that instead of an ``extension of the human body'', the Silent Drum can be thought of as extending Latour's network \textit{with} the body, that is, it grants affectivity to the network. In this sense, this infraskin is not a layer that separates the interior form the exterior of a body, or a surface under which or over which two selves can connect or extend. This infraskin is not just the promise of a new cultural algorithm (in Manovich's sense). This skin is not a surface, but it can be thought of as a texture; not a layer, but an interweaving of elastic fibers \fsee{expand} that, in their own locality are fragile, but due to their reticulated structure expand into a redundancy of fragilities. And we can listen to it.  

\paragraph{Database Community}
With resonant networks as an instantiation of a process of unworking, we can think of database communities. That is to say, the database and the databaser engage in a form of touch. This touch is easy to feel in the case of the Silent Drum because of the elasticity of the drumhead resisting the pressure of the hand. The material resistance of the skin represents, thus, the evidence of the strength of connectivity itself: ``strength does not come from concentration, purity and unity, but from dissemination, heterogeneity and the careful plaiting of weak ties'' \parencite[3]{Lat90:On}. In this sense, if one removes the skin one would be left with no resistance. Hence, the Silent Drum would be rendered weak. But, this infraskin or its resistance does not need to be visible for the community of the human and the non to exist. This is the case of the MANO controller \parencite{DBLP:conf/icmc/OliverJ10}, where the elastic tissue that was tracked in the Silent Drum is removed, and what is tracked is now the skin of the hand (\textit{mano} in spanish) directly. This physical elision points equally to the (human) invisibility of the infraskin, as to the sonorous presence of its resonance. In making this controller, it can be argued that Oliver La Rosa was precisely making audible not only the gap between the sensor and the dark background interrupted by manipulation, and not only the different gestures recognized and tracked by the algorithm, but also the gap (the limit) between the human and the nonhuman, what I am calling here the infraskin. In making this skin resonate with a database, the result is a sonic event that instantiates the community of this limit. That is to say, all the elements that compose the multiplicity of the construction (sensor, video stream, C++, C, Pure Data, the parameters obtained from video analysis, etc.) enter into resonance with each other, and thus manifests the sound of a community. The ``trace of the edge of the hand'' \parencite[2]{DBLP:conf/icmc/OliverJ10} becomes the trace of the resonance of a database community. Or simply, database music.

In a database community, database music can be understood as a hybridly social and communicative event. As an unworking of databasers and databases, it can be considered as the infraskin upon which we (human and nonhuman) resonate, as well as the trace and the tracing of its resonance. The qualities of the music work that result from it can also relate to incompleteness, suspension, as well as to fracture, instability, and interruption. In this sense, databaser and a database can be heard in resonance with each other, as well as in communication with each other, but it is a resonance or a communication that is not teleological. Communication, in this context, refers not to language, or to information, but to a property of being in common. In Nancy's sense, the being in common of singularities; in Latour's sense, the connectivity and associations of actors. For Nancy, ``communication is the unworking of work that is social, economic, technical, and institutional \parencite[31]{Nan91:The}. Communication, in this sense, is also the unworking of work that is resonant, and thus it can be considered as a music unwork. I will refer to this in later sections \see{anarchy}. In what follows, I will focus on the `trace' of this resonance.

