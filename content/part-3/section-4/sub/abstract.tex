% The concepts exposed in \nameref{chapter:Database_Aesthetics} affect and reconfigure transversally the practices of composition and databasing. Traditionally, music composition was considered a single author practice, in which the composer's technique or aesthetic intuition is the sole agent, romanticizing the artist as an ``involuntary vessel through which inspiration flows'' \parencite{Bor95:Rat}. As I have outlined in , this is no longer the case, since understood in terms of its resonance and of its performativity, composition explodes the name of the composer, leaving as many spectral remains of its trace as can be imagined. Conversely, databasing is already embedded in a networked structure that only allows partial and temporary allocation of authors (databaser), since in the structural database tree exist multi-authored branches that renew themselves, outgrowing themselves in perpetual difference and instability. The notions of stability and authority can only be related to snapshots in the history of a software. However, the institutional quality of both databasing and composition is still at play, namely in the many cases of proprietary software and in the composer's name, that is, in the commercial release and the objectification of music work. Less than focusing on general criticism, in this section I argue that, since the agency of the database reveals itself as aesthetic experience, then it is the dynamics of this agency need to be addressed. I claim that this agency, when contextualized within music composition, specifically composing with computers, it has the form and the politics of a music listening to itself. 