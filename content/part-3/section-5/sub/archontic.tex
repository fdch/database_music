Derrida's broader notion of `writing' relates to what he calls an `originary' trace. As I described earlier, the inscription of the trace contains from the start its own erasure. This means that, in a reciprocal motion, the origin of the trace is itself originary and non-originary: the origin of a disappearance and the disappearance of the origin. This apparent contradiction can be approached by the actual process of writing something on a page: where or when does the writting begin, when the pen touches the paper or when the paper lends itself for writing? The same question applies to the crossfade between the `I's in the case of Alvin Lucier's work \see{network}, where or when does the `I' begin `sitting' in the loop? Further, no matter how many metaphors we may give to this paradox, it simply falls out of empirical quests. Therefore, this paradox of the trace is understood by the concept of \textit{différance} that we outlined above. The repercussions of this word that ``is not a word and it not a concept'' have now a long history that does not interest us fully, therefore we can will simply jump thirty years ahead in time, after computers entered the scene, and after the Internet became popularized by the World Wide Web.


\textcite{Der95:Arc} exemplified the intricacies that come out of the archivization of a private place with the Sigmund Freud Museum, which is located in the same house where Freud lived.\footnote{\url{https://www.freud-museum.at/en/}} Derrida delineated an economy of archives into he calls the \textit{archontic} principle. `Economy' is related to the greek word \textit{oikos} (house), and \textit{nomos} (man-made law). The archive condenses this economy within the confines of a public place. The word `archive' comes from the greek \textit{archē} which relates, on the one hand, to the originary, as well as to the ruling. Thus, the archontic principle is a type of authority that the archive exerts, which can be understood as a powerful dictating rule that is before anything else. Hence, its categorization as principle, which is also related to the origin (e.g., the latin root \textit{principium} which refers to the beginning) and to the figure of the ruler (principal or prince). 

As hypomnesis, archives represent for Derrida another instance of the movement of technology. This movement consists in ``a transformation of the techniques of archivization, of printing, of inscription\dots'' \parencite[16]{Der95:Arc}. These `archival machines' which developed in the thirty years since he first considered Freud's \textit{Note on the Mystic Pad} (as a technical agent in Freud's structure of the psychich apparatus), had reached Derrida's e-mail inbox. His comments on how e-mail technology would have (and will) affect psychoanalysis attest to this fact. Further, they pose a question that he leaves nonetheless unanswered, which is why I consider his discussion of the archive relevant to our discussion of database aesthetics. Derrida finds in Freud's metaphor for the psychic apparatus a point of departure to the question of how psychoanalisis changed by the presence of a technological device. In this case, the device was a toy called the Mystic Pad, a children's writing board made of wax and a thin layer on top: upon impression it leaves a trace; when one lifts the layer, the trace erases. The structure of the psyche for Freud could thus be understood by using both hands simultaneously, one pressing (writing), the other lifting (erasing). Writting in 1995, Derrida poses the following question

\begin{quote}
	Is the psychic apparatus better represented or is it affected differently by all the technical mechanisms for archivization and for reproduction, for prostheses of so-called live memory, for simulacrums of living things which already are, and will increasingly be, more refined, complicated, powerful than the `mystic pad' (microcomputing, electronization, computerization, etc.)? \parencite[16]{Der95:Arc} 
\end{quote}

% I will begin to approach this question in two ways, but first I would like to narrow down the scope of its premises. On the one hand, the `psychic apparatus' only interests us in relation to what I have mentioned of memory, and more precisely of a certain quality of the unconscious (death drive). On the other, the techno-science or technical mechanism that is in question here is that of the database. Thus, I will proceed to break the question appart, and then turn it upside down, so as to bring it into the sphere of database aesthetics. 

``Is the psychic apparatus better represented or is it affected differently?'' The structure of this question points to two possible answers. Either technology is a representation of the psyche, or the understanding of the psyche is changed by technology. On one hand, the question refers to how distant we are in the movement with which we can arrive at the object (psychic apparatus). This is a temporal movement which relates to deferral: how much longer until we get our object? The answer, as we have seen, is simply never. In order for the represented to be fully represented it must become present. On the other hand, the `technical device' has the potential to affect the object: it differs from and might reformulate thus the psychic apparatus. To a certain extent, this suggests that these `prosthesis' of `live' memory, or `simulacrums of living things' represent some form of (disembodied) nonhuman (life) against which the question is asked. We can only guess. In any case, Derrida claims it is a question of progress or evolution, in which ``neither of these hypothesis can be reduced to the other'' \parencite[16]{Der95:Arc}. It appears, therefore, that this quest reaches a dead end. Derrida does not continue on this quest, and instead takes the text in directions that are not pertinent for our purposes here. Therefore we can stress on this question for a while. 

The structure of the memory (``the essence of the psyche'') in opposition to the archive can be understood with the opposition of \textit{anamnesis} (the act of recalling, remembering) and \textit{hypomnesis} (the technical storage device); the former internal, the latter external. We can advance now that, like the archive, the database is hypomnesis and external (to the psyche), just as well as it exerts the principles of the archontic. As I have described earlier \see{chapter:part-2}, a database comprises the partition within computer memory where data is stored. However, in order to store data, one has to assign data types and strucures, thus providing with the necessary structural frame that indicates how to access the data. One of the key concepts of the `archontic principle' is `consignation,' which has at least three meanings: assigning residence, entrusting something in reserve or deposit, and what Derrida calls ``gathering together signs'' \parencite[10]{Der95:Arc}. All of these can be represented with memory management, that is, by `declaring' and `initializing' variables, as well as by making `unions' or more complex data structures. Consignation in archives has, for Derrida, a presupposing aim: to ``coordinate a single corpus'' \parencite[10]{Der95:Arc}. This is exactly the case of a database, which extends this coordination with the posibility of networks. The various database models that I have shown in previous sections point to the different ways with which to coordinate the economy of database systems. In order to flip Derrida's question upside down, we can ask ourselves what to do with all these technical mechanisms that microcomputing enables? How do we make something out of them? And further, what is in them that we can call aesthetic? Instead of thinking how memory is represented in the database, we can ask how we can represent the database within memory. Further, instead of asking how memory is affected by the database, we can ask ourselves how can memory affect databases. The theoretical and aesthetical aspects of this reversal can be seen as follows. On one hand, this reversal points to a reconceptualization of databases within the ``evolution of archival techno-science'' \parencite[16]{Der95:Arc}. On the other, this reversal calls for a restructuring or a technical reconstruction of the database in relation to memory. 



