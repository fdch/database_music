%  The audiovisual archives are thus the real interplay of traditional and digital archives. Analog technological storage devices (such as magnetic tape) operate, anarchivally, in the material sphere of magnet spots and electro- magnetic induction (the symbolic ordering, for instance, the counter on a video recorder, is extrinsic and has to be mechanically added). Computer matrix memories, in comparison, are closer to the symbolic ordering of the classical archive, with a clear address structure: they are microarchives and similar to the digital library, where the phonograph record and film were the previous alternatives to the alphabetical library. Alphanumerics herald the advent of a new kind of library expressed in the informatic concept of pro- gram libraries. ENRST 92


% The phonograph (Emil Berliner’s gramophone) registers the whole range of acoustic events. Whereas in musical notation (developed by the Greeks in analogy to the alphabet and later differently by Guido of Arezzo) a symbolic recording takes place, the phonograph registers the physically real frequency. The alphabetic symbolism reduces acoustic events to the “musical” (harmon- ical order), whereas the register of the real encompasses the sonic (includingnoise, arhythmical temporal phase shifting, the “swing,” differing amplitudes and frequencies)—an anarchive of sound in technological storage as opposed to the archival order of musical notation. ERNST 	173-4



\paragraph{Anarchic Records}
In his conceptualization of the \textit{anarchoarchive}, Wolfgang \textcite{Ern13:Dig} opposes technical recording with symbolic transcription. A microphone captures the entire sonic environment forming an involuntary memory that becomes a form of anarchic archive, or \textit{an-archive}. 
%that is, an anarchive that...
% \footnote{of ``past acoustic, not intended for tradition: a noisy memory'' \parencite[174]{Ern13:Dig}}
% 

As an example, Ernst claims that Bela Bartok's transcriptions to musical notation of Milman Parry's Serbian epic song recordings\footnote{\url{https://mpc.chs.harvard.edu/}} constitute an archivization process by which symbolic transcription leads to an ordered archive, i.e., a score.\footnote{Another example Ernst provides of the anarchive is the Internet itself: `` [The Internet] is a collection not just of unforeseen texts but of sound and images as well, an anarchive of sensory data for which no genuine archival culture has been developed so far in the occident'' \parencite[139]{Ern13:Dig}.} In contrast, a sound file storing samples of the recorded world is anarchic because of the unfiltered (`pure') way in which the world is inscribed by the recorder. Ernst bases this idea on the work of media theorist Sven Spieker, who claims that, following Derrida's conceptualization of the archive, a central feature of archives is not memory, but a need to ``discard, erase, eliminate'' that which is not intended for archivization \parencite[113]{Ern13:Dig}. In other words, archives filter and frame. 
%
% This needs to be rewritten. perhaps in a single strong sentence.
The problem is that, the moment Ernst deposits on the recording technology the disappearance of the body, he dislocates framing altogether, and this is how the concept of `pure' recording comes in. Therefore, if for Ernst, Parry's audio recordings are anarchic, this is because they are constructed upon a Kittlerian idea of purity that leaves the human out of the equation.

\paragraph{Memory and Framing}
% rewrite for clarity
If we consider the body's capacity to create images from the world, then framing is also a possibility condition for human memory. 
%
% again, needs clarity, you need make sure what frames what
In fact, we have seen with the Derridean trace that it is a resisting force what first enables the violent rupture of traces, and thus, this force can also be considered as an instance of framing itself. 
%
% as -- as
Resistance of the psyche as creativity taking action as the moment of framing. 
% the arrow seems a bit abstract. the example is not contributing to clarity in this case. ((MISSING PSEUDOCODE LISTING))
Furthermore, framing can also be inversely considered as the arrow between the private and the public in the case of archives, since, for example, in the case of a contract, only a limited amount of signatures are needed, thus a frame needs to be an active part of the process of archivization.
%  CONCLUSION::: framing is in between the public and the private



\paragraph{Nonhuman Tympans}
Audio recording can be seen as either memory or archive depending on their role in mediating private and public spheres. The world can be considered public because it is in a constant state of availability at any time, but a microphone can be considered an actor of privacy, since it prevents some sounds from being recorded while allowing  others to pass through. In this sense, a transducer's filtering capacity enacts the passage from the public to the private and in the case of speakers from the private to the public. For Jonathan \textcite{Ste03:Aud}: ``every apparatus of sound reproduction has a tympanic function at precisely the point where it turns sound into something else\dots and when it turns something else into sound'' \parencite[34]{Ste03:Aud}. Therefore, considering these nonhuman eardrums (tympanus) as as actors of privacy and publicness, audio reproduction technology can be compared to the structure of memory and archives. 
%loudspeakers are not eardrums proper
% you use memory, but computer or human?

% somewhat unlcear
For example, audio recording can be conceived as a form of memory, only to the extent that its inscription is singular yet reproducible. This is because a magnetic tape or a hard drive cannot be considered in the same (plural) way as Derridean traces, since once data is stored, it exists only once in discrete, limited space, unless copied to another location, in which case it becomes a duplicate, an identical clone of itself. In contrast, audio playback can be considered a form of archive, since it consists of the passage from the private stored waves to the publicly reproduced air pressure waves in space.
% don't say what needs to be done, just do it! (solved)

\paragraph{Spectrality of Archives}
Transducers as nonhuman eardrums constitute the limit between the sonic world and the binary world of databases.\footnote{Transducers can also be understood as the limit be between databases and the tactile world \parencite[223]{Eck13:Bet}} Comparing these transducers to memory and archives suggests a hybrid object: one that, on the one side, becomes a private and singular `trace', and, on the other, becomes a public, resonant `space.' Thus, databases represent neither a trace nor a space. According to Derrida, ``the structure of the archive is \textit{spectral}. It is spectral \textit{a priori}: neither present nor absent ``in the flesh,'' neither visible nor invisible, a trace always referring to another whose eyes can never be met\dots'' \parencite[54]{Der95:Arc}. The hybridity that the database projects when compared to memory or archives points to a certain uncanniness, that is, precisely to the hauntedness that comes from its spectrality. 
% In this context, spectral means ….

Derrida claims that, addressing a phantom is a ``transaction of signs and values, but also of some familial domesticity'' \parencite[55]{Der95:Arc}, meaning, on the one hand, that in the uncanny encounter with a ghost, there is familiarity, that is, there emerge feelings of what is known to be close to us, but also that which composes the authority of that closeness. Therefore, embedded in this familiarity is the archontic, the oedipal, etc., and thus the expression of power that this apparition brings forth. On the other hand, the familiar is also related to an economy, that is, to the passing through (trans-action) of signs, but also of translation ---or better, the `transduction' of things. This is why Derrida considers any encounter with the spectral to be an instance of `addressing,' which implies this uncanniness of the ghost entails is nothing other than the haunting itself, as Derrida writes: ``haunting implies places, a habitation, and always a haunted house'' \parencite[55]{Der95:Arc}. Like Funes' acousmatic voice, from the shadows as a shadow, the transduced publishes and the database haunts.\footnote{For a detailed revision of acousmatic history, see \textcite{Kan14:Sou}} Furthermore, it can be argued that considering the database in such way is plain and simple delusion, that is, insanity at its best. Not surprisingly, this is the point exactly; within this delusion exists truth:

\begin{quote}
	\dots it resists and \textit{returns}, as such, as the spectral truth of delusion or of hauntedness. It \textit{returns}, it belongs, it comes down to spectral truth. Delusion or insanity, hauntedness is not only haunted by this or that ghost\dots but by the specter of the truth which has been thus repressed. The truth is spectral, and this is its part of truth which is irreducible by explanation\dots \parencite[54-56]{Der95:Arc}
\end{quote} % 3/9/2018 9:24:10

\paragraph{Spectrality of Databases}
Building on Derrida's views on the structure of the archive as spectral, I suggest we can consider databases spectral too but in a different sense. The spectrality of the database comes not from its hauntedness,
 % but from …… 

 % ---that is, it is not domiciled since, 

As humans, we have no direct access to data space with our bodies and thus we cannot engage in transaction with the specter directly: we need transducers. In the case of audio playback, 
%
%I see playback and recorded, but what about synthesis and other techniques?
%
we need loudspeakers to recreate the recorded image. 

% These two sentences need a bit more work to connect to the prevous section
%%%%%%%%%%%%%%%%%%%%%%%%%%%%%%%%%%%%%%%%%%%%%%%%%%%%%%%%%%%%%%%
In consequence, while embodying stored data, we embody its specter.  %%????
Therefore, the hauntedness needs to be `transduced' into space, and with this transduction the agency of the database can be effectively felt. 
%%%%%%%%%%%%%%%%%%%%%%%%%%%%%%%%%%%%%%%%%%%%%%%%%%%%%%%%%%%%%%%%%%%%%
% Very unclear ...
Likewise, the uncanniness of the case of audio recording can be felt as the ghost that will `remember us,' because of this computer `memory' that will keep (to its own privacy) the sonic environment that attests to our presence in space ---because we sing, we make sounds, etc. 

% This sentence does not make sense!
Hence, the haunting that resides in-memory, together with the stored data and pointers.



\paragraph{Agency of the Uncanny}
% unclear

In recognizing this presence, which is the archontic specter of the database, comes the recognition of the agency of the database itself, and, furthermore, of the quality of the aesthetic experience that we encounter whenever there is a database in art. As Timothy Morton \parencite{Mor13:Hyp} writes in relation to what he calls hyperobjects:

\begin{quote}
	Recognition of the uncanny nonhuman must by definition first consist of a terrifying glimpse of ghosts, a glimpse that makes one's physicality resonate (suggesting the Latin \textit{horreo}, I bristle): as Adorno says, the primordial aesthetic experience is goose bumps. Yet this is precisely the aesthetic experience of the hyperobject, which can only be detected as a ghostly spectrality that comes in and out of phase with normalized human spacetime. \parencite[169]{Mor13:Hyp}
\end{quote}

Insofar as databases are spectral, they can be considered hyperobjects in Morton's sense, and thus, as agents translating through aesthetic networks.
%%??
However, I will not dwell much longer on this definition in this text. 
%
%criptic
If we consider the ``recorded movement of a thing'' with which Latour identified his concept of the network, % re-phrase this part of the sentence
databases are agents not only of recording, also of motility, and of thing-hood itself. 
%
% criptic explanation
This is to say that, in the constitution of networks, the spectrality of the database expands in multiple directions. 
%
The illusory violin in the acousmatic concert I mentioned earlier \see{resonance_of_a_return}, exemplifies Hansen's concept of the creation of auditory images our brain's capacity for virtuality has: our imagination. The sound of a violin can be recorded or synthesized into the privacy of a database. Then, it can be played back with loudspeakers located in such a way that they emulate the location of an actual violin player. As a result, the listener could very likely imagine a physically present violin in the room, a sound without a body, that is, a ghost. This ghost comes in as the phantom of a human player; of the violin; of the histories and traditions that those two elements bring forth; of the presence of the nonhuman that the database implies; of the privacy that is not human but is still uncannily private; of the plurality % is this explained before?
that is embedded in the construction of databases; of the hauntedness of the archontic that the above sets forth; and so on. In this way, the spectrality of the database attests to its relation to memory and archives, and, thus, to its aesthetic resonance within our experience.

% This needs reworking.
% you could use the term embody and materialize in this context as you have used these words before.
This hauntedness can be indeed instantiated into appearance, not only in the form of authority, as I have shown in the case of its archontic presence, but also in the form of style, and most important, gender: the performativity of the database.
