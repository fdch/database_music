
\paragraph{Anarchic Records}
In his conceptualization of the \textit{anarchoarchive}, Wolfgang \textcite{Ern13:Dig} opposes technical recording with symbolic transcription. A microphone captures the entire sonic environment forming an involuntary memory that becomes a form of anarchic archive, or \textit{an-archive}. For Ernst, an `anarchive' presents no intrinsic symbolic ordering, in contrast to the highly structured ordering inherent to archivization. From this distinction he draws two comparisons. First, he compares analog and digital recording technologies: the former are anarchic because they ``operate\dots in the material sphere of magnet spots and electromagnetic induction;'' the latter are ``microarchives'' due to their ``clear address structure'' \textcite[92]{Ern13:Dig}. Second, he finds in musical transcription another expression of archival order. For example, Bela Bartok's transcriptions to musical notation of Milman Parry's Serbian epic song recordings represent an archivization process by which symbolic transcription leads to the ordered archive that constitutes a score: ``an anarchive of sound in technological storage as opposed to the archival order of musical notation'' \parencite[174]{Ern13:Dig}. What these oppositions point to is the middle term that constitutes the object of the ``media archaeological'' project, whose central focus is an awareness of the mediating device: ``at each technologically given moment we are dealing with media not humans'' \parencite[183]{Ern13:Dig}. Ernst bases much of his considerations of the archive on the work of media theorist Sven Spieker, who analyzes Derrida's conceptualization of the archive and its destruction drive. For Spieker, the central feature of archives is not its hypomnesic quality, but a certain need to ``discard, erase, eliminate'' everything not intended for archivization \parencite[113]{Ern13:Dig}. In other words, archives filter and frame, and thus any archivization project is tailored to the technical intricacies of the media involved. However, Ernst radically reminds media archaeologists: ``we are not speaking with the dead but dealing with dead media that operate'' \parencite[183]{Ern13:Dig}.   

At the core of archivization projects, as I have mentioned earlier, is an institutional passage from the private to the public. The force that constitutes this passage is the destructive drive, the archive fever, which exists as the very threshold of the inside and the outside of the archive. The passage (and not the force) is thus marked on the mediating technology. The reverse happens in the psychic apparatus. The public becomes private by the effraction of the trace, which is also an expression of the same destructive force that is the death drive. The traces are thus the marks of the psychic activity. To a certain extent, we can understand this drive with transducers.

\paragraph{Nonhuman Eardrums}
Audio recording can be seen as either memory or archive depending on their role in mediating private and public spheres. The world can be considered public because it is in a constant state of availability at any time, but a microphone can be considered an actor of privacy, since it prevents some sounds from being recorded while allowing others to pass through. In this sense, a transducer's filtering capacity enacts the passage from the public to the private and in the case of speakers from the private to the public. For Jonathan \textcite{Ste03:Aud}: ``every apparatus of sound reproduction has a tympanic function at precisely the point where it turns sound into something else\dots and when it turns something else into sound'' \parencite[34]{Ste03:Aud}. Therefore, considering these nonhuman eardrums (tympanic membranes) as actors of privacy and publicness, audio reproduction technology can be compared to the structure of human memory and archives. A microphone becomes the threshold from the public to the private, from the outside to an inside that as far as we can tell for our knowledge of our own memories, is made of psychic activity, that is, memory. A loudspeaker, as a reversed eardrum would publish all there is to sound of an archive, the privately stored waves to the publicly reproduced air pressure waves in space. Perhaps this function of the transducer is in itself so representative of the very own death drive, that when in front of them, or around them, we feel a certain call for performance, and a certain disposition towards listening, which might have something to do with something other than just media. That is to say, even though they are ``media and not human'' transducers engage us humans with a ghost, neither dead or alive, neither material or immaterial: in lack of a better word, what is known as \textit{spectral}. 

\paragraph{Spectrality of Archives}
Transducers as nonhuman eardrums constitute the limit between the sonic world and the binary world of databases.\footnote{Transducers can also be understood as the limit be between databases and the tactile world \parencite[223]{Eck13:Bet}} Comparing these transducers to memory and archives suggests a hybrid object: one that, on the one side, becomes a private and singular `trace,' and on the other, becomes a public and resonant `space.' Thus, databases represent neither a trace nor a space. According to Derrida, ``the structure of the archive is \textit{spectral}. It is spectral \textit{a priori}: neither present nor absent `in the flesh,' neither visible nor invisible, a trace always referring to another whose eyes can never be met\dots'' \parencite[54]{Der95:Arc}. The hybridity that the database projects when compared to memory or archives points to a certain uncanniness, that is, precisely to the hauntedness that comes from its spectrality.\footnote{Not to be confused with the Fourier-based French `spectralists' of the 1970s.} Derrida claims that, addressing a phantom is a ``transaction of signs and values, but also of some familial domesticity'' \parencite[55]{Der95:Arc}, meaning, on the one hand, that in the uncanny encounter with a ghost, there is familiarity, that is, there emerge feelings of what is known to be close to us, but also that which composes the authority of that closeness. Therefore, embedded in this familiarity is the archontic, the Oedipal, etc., and thus the expression of power that this apparition brings forth. On the other hand, the familiar is also related to an economy, that is, to the passing through (trans-action) of signs, but also of translation ---or better, the `transduction' of things. This is why Derrida considers any encounter with the spectral to be an instance of `addressing,' which speaks of a certain uncanniness coming from the hauntedness of that which is at once familiar and unfamiliar: ``haunting implies places, a habitation, and always a haunted house'' \parencite[55]{Der95:Arc}. Like Funes' acousmatic voice, from the shadows as a shadow, the transduced publishes and the database haunts.\footnote{For a detailed revision of acousmatic history, see \textcite{Kan14:Sou}} Furthermore, it can be argued that considering the database in such way is plain and simple delusion, that is, insanity at its best. Not surprisingly, this is the point exactly; within this delusion exists a certain truth:

\begin{quote}
	\dots it resists and \textit{returns}, as such, as the spectral truth of delusion or of hauntedness. It \textit{returns}, it belongs, it comes down to spectral truth. Delusion or insanity, hauntedness is not only haunted by this or that ghost\dots but by the specter of the truth which has been thus repressed. The truth is spectral, and this is its part of truth which is irreducible by explanation\dots \parencite[54-56]{Der95:Arc}
\end{quote} % 3/9/2018 9:24:10

\paragraph{Agency of the Uncanny}
Building on Derrida's views on the structure of the archive as spectral, I suggest we can consider databases spectral too but in a different sense. The spectrality of the database comes when we (databasers) become part of the loop by databasing. The loop, that is, the circuit, the access, the programs, the transducers, the feedback, etc., in performance, in listening, in composing, etc. We engage with spectrality at this loop, that I also have called skin (or infraskin), and I also referred to as resonant network. As humans, we have no direct access to data space with our bodies and thus we cannot engage in transaction with the specter directly: we need transducers, we need media. Once we begin residing in the loop, we begin resonating with the database, but we also begin resonating with its spectrality. However, this does not mean that we have become an `actual' ghost, or a spirit, or that we have suddenly become enlightened with a transcendental \textit{deus ex machina}. Upon resonating in this spectral loop we can recognize a certain aesthetic agency of the database, one we can encounter whenever there is a database in art: ``Recognition of the uncanny nonhuman must by definition first consist of a terrifying glimpse of ghosts, a glimpse that makes one's physicality resonate (suggesting the Latin \textit{horreo}, I bristle)\dots  \parencite[169]{Mor13:Hyp} The illusory violin in the acousmatic concert I mentioned earlier \see{resonance_of_a_return}, exemplifies Hansen's concept of the creation of auditory images our brain's capacity for virtuality has: our imagination. The sound of a violin can be recorded or synthesized into the privacy of a database. Then, it can be played back with loudspeakers located in such a way that they emulate the location of an actual violin player. As a result, the listener could very likely imagine a physically present violin in the room, a sound without a body, that is, a ghost. This ghost comes in as the phantom of a human player; of the violin; of the histories and traditions that those two elements bring forth; of the presence of the nonhuman that the database implies; of the privacy that is not human but is still uncannily private; of the hauntedness of the archontic that the above sets forth; and so on. In this way, the spectrality of the database attests to its relation to memory and archives, and, thus, to its aesthetic resonance within our experience.

This hauntedness can be indeed embodied, not only in the form of authority, as I have shown in the case of its archontic presence, but also in the form of a style, and as we will see, within the constitution of gender.



% Another example Ernst provides of the anarchive is the Internet itself: ``a collection not just of unforeseen texts but of sound and images as well, an anarchive of sensory data for which no genuine archival culture has been developed so far in the occident'' \parencite[139]{Ern13:Dig}. The difference between the Internet and classical archives is that it is dynamic, that is, a ``fluid intermediary random access memory'' which is nonetheless subject to archivization.
