% In this interlude I take the concept of an absolute human memory to be at an intersection between disembodied theories of information and, precisely, the concept of an embodied memory. The aim is to introduce and differentiate between human and nonhuman in terms of memory and databases, so as to provide a link between listening \see{chapter:section-1} and performance \see{chapter:section-3} within the context of databasing and database music. I understand memory by bringing Jacques Derrida's concept of `trace,' `archives,' and `spectrality.' All three concepts take on a psychoanalytic perspective with Derrida's reading of Freud, which I bring to provide some insight into the effects of the database's relation to human memory. That is to say, the wonder, admiration, but also the fear and mystery that Funes awakens in the narrator, can speak for the uncanny feeling that occurs whenever databases are involved, and thus can speak for the agency of the database. I understand this feeling as what accounts for the aesthetic experience of database music more broadly. A central aim in this section is to prove that, on the one hand, databases (as disembodied memory) belong to the realm of what Derrida calls the `archontic,' which relates to \textit{archē} (the origin and the rule). On the other hand, human (embodied) memory are anarchic (withdrawn from the origin and from ruling). Therefore, by first considering the essence of the psyche (human --- memory) and then the externalization of the psyche in hypomnesis (nonhuman --- archives, databases), I arrive at a spectral quality in between these two, what I define as the spectrality of the database (the uncanny).

According to Jacques \textcite{Der78:Wri}, Freud understood memory as the essence of the psyche: ``Memory\dots is not a psychical property among others; it is the very essence of the psyche: resistance, and precisely, thereby, an opening to the effraction of the trace'' \parencite[201]{Der78:Wri}. `Effraction' is a legal term that refers to making a forcible entry into a house. Derrida uses it in relation to the process of impression (pressing \textit{in}) with which memories are unconsciously inscribed. The inscription of a trace relates thusly to a certain violence. This does not mean that memories are violent in themselves, but that in order for something to leave a mark (a trace), there has to be a force acting against a certain resistance. If traces can be thought of as `paths' or tracks along which memories are inscribed, then making these paths is a form of `breaching' or path breaking. Derrida points to a crucial issue with Freud's idea of opposing forces: if resisting forces met equally strong resistive forces, the result would be a `paralysis' of memory. Upon this possibility, he recognizes that ``trace as memory is not a pure breaching\dots it is rather the ungraspable and invisible difference between breaches [traces]'' \parencite[201]{Der78:Wri}, a difference that exists both in space and in time. This spatio-temporal play of difference is what Derrida calls \textit{différance}, and how he understands the ``work of memory'' itself \parencite[226]{Der78:Wri}. \textit{Différance} has many names throughout Derrida's text, making any attempt to address it an extensive process.\footnote{For further reference on \textit{différance}, see \parencites[71-72]{Gra15:The}[219]{Der78:Wri}{Der82:Mar}} In what follows, I will briefly delineate what I consider its most important qualities for the purpose of discussing the database in relation to memory.

\paragraph{Différance}
The play of difference with which Derrida describes the psyche can be understood in terms of how it applies to language. For example, in order to describe an object (`database'), one needs another word (e.g. `data,' `algorithm,' `structure'), which in turn demands yet another word (e.g. `bit', `computer,' `model'). Up to this point, we still don't know what the word means in the first place, or if one means \textit{this} or \textit{that} database, or if one will ever describe the object in question. To explain fully the point, for instance, the reference would need to \textit{become} the object. In fact, the reference will \textit{never} become the object, hence the moment of deferral that plays \textit{in time}. Likewise, the word `database' is not the same as, for example, `archive,' `library,' or `dictionary,' pointing at the differing references that play \textit{spatially}, across the multiple meanings to which each word points. In any case, the play of difference engages us with an adventure that makes us think about `database' without necessarily aiming at one answer, but might help us `traverse' its trace. Within this traversal we can find the two meanings of the the French word for difference: to defer and to differ. Derrida makes this distinction `evident' with the use of the `a' on the spelling of the word `différance,' which in French makes no audible change. In doing this, he points to the distinction between spoken and written words, while emphasizing the dual meaning of the play in question: deferring-differing. In relation to the psyche, the play is between the traces. On one hand, the `spacing' of memory traces relates to the difference among traces, how they relate to each other in terms of their identity, and also to the difference between consciousness and unconsciousness. On the other, difference relates to the deferment, delays, or the postponement of the inscription of traces. In this sense, he understands Freud's concept of `death drive,' that is, the gravitational pull that the inorganic exerts on the organic: ``death at the origin of life which can defend itself against death only through an \textit{economy} of death, through deferment\dots'' \parencite[202]{Der78:Wri}.

\paragraph{Funes}
I would like to return to one more aspect of Funes' story that relates to \textit{différance}. Consider how the narrator writes about his own unfinished project of learning Latin: ``The truth is that we all live by \textit{leaving behind}; no doubt we all profoundly know that we are immortal and that sooner or later every man will do all things and know everything'' \im \parencite[113]{Ker94:Fun}. A more literal translation of the first sentence would read: \textit{what is certain is that we live deferring all that can be deferred} [Lo cierto es que vivimos postergando todo lo postergable \parencite{Bor42:Fun}] The narrator touches upon two crucial points with which we can understand \textit{différance}. On one hand, the postponement that is assigned to `every man' but that is deprived from Funes who cannot afford to defer given his condition of total temporality. On the other hand, the absolute knowledge that the narrator assigns to the multiplicity of man and the multiplicity of things, that is nonetheless made present in a certain sameness that Funes represent (a cancelling out of that which differs). Thus, extending Oviedo's consideration of Funes as the `antithesis of the writer,' Funes can be further thought of as comprising an antithesis of \textit{différance}, a man whose psyche has no resisting unconscious. For Spivak, what Derrida's reading of Freud allows him is to understand how forgetfulness is ``active in the shaping of our `selves' in spite of `ourselves''' \parencite[xlv]{Der76:Of}. That is to say, because of the unknowable forces of the unconscious, we have no control on neither tracing nor the undoing of traces: ``we are surrendered to its inscription'' \parencite[xlv]{Der76:Of}. Furthermore, in contrast to a Nietzschean view of an active search for forgetfulness ``or the love of chance,'' what Derrida finds is that ``we are the play of chance and necessity [and] there is no harm in the will to knowledge \parencite[xliv]{Der76:Of}. Unlike Funes, we are already immersed in (and shaped by) a world of \textit{différance}. Therefore, we can now ask ourselves in what way does \textit{différance} change our understanding of the database? Databasing can perhaps be understood as one way of traversing \textit{différance}, not as an aim, but simply as something you are: in databasing, you \textit{are} databasing. Funes, in this sense, it not databasing, he simply is the database. To be databasing one must be in a loop, in resonance, differed-deferred in time and space \textit{with} a database.

\paragraph{Writing and Databasing}
As Derrida points out, western philosophy has construed writing as a process of \textit{hypomnesis} or an externalization of memory \parencite[221]{Der78:Wri}. Writing media can be considered hypomnesic or an externalization of our memory that on one hand materializes our memories and on the other constitutes a operable tool. Furthermore, Freud's metaphor of writing to define the apparatus of the psyche as tracing and erasure indicated that ```Writing\dots is the name of the structure always already inhabited by the trace,'' meaning that it is a broader concept that goes beyond writing itself, that is, beyond a ``system of notations on a material sub­stance'' \parencite[xxxix]{Der76:Of}. How does `writing' relate to the resonant network I described earlier? We can approach this question in two ways. On the one hand, we can find in Latour's `tracing' of the network a relation to the `trace' as described by Derrida, insofar as they both relate to meaning. Latour's project, however, leaves behind the (human) body so as to only maintain this `tracing' of the network, thus making it exceptional to think of memory in terms of networks. Nevertheless, there is an activity of tracing that takes place in the resisting forces of the unconscious. On the other hand, with his conceptualization of memory as writing, Derrida reconfigures the notion of authorship within writing. In this sense, when memory is thought of as writing, the classical notion of `self' begins to disappear, opening up the space for the nonhuman. In what sense can this disappearance of the self be accounted for in databasing? Along the trace of the resonant network (what I called infraskin) every resonant node of its trace can be considered an agent in the constitution of selves. The database becomes an agent of selfhood as well as an agent of authorship relating to the resulting work of database music. Further, as an instance of hypomnesis, the database appropriates the qualities relating to memory and \textit{différance}. Thus, not only can the nonhuman be reconceptualized within these qualities, also the human itself becomes reconfigured when faced upon this common linkage. This is how a further step into the conceptualization of memory can be of aid, one that extends `tracing' or `writing' into what I am calling `databasing.' `Writing' can thus function as a link between human and nonhuman memory. For example, from the beginning of the process of impression, traces are ``constituted by the double force of repetition and erasure'' \parencite[226]{Der78:Wri}. Memories are inscribed with the very structure that enables their own effacing, they ``produce the space of their inscription only by acceding to the period of their erasure'' \parencite[226]{Der78:Wri}. Programming languages imply writing in terms of symbols (words and characters) that call certain functions. However, the structures that can be instantiated with code, such as data structures, appropriate the concepts of writing and erasing as well. This is known as memory management. Erasure is embedded within the structure of coding. For instance, C++ classes include within their own data structure a call to their \texttt{destructor}. This means that, whether explicitly or implicitly, all classes ---i.e., all data structures which correspond to instantiated objects--- have a way to self-erase, or self-destruct after the object is no longer needed. This self-destruction means, precisely, releasing the object's resources, that is, to free the physical memory space that it has occupied throughout its `lifetime.' This comparison between \texttt{destructor}s and erasure serves as a starting point to determine the extent to which computer memory (data structure handling in restricted, discrete space) can be thought of as human memory itself, and, if so, the extent to which it also constitutes an instance of an unconscious structure. We can imagine a (useless?) program written in C++ with self-destructing classes that instantiate and destroy themselves at their own pace. Considering the \texttt{destructor} itself, it is simply an automation of the otherwise manual memory allocation or deallocation. In a practical sense, allocating memory is like calculating how much and what kind of paper you will need to write your next short story, ordering it online, and then throwing it in the recycling bin, I presume, only after you have written and digitized your story. In any case, we can think of this `paper' as the space that is needed for writing, as well as the resisting force against the pen, but also as a base upon which data is stored in relation to its size. Memory management is a feature with which databasing relates at the level of writing code.

 % In what follows, I address the extent to which databases can be understood in terms of Derrida's more broad concept of trace.

% , taking German media theorist Wolfgang Ernst's concept of the \textit{anarchoarchive} to a different dimension \see{archontic}. 

