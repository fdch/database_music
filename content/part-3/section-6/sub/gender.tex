\begin{quote}
	Gender is not passively scripted on the body, and neither is it determined by nature, language, the symbolic, or the overwhelming history of patriarchy. Gender is what is put on, invariably, under constraint, daily and incessantly, with anxiety and pleasure, but if this continuous act is mistaken for a natural or linguistic given, power is relinquished to expand the cultural field bodily through subversive performances of various kinds. \parencite[531]{But88:Per}
\end{quote}

Philosopher Judith \textcite{But88:Per} distinguishes between an expressive and performative self. The former comes from an essentialist view from the self as being `inside' and displaying itself on the outside. The latter is an illusory self, strictly outside and unrelated to the ``natural or linguistic given.'' She understands gender within this performativity of the self. Like the self, gender emerges temporally, at the surface level of the skin of the body. This notion of gender relates with Jean-Luc Nancy's notion of resonance and the self \see{chapter:section-1}.

\paragraph{Skin of the Database}
In the performativity of databasing resides the possibility for the what I have called infraskin of the database to emerge. The prefix `infra' that I have added to this skin, however, does not indicate interiority. It simply suggests a positioning that cancels any hierarchical order in the resonance among the human and the nonhuman. It is `infra' in relation to the dynamics of an interaction along the interweaving of its texture. On the one hand, this skin is this spectral texture of the database's illusory self. On the other, it is the limit upon which the human and the nonhuman engage in resonance. The skin of the database carries the mark of a style and the possibility of its gender. That is to say, in defining style as a repetition of acts, it is a form of embodiment that is ascribed to databases. This enactment can be understood as the enactment of a gender ``which constructs the social fiction of its own psychological interiority'' \parencite[528]{But88:Per}. Therefore, the database has as many genders as there are `social fictions,' in a permanent play of difference suspended on its limit. We can understand ``its own psychological interiority,'' in this sense, as referring precisely to the gendered self of a database, one that is established in its own historical sense: the history of its resonance, authority, and transformations. Thus, the gendered database participates aesthetically, dramatically, and with its own authority in the history of its practice. The infraskin is gendered at any point, in any time, in any way, and the uncanniness of its appearance redefines our own social categories, and our own reality. Databasing, as the performative condition of databases, is gendered, and it expresses nothing: it exposes us to the gendered resonances of our acts.

\paragraph{Expressing Nothing}
Butler sets forth a critical genealogy of gender which relies on a ``phenomenological set of presuppositions, most important among them the expanded conception of an `act' which is both socially shared and historically constituted, and which is performative\dots'' \parencite[530]{But88:Per}. The difficulty Butler recognizes in this view of gender, is that ``we need to think a world in which acts, gestures, the visual body, the clothed body, the various physical attributes usually associated with gender, \textit{express nothing}'' \parencite[530]{But88:Per}. Therefore, how can the database itself be conceived in these terms of performativity, to the point that, while having the capacity to store millions of data, a database can indeed express nothing? 

\paragraph{A Historical Situation}
Butler defined gender identity as a historical situation, distinguishing between physiological facticity of the body (sex) and the cultural significance of such facticity in terms of gender. Added to this distance between body and identity, Butler speaks of the body as a performative process of embodying cultural and historical possibilities. These possibilities, which are delimited by historical conventions, are thus materialized on the body: ``one does one's body differently from one's contemporaries and from one's embodied predecessors and successors'' \parencite[521]{But88:Per}. Therefore, the body comes to be a ``historical situation'' that results from the performativity of embodiment itself. In other words, the actions related to what Butler calls the ``structures of embodiment'' constitute an ontological sphere of present participles, such as `doing,' `dramatizing,' and `reproducing.' Furthermore, what this structure of embodiment entails is the constitution of not only gender, but also style. Since gender is constituted temporally, it is necessarily historical:

\begin{quote}
	to be a woman is to have \textit{become} a woman, to compel the body to conform to an historical idea of 'woman,' to induce the body to become a cultural sign, to materialize oneself in obedience to an historically delimited possibility, and to do this as a sustained and repeated corporeal project. \parencite[521]{But88:Per}
\end{quote}

\paragraph{Subversive Repetition}
Far from being a prescribed given, the constitution of gender on the body is itself a result of mediated history. In other words, gender is a creative act of interpretation and reinterpretation that reveals itself on the body, not as an expression that comes from within, but as the sedimented layers that deposit themselves in time. Furthermore, within this notion of temporality there is a need for repetition that is susceptible to breakage, or what Butler refers to as ``subversive repetition'' \parencite[520]{But88:Per}. In being a temporal identity which reveals itself through a ``stylized repetition of acts'' gender constitutes an ``illusion'' of a gendered self. These acts take place \textit{on} the body, by the mundane instantiation of bodily gestures, movements, and enactments. Furthermore, these acts are necessarily discontinuous, and it is because of this discontinuity that exists the possibility of gender transformation. In this sense, gender performance is neither linear or nonlinear. It resides along an anarchic temporality that replaces teleology with the multiplicity of resonant nows. It is an inline iterative function with random breaks. 

\paragraph{Gendered Database}
The database is a collection of facts. This is what Butler's gendered self can teach about databases: in performing the database, the database appears like gender, as a historical situation. Its body is felt neither as the database body ---as if the materiality of the computer's architecture could come as a proxy for the nonhuman body--- nor as the extension of the embodying databaser, that is, as a prosthesis that expands the databaser in an expressive way. The body of the database emerges as a phantom, as spectrality itself, and it is this nonhuman presence that engages in the publicness of performative acts. The specter of the database must not be understood spiritually, or as a \textit{deus ex machina}, or as a soul, or singularity that begins to act \textit{as} human and, by extension, supersedes the human. It is simply a nonhuman fabrication of selfhood: there, around, making its way through the rupture of the permanent condition of performativity to which we (humans and nonhumans) are phenomenologically bound. This is how the style of the database appears. This nonhuman self, like Butler's gendered self, is equally `outside;' ``constituted in social discourse'' \parencite[528]{But88:Per}. In other words, the skin of the database ---what I called infraskin \see{inoperativity}--- is open for perception outside of itself, and in fact, nothing about of the database can be considered expressive. Inside the database there is literally nothing but zeros and ones, nothing but data; in the same way, nothing is inside of the body but flesh, bones, and veins. When considered as internal, inherent, or essential, the classical notion of the self, in its heteronormativity, is seen as a ``publically regulated and sanctioned form of essence fabrication'' \parencite[528]{But88:Per}. In this state of being fabricated, expressivity serves as the foundation for what Butler refers to as the `punitive' aspects of wrong gender performance. In this sense, the social quality of acts that fall outside the regulated binary gender construction work their way into punishing the body, incarcerating it, severing it, as is, for example the famous case of Alan Turing:

\begin{quote}
	Turing's later embroilment with the police and court system over the question of his homosexuality played out, in a different key, the assumptions embodied in the Turing test. His conviction and the court-ordered hormone treatments for his homosexuality tragically demonstrated the importance of \textit{doing} over \textit{saying} in the coercive order of a homophobic society with the power to enforce its will upon the bodies of its citizens. \parencite[xii]{Hay99:How}
\end{quote}
