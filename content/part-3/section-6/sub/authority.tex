\begin{quote}
	Gender is instituted through the stylization of the body and, hence, must be understood as the mundane way in which bodily gestures, movements, and enactments of various kinds \textit{constitute the illusion of an abiding gendered self}. \im \parencite[519]{But88:Per}
\end{quote}

The figure of the author (composer/databaser) is, to a certain extent, expanded through the network by the complexity of the system: the composer's agency and compositional authority is distributed to the various agents of the network (database, interface, sounds, etc). However, authority is reified into the `name' of the author because of the interplay among work, productivity, and product. In this section I attempt an approach to the `name' of the composer not by its work, but from the illusory perspective of authority. However composable all Vaggonian primitives can be, the structure of the database tree is so vast that any attempt to comprehend it as a whole would extend it even further \see{network}. However, this determines neither the extent of the performativity of databasing, nor the agency of the human. Quite the contrary, expansion through the network can be considered as the trace of the author, or better, the elongation of the spectral shape of an author. Further, with the performativity of databasing, the databaser too becomes incomplete.

\paragraph{The Name}
The infinitude in the fractality of databasing, however, is at some point reified in a figure or a name. This figure is the place where authority is condensed, and it responds to traditionally essentialist conceptualizations of the romantic author which, despite the many attempts during 20th century,\footnote{See for example Roland Barthe's 1967 \textit{Death of the Author}, or Michel Foucault's 1969 text \textit{What is an author?}, both of them commented on in \parencite{Dan07:The}.} are still in effect today, specifically in the field of music composition. It is not the purpose of this section to criticize this tradition, namely because I don't consider it relevant for the purposes of databasing. Focusing on it would be missing the point. That is to say, in the case of databasing, such figure of an essential author is simply dislocated and forced upon the structure of the network, and it is anachronic because it constitutes a temporality set against the temporality of networks. Databasing, as resonant performativity already exists beyond this traditional figure of the author. However, in its spectrality that stems from the archontic \see{archontic}, authority can be seen as the illusory resonance of an author. It is this illusion that I attempt to address here; a ghost that haunts music composition.

% I bring here the name `Vaggione,' at least to the extent that he, as composer and author, but also as a spectral voice in his music and his writings, can exemplify the figure of authority in composition. It is not a coincidence because it is a name that I have created over the years, as I am sure there are as many Vaggiones as there are grains in his music. The Vaggione that I have created is one that haunts me personally, because we both come from the same place (Córdoba, Argentina), attended the same university (National University of Córdoba), facts that, for my own situation as young composer, resonated deeply in the music and research that I pursued over the years.\footnote{As I have described above \see{spectrality}, Derrida claims that addressing phantoms is a transaction that is familial and domestic; thus is how I feel when thinking of the name Vaggione.} Further, this discussion of authority of the name Vaggione does not intend neither to criticize, nor to deconstruct the name in itself, it is only an approach to to the spectrality of the author.

\paragraph{Dictionaries}
Consider how style is used in some cases of \gls{cac}. David Cope's \gls{emi} \parencite{DBLP:conf/icmc/Cope87}, for example, can be considered a formalization of compositional authority. That is to say, intentional stylization ``based on a large database of style descriptions, or rules, of different compositional strategies'' \parencite[3]{Mau99:Abr}. Written in the functional programming language \gls{lisp}, \gls{emi}'s focus is ``style imitation'' in order to assist the composer when in front of a ``composing block,'' provoking the ``author into almost immediate action. Any blank moments along the way are immediately filled by simple queries\dots'' \parencite[38]{Cop87:AnE}. Cope's approach is inherently hierarchical, and thus based on the premise that music is a language. Therefore, Cope designed dictionaries (databases) of \gls{midi} scores representing the internal relations between composed elements. From items in the dictionary, logically correct inferences are drawn (predicate calculus) \parencite[1]{DBLP:conf/icmc/Cope87}. Thus, \gls{emi} is aimed at generalizations that reify the authority of the composer as style: 

\begin{quote}
	Years of consistent interactive use have resulted in dictionaries which so complement the author's own style that compositions show little evidence of the origins (man/machine) of the music. \parencite[179]{DBLP:conf/icmc/Cope87}
\end{quote}





 % is not only that 
 % 	by performing this type of machine learning techniques 
 % 		databasing becomes at once detached from the human, 

 % it also 
 % 	replicates a notion of the idealized work of music. 

 % 		Since, even 
 % 			if 
 % 				the data was inputted by Bach himself, 
 % 			the output would not only be entirely deterministic, 
 % 				it would be the result of globally defined operativity. 
 % Hence, 
 % it would be impossible to think of 
 % 	this formalization of style 
 % 	as an 
 % 	instance of resonance, for example, 

 % 	since 
 % 		there is 
 % 			no moment of resonance at all, 
 % 			no exposure between the nonhuman and the human, 
 % 			no communication.


\paragraph{Artistry}
Vaggione, in response to a formalized approach to music ---among many that exist in the literature \parencites{Hil59:Exp}{Xen92:For}{Tru76:ACo}{Ari05:Ano}---, proposes the equal role of the informal craftsmanship of the composer using computers. In a very different case of the use of databases, consider Roads' account of Vaggione's workflow when composing the work \textit{SHALL}:

\begin{quote}
	These involved arranging microsounds\footnote{The word `microsound' refers to sonic events shaped below the threshold of the `note.' See \parencite{Roa04:Mic}} using a sound mixing program with a graphical time-line interface. He would load a \textit{catalog of pre-edited microsound} into the program's library then select items and paste them onto a track at specific points on the timeline running from left to right across the screen. By pasting a single particle multiple times, it became a sound entity of a higher temporal order. Each paste operation was like a stroke of a brush in a painting, adding a touch more color over the blank space of the canvas. In this case, \textit{the collection of microsounds in the library can be thought of as a palette}. Since the program allowed the user to zoom in or out in time, the composer could paste and edit on different time scales. The program offered multiple simultaneous tracks on which to paste, permitting a rich interplay of microevents. \im \parencite[313-314]{Roa04:Mic}
\end{quote}

While this workflow is only representative of certain aspect of the piece in question, it does serve as an example of his concept of craftsmanship. Craftsmanship refers to the manual and direct action of the hand of the composer. The hand, as Makis Solomos very well points out, is not to be understood as being without the tool (mouse) that it needs to use in order to precisely locate sounds on the timeline interface \parencite[4]{Sol05:AnI}. Craftsmanship might be better understood, however, as `artistry,' thus keeping its relation to hand-made crafts, while maintaining a link with articulation, one of Vaggione's crucial concepts. While articulation relates to the composer's operativity on multiple time scales, artistry relates to the arbitrariness of choice. It is thus a reaction to the abundance of radical formalism and automation in \gls{cac} \parencite[3]{Sol05:AnI}. Therefore, Vaggione writes, ``to write music `manually', note by note, partial by partial, or grain by grain, is an approach proper to a composer, and he should not be embarrassed about using this aspect of his craftsmanship''' \parencite[3]{Sol05:AnI}. Vaggione built his terminology not in opposition, but in the spirit of reconfiguring \gls{cac} from an embodied stance coming from outside information theory. This stance is not only evident in Vaggione's writings and music. To a debatable extent, this stance is a point of departure to think of a branch of Argentinian electroacoustic identity that developed in France.\footnote{For example, in the work of Beatriz Ferreyra, Elsa Justel, Mario Mary, to name a few. For an approach to Justel's timeline-based spatialization techniques, see \parencite{fdch/papers/elsa}.} 


\paragraph{The Work of Mice} 
For Vaggione, instead of relying on the rule-based programming of formalization processes alone (keyboard-based input), the artistry of the composer resides in the use of the mouse. The timeline of the sequence interface, and its workflow depends on the mouse pointer. The presence of the composer's hand is evidenced by the trajectory or course of the pointer. The mouse, along with the history of clicks and drag-n-drop motions suggests the spectral presence of the author. The mouse pointer, the `stilus', like the writing device, becomes that with which we resonate as listeners. Thus, we perceive the marks of an authorial skin in database music. The Vaggionian singularity-based approach to authority embeds composers and computers in a complex system or network, that renders the world of music with computers as a hybrid between human and nonhuman. This is how the specter of the author coexists with the specter of the database, and thus, how databasing and composition reveal themselves to be instances of a performativity that resonate aesthetically through the work of music. 
