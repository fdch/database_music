% In this section, I draw from performance and gender studies to analyze database practice as a performative activity. I use Judith Butler's concept of gender \parencite{But88:Per} to analyze the extent to which authority in database practices can be understood in terms of style.  The databaser, which is the human subject in this case, begins resonanting with the database, the nonhuman subject, in a form of feedback loop. This resonant loop is only possible through the performance of the database. I thus locate the origin of the database as listening subject the moment its performance begins. In this moment of performance, both human and nonhuman listening subjects are resounded upon their limit. I consider this limit to be a surface in between the human and the nonhuman, and I like to think of it as the `skin' of the database. I argue that this skin of the database is the possibility condition for illusions of style ---as in the style that is visible in one's clothes, or in one's decoration of the skin--- and authority ---as in the appearance of a subject who has some sort of power: the subject of the listening database. I thus analyze these illusions as belonging to the sphere of Agency that the Database presents, and assess the extent to which they affect the aesthetics of Database practice.
%
% EH:
% designial What do you mean by this?
% 
% My argument is that since the aesthetics of the database presents itself through performance --designial and navigational--, I claim that the resonance of each agent in this system can be delineated as a feedback loop traversing all the nodes in the network, in any direction and in any instance. For example, a programming decision can change the sonic outcome of a music work with the same ease as the design of a photo sensor can influence the spatialization technique used in a multimedia work.