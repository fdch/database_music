\begin{quote}
	\dots style, supplementing timbre, tends to repeat the event of pure presence, the singularity of the source present in what it produces, supposing again that the unity of a timbre ---immediately it is identifiable--- ever has the purity of an event\dots The timbre of my voice, the style of my writing are that which for (a) me never will have been present. I neither hear nor recognize the timbre of my voice. If my style marks itself, it is only on a surface which remains invisible and illegible for me. \parencite[296]{Der82:Mar}

\end{quote}

A database without performance represents a disembodied `base', that is, the spatially ordered set of computer hardware together with the software routines that it embeds. It is its most basic level, a foundation upon which the database tree can be performed. This `base' in database comes as a stage for databasing itself: a stage without performance is an empty stage, extension of space. Databasing projects its own style as a result of its performance, and through this projection comes the exposure of its skin. The ``stylized repetition of acts''  \parencite[519]{But88:Per} in the dramatic case of the gendered database is now revealed as style. Like skin and voice, singularity emerges as style and timbre.

\paragraph{Style and Timbre}
`Style' comes from the latin \textit{stilus}, meaning a sharp object with which you can write: like the stylus of a record player, it is a writing tool. Its meaning extends through writing to the manner in which writing is carried out: the variations and oscillations of the pen and of the text. Hence, these variations result in the style of a certain text or, for that matter, in certain programming or music styles. Both of these stylistic idiosyncracies of the written text extend to the style of an author, programmer, or composer; and, to the style with which we can identify certain literary, programming, or musical peridos. Furthermore, style can also be seen in the way in which the human body moves and looks. Thus, style is a manifestation of the singular. In the sense that style does not lend itself to duplication, and provided that it happens as the apparition of an event, it exposes singularity as such. Style is thus comparable to the voice of a certain author, and also to the sound of the voice: timbre. That is to say, style and timbre can be understood equally as the presence of the singular: the signature of a unique and irreproducible quality:\footnote{In signal processing terms, the sound of a voice might be approached with timbre stamps or vocoders, a type of Fourier-based filter in which ``the spectrum of one sound is used to derive a filter for another'' \parencite{DBLP:conf/icmc/Puckette07}.} 

\begin{quote}
	In its irreplaceable quality, the timbre of the voice marks the event of language. By virtue of this fact, timbre has greater import than the \textit{forms} of signs and the \textit{content} of meaning. In any event, timbre cannot be summarized by form and content, since at the very least they share the capacity to be repeated, to be imitated in their identity as objects, that is, in their ideality. \parencite[296]{Der82:Mar}
\end{quote}

\paragraph{Endless Databases}
The skin of the database unfolds in the duration of the performative act. What is exposed as its singularity is the ruggedness of the traces of which it is composed. That is to say, the discontinuities of its reticulated constitution of style. The length of this skin can only be estimated: there is no possibility of rendering it complete. In this fractured state it points to infinity. In this sense, databasing means participating in the infinite, taking a small part of the infinite: performing the infinite within the limits of our own embodiment. Furthermore, the contingent situation of resonance within the frayed spatio-temporal configuration of networks relates to the concept of chaos. I have mentioned earlier the relation between computers and users as understood in terms of complex systems \see{programming}. Considering databasing as chaotic systems brings yet another aspect to the contingency of style. 

\paragraph{Database and Chaos}
Given that this style can be considered as an emergent singularity of databasing, we can hink of this singularity as deterministic. In mathematics, determinism refers to the capacity to predict results, specifically by solving differential equations. This is the case of dynamic systems studied within Chaos Theory. For example, the Lorenz attractor is a system of differential equations discovered by Edward N. Lorenz in 1963, following experiments on weather conditions prediction. The attractor is most famously recognized by the butterfly-like appearance of its visualization and, most interestingly, for certain fractal properties that are evienced in its graph \fsee{lorenz_plotter}.\footnote{This butterfly shape also relates to what is known as the `butterfly effect,' a concept of chaos brought forth by Lorenz in his 1972 lecture named: ``Predictability: Does The Flap Of A Butterfly's Wings Brazil Set Off A Tornado In Texas?'' \parencite[181]{Lor93:Ess}} The Lorenz attractor is a dynamic system, which means that it can render very different and quite unpredictable results by minimal changes on their initial conditions, despite the fact that it is indeed a deterministic system. Considering databasing as a dynamic system two performances can be exactly the same if given the same initial conditions and states. In this case, databasing would be closer to the performance of digital fixed media works, in which at least at the sample level every `bit' of it is exactly the same as the original. 

\img{lorenz_plotter}{0.7}{
	Plotting of the Lorenz system in Pure Data.
}{Lorenz Attractor}

\paragraph{Fractality}
However, identifying predictability in this way means falling in a cybernetic trap, of which Hayles already warned about when considering Turing's Test. Hayles reads Turing's test as a game which, in order to play you are already part of its outcome because you accept its predicates as a condition for playing. In Turing's case, the moment you enter into the disembodied place where the screen is the only thing you see, you are already a cyborg, and the definition of the human and the nonhuman is already laid out in principle. On the one hand, by equating fidelity of data storage with fidelity of performance, one is already removing the human out of the concert stage, and the question of performance altogether, leaving only the idealist and romantic notion of the work of art in its pure and objective state. On the other, in order to allow for the style of databasing (skin) to emerge, one has to consider not only the actual staging of performance, also the staging of listening, allowing the resonant subject of the database to emerge as the communicative apparition of a skin. Therefore, the contingency of style (as chaotic state) can only emerge out of the unpredictable agency of the unfolding. This is how I consider databasing and the contingency of style: the unpredictability of databasing has the qualities of a fractal. Because of the fractal dimension, it expands the definition of geometric figures to the infinite. In this sense, it presents an unfolding symmetry (self-similarity), which relates to their shapes being replicated nearly exactly in different scales. 

\paragraph{A Music Work as a Singularity}
The nature of the aesthetic experience of database music slips through the cracks of traditional conceptualizations of the work of music as a result of stylistic or stipulated constraints on the part of the composer or of stochastic procedures applied to musical structures. For example, composer Horacio Vaggione goes to great lengths to prove that the musical work affirms itself as a singularity, in the particular sense that its rules are only prescribed from within, and always in an ``action-perception loop'' with the composer \parencite{Vag01:Som}. What Vaggione is arguing against is the tendency of formalized musical processes that epitomize the black-box approach towards composition in \gls{cac}: ``a composer [unlike a scientist] knows how to generate singular events, and how to articulate them in bigger and bigger chunks without losing the control of the singularities'' \parencite[97]{Vag93:Det}. However, there is a fundamental concern that needs to be addressed in relation to the contingency of style. For Vaggione, style comes to represent the reified status of the rules within a work, insofar as this reification is taken as the starting point from which to compose, and not the result of a composed thing. Consider this quote from an earlier text:

\begin{quote}
	Here lies what seems to be one of the sources of confusion regarding the nature of music composition processes: on the one hand, we must make as careful a distinction as possible between the collective rules and the composer's own constraints; on the other, this distinction seems irrelevant [because] any primitive (coming from a common practice or postulated ad hoc) is to be considered as a part of what is to be composed, \textit{in order to produce a musical work affirming itself as a singularity, beyond an exercise in style}. \im \parencite[59]{Vag01:Som}
	% Adorno was of course conscious of this dialectic: his statement about sound material considered not as something ``given'' but as a ``result'' of a musical thesis clearly points to this fact. \parencite[59]{Vag01:Som}
\end{quote}


\paragraph{Arbitrariness}
Distinguishing between rules and constraints, that is, between socially and historically established canons as stylistic conventions, and locally established postulates to be carried out by the composer as constraints, is crucial, simultaneously, in defining style in composition and databasing. However, conventions and constraints collapse into the realm of compositional arbitrariness for, if any ``primitive'' of the composition is to be considered ``part of what is to be composed,'' then style itself becomes a result. This is what Vaggione means by ``beyond an \textit{exercise} in style:'' it is not an exercise in the sense of a draft, in the military context of training (practice for the sake of training). Style is not an exercise because it cannot be operative in the sense that it is considered a product of work. Style is contingent, emerging from the performative action of databasing. Style would only result in closure if considered teleologically: a closed object, stipulated from the start as a law to which every composable element abides. Vaggione calls these overarching laws `global laws,' and he compares them with a marching army following a `one-two' directive, where ``no singularities\dots are\dots allowed'' \parencite[101]{Vag93:Det}.


% I know this paragraph is pushing poetic qualities, but you will leave an important section in an unclear situation if you don't manage to explain what you mean so you can build on things
% This whole paragraph is beautifully poetic, but hard to understand. You have created and inherited so many terms throughout your dissertation that it is hard to read a sentence. 

\paragraph{Inoperative Style}
As mentioned earlier, inoperativity can be understood as a feature of the activity of work that allows the music work to distinguish from notions of production, product, and completion. An inoperative style can be understood, therefore, as a contingency that appears in the form of exposure, not as a closed object, but as an unclosed object; some \textit{thing} that is exposed and bound to exposure; a thing that exposes us in the same resonance of its touch. Another word I have given to this `thing' in exposition is `infraskin,' which is where this inoperative style would be imprinted. Inoperative style does not mean it is a passive style. As I described before, activity is what defines style. Therefore, to speak of an inoperative style means to place ontology at the limit: whatever style databasers perform becomes the style that defines them but, this definition is never achieved, it simply leaves a trace. Like the marks on our skin, like its wounds; like the cracks of an old house, like debris, wreckages, or any form of residual mark that is the evidence of an event; with forensic intimacy, the contingent style of a musical unwork reveals itself as communication. This skin is what connects aesthetic experience of style with forensic or after-the-fact musical analysis as well as with an encounter with the spectral. Furthermore, this is how the spectral cannot be but a result of the inoperative, of that which escapes the limits of the work. Like the timbre of Lucier's voice that, releasing from itself into the room, then returns back as the resonance of a self. This `voice' of the unwork is what is `invisible' to the work. Invisible, because neither the inner voice in one's head, nor the actual timbre of the voice as one hears it are accessible to us. We can only hear this voice transduced, and from the perspective of others. It is what we can never listen and yet, in being hidden or silenced from us is how it becomes available for listening, what begins listening at the first staging of the waves: the strength of the first `I' in Lucier's work with no first breath. Severing the voice from the impulse of the body requires an insurmountable amount of activity, even if it means cutting a magnetic tape, or applying an offset when reading a sample buffer. An inoperative style depends on this excess of activity.
