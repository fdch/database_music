\paragraph{Communities of Skin}
The limit of the database, as performative, spectral skin, allows for a community to emerge between the human and the nonhuman. This means that the agency locus of the database needs to be placed precisely on its skin, because it is what becomes public of itself. In other words, given that this skin is available to the perception of others, it becomes touchable, it reaches our own limit as databasers. By exposing our own limits to ourselves and to each other, the database changes our definition and delimits the extent of our own singularity. However, this does not mean that this skin stands in the way of our performativity, or worse, that it precludes or determines ourselves. If this were true, we would be once again subject to technical determinism, essence fabrication, etc., and falling out of considering any possibility of community between anything that is not human. As I have already commented, this skin is human and nonhuman. Thus, the fact that the skin of the database changes our own skin simply means that we are already in communication with it, that is, in community, and also in a state of resonance with it. This is the function of the skin of the database: like the skin of a drum, or the skin of a loudspeaker, the skin of the database resonates with our own skin, engaging the resonant body with a resonant spectrality. In sum, what this infraskin allows is for a community of resonance, which has no purpose, no intentionality, and no essence; only appearance and motility, performance and repetition.

\paragraph{Hybrid Pluralities}
Database models tend to reside next to each other, either within a single database system or within an interconnected networked system. Databasers have access to the many features that each model offers, focusing on those features that are suitable for their needs. The skin of the database is as fluid as the constitution of gender, and if this is true, then the fluidity of databasing itself comes to represent the constitution of gender through the performativity of databasers. By resonating in such performativity, databasers approach the limit of the database. This approach to the skin of the database mutually exposes database and databaser. What this exposure amounts to is not, however, an opposition of forces. It results in the fragmented state of community that resides in the different spaces opened by this exposure. In other words, this exposure is of a hybrid plurality that resonates at the limit. Engaging with the touch of the spectral database means reconfiguring, resounding, and remembering our own sense of touch, just as well as our own sense of self.

