I would like to refer once again to Jean-Luc Nancy's concept of inoperativity \see{inoperativity}, this time in relation to what I call the severed music object. This object is different from Pierre Schaeffer's music or sound object, which comes to represent material with which to work. Neither is it related to Vaggione's concept of an object, which comes from object-oriented programming, meaning every composable primitive, from the micro to the macro. In both these authors, the object is used to provide, though not without their author's intervention, a notion of \textit{coherence} to the work. 

\paragraph{Remnants of Listening}
The object I am referring to resides in memory, as what remains after the event of an exposure. It is inherently linked to the fractured way in which our own memory works, and it is impossible to define, since it has no beginning and no end. Its dimensionality includes both beginning and ending simultaneously. This object is the spectral evidence of a musical event or, better, of the happening that takes place in listening. In being evidence, it becomes a topic for analysis; it is forensic. In being fractured, this object is the evidence of a destruction. In being severed, and this is the central aspect that I would like to focus on, risking simultaneously the severing of the object, it becomes the evidence of a sacrifice. If it can be said that the music object is a severed object, then the question of its severing necessarily relates to the question of listening. Therefore, by listening ---and, by this, I mean entering in resonance with resonance, exposing the self to that which returns to itself--- I participate in this severing, because I choose what to listen in spite of being already deprived from that choice.

\paragraph{Sources and Sorcerers}
% The sounds onstage exist always before and after the staging. 
The severed object of music is what we as listeners grab from the stage, what we choose to rip from the sounding waves, and also what we cannot help but feeling so much a part of us before noticing it is happening. Severing is yet another way of thinking the aesthetic experience of listening, but it is not as passive as it seems. Severing empowers the listener; it is the tool of listening; the reversed stilus; the inverted mouse; the part of the human that necessarily is nonhuman. With it, we can make the world appear, but only as a fraction, because `it' can never be \textit{completely}. As Brian Kane writes of Nancy's work of art: ``a work that refuses to create itself as a total work'' \parencite[29]{Gra15:The}. The severed object of music is always severed, but never in the same way, since there are as many severings as there are listeners, and as many listenings as there are moments. Composers have been traditionally considered a `source' of this object or, better, the one at the door, the key keeper that has access to the door that opens up the flow of inspiration. The composer, but also the programmer with access to the source code, which unless it is opened, is hidden to the rest; and, unless you know the language, it is complete pseudo-linguistic nonsense with weird punctuation marks, sometimes closer to poetry than it is to extreme formalism. For example, consider the following piece of code that can be read as a simple poem, but when run from a terminal would repeat in a computerized voice a famous line from Gertrude Stein's poem ``Sacred Emily:''\footnote{To run the code (linux or macos shell) simply copy the text onto a plain text file you can name \texttt{gert.sh} and run it with \texttt{sh gert.sh}. Note: it will never stop on its own.}

\begin{flushleft}
\small
\begin{lstlisting}[caption={Little words that do things.},captionpos=b,language=bash,mathescape=false]
#!/bin/bash

# Palabritas que hacen cosas

while true
do
	for ever in rose is a
	do 
		say $ever
		sleep $((RANDOM/10000))
	done
done

\end{lstlisting}
\end{flushleft}

% I thought I understood severing. Is it not a erpceptual process where the listener retains certain aspects of what was listened and filters others out, thus being left with an incomplete memory of an event? so why are the composers the first to sever?

In this access to the source, the programmer and the composer are traditionally kept at a distance, as if their listening were of some other sort, engaging with the very essence of the source, drinking the water from the originary fountain, satisfying an originary thirst. Therefore, if this is the role of the composer and the programmer, if this is their relation to the source, then, they are the first to perform the severing. In the hierarchy of the consequent severings, they are at the top. Further, if they are the first severers, they are the first who perform the first listening. They are the listeners at the top of the mountain, next to the source of all fountains. On the way in and out of the world, the sorcerers of condensation.

\paragraph{Naming}
I would like to point out now that it is not my intention here to sever the head of the sorcerer because it is an illusion that does not allow me to do so. It is not my illusion, although I have described how I interpret it, and it comes as a product of a reification of the composer, but also of the human as the one and only owner of the world. In being in resonance, listeners become the resonant world, that is, the self begins to resonate as space. In this sense, it is the world what is listened to, and it is a world that has no apparent origin. However, the composition ---the written score, like the written code--- propose their own origin ---the composer, the programmer. Thus, they give an origin to the world by providing an answer (a name) to the question of creation: Who created this music? \textit{this} composer. The answer, therefore, has a `this' that comes in the form of the name of the composer. This name becomes attached to the flowing of the source. Therefore, the name of the composer is like a timbre stamp that is applied to the listening experience. Furthermore, the severing style now can be named. How many different names or anagrams would it take for Click Nilson's style to dilute or to arrive at the unclaimable work of art?: ``This is why for some years I have experimented with releasing music under other people's names, so as to dilute their style, and under multiple versions of my own name, to cast doubt on any claim to the future'' \parencite{Col15:Col}. The name of the composer becomes a synecdoche of the source, directly naming part of the source. This applies, quite literally in some cases, to the name of the program and the name of the programmer (the `max' in Max Mathews and the `smith' in `msp').\footnote{`Max' is named after the `father' of computer music Max Mathews, and \gls{max/msp} contains Miller Smith Puckettes's initials. Friendly gestures, most probably, but also pointers to originary sources, sources of inspiration, historical references that contextualize computer music software within broader social and environmental structures.}

\paragraph{Dynamics}
Furthermore, the activity of the sorcerer lends itself to its signature. In other words, the manner in which the composer defines the music from beginning to end becomes the shape of the music. We can understand `shape' or `form' as something that is at once behind and in front of the singularity of the listened music. It is behind because it is the activity of sound sources ---speakers, musical instruments, or media in general--- the movement of air pressure. It is in front because it filters the memory of the activity of sound sources. However, a composed shape and a singularity act together in the moment of listening. The question is, then, regarding the dynamics of this activity. Given that this activity happens during listening, what I address now is precisely how the shape of the music interacts with the listening experience. Interaction, here, refers to the shared activity that occurs `inside' listening, and it happens `inside' because of the severing that needed to occur prior ---or immediately at--- the resonant oscillation of air pressure. However, once this severing has occurred, and within its momentum, it is the internal dynamics that enter into play, and it is the shape of the music what begins to delineate the shape of the listened.

\paragraph{Masterwork}
The shape of the music is a force that produces a certain listening experience. Therefore, the internal dynamics are already prescribed. The singularity of the listened becomes (almost) one and the same with the shape of the music. `Almost,' because it is not that the listened brings no resistance to this `ideal' force. The singularity of the listened is resistance, it acts as resistance, but its force is not enough to resist the command of an excellent work. This is the very presence of the masterwork: the work of a master that requires a slave. This `slave' is not the other music works that have not reached the necessary status of a master. Slavery exists among the outshunned singularities that have been muted by the very presence of a master. `Almost,' in the hope that the work of this masterwork can be relativized and disarticulated; disentangled from the source of sources; brought down the stream to the place where singularities can resonate in endless forms of matter. The problem is now of a different sort. Even if resisting forces match those of the masterwork, then, like Derrida's paralysis of memory, we can encounter a paralysis of listening as such. This paralysis might (also) be what Szendy means by the inattentive listener that falls away. But, it is not a paralysis caused by distraction, it is a paralysis caused by the very effort that is needed to match the force of the masterwork. Thus, it is a paralysis that is directly called for from `outside,' preventing any further listening. This is what is called for by the work of the masterwork: pure ---and utterly ideal--- silence.

\paragraph{Architecture of Obedience}
Therefore, within these dynamics of work, what results is a function of the predicates, it is an architecture of obedience that is written in the form of a music work, with the one and only aim which is for it to `work.' Thus, the composer engaging with this dynamics of the work, becomes the architect of the listened, the creator of a listening of which he himself is the only chief. The sorcerer in charge of quenching a thirst that is only there because it  instantiates with its creation. The question now is how can this dynamics be approached once we have recognized it. How can music composition continue? A composition not participating in this dynamics? A composition that is not a force? A composition that is not `really' or `entirely' a composition? A composition that does not impose its shape? A music work that is not a work but that still resonates within listening?


% 
% The listening experience that 

% I argue that, given that the inoperativity of the listening experience can be thought of as the interaction between resonance ---as the \textit{différance} within sense and sensuality--- and the unworking of the network, its resulting object, instead of being a complete whole ---a finished, integral `thing', or even, a `piece'\footnote{Since, the notion of a `piece' presupposes that of the whole to which it belongs.}--- it becomes a severed music object. 