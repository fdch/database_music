Inoperativity characterizes the aesthetic dimension in the severed music object of the composition that does not impose its own listening. In this sense, the practice of music composition can be understood in terms of Nancy's positive, active force of unworking. The condition of unworking in relation to works of art is exposed by a certain resistance present in the unwork of art. This resistance is a force of interruption and suspension that prevents the notion of a whole to reach completion.

\paragraph{Place in Common}
An unwork differs radically from the notion of an open work as is the case, for example, of Umberto Eco's famous formulation that ``the work of art is a complete and closed form in its uniqueness as a balanced organic whole, while at the same time constituting an open product on account of its susceptibility to countless different interpretations\dots'' \parencite{Eco04:The}. Instead of openness being located in the interpretation, the openness is inherent to the hybridity of its construction. The construction, in turn, is a result of the reticulated and fragmented state of exposure between the human and the nonhuman. %In this sense, the limit of the unwork is the exposure of exposure itself, that is, an instantiation of the place in common.

\paragraph{Disintegrated Imperative}
I would like to analyze the inoperativity of the music in relation to the dynamics of the shape of the unwork and the singularity of the listened. The former, in being a disintegrated imperative ---i.e., without the integrity that is required of the imperative for it to work as command and instruction---, cannot behave as a force in its own right. This is not to mean that it `fails' as a force. At this point it would be useful to revise Kim Cascone's consideration of the aesthetics of failure \parencite{Cas00:The}. In his analysis of the `post-digital' culture of the late 1990s, Cascone identified electronic music outside academia as one related to the unintended uses of computer music software, also known as glitch art:

\begin{quote}
	It is from the `failure' of digital technology that this new work has emerged: glitches, bugs, application errors, system crashes, clipping, aliasing, distortion, quantization noise, and even the noise floor of computer sound cards are the raw materials composers seek to incorporate into their music. \parencite[13]{Cas00:The}
\end{quote}

\paragraph{Blind Experimentation}
Within what he called the ``cultural feedback loop in the circuit of the Internet'' ---where artists engage with download and upload of software tools and artworks--- Cascone describes a `modular' approach regarding music creation as being grounded in the use of (recorded) samples and later mixing \parencite[17]{Cas00:The}. His argument is that ``electronica DJs typically view individual tracks as \textit{pieces} that can be layered and mixed freely'' \im \parencite[17]{Cas00:The}. In atomizing this use of samples, glitch art descended to the micro-level, but precisely by this descent, it sacrificed the whole for the parts, that is, it became a case of extreme modularity that ``affected the listening habits of electronica aficionados'' \parencite[17]{Cas00:The}. Cascone's conclusion is to call for new tools ``built with an educational bent in mind'' \parencite[17]{Cas00:The}, bridging the gap between academic and non-academic electronic music and, therefore, illuminating glitch music ``past its initial stage of blind experimentation'' \parencite[17]{Cas00:The}. 

\paragraph{Doctoring the Glitch}
It must be noted that Cascone's inclination towards bringing academic knowledge to the academy of the Internet refers not only to computer music software. Professors, generally of computer music, in several universities across the USA have been openly uploading class materials, patches, softwares, and many other highly useful technical information; not to mention the free and open online publishing of conference proceedings that have spawned in the last 20 years. Cascone's rendering of this educational turn can be understood with an authoritative and dated tilt on his end. Particularly, consider what he writes in relation to the form of glitch music, which is the last arguing moment before his claim for education: `\dots contemporary computer music has become fragmented, it is composed of stratified layers that intermingle and defer meaning until the listener takes an active role in the production of meaning'' \parencite[17]{Cas00:The}. How are we to interpret Cascone's call for education? What is the center of this education: music technology, composition, or listening? If fragmentation, modularity, stratification, and deferred meaning are affecting listening habits, are these habits themselves what needs to be taught? Or is it the structure of the music that is under distress? I understand the ambiguities in his argument as coming out of the main premise of the text, that of extending the concept of failure from technology to the analysis of the artwork. Thus, in Cascone's view, the aesthetics of failure of the late 1990s are still failing to enter academia because they fail to achieve the same standards of formal cohesion that are required by the modern conception of the music work. Therefore, instead of finding an academic cure for blind experimentalism, I claim that failure is itself an unnecessary blindfold. Failure is only possible within the structure of teleology. That is to say, both failure and success are measured against the expected outcome of a project. In success, the aim is accomplished. In failure, there is nothing achieved since the task at hand is neither neglected nor omitted. Therefore, if our project is a music work and we consider it outside the arrow of teleology, then failure has no place in it.  

\paragraph{Spectral Traces}
The unwork cannot behave like a force, but it can be considered the spectral traces of a force. In this sense, if there is an illusion of a force, it must appear as wreckage; an after dream; a mirror that shows us a skin of the past; the ruins of an empire; or the humidity creeping through the cracks of an old house. However, and this is a big however, these allusions to vessels, to the psyche, to architecture, and to the presence of the past altogether, must be addressed with the same strength as one would address a phantom. The unwork makes us feel the uncanny presence of the past in the now, of the overpowering ghost that brings with it the archontic, in the shape of our own selves that has been revealed to us as `not us,' but as yet again us. This is the moment that the unwork carries with it the most crucial aspect of all: it has nothing to give. It expresses nothing. And this is when listening finds us without anything to hold on to but our resonance. Our very own listening to ourselves listening. The moment where we realize it is our own self that is returning to us. This is our resistance.

\paragraph{Macroforma}
The resonance of a return. This is why the unwork depends so extremely on its very state of fragility: even the softest sounds have this self-referencial power. The moment this fragility is forgotten is when composers, performers, improvisors, programmers ---humans and nonhuman listeners, in the broadest sense possible--- enable an operative \texttt{macro} that has a political agency in the shaping of singularities. In order to provide some insight into the difficulties that arise from this conceptualization of the unwork, I would like to bring Vaggione again. When he writes of the shaping of singularities, he refers to the arbitrariness of the composer. However, he intentionally maintains formal coherence by extending the singularity of a grain (conceptually) to the singularity of a work. Therefore, in expanding this singularity he is ultimately arriving at a very unique and delimited shape that is the work. The contradiction I see here is that, in an attempt to propose a bottom-up approach in which, like Lewis' work, local actions percolate up to global behavior, Vaggione grants his work with an inevitable global behavior that is extremely operative: Vaggione himself. Without a doubt Vaggione (self) \textit{is} singular, and the value of his music is not put into question. I bring this as an example of the name of the composer and its impression on the music. In this case, the singularity of the composer impresses its own shape, style, and trace on the music. The problem is that the work now engages with its own operativity, with its integrity, and begins to dictate the shape of its own listening.

\paragraph{Overfitting}
Anarchy is a paradoxically productive force. As I have outlined before, databasing and composition bring forth their relation to the archive, and by doing so, they are bound to the origin and the rule. Like the name of the composer which is written on the shape of the music, the database has too the potential of becoming a source. Databasing becomes an activity of this source, and thus embeds the databaser with a specter of authority. Claiming, therefore, that composition can be identified with, or understood as, databasing means translating the `archic' not only to the performativity of composition, but also to the product of composing, to the composer and the composed; to the shape of the music and to the singularity of the listened. An unwork, therefore, would be necessarily an an-archic work. It is still a work, however, in the sense that it demands from the composer, from the databaser, and from every node in the scope of its network, an incessant operativity. That is to say, the `un' of the unwork does not come from inactivity, from passivity, from an escape of any form of action. Quite the contrary, it is a result of the constant impression of the work, the accumulated efforts towards the `un' of the thing; an extreme operativity that goes beyond the threshold of its own making so that it reaches a point of inflexion, a bent, an overflow. There is a point in statistics where learning algorithms, given a dataset, tend to adapt themselves too closely to the dataset, thus failing to render future predictions reliably. This is known as overfitting. Despite its uselessness (or, better, because of it) I believe this to be a suitable metaphor for the thinking about the unwork: precisely by overworking the work, one can find some insight into the `un,' and thus, one can begin to approach the anarchic in music composition. However, this approach comes not without its warnings, since it means at once, to eradicate the archic with the `an', which means to introduce a bug in the Oedipal loop that could result in unheard-of musical behaviors.
