
\paragraph{The Work Problem}
What does the problem of the music work consist of? and, why is it a problem? As I have already described in relation to community \see{inoperativity}, work can be thought of in two ways: work as teleology or work as ontology. In both cases, the word `work' uses its two meanings: the first meaning is that of the activity of working (labor, effort), which points to a series of meanings that I will explain below. The second meaning is that of the finished activity, what is traditionally referred to as a music `work' (product, composition). When a music work is understood teleologically, work acquires an aim, and it is measured in terms of this aim. Therefore, one can say `this music works,' `it sustains itself,' `it is very well structured;' or `this just doesn't work,' `it fell apart.' These expressions generally refer to what the music `proposes' and what it ultimaterly produces, and how these two (proposition, product) relate. We can understand this as what Peter \textcite{Sze08:Lis} calls the ``modernist regime of listening,'' in which the music work shapes its listeners into an ideal listener: ``the work listens to itself'' \parencite[127]{Sze08:Lis}. What this ``listening without listener'' refers to is a certain gap between the listener and the listened: between the subject and the object. For Szendy, this modernist regime is based on a more fundamental aspect: the absorption of the listener by the work. In this dynamics of absorption, ``distracted'' listeners ``fall away like a dead limb'' and ``bring nothing to the great \textit{corpus} of the work'' \parencite[127]{Sze08:Lis}. On the one hand, `distraction' in the listener relates to an inability to listen structurally, to maintain and analyze the relations of the different elements of the musical discourse. However, this `inability' is measured against the standards of the music work, which `tells' you how to listen. Therefore, a distracted listener pays no attention to the way the work should be listened, and it is left `outside.' On the other, the notion of a corpus of music work (understood broadly the \textit{oevre} throughout the composer's lifetime) relates to the concept of the archive. The fact that a listener would remain `outside' speaks of the filtering activity of archives. 
In sum, the moment the music work begins to act as `work,' its listening is predetermined neither by the physicality of the waves in media, nor by the virtuality inherent in perception, but by a teleology of work. 

\paragraph{Music Unwork}
When understood ontologically, work has no purpose other than being. `Work,' in relation to music, begins to separate from itself. What is this `itself'? Its productivity, its objectification, its convergence, its completion, integrity, and organicity, its unification, and its consignation. First, the productivity of work stops, not in the sense that there is no longer any activity of `production,' but rather because there is no longer a notion of a finished product. Second, the objectification of the music work (the work-as-object) looses its retaliation, that is, the `object' stops `absorbing' the listener. Third, instead of convergence into the `one' of the work, we have a certain divergence that instead of being `more' than one, it becomes always `less.' Fourth, the work of music is no longer a whole: it is not an `organicity,' but an `inorganicity' that relates to the destruction drive of archives. Both listener and work become incomplete (like Nancy's self, interrupted and suspended), an in such way we can understand their disintegration: the `dead limb' is not the innatentive listener any more, but rather, the concept of work itself. Fifth, the music work stops pursuing unity, and it is instead segregated. What this entails is that, in its core, its disintegration is a way for the music work to sever itself from its own historical constitution. In this sense, we can speak of a break of music composition with its past. This severing constitutes a break because, at once, it erases and inscribes its consignation. That is to say, in being an anarchic breakage, the ontological understanding of work opens up an asthetic space for imagination, while nonetheless still remaining under the spell of archives, under their constitution. In this sense, a certain nostalgia of the unwork should not misguide us into inactiviy. On the contrary, in music composition today we can still engage in resonance with this spectral `feature' of the music work, we can still address the powerful force that drives the archive, and this addressing is something that occurs in databasing. We can feel this ifraskin common to humans and nonhumans in database music. 

% (is there an archive of listening?)

\paragraph{A Space of Difference}
In thinking listening this way, the concept of the work is relieved of its duties, discharged, fired, it becomes unemployed. The regime of listening becomes a listening space, but a space not of equality: a space of difference. Within this space where difference resonates, the music work no longer `works', it `unworks' \see{inoperativity}. That is to say, the relations between the different resonant points in the composition network expose themselves in a state of suspension, or interruption, creating space with the space of their own incompleteness by the fractality of their fracture. Thus, inoperativity is creation, it is techne, but it is a creativity that is necessarily indefinite, incomplete: the moment it becomes a thing it begins to work in the realm of the `archi'; the moment it remains suspended upon its limit, it unworks in negation of the `archi'. One is tempted to place this inoperativity in utopia, in the very instance of the non-place itself, but then one would forget what is already `there', the fluid medium, as well as gravity itself, which was until recent studies, thought of as unrelated to sound.\footnote{In a recent study, sound itself proven to make (tiny) gravitational fields: ``We show that, in fact, sound waves do carry mass ---in particular, gravitational mass. This implies that a sound wave not only is affected by gravity but also generates a tiny gravitational field, an aspect not appreciated thus far'' \parencite{PhysRevLett.122.084501}.} One would be tempted, equally, to place this inoperativity outside temporality itself, but then one would forget forgetfulness itself. Inoperativity is within the resonant space of an always.

\paragraph{A Severed Work}
What constitutes, then, that moment when the music work becomes a work? How is it possible for the work to become a thing, for the object to become the ruler, for the regime to be built on the first place, if the resonant space is already an inoperative space, interrupted and suspended? I would like to revert Szendy's metaphor of the amputated limb, and propose that it is the music work itself what is amputated, what falls off, the moment that it becomes a finished thing. Thus, just like the modern inattentive listener falls in the uselessness of its excess, sound `outside' the work is cut from the work, fading out in the uselessness of its excess. Like the human in Kittler's digitally converged apocalypse, redundancy is out of the question, it is left at the gates of the majestic concert hall, with the rest of the (useless) humans: it is literally and conceptually placed outside architecture itself. The created work, in its essential nature of being a cohesive, coherent whole, separates itself from the world of mechanical waves, and forms the one and only work: the piece of music. A `piece' not because it is in itself incomplete, but because it is the piece of the whole of the work of the composer.

\paragraph{Absorption}
I would like to add to this worldview another concept brought by Szendy, that of `absorption.' He claims that it is the absorption of the listener in the work what is the ultimate aim of this modern regime of listening. Not surprisingly, `absorption' is the key concept in Iannis Xenakis' narrative of the fours stages of degradation of Western Music's ``outside-time structures,'' article \textit{Towards a Metamusic} (1967): ``we can see a phenomenon of absorption of the ancient enharmonic by the diatonic. This must have taken place during the first centuries of Christianity, as part of the Church fathers' struggle against paganism and certain of its manifestations in the arts\dots'' Later, referring to larger structural groupings: ``this phenomenon of absorption is comparable to that of the scales (or modes) of the Renaissance by the major diatonic scale, which perpetuates the ancient syntonon diatonic\dots'' Finally, ``one can observe the phenomenon of the absorption of imperfect octaves by the perfect octave by virtue of the basic rules of consonance'' \parencite[189-190]{Xen92:For}. The final stage of this process of absorption and degradation comes with atonalism, which ``practically abandoned all outside-time structure'' \parencite[193]{Xen92:For}. However, Xenakis' narrative contextualizes his sieve theory, devised as a means to ``establish for the first time an axiomatic system, and to bring forth a formalization which will unify the ancient past, the present, and the future'' \parencite[182]{Xen92:For}. Further, Xenakis formulated this theory with computers in mind, that is, with its concrete application in computer programs, under the subtitle ``suprastructures'' \parencite[200]{Xen92:For}. Therefore, considering the organicity of the work in Xenakis's overly modern gesture towards unity, metastructure, and mechanization, which was built in reaction to the ``poison that is discharged into our ears'' as he witnessed the ``industrialization of music [that] already floods our ears in many public places, shops, radio, TV, and airlines, the world over'' \parencite[200]{Xen92:For}, how can these notions of inoperativity be found together within architecture? How would this archi-techne be designed? Would it still be the product of the `archi'? Where is the poison that Xenakis the architect and  composer, was identifying with `industrialized' music? Is it not a product of modernity itself, as the working that listened to itself to the point of working out a Xenakis-listener-node to the extreme? 
