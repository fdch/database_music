\begin{quote}
	I am motivated to present this architecture, which is linked to antiquity and doubtless to other cultures, because it is an elegant and lively witness to what I have tried to define as an outside-time category, \textit{algebra}, or structure of music, as opposed to its other two categories, in-time and temporal. \im \parencite[192]{Xen92:For}
\end{quote}

\begin{quote}
	With [the relational] model any formatted data base is viewed as a collection of time-varying relations of assorted degrees\dots this collection is called a \textit{relational algebra}\dots a query language could be directly based on it\dots The primary purpose of [relational] algebra is to provide a collection of operations on relations of all degrees\dots suitable for selecting data from a relational data base. \im \parencite[1-5]{Codd72relationalcompleteness}
\end{quote}

\paragraph{Querying the Sieves}
If we consider pitches as an outside-time (relational) database, one way of understanding Xenakis' sieve theory is as a query method, for which E.F. Codd's \nameref{model:relational} model would fit perfectly. The nature of this consideration stems from the application of algebra as a programmable selection mechanism or simply, filters. Both concepts (sieves and relational algebra) have a common link, which is, not surprisingly, the computer itself, and not just any computer, the \gls{ibm-7090}.\footnote{Among other things, the \gls{ibm-7090} computer was used in the computation of the first 100,000 digits of $\pi$ \parencite{picalc}, Roger Shepard's computation of the homonymous `shepard' tone \parencite{shepard}, Alexander Hurwitz's computation of the 19th and 20th mersenne prime numbers,\footnote{\url{https://www.mersenne.org/primes/}} and Peter Sellers' plot-twisting moment in Stanley Kubrick's ``Dr. Strangelove or: How I Learned to Stop Worrying and Love the Bomb:'' \url{https://en.wikipedia.org/wiki/Dr._Strangelove}} While Xenakis' experiments were carried out on the \gls{ibm-7090} mainframe computer located at IBM-France in Paris, Codd himself worked at the IBM Research Laboratory in San Jose, California. Furthermore, this same computer was used by Hiller and Baker in their realization of \gls{musicomp}, a pioneering language for algorithmic composition \parencite[44]{Ari05:Ano}. Most important, the \gls{ibm-7090} used the programming language FORTRAN IV, as can be seen by the printed FORTRAN routines for Xenakis' 1962 work \textit{Atrées (ST/10-3 060962)} \parencite[145]{Xen92:For}.\footnote{Interestingly, given that \citeauthor{arizaSieves} finds Xenakis' sieves code unusable \parencite[1]{arizaSieves}, chances are that the printed code for the ST/10-3 composition is likewise useless.} Xenakis's work on sieves came a few years after his experiments on the \gls{ibm-7090}, and his sieves program was written in Basic and then in C. However, the experience with FORTRAN IV at the \gls{ibm-7090} serves nonetheless as a common ancestor to both Xenakis and Codd. For example, Xenakis' transcriptions in early \gls{cac} systems were performed with tables of outputted computer data. Further, \citeauthor{Ari05:Ano} \parencite{Ari05:Ano} writes how ``the early systems of Hiller, Xenakis, and Koenig all required manual transcription of computer output into Western notation. The computer output of these early systems was in the form of \textit{alpha-numeric data tables}: each row represents an event, each column represents an event parameter value'' \im \parencite[94]{Ari05:Ano}. In this sense, performance in \gls{cac} meant interpreting results out of a database.\footnote{For further reference on the early uses of computers in \gls{cac}, I refer the reader to \citeauthor{Ari05:Ano}'s PhD thesis \parencite{Ari05:Ano}.}


% This should be a footnote as it is not directly related to your argument byt only as a reference for speed. or am I reading something wrong?

\paragraph{Sound Synthesis Parenthesis}
(Before continuing, a sound synthesis parenthesis must be opened. While Xenakis praised the speed at which the \gls{ibm-7090} could perform computations, Max Mathews \parencite{Mat63:The}, then director of the Behavioral Research Laboratory at Bell Telephone Laboratories, wrote:

\begin{quote}
	A high-speed machine such as the \gls{ibm-7090}, using the programs described later in this article, can compute \textit{only about 5000 numbers per second} when generating a reasonably complex sound. However, the numbers can be temporarily stored on one of the computer's digital magnetic tapes, and this tape can subsequently be replayed at rates up to 30,000 numbers per second (each number being a 12-bit binary number). \im \parencite[553]{Mat63:The}
\end{quote}

Mathews' concern for speed was grounded on the need to achieve sound synthesis, which meant fast computations of the sample theorem. Initially, the first synthesized sound was obtained in 1957, with the (assembly-code written) MUSIC 1 program with the IBM-704 (a predecessor of the \gls{ibm-7090}). Later, when Bell Labs obtained the IBM-7094 ---which ``was a very, very effective machine'' \parencite[16]{Roa80:Int}---, and in combination with the (then) widely available FORTRAN compiler, Mathews could develop the MUSIC I program, into MUSIC V, which became the first portable computer music language designed for computer music synthesis.\footnote{As an example, I would refer the reader to James Tenney's work from 1962 ``Five Stochastic Studies,'' which can be found on a YouTube account on his name: \url{https://www.youtube.com/channel/UCEzSaoPnxCJVzXxA9obuRWg/videos}. Roads, while interviewing Matthews recalls this piece to be named ``Noise Studies'' \parencite[18]{Roa80:Int}, which fades out the reference to Xenakis' music.} I will close this parenthesis, not without returning to this discussion in the following section \see{improv})

\paragraph{Algebraic Abstractions for Freedom}
Xenakis' and Codd's papers came out around the same time: Xenakis' english publication of \textit{Towards a Metamusic} was in 1970, Codd's papers were published in 1970 and 1972. While sieve theory was aimed at providing a plethora of computable sets (or relations) of pitches, according to different temperings of the smallest displacement unit and the selected value for the modulo operator, Codd's relational algebra was meant the internal structure of a query language for selecting elements based on their relations. Both of these can be considered algebraic abstractions of a selection process. In the case of Xenakis, the abstraction was one held outside-time. This meant that the composer could make a snapshot, or a tomography of the pitch space in order to analyze it, extrapolating structural relations. In Codd's case, the abstraction was spatial: the query language would be separated from the database itself, allowing a distance between a `backend' and a `frontend,' allowing databasers to perform queries without worrying about internal data structures, memory allocation, since these operations would occur in the background. Both methods came as an extension of freedom on the human operator: by black-boxing hardware-specific programming, the human operator could devise any kind of algebraic queries, thus operating at a higher level of abstraction, enabling a less problematic kind of envisioning. Conversely, Xenakis writes: ``freed from tedious calculations, the composer is able to devote himself to the general problems that the new musical form poses, and to explore the nooks and crannies of this form while \textit{modifying the values of the input data}\dots'' \im \parencite[144]{Xen92:For}.

\paragraph{A Cosmic Vessel and an Armchair}
Therefore, the composer delegates to the computer the minutiae of arduous iterative computations: precisely what the computer is better at than the human. As a result, in Xenakis' view, and in resonance with programmer Charles Bachman's claim for the \textit{The Programmer as Navigator} \parencite{Bachman:1973:PN:355611.362534}, the composer became a pilot:

\begin{quote}
	With the aid of electronic computers the composer becomes a sort of pilot: he presses the buttons, introduces coordinates, and supervises the controls of \textit{a cosmic vessel sailing in the space of sound}, across sonic constellations and galaxies that he could formerly glimpse only as a distant dream. \textit{Now he can explore them at his ease, seated in an armchair} \im \parencite[144]{Xen92:For}
\end{quote}

Codd's and Xenakis' propositions were abstractions deeply rooted in and contextualized against a backdrop of their own fields. Xenakis wrote against the current state of Western Music with its ``degradation of outside-time structures'', the ``followers of information theory'' and the ``intuitionists.'' Codd wrote against the previously developed hierarchical and network database models. Most important, these tools and their development had the human operator's considerations in mind. The composer, like the databaser, would engage in a rudimentary and limited, but still present, feedback process at the \textit{input} level. That is to say, unless rewriting the code, which consisted in a very long and economically expensive process combining punch cards and magnetic tapes, the composer and the databaser could change the input several times, achieving different outputs in a matter of hours.\footnote{As a reference, the computation of the first 100,000 values of $\pi$ took about eight and a half hours \parencite{picalc}.} For example, queries made on the relational model would appear on screen at a very fast rate, thus enabling better tuning of the input in relation to a wanted output. Likewise, the composer could modify the input values to highly complex calculations that would otherwise take a long time, or be error prone. The limitation, of course, is the level of intervention with the code itself, which the overall circuitry would itself complicate; criticism on account of this shortcoming of the circuit would thus be rendered anachronic, but recalling these limitations places composition and databasing in perspective.

