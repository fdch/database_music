I would like to take an improvisational turn that would make Xenakis fall off his armchair. Xenakis' fall would be contemplated against the spirit of the later discussions on interaction that came with George Lewis and \textit{Voyager} \parencite{Lew93:Put, Lew99:Int, Lew00:Too}. Lewis called his approach an ``improvisational, nonhierarchical, subject-subject model of discourse, rather than a stimulus/response setup'' \parencite[104]{Lew99:Int}. Thus, the activity of the composer was reconfigured in a networked relation \textit{with} the computer. That is to say, Xenakis' metaphor of the composer as pilot turns upside down, altogether reconfiguring the navigational metaphor: the ship begins to navigate itself.

\paragraph{The Computer as a Musical Instrument}
It is now pertintent to bring back Max Mathews paper \citetitle{Mat63:The} \parencite{Mat63:The}. The architecture of \gls{music-v} is founded upon the concept of the computer as an instrument that the composer executes by providing it a score. The three stages of data flow (reading, sorting, and executing) can be understood as modeled with three music concepts: score, conductor, and instrument. Therefore, it can be argued that \gls{music-v} left composers, performers, and programmers on the margins of its data flow, and that by this elision a new identity was in the makings. On the one hand, the hybrid musical instrument that the computer represented already subsumed three concepts into one, resulting in a hybrid score/conductor/instrument. On the other, by this elision, the three missing human terms have now been subsumed anew, forming a hybrid definition of composer/performer/programmer. In any case, this hybridity is evident in the music works that I have cited in earlier sections \see{applications}.

\begin{quote}
	So far I have described use of the computer solely as a musical instrument. The composer writes one line of parameters for each note he wishes played and hence has complete control of the note. He is omnipotent, except for lack of control over the noise produced by the random-number unit generators. \textit{Here a minor liberty is allowed the computer}. \im \parencite[557]{Mat63:The}
\end{quote}

\paragraph{A Minor Liberty}
As can be read at the end of the introduction to \gls{music-v} \parencite{Mat63:The}, the extent of this ``minor liberty'' was measured against Hiller and Isaacson's previous work \parencite{Hil59:Exp}, which Mathews describes as an extreme case of the computer as composer: ``the computer can be given a set of rules, plus a random-number generator, and can simply be \textit{turned on} to generate any amount of music'' \parencite[557]{Mat63:The}. On the one hand, Mathews' argument is based on the ``omnipotence'' of the composer in front of the computer. Control of the music work is not something that can be delegated to the computer, unless it comprises lengthy calculations of pseudorandomness. On the other hand, as Ariza has shown, the computer output of early \gls{cac} has been often misconceived in the literature as directly musical output, disregarding the extensive transcription work on the part of composers \parencite{Ari05:Ano}. Nonetheless, Mathews' ``minor liberty'' can be considered as a reassurance for the reader that computers would not take control over music, let alone over the world. As I see it, arguing for control while granting some liberty relates to a negotiation between a composer's work and computer time. Because of the correlation between sonic complexity and parameter input, ``the composer must make his own compromise between interest, cost, and work'' \parencite[555]{Mat63:The}. Pseudorandom generators introduced complexity in an efficient way \parencite{fdch/papers/spectral}. Therefore, in an economical choice, arguing for omnipotence allowed for some aesthetics agency to come from computers. 

\paragraph{The Computer as a Player}

\citeauthor{Row92:Int} \parencite{Row92:Int} identified two paradigms within interactive systems: \textit{instrument} and \textit{player}. The instrument paradigm comprises systems in which performance gestures are sensed (collecting gestural data), processed (reading and interpreting data), and then a sonic output is elaborated in the form of a response. The player paradigm comprises the creation of ``an artificial player, a musical presence with a personality and behavior of its own\dots'' \parencite[Chapter~1]{Row92:Int}. Therefore, the system contains embedded processes that grant a great level of independence. This means that the composer intentionally relinquishes control of the artwork's structure to the system. Like Vaggione's concept of the computer as a complex system in which the composer ``is imbedded in a network within which he or she can act, design, and experience concrete tools and (meaningful) musical situations'' \parencite{Vag01:Som}, the human node breaks the traditionally hierarchical structure of composer-work, arriving at a distributed authority of the work among the elements of the system. For example, in George \textcite{Lew99:Int}'s \textit{Voyager}, ``the computer system is not an instrument, and therefore cannot be controlled by a performer. Rather, the system is a multi-instrumental player with its own instrument'' \parencite[103]{Lew99:Int}. The computer becomes an improvisation partner. While the limitations of computer capabilities precluded more complex conceptualizations of the type of interactivity between computer and composer in \gls{music-v}, as personal computers became affordable the type of negotiations no longer depended on economic decisions. For Lewis, this  negotiation existed sonically between computer and improviser:

\begin{quote}
	There is no built-in hierarchy of human leader/computer follower, no `veto' buttons, pedals, or cues. All communication between the system and the improviser takes place sonically. A performance of Voyager is in a very real sense the result of a process of negotiation between the computer and the improviser. \parencite[104]{Lew99:Int}
\end{quote}

\paragraph{Programming Decisions}
However, in order to implement concepts coming from artificial intelligence such as machine listening and learning, the complexity of the program increases exponentially. In light of the difficulties arising from programming large software, and in response to Lewis' criticism of the \gls{max} patching paradigm rooted on trigger-based interactivity, Miller Puckette responds: ``If you wish your computer to be more than just a musical instrument ---if you want it to be an improvisation partner, for instance--- you need a programming language. One thing people in this situation might want to do is write \gls{max} external C procedure'' \parencite[8]{Lew93:Put}. As Rowe writes:

\begin{quote}
	To arrive at a more sophisticated interaction, or \textit{cooperation}, the system must be able to understand the directions and goals of a human counterpart sufficiently to predict where those directions will lead and must know enough about composition to be able to reinforce the goals at the same moment as they are achieved in the human performance. \parencite[Chapter~8]{Row92:Int}
\end{quote}

The player paradigm and its subsequence reconfiguration of compositional authority is possible by means of a database: the computer stores features during the course of the performance, which are then analyzed over time, and which serve as guides for the sonic outcome on the part of the computer. As I mentioned earlier, while the guidance of the database provides paths through uncharted territories, it also hides other paths \see{computer:free}. \textit{Voyager} indeed brings interactivity between humans and nonhumans to another stage, and because of it, music composition can be seen differently. However, the intricacies of programming decisions still play a role in the musical outcome, specifically in the modelling of musical concepts within data structures.

\paragraph{Anachronistic Composers}
This notion of interactivity differs greatly from Xenakis' model of the (modern) composer. He is sitting quietly in his armchair pressing buttons in 1962. By pressing them and inputting certain values, he controls the output, since he knows beforehand the internal mechanisms that are embedded in the software. This image of the modern composer in front of computer technology can also be found in, for example, Edgar Varèse: ``The computing machine is a marvelous invention and seems almost superhuman. But, in reality, it is as limited as the mind of the individual who feeds it material'' \parencite[20]{Var04:The}. Varèse's words, however, refer to the creative limit that a computer might have, which is always a function of the input and, by extension, of material itself. Furthermore, in relation to electronic technology, he writes: ``like the computer, the machines we use for making music can only give back what we put into them'' \parencite[20]{Var04:The}. Therefore, from these images of Varese-composer and Xenakis-composer, two axioms can be extrapolated: first, that composers do not lose control of the output; second, that the way to interact with computers is precisely by telling them what and when to do it, so that the user is in total operative control. It is against these two axioms of computers and composition that Lewis' work in the late 1980s and 1990s can be contextualized. More precisely, it is because of the anachronic presence of the modern `eurocentric' composer, and of its popularity among computer music history, that Lewis brings into surface the question of interactivity. 

\paragraph{\texttt{[bang(}}
Placing \gls{max} into perspective by commenting on the social and cultural environment of computer music of the late 1980s, Lewis writes:

\begin{quote}
	[Lewis:] `interaction' in computer music has moved from being considered the province of kooks and charlatans (I'm proud to have been one of those), to a position where composers now feel obliged to `go interactive' in order to stay abreast of newer developments in the field \parencite[11]{Lew93:Put}.
\end{quote}

The way in which interactivity was conceived in the `interactive' music made with \gls{max} was, for Lewis, determined by a fundamental feature of program ---the `trigger'---, which, in turn, was grounded on a more general programming concept: the conception of the patching window as a digital equivalent to the analog synthesizer's patching mechanism, where graphic cords are equivalent to cables, equating data flow with voltage flow. Nonetheless, the trigger (`bang') is a feature, not a bug, unless it is used as an extension of the stimulus/response paradigm of interactivity. In other words, in resonance with Vaggione \see{style}, subordinating music events to triggers by a human operator brings out a certain military metaphor which Lewis calls ``hear and obey'' \parencite[11]{Lew93:Put}. This metaphor can easily be extended to that of weaponry, and to `bang,' the unfortunate naming of the method which (generally) triggers an object's core routine.\footnote{However unfortunate this `bang' name is, the conputer itself makes one think back to the 1946 setting of the UNIVAC computer, in the military context of the Manhattan Project, for which it was used to get closer to the `H' bomb. That is to say, even if `bang' was named differently, the computer itself would be inevitably linked to this particularly \textit{big} bang.} In order to address this shortcoming of interactivity, Lewis relates the trigger-based type of interactivity that was in vogue in the early 1990s to rudimentary mental processes or, as he puts it, to ``amoeba- or roach-like automata'' \parencite[11]{Lew93:Put}. In this sense, not only interactivity is at stake, it is rather the empowering of the image of the composer by the presence of a simple model of interaction. This intentionally (very) simple automaton promotes two fundamentally hierarchical notions that Lewis attempts to deconstruct. On the one hand, the composer as controller who would never relinquish control of the music work, that is, the modern (eurological) image of the composer, and the old ghost train that comes with it: ``The social, cultural, and gender isolation of the computer music fraternity (for that is what it is)'' \parencite[11]{Lew93:Put}. This image leaves improvisation, together with non-eurological thinking out of the scope of contemporary music research. On the other hand, the human operator, as the higher (architectural) mind that would not allow for the nonhuman to become an operational agent beyond the instructions for which it was designed. In this sense, the simple-level automaton is a symbolic restraint representing the classical concept of the human, which allows a non-threatening relation between man and machine that can be considered functional, productive, and operative.

\paragraph{Nonhuman composers}
One is tempted to claim that the first of these images ---the reified composer--- is determined by the second ---the reified human---, and that their relation is a matter of depth or inheritance. Thus, in order to redefine the composer one would have to redefine the human. In turn, this depth would be measured against that which is nonhuman, and by extension, that which is non-composer. We can understand, therefore, Lewis' narrative as the redefinition of composition itself by making the non-composer (e.g., what was eurologically considered the `improviser' or the `performer') resound back into composition, regrouping the concept `composer,' but not as a whole, since now the extent of its terms have found places within a networked system. This is precisely what he does in \textit{Voyager}. The composer, like the human, became regrouped in a certain hybridity where interactive computer music is \textit{not entirely} driven by (human) input, because the system proposes an `input' of its own. 

\paragraph{Fractured Works}

\begin{quote}
	[Lewis:] The composer therewith relinquishes some degree of low-level control over every single bloop and bleep in order to obtain more complex macrostructural behavior from the total musical system. The output of such entities might be influenced by input, but \textit{not entirely} driven by it. \im \parencite[11]{Lew93:Put}
\end{quote}

It is precisely this `not entirely,' as a negation of wholeness, what begins to question the basis upon which our general concept of the human is built, and by extension, what begins to announce the presence and the agency of everything that falls outside of its definition. It is the beginning of a breakage, a crack on the foundation of Xenakis' (old) armchair, from which the state of suspension of the concept of the music work can also be understood:

\begin{quote}
	[Lewis:] With this in mind, it becomes easier to see that Voyager is \textit{not really a `work'} in the modernist sense ---heroic, visionary, unique\dots I choose to explore allegory and metatextuality, the programmatic, the depictive--- and through embedded indeterminacy [pseudorandom generators], the contingent. Ultimately, the subject of Voyager is not technology or computers at all, but musicality itself. \im \parencite[110]{Lew99:Int}
\end{quote}

What this fracture in the constitution of the concept of the work reveals is the hybrid nature of reality and virtuality. Understood traditionally or under the stipulations of first wave cyberneticians, the composer, being the real factor in the constitution of the (modern) image of the composer, is faced with the virtuality of the computer. Upon this encounter, the virtual comes as a form of threat to that image, and thus to the reality of the composer: the computer's virtuality would replace first the imaginary and then the real. However, as Lewis claims, interactivity between the composer and the computer allows both the virtual and the real, ``virtuality and physicality,'' to engage in the production of a hybrid that ``strengthens on a human scale.'' ``Seen in this light,'' Lewis continues, `` virtuality should enhance, not interfere, with communication between us'' \parencite[110]{Lew99:Int}. Considering the role of virtuality after embodied new media theory, the computer reveals to the human ---composer, improviser, performer--- the very condition of its own capability for virtuality, and thus redefines reality for the composer, and in turn, a new image of the composer emerges. This new image is no longer a threat, but rather a reflection of a new agency within music composition. In the case of \textit{Voyager}, this virtuality is sonic, it comes as the ``emotional transduction'' that Lewis aims for with this computer system. Therefore, Lewis is right in claiming that \textit{Voyaguer} is not `really' a work, because it is virtuality as such: a virtual composer, improviser, and performer, and in sum, a virtual listener.

\paragraph{Databasing Vessel}
Understood as a listener, \textit{Voyager} engages not only with signal processing at the lower level, it engages with the resonant process of the relation to self. Furthermore, the computer is not only listening, it is \textit{databasing}, because it is keeping record of the listened features, and in so doing, it becomes empowered with the database. This database of actions, however, is the sonic trace of the performance, which is what is most surprising of its agency, and what resounds most in time. Therefore, far from being `really a work', but also far from Lewis' notions of narrative in the sense of ``allegory and metatextuality, the programmatic, the depictive'' \parencite[110]{Lew99:Int}, I consider \textit{Voyager} an unwork of music, because it questions the operativity of the music work. However, certain notions of productivity and cohesion are still present within Lewis' music and texts, and thus \textit{Voyaguer} is still considered a `work;' a destiny that somehow manages to persist within the practice of composition, which is why the metaphor of the unwork comes not without resistance. Nonetheless, and without a doubt, Lewis' claim for a ``non-eurocentric computer music'' \parencite[107]{Lew99:Int} can be a starting point to the conceptualization of the unwork.

