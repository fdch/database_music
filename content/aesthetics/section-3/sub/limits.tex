
\paragraph{Exposure}
The performativity of databasing can be understood in terms of what Nancy calls exposure \parencite{Nan91:The}. Exposure is the appearance of a limit and the finitud of a singularity. With this limit instantiated in the (public) moment of the performative act, is how communication emerges as that which is in common among singularities. That is to say, because it is itself nothing (``neither a ground, nor an essence, nor a substance'' \parencite[31]{Nan91:The}) Nancy considers finitud to just appear in the form of communication: ``it presents itself, it exposes itself, and thus it exists as communication'' \parencite[31]{Nan91:The}. His emphasis on communication as exposure marks a crucial distinction on the concept of community. For Nancy, as I have described above \see{inoperativity}, community cannot come from an instance of work: it emerges as an instance of the communicative action which is ``the unworking of a work that is social, economic, technical, and institutional'' \parencite[31]{Nan91:The}. In other words, the performativity of this communicative act, the publicness and dramatic qualities with which it unfolds, result in exposure. This is why, in the case of databasing, what is exposed at the limit of its performance is the finitud of the database itself. Through this exposition of the limit is how the singularity of the database can be communicated; or, better, through this exposure is how communication expands our concept of community towards one that includes the database as an agent of community itself.

\paragraph{Anarchic Touch}
As Nancy writes, ``a singular being does not emerge or rise up against the background of a chaotic, undifferentiated identity of beings'' \cite[28]{Nan91:The}. This is to say that, like Butler's gendered self, there is no substance within singularity. The appearance of a database as finitud comes not from an originary \textit{archē} which would impose its archontic power as is the case of archives. Further, the limit of the database is not instantiated out of the one, a ``unitary assumption'' \cite[28]{Nan91:The}, or the wholeness of a single one. This finitud does not come from intentionality or any essentialist notions: it simply appears, for Nancy, in the form of touch: ``\dots as finitude itself: at the end (or at the beginning), with the contact of the \textit{skin} (or the heart) of another singular being, at the confines of the same singularity that is, as such, always other, always shared, always exposed\dots'' \im \cite[28]{Nan91:The}. The temporality of this touch, like that of gender, is anarchic.

\paragraph{Communities of Skin}
The limit of the database, as performative, spectral skin, is the condition for community to emerges between the human and the nonhuman. This means that the agency locus of the database needs to be placed precisely on its skin, because it is what becomes public of itself. In other words, given that this skin is available to the perception of others, it becomes touchable, it reaches our own limit as databasers. By doing so, that is, by exposing our own limits to ourselves and to each other, the database changes our definition, or, better, delimits the extent of our own singularity. However, this does not mean that it stands in the way of our performativity, or worse, that it precludes or determines ourselves. If this were true, we would be once again subject to technical determinism, essence fabrication, etc., and falling out of considering any possibility of community between anything that is not human ---or, more accurately, anything that is not `man.' As Nancy rightfully claims, `` it is not obvious that the community of singularities is limited to `man''' \cite[28]{Nan91:The}. Thus, the fact that the skin of the database changes our own skin simply means that we are already in communication with it, that is, in community, and also in a state of resonance with it. This is the function of the skin of the database: like the skin of a drum, or the skin of a loudspeaker, the skin of the database resonates with our own skin, engaging the resonant body with the resonant spectrality. This is the community of resonance, the community of the resonant network, which has no purpose, no intentionality behind, no essence; only appearance and motility, performance and repetition: an activity that is the unworking with which community exposes itself. 

\paragraph{Hybrid Pluralities}
Database models tend to reside next to each other, either within a single database system or within an interconnected networked system. With this plurality of the model, databasers have access to the many features that each model offers, focusing on those features that are suitable for their needs.  The skin of the database is as fluid as the constitution of gender, and if this is true, then the fluidity of databasing itself comes to represent the constitution of gender through the performativity of databasers. By resonating in such performativity, databasers approach (but do not reach) the limit of the database. This approach to the skin of the database exposes simultaneously the skin of the databaser to the database. What this exposure amounts to is not, however, an opposition of forces. It results in the fragmented state of community that resides in the different degrees of this exposure. In other words, this exposure is of a hybrid plurality that resonates at the limit. Engaging with the touch of the spectral database means reconfiguring, resounding, and remembering our own sense of touch, just as well as our own sense of self.