Delineating the agency of the Database in the practice of music composition, I discuss the aesthetics of Database Music, developing the concepts of listening, memory, and performance. 

First, I analyze the extent to which the Database can be a listening subject which promotes illusions of style and authority. I consider style and authority as central aspects of the sphere of aesthetic agency of the Database. I then focus on a form of collective `listening' and I arrive at my conception of the Database as an inherently deterministic system. This system is shaped as a network of nonhuman agents, whose `resonance' is fundamental to its definition. I use this resonant network to further analyze the agency of the Database, in terms of how authorial qualities percolate through the network. I use Jean-Luc Nancy's ontology of sound to understand how the database can be a listening subject. In Brian Kane's reading \parencite[143-144]{Gra15:The} of Nancy's work \parencite{Nan07:Lis}, he presents the this ontology ---i.e., what Nancy calls \textit{resonance}---, considering it as a process constitutive of a phenomenology of the self. 

% EH:
% Listening to itself? or just listening?
% 
% 

Second, in order to narrow the gap between human and nonhuman agency, I assess the extent to which computer memory resembles human memory. On the one hand, I compare memory and writing with digital information storing, and thus arrive at databasing as a form of memory. On the other hand, I consider archives as collective memory, which serves to to explain how the Database can also be a form of collective memory. 

Finally, I focus on the performativity of the database. On the one hand, I claim that the database is gendered. I argue that the notion of `style' is what promotes the illusion of a gendered subject in the Database. I argue that since both the performance and the directionality of the `styling process' remain strictly on the virtual skin of the database, the database's authorial subject, like the gendered self, remains in the spectrum of the illusory. On the other hand, I claim that the limit of the Database resides on its performativity. I consider the technical aspects of databases and define computer systems as networks of interconnected-but-independent databases. This definition serves to extend the performatic limit of databases to computers, and therefore to link the performance of the database to the performance of the computer. My goal in this final section is to lead the way to the connection between Database performance and Music Composition: the performatic limit of the Database is also the limit of Music Composition.

In search of understanding the political in Database and Composition practices, I question the established concept of music composition and arrive to new definitions of the music work, practice, and authorship. 

First, I consider the concepts developed in the previous chapter to understand Music Composition as Database Performance. I propose that the ontology of Composition needs to be redefined in terms of the agency of the Database. My goal in this section is to reveal that the Database agency, when contextualized within Music Composition, has the form and the politics of a music listening to itself. 

Second, I use Nancy's concept of inoperativity to redefine the music object. I argue that the inoperativity of the listening experience, which resides on the delay between sense and sensuality, provides insight on the type of unworking that affects music composition. I thus redefine the outcome of music composition as the \textit{severed music object}, emphasizing its inoperative status of suspension, withdrawal, and its inherent state non-completeness. I then consider how this state of suspension of the severed music object can be analyzed in terms of a Community of artists, database performers, composers, etc., mutually exposed to each other (Nancy 1991). Therefore, in order to understand the dynamics of this transversal community of Database and Composition, I analyze the paradox of anarchy and reflect on the consequences of both the anarchic and the inoperative in Database and Composition practices. 

Finally, I present my view on collaboration, and propose a redefinition of the term uprooting it from the traditional union of forces forming a whole. I claim that the new form of collaboration can be understood as a form of collective, or \textit{trans-inoperation}, consisting in the mutual exposure of the limits of singular, performing beings. As a consequence of this form of collective inoperance, I claim that a new politics of authorship needs to be analyzed, particularly in terms of the spectral in the Database. I question the power of this illusory figure in terms of the effectiveness of the archontic principle that is present in \textit{trans-inoperant} works of art. I believe the specter of the author loses the sensuality and the sense of the listening subjects in state of trans-inoperance, and thus the power of the author ceases to take place.

