
% EH:
% 
% Why does Irineo's isolation preclude a replaying of any infinitesimal
% moment? Your story is poetic and powerfully metaphoric such that I get
% your point without fully understanding how the story demonstrates it
% alas. Thanks!






\begin{quote}
	I suspect, nevertheless, that he was not very capable of thought. To think is to forget differences, to generalize, to abstract. \parencite[2]{Bor42:Fun}
\end{quote} % 4/7/2018 10:31:06

The importance of memory ---and forgetfulness--- can be represented by Jorge Luis Borges's famous 1942 short story, \textit{Funes, the memorious} \parencite{Bor42:Fun}. Due to an unfortunate accident, the young Irineo Funes was ---``blessed or cursed'' as Hayles points out \parencite[156]{Hay93:The}--- with an ability to ``remember every sensation and thought in all its particularity and uniqueness '' \parencite{Hay93:The}. A blessing, since a capacity to remember with great detail is certainly a virtue and a useful resource for life in general; a curse, because he was unable to forget and, as a consequence, he was unable to think, to remember, to dream, to imagine. Throughout the years, he became condemned to absolute memory, and so to its consequence, insomnia:\footnote{In the prologue to \textit{Ficciones}, Borges writes that this story is a long metaphor of insomnia: ``Una larga metáfora del insomnio'' \parencite{Ovi19:Mem}.} he was secluded in a dark and enclosed space so as not to perceive the world.\footnote{Within this fictional universe, the only way for him to sleep was to imagine the opaqueness of an unknowable future\dots} Hayles focuses on one aspect of the story, namely, the fact that Funes invented ---and begun performing--- the infinite task of naming all integers, that is, of giving a unique name ---and sometimes, last name--- to each number without any sequential reference. According to how Hayles describes it, by carrying out his number scheme, Funes epitomizes the impossibilities that disembodiment brings forth, to the point that:

\begin{quote}
	If embodiment could be articulated separately from the body ---an impossibility for several reasons, not least because articulation systematizes and normalizes experiences in the act of naming them--- it would be like Funes's numbers, \textit{a froth of discrete utterances registering the continuous and infinite play of difference}. \im \parencite[156-159]{Hay93:The}
\end{quote} % 4/17/2018 5:09:57

Therefore, the point that she is touching is that of the limits and fragility of embodied memory. In that Manovichian world which ``appears to us as an endless and unstructured collection of images, texts, and other data records'' \parencite[219]{Man01:The}, this idea would be perfectly viable. Indeed, data banks have already been growing exponentially much in the same way as Borges' 1942 character's mind was aiming at. This capability of accumulation without the need of erasure is enabled by the database structure inherent in computers.\footnote{In fact, the demand when it comes to computers is less its ability to erase ---or even compress data--- than storage space, a hardware-dependent commodity that has circulated ever since Von Neumann's architecture came into the picture \see{programming}.} However, the distinction that Hayles presents ---which has been discussed before \see{bodiless_information}--- is crucial: data is not information because information needs to be embodied. Therefore, on one hand, a disembodied data bank can have all the uniqueness and difference that is available by the sum of all cloud computing and storage to date; however, on the other hand, an embodied memory is only available by the human capacity to forget. 

Matías Borg Oviedo \parencite{Ovi19:Mem} relates this incapacity for thought precisely to the negation of narrativity itself, thus finding in the image of Funes a hyperbole for contemporary subjectivity \parencite[5]{Ovi19:Mem}, where there is no room for narration, only accumulation of data. In this sense, narrativity can be seen as that which resides in the threshold between knowledge (i.e., memory) and storage (i.e., archives, databases). I believe this distinction stems precisely from the difference between information and data. The process of information, of giving form, requires a certain temporality that is not that of the immediate and extremely operative zero-time of the (computer) processor. Within the zero-time of computer operations, there simply is no time for narrative, only for addition, for an increment. With this in mind, I would like to question Manovich's opposition of narrative and database, precisely on the grounds that narrative is temporal ---happening as a historical process--- and algorithms are atemporal ---operating in an effervescent now. Therefore, Funes' accumulative memory represents the overflow of the now that precludes narration: neither data structures nor algorithms can forget to count.

Studies in cognitive psychology might have something to add here. In Wessel and Moulds' commentary \parencite{Wes08:How} on Paul Connerton's \textit{Seven Types of Forgetting}, the authors consider forgetting to be the ``failure'' of certain search processes on account of an inability to recall information from memory. While pointing to current psychological models of memory which consider forgetting to be an adaptive and functional activity, the authors acknowledge the mystery of certain aspects of memory: ``In human memory, it is unclear what really happens to old, disused or deliberately ignored memory traces ---they might be retrievable, they might be lost, but no-one can tell'' \parencite[292]{Wes08:How}.

Thus, considering this distinction between embodied and disembodied memory, I propose an imaginary experiment, one that I believe was missing from Borges' short story, however utterly fantastic his writing was. 

One thing that can be read from the story is that, in order to seclude himself from perceiving the world, or better, in order to forget the world altogether, Irineo stayed in the dark. This is how he cancelled light, a quite powerful stimuli if memory-space is to be optimized ---for the purpose of, say, getting some sleep.\footnote{For example, one of Irineo's concerns was to reduce the amount of memories on a single day, which he downsized to about seventy thousand\dots} However, there is little to no mention of the sonic environment in which Funes was embedded ---probably in the outskirts of the quiet Uruguayan city of Fray Bentos. In fact, the only sonic references are focused on the narrator's perspective, referring to Funes' high-pitched and ---due to his being in the darkness--- acousmatic voice.\footnote{This acousmatic quality of Funes' voice will not be touched here, but it is indeed a good point of departure for an essay.} Therefore, focusing on Funes' listening, by locking himself inside a room he would have managed to attenuate sound waves coming in from outside. Notwithstanding his isolation ---or, better, his self-imprisonment---, sound waves are actually very difficult to cancel. An interesting experiment would have been to have John Cage take Irineo to an anechoic chamber and ask him what he can remember then. From Cage's own experience, we can guess that Funes would effectively remember his own sounding body.

(It is interesting to compare Funes' search for filtering out the world with John Cage's search for silence. Kim Cascone writes that ``[Cage's] experience in an anechoic chamber at Harvard University prior to composing 4'33'' shattered the belief that silence was obtainable and revealed that the state of `nothing' was a condition filled with everything we filtered out'' \parencite[14]{Cas00:The}. It is interesting to place an 80 year-old Irineo in David Tudor's premiere at Maverick Concert Hall in Woodstock, NY, infinitely listening to 4'33'')

However, it is very unlikely ---but nonetheless possible--- that Borges was aware of American acoustician Leo Beranek's research for the US Army during World War II, that is, when the first anechoic chamber was built.\footnote{\url{https://en.wikipedia.org/wiki/Leo_Beranek}} Furthermore, even if he managed to isolate himself perfectly from the world, cancelling perception altogether, Funes would have been with his memories, which are not discrete, but continuous iterations of the world he had accumulated over the years. What this means is that all the sounds he had listened to would be available to his imagination.

As far as we can learn from the narrator, while \textit{smell} is referenced to in the story, \textit{sound} was completely out of Funes' concerns. Therefore, the question is how would the world sound for Irineo Funes? The task is not difficult to imagine: the world would be inscribed in poor Irineo's memory in such an infinitely continuous way that each fraction of wave oscillation would be different, unique, leaving no space for repetition of any kind. All sounds would be listened completely, with every infinitesimal fraction of oscillation of the waves pointing to the most utterly complete scope of imaginable references. It would not be inaccurate to compare this type of memory saturation with CPU saturation, for example, the way a computer would ---even the most gigantic multi-core imagined---, if it was commanded to compute, with accuracy, the wave equation. In this complete state of listening, there would be no possibility for thought, no processing of any kind, only infinite accumulation and storage.

In this sense, an infinitesimal incorporation of sound is unthinkable. This is not to be confused with the infinite structure of referrals and deferrals that difference is made of. The problem here ---and this is evident in the story itself--- is in the intersection of the finite with the infinite. While the structure of sound itself is infinite, in the sense that it is an ongoing process ---i.e., circular loop--- of difference, the singularity of the listening subject is finite: its limit exists as its possibility condition. This is to say that, given such an ontology of sound, the emergence of a resonant subject occurs \textit{at} the limit, that is, at the liminality of an exposure to itself, to others, to waves, etc. In the case of the story, while Funes' nonhuman qualities correspond to dynamics of the infinite, his human body is just like any other that was thrown in the world. Therefore, no matter what the narrator has the reader believe, Funes is not deprived of this liminality. Throughout the story, Funes' liminality grows more and more evidently, all the way until the end ---an ending that, despite all his nonhumanly infinite qualities, is, nonetheless ---be it a blessing or a curse---, utterly human.

Less than focusing on literary analysis, in this interlude I take the concept of an absolute memory within a human being to be at an intersection between disembodied theories of information and, precisely, the concept of an embodied memory. The aim is to differentiate between human and nonhuman in terms of memory and databases, so as to provide a link between Latour's `recorded movement' of the network, and Derrida's concept of the archive. 
