% There is yet another substitution that can be ammended to Latour's definition of the network. If the network is `recorded movement,' that is, a trace, a trajectory, this means that its existance is evidenced by way of not only motion in itself, in the sense that the very same oscillatory motion of reference in between the nodes creates its defining gesture. It is also the case that the recollection of movement constitutes an structurally inseparable part of the definition of the network. Therefore, in this chapter, I understand the database in its relation with memory, understanding memory from three points of view: the human, the nonhuman, and the spectral.

% In section one, I analyze memory as process of embodiment and relate it to resonant networks.

% In section two, I analyze the concept of the archive and its relation to the database, specifically, to databasing as a collective form of memory. In this sense, the nonhuman comes as an instantiation of memory outside the human. I assess the extent to which resonant networks can be considered under the scope of the concept of the archive. 

% In section three, I understand the dynamics of resonant networks spectrally, that is, as an expression of a force, or a power, that comes out of the spectrality that results out of the resonance of the human and the nonhuman. By considering the spectrality of the database, I asses the extent of its aesthetic agency in terms of a \textit{haunting} force. Therefore, if database music is indeed haunted by the specter of the database, how does this affect the aesthetic result? If the ghost of the database can be understood as an image of the nonhuman, that is, an image emerging out of the plurality of memory, then, to what extent can it be considered a singularity, a self in itself? Finally, if, as databasers, we are engaged with this force in creative action, how does the music made with databases sound, and what is making it?


% % EH:
% % This paragraph above is provocative, but I wonder whether you can more
% % persuasively explain why the plurality makes it "non-human."





