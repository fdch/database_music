\begin{quote}
	[The] Heideggerian `work of art’ is able to present a unified picture that may be used for political purposes [it] is only what it is in the world that it opens\dots Nancy is seeking a `workless’ or `unworking’ work, \textit{a work that refuses to create itself as a total work}. Hence, Nancy proposes an artwork that would offer itself as a permanently open whole, the concept of art remaining undecided and lacking anything that might unify it. \im \parencite{Gra15:The}
\end{quote}

\paragraph{An Incomplete Object}
I would like to refer once again to Jean-Luc Nancy's concept of inoperativity \see{inoperativity}, this time in relation to the music object. I argue that, given that the inoperativity of the listening experience reveals itself as the interaction between resonance ---as the \textit{différance} within sense and sensuality--- and the unworking of the network, its resulting object, instead of being a complete whole ---a finished, integral `thing', or even, a `piece'\footnote{Since, the notion of a `piece' presupposes that of the whole to which it belongs.}---, it becomes a severed music object. This object is different from Pierre Schaeffer's music or sound object, which comes to represent material with which to work. Neither it is related to Vaggione's concept of object, which comes from object-oriented programming, meaning every composable primitive, from the micro to the macro. In both of the above, the object is used to provide, though not without their author's intervention, a notion of \textit{coherence} to the work. 

\paragraph{Remains of Listening}
The object I am referring to resides in memory, as the remains of the event of an exposure. It is inherently linked to the fractured way in which our own memory works, and it is impossible to define, since it has no beginning and no end. Its dimensionality includes both beginning and ending simultaneously. This object is the spectral evidence of a musical event, or better, of the happening that takes place in listening. In being evidence, it becomes subject of analysis, it is forensic. In being fractured, it is the evidence of a destruction. In being severed, and this is the central aspect that I would like to focus on, risking simultaneously the severing of the object itself, it becomes the evidence of a sacrifice. If it can be said that the music object is a severed object, then the question of its severing necessarily relates to the question of listening. Therefore, by listening ---and, by this, I mean entering in resonance with resonance itself, exposing the self to that which returns to itself--- I participate in this severing, because in listening I choose what to listen in spite of being already deprived from that choice. 

\paragraph{Sources and Sorcerers}
The sounds onstage are always before and after the staging. The severed object of music is what, as listeners, we grab from the stage, what we choose to rip from the sounding waves, and also what we cannot help but feeling so much a part of us before noticing it is happening. Severing is yet another way of thinking the aesthetic experience of listening, but it is not as passive as it seems. Severing empowers the listener, it is the tool of listening, the reversed stilus, the inverted mouse, the part of the human that necessarily is nonhuman. With it, we can make the world appear, but only as a fraction, because `it' can never be \textit{completely}. The severed object of music is always severed, but never in the same way, since there are as many severings as there are listeners, and as many listenings as there are moments. In this difference, what is resonating is the object of music, which is never one and the same because it is a singularity resonating in plurality. Composers have traditionally been considered a `source' of this object, or better, the one at the door, the key keeper that has access to the door that opens up the flow of inspiration. The composer, but also the programmer with access to the source code, which unless it is opened, is hidden to the rest; and, unless you know the language, it is complete pseudo-linguistic nonsense with weird punctuation marks, sometimes closer to poetry than it is to extreme formalism. 

\begin{flushleft}
\small
\lstset{language=bash,mathescape=false}
\begin{lstlisting}[caption={Little words doing things},captionpos=b]
#!/bin/bash

# Palabritas que hacen cosas

while true
do
	for ever in rose is a
	do 
		say $ever
		sleep $((RANDOM/10000))
	done
done

\end{lstlisting}
\end{flushleft}

In this access to the source, the programmer and the composer are traditionally kept at a distance, as if their listening were of some other sort, engaging with the very essence of the source, drinking the water from the originary fountain, satisfying an originary thirst. Therefore, if this is the role of the composer and the programer, if this is their relation to the source, then, they are the first to perform the severing. In the hierarchy of the consequent severings, they are at the top. Further, if they are the first severers, they are the first who perform the first listening. They are the listeners at the top of the mountain, next to the source of all fountains. On the way in and out of the world, the sorcerers of condensation.

\paragraph{Naming}
I would like to point out now, that it is not my intention here to sever the head of the sorcerer, because it is an illusion that does not allow me to do so. It is not my illusion, although I have described how I interpret it, and it comes as a product of a reification of the composer, but also of the human itself as the one and only owner of the world ---that is, owner of the mountain itself, and of the water, and every particle of the one and only universe. In being in resonance, listeners become the resonating world, that is, the self begins to resonate as space. In this sense, it is the world what is listened to, and it is a world that has no apparent origin. However, the composition ---the written score, like the written code--- propose their own origin ---the composer, the programmer. Thus, they give an origin to the world itself by providing an answer (a name) to the question of creation: Who created this music? \textit{this} composer. The answer, therefore, has a `this' that comes in the form of the name of the composer. This name becomes attached to the flowing of the source. Therefore, the name of the composer is like a timbre stamp that is applied to the listening experience itself, and further, it is the severing style itself that can be named. The name of the composer becomes a synecdoche of the source itself, directly naming part of the source. This applies, quite literally in some cases, to the name of the program and the name of the programmer.\footnote{`Max' is named after the `father' of computer music Max Mathews, and MAX/MSP contains Miller Puckettes's initials. Friendly gestures, most probably, but also pointers to originary sources, sources of inspiration, historical references that contextualize computer music software within broader social and environmental structures.}

\paragraph{Dynamics}
Furthermore, the activity of the sorcerer lends itself to its signature. In other words, the manner in which the composer defines the music, from beginning to end, becomes the shape of the music, understanding `shape' or `form' as something that is at once behind and in front of the singularity of the listened music. It is behind, because it is the activity of sound sources ---speakers, musical instruments, or simply media in general---, the movement of air pressure. It is in front, because it filters the memory of the activity of sound sources. However, this composed shape and the singularity act together in the moment of listening. The question is, then, regarding the dynamics of this activity. Given that this activity happens during listening, what I addressing now is precisely how the shape of the music interacts with the listening itself. That is to say, the interaction between shape ---but also the form, the idea--- and the singularity of the listened. Interaction, here, refers to the shared activity that occurs `inside' listening itself, and it happens `inside' because of the severing that needed to occur prior ---or immediately at--- the resonating oscillation of air pressure. This is what I consider the moment of listening that is none other than listening to music. However, once this severing has occurred, and within its momentum, it is the internal dynamics that enter into play, and it is the shape of the music what begins to delineate the shape of the listened.

\paragraph{Masterwork}
Understood in this way, that is, the shape of the music as a force that produces a certain listening experience, therefore, the internal dynamics are already written. The singularity of the listened becomes (almost) one and the same with the shape of the music. `Almost,' because it is not that the listened brings no resistance to this ideal force. The singularity of the listened is resistance itself, like I have mentioned before in relation to the trace \see{human}. It acts as resistance itself, and its force is not enough to resist the command of the excellent work. This is the very presence of the masterwork, at work, the work of a master that requires the slave ---a slave that is not the rest of the works but the outshunned singularities that have been muted by its very own presence. `Almost,' in the hope that its work can be relativized, disarticulated, disentangled from the source of sources, brought down the stream to the place where singularities can resonate in endless forms of matter. However, the problem is now of a different sort. Even if resisting forces match those of the masterwork, then, like Derrida's concept of a paralysis of memory, we can encounter a paralysis of listening itself. This paralysis. This might (also) be what Szendy means, as well, by the cutting loose of the inattentive listener in modernity, but in a different way. It is not a paralysis caused by distraction, it is a paralysis caused by the very force that is needed to match the force of the master work. It is a paralysis that is directly called for from outside ---from the shape of the music itself---, one which prevents any further listening. This is what is called for by the work of the masterwork: pure ---and utterly ideal--- silence.

\paragraph{Architecture of Obedience}
Therefore, within these dynamics of work, what results is a function of the predicates, it is the architecture of obedience that is written in the form of a music work, with the one and only aim which is for it to `work.' Thus, the composer engaging with this dynamics of working out the work, of creating the structures, becomes the architect of the listened, the creator of a listening that of which he himself is the only chief. The sorcerer in charge of quenching a thirst that is only there because it is always already there, beforehand, instantiated with its own creation. The question now is how can this dynamics be approached once that I have recognized that it is there. How can composition continue, a composition that does not participate in this dynamics? A composition that is not a force? A composition that is not `really' or `entirely' a composition? A composition that does not impose its shape? A music work that is not a work but that still resonates within listening?


