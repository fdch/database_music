% How does thinking of database music affect the practice of music composition? One would have to begin at the origin, start at the beginning of composition, that is, at the moment of performative action that I am calling databasing. Identifying music composition with databasing would mean to interpret both practices under the scope of computer practices. That is, not just databasing, but computer-based databasing; and, likewise, not just composition, but computer-based composition. 

\paragraph{Imagining Composers}
In today's composition and databasing practices, the probabilities of a composer or a databaser working without computers are very slim. Databasing or composition outside the digital seems rather fictional. However, the very image of a `composer,' which traditionally stems from romantic standards, is already outside the world of computers. This image of composing can be painted as follows: the composer at work, quietly on a desk with pen and paper, transcribing, arranging, making parts, drawing line after line, dot after dot, notating instructions for the performance of an imagined music. Where is the computer in this image of composition? Certainly, placing a computer on this idyllic desk would be anachronic and obtrusive; anachronic, since the romantic quality of the scene would point to the fact that personal desktop computers were not available until late in the 20th century; obtrusive, in the sense that it would attempt against this composer, by interfering with the `ethereal' link between imagination and notation. This reification of the composer already precludes not only the digital, also the many technological devices that have entered music composition over the years, such as tape recorders, or electronics in general. These technological devices have redefined the composer in many ways. 

\paragraph{Composers and Technology}
Georgina Born's ethnography of \gls{ircam} \parencite{Bor95:Rat}, captured how the institutionalization of music composition and technology resulted in hierarchical structures of work dynamics, and how these were coated with false notions of collaboration. Inequalities of social, economical, and political status among technicians and composers within \gls{ircam} became privately evident. Knowing how to use computers and knowing how to compose comprised two irreconcilable poles in the institutional structure. For example, Born describes internal hierarchies such as `superuser' password knowledge, source code access, software licences, and, in some cases, she showed how these hierarchies reflected on internal privacy issues: ``workers concocted their various informal ways of protecting privacy and retaining secrecy: blocking the glass walls of their studies, working at night to prevent others knowing what they were doing or even whether they were working at all'' \parencite[272]{Bor95:Rat}. 

\paragraph{Playing with Shadows}
On the one hand, it is tempting to link this irreconciliation to the extreme reification of the name Pierre Boulez. The obscure dynamics behind this reification, however privately and secretly they were kept within the institution, can be nonetheless seen as the shadow of the more general specter of the music maker. Born's mysterious but telling anonymization of anyone but Boulez on her transcriptions might attest to this shadow. The music maker has been traditionally considered an outsider, marginalized by society, but simultaneously an integrator of society itself \parencite[12]{Att77:Noi}. On the other hand, this shadow might also be that of the computer itself, the structural presence of a fictional intelligence constructed upon first wave cybernetics. That is to say, precisely because the computer projects an insurmountable power that comes from its calculations, the human is inevitably bound to be subordinate, and with this subordination comes the subordination of the composer, and (perhaps) the end of music. This hyperbolic reaction would explain the need for privacy and secrecy of information, the undocumented ``oral culture'' of Born's \gls{ircam}, as well as the reversal of the human-computer subordination evidenced in the social strata of the institution. In any case, a composer without computers cannot be imagined today, but this is not due to the practice of composition itself. My argument here is that in any given situation, it is hardly possible to imagine a human without computers at all. This is what media studies has to teach us about the posthuman condition in which we hybridly live, where humans and technology, humans and nonhumans, unfold as interminably networked traces.

\paragraph{Composers Without Computers}
Composing with or without computers cannot be seen as poles on a continuum upon which the name of the composer writes and rewrites itself. Composing cannot be separated from computers, because the human cannot be separated from the non. At the risk of drawing a straw-man out of this computer-less composer, it is very unlikely in today's world to imagine a composer that has not googled `clarinet multiphonics' for more than a few YouTube tutorials on the topic. The same can be said for digitized music listening, which, in order to escape it, one has to go to great cult-like lengths to do so: going to instrumental performances, getting a vinyl record or a tape player, etc. To have a concert, therefore, a composer without computers today would need to whisper the score to the performers who would, in turn, play by ear. (`By ear', in the sense that they would need to play from memory, since no printed score would exist, for even if the composer wrote the parts, the score would have to be inscribed on a paper, and somewhere along paper networks there is at least one computer.) The composer should also whisper invitations to a few neighbors to be part of the audience. The composer should also demand no recordings whatsoever, while performing for an audience that has been kindly reminded not to bring their cellphones. Even then, the concert would need to take place on an amphitheater to avoid architectural networks, and \gls{autocad}; before the sun sets, to avoid electricity networks altogether while we are at it; away from cities, a car driving by would be unforgivable; so far away that we would, in fact, need to bring non-perishables for the pilgrimage, and even then, packaging networks or agriculture networks would be almost impossible to avoid. And this is precisely the point: in attempting to avoid it, the pilgrimage exists not in space, but in time, and thus it enters into the realm of fiction. The same can be applied to the overloaded case of a composer totally \textit{with} computers, that is, a computer composer, that would not need the human to write music. 

\paragraph{Databasing Without Computers}
The same applies to databasing itself. Removing computers altogether from databasing takes us to the world of libraries, encyclopedias, collectors, gatherers. Most important, it takes us to the place databasing occupies within society: to museums, but also to the dynamics of civilization, to church, put simply: to institutionalization itself. That is to say, it relates performance with the archontic, with the oedipal drive to re-place \see{archontic}, to an infinite return that the structure of the archive imposes upon us. So, if we imagine a computer-less census, we'd have to picture a gatherer of names walking around town, asking out loud for each person's name and place of residence. Getting rid of networks which might have computers ---as in the case of the above painted computer-less composer---, it is clear that the only suitable person for the job would be Irineo Funes \see{funeslude}, and the only possible storage medium would be his memory. Hopefully, the reader would consider this resort to hyperbolic fictions less as a means of justification of the hybrid condition of composition and databasing, and more as an absurd parenthesis that brings nothing of criticism to the ---still valid--- efforts of working `outside' the digital. These efforts are not questioned in regards of their validity, only in terms of their definition, which, for the purposes of this text, is understood as built upon organicist notions of the human: the human as the one, indivisible, complete whole. These notions are precisely those that help reify the image of the composer in its essentiality.