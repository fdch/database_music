\begin{quote}
	And the listening in question here is not that of a given listener, or of a category of listeners one has to take into account; it is rather structural listening in Adorno's sense ---or even, beyond Adorno, a \textit{listening without listener in which the work listens to itself}. \dots we hear an organicist concept of the work being strongly articulated [by Schoenberg] (where the work is a whole that doesn't allow any cuts) and a regime of listening whose ultimate, ideal aim is the absorption or resorption of the listener in the work. A listener who is somewhat distracted, inattentive, who would skip over a few tracks daydreaming ---\textit{such a listener could fall away like a dead limb}. \textit{Useless}. Bringing nothing to the great corpus of the work. This organicism, in the radical (or structural) tendency that Schoenberg gives it, forms the cornerstone of the construction of a modernist regime of listening. \im \parencite[127]{Sze08:Lis}
\end{quote}
In Peter Szendy's discussion on Schoenberg's modern organicism and what he calls ``the modernist regime of listening'' \parencite{Sze08:Lis}, `listening' and `work' collapse into each other, and the problem of the music work can be articulated. The image of the composer and the practice of composition can be understood differently, and by extension, databasing will be reconfigured as well, and the database as that which is a product (or work) of databasing will in turn be seen differently.

\paragraph{The Work Problem}
In what does this articulation of the problem of the music work consist of? First of all, why is it a problem? As Szendy suggests with the metaphor of the the self-amputation of the listener, we as the body of listeners (the listening body) would be severed. Put differently, listening itself would be delineated from outside itself. That is to say, with the presence of an object (music work) which, in its interest of perfecting, polishing, and thus giving itself a `finish,' would shape and reshape listening until an ideal listening was achieved. The work would work out listening as work. This is the point: the moment the work of music begins to act as `work' itself, its listening is worked out as well, not by the physicality of the waves in media, nor by the virtuality of perception, but by the concept of `work' itself. The ``modern regime of listening'' played this through and through, shaping its listeners into an idealized listener. In this sense, this listener displays a measured listening, tailored, developed into different degrees of listening like a \textit{gradus ad auscultare}, one existing beyond any psychoacoustic measuring.

\paragraph{Working Rules}
From the shaping of listening by the work, and from the working activity that is performed by the work itself, the presence of the work as an object can be thus traced. In other words, the work of the music work can be considered as the work of an agent in the composition network: the music-work-as-object and the listening-as-object become nodes. In the modernist regime, the hierarchy priorizes the ruler (work-node), and all that can be identified with listening-node is arrived at by subordinating the relations to the work-node, and by restricting the directionality of this relation to being \obj{work} $\rightarrow$ \obj{listening}. The only exception is, of course, the extremely cultivated case of composers, which revert the arrow. This exception, however, is not so much an exception, as it is the prescription of the rule and of ruling itself, since it is this reversal what enables the structure in the first place: the ruling of the exception. This is what occurs when the work begins to listen to itself. Which is radically different from the case of a \textit{listening} that listens to itself. Jean-Luc Nancy, in the foreword to Szendy's text, considers that ``music places us outside of ourselves,'' because ``\dots what listens to itself is not just what resounds in the self and what rebounds to the self: this same movement, and this very movement, places it outside of self and makes its rebound overflow. \parencite[xii]{Sze08:Lis}'' As I have explained before using Nancy's concept of the resonance of a return \see{resonance_of_a_return}, listening is an approach to the relationship in self. Implementing this relationship within the dynamics of listening and work, the previous graph can be revised as follows. `Work' and `listening' would exist as well in relation, with the difference now that it is a relation that exists in a permanent state of overload, redundancy, or excess: \obj{work} $\leftarrow\rightarrow$ \obj{listening}. 

\paragraph{A Space of Difference}
In thinking listening this way, the concept of the work is relieved of its duties, discharged, fired, it becomes unemployed. The regime of listening becomes a listening space, but a space not of equality: a space of difference. Within this space where difference resonates, the music work no longer `works', it `unworks' \see{inoperativity}. That is to say, the relations between the different resonating points in the composition network expose themselves in a state of suspension, or interruption, creating space with the space of their own incompleteness by the fractality of their fracture. Thus, inoperativity is creation, it is techne, but it is a creativity that is necessarily indefinite, incomplete: the moment it becomes a thing it begins to work in the realm of the `archi'; the moment it remains suspended upon its limit, it unworks in negation of the `archi'. One is tempted to place this inoperativity in utopia, in the very instance of the non-place itself, but then one would forget what is already `there', the fluid medium, as well as gravity itself, which was until recent studies, thought of as unrelated to sound.\footnote{In a recent study, sound itself proven to make (tiny) gravitational fields: ``We show that, in fact, sound waves do carry mass ---in particular, gravitational mass. This implies that a sound wave not only is affected by gravity but also generates a tiny gravitational field, an aspect not appreciated thus far'' \parencite{PhysRevLett.122.084501}.} One would be tempted, equally, to place this inoperativity outside temporality itself, but then one would forget forgetfulness itself. Inoperativity is within the resonating space of an always.

\paragraph{A Severed Work}
What constitutes, then, that moment when the music work becomes a work? How is it possible for the work to become a thing, for the object to become the ruler, for the regime to be built on the first place, if the resonating space is already an inoperative space, interrupted and suspended? I would like to revert Szendy's metaphor of the amputated limb, and propose that it is the music work itself what is amputated, what falls off, the moment that it becomes a finished thing. Thus, just like the modern inattentive listener falls in the uselessness of its excess, sound `outside' the work is cut from the work, fading out in the uselessness of its excess. Like the human in Kittler's digitally converged apocalypse, redundancy is out of the question, it is left at the gates of the majestic concert hall, with the rest of the (useless) humans: it is literally and conceptually placed outside architecture itself. The created work, in its essential nature of being a cohesive, coherent whole, separates itself from the world of mechanical waves, and forms the one and only work: the piece of music. A `piece' not because it is in itself incomplete, but because it is the piece of the whole of the work of the composer.

\paragraph{Absorbption}
I would like to add to this worldview another concept brought by Szendy, that of `absorption.' He claims that it is the absorption of the listener in the work what is the ultimate aim of this modern regime of listening. Not surprisingly, `absorption' is the key concept in Iannis Xenakis' narrative of the fours stages of degradation of Western Music's ``outside-time structures,'' article \textit{Towards a Metamusic} (1967): ``we can see a phenomenon of absorption of the ancient enharmonic by the diatonic. This must have taken place during the first centuries of Christianity, as part of the Church fathers' struggle against paganism and certain of its manifestations in the arts\dots'' Later, referring to larger structural groupings: ``this phenomenon of absorption is comparable to that of the scales (or modes) of the Renaissance by the major diatonic scale, which perpetuates the ancient syntonon diatonic\dots'' Finally, ``one can observe the phenomenon of the absorption of imperfect octaves by the perfect octave by virtue of the basic rules of consonance'' \parencite[189-190]{Xen92:For}. The final stage of this process of absorption and degradation comes with atonalism, which ``practically abandoned all outside-time structure'' \parencite[193]{Xen92:For}. However, Xenakis' narrative contextualizes his sieve theory, devised as a means to ``establish for the first time an axiomatic system, and to bring forth a formalization which will unify the ancient past, the present, and the future'' \parencite[182]{Xen92:For}. Further, Xenakis formulated this theory with computers in mind, that is, with its concrete application in computer programs, under the subtitle ``suprastructures'' \parencite[200]{Xen92:For}. Therefore, considering the organicity of the work in Xenakis's overly modern gesture towards unity, metastructure, and mechanization, which was built in reaction to the ``poison that is discharged into our ears'' as he witnessed the ``industrialization of music [that] already floods our ears in many public places, shops, radio, TV, and airlines, the world over'' \parencite[200]{Xen92:For}, how can these notions of inoperativity be found together within architecture? How would this archi-techne be designed? Would it still be the product of the `archi'? Where is the poison that Xenakis the architect and  composer, was identifying with `industrialized' music? Is it not a product of modernity itself, as the working that listened to itself to the point of working out a Xenakis-listener-node to the extreme? 
