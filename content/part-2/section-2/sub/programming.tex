The common use of the word `database' within computer science came around the 1960s, when computers became available to companies throughout the United States of America. For the purpose of data processing, software developers began designing \gls{dbms}, which are still used in great demand by multiple contemporary companies. The computer's capability for data processing and storage is inherent in the constitution of database systems. In fields such as \gls{cac}, working with computers meant being part of a system. The human operator has been regarded, for example, as a co-operator \parencite{Mat63:The}. A further approach understands humans operating with computers as another component of complex systems \parencite{Vag01:Som}. In this section, I describe the different levels of database systems as a tree \fsee{dbtree}, starting from basic data structures to more elaborate database systems, and then present a brief history of how databases were designed.

\img{dbtree}{0.4}{
	A very simple sketch of a tree representing the database tree of computer evolution
}{A Database Tree}

\paragraph{Soil}
The tree is built on different interpretations of the Von Neumann architecture. That is to say, while this architecture went through several optimizations over the years, its three central aspects remained. Therefore, despite the fact that different industry standards for hardware construction resulted in different kinds of operating systems, the core elements of the architecture remained the same: memory (for data and program/code), central processing unit, and input/output interfaces.

\paragraph{Roots}
The below-the-soil level is accessed through machine and assembly code, which constitutes the core of low-level programming languages and are, to a certain extent, humanly un-readable: the world of bits. Above the soil, readability by humans is the main feature.

\paragraph{Macros}
\label{portability}
The database tree metaphor relates to the concept of portability. The database tree only takes the form of a tree once it is instantiated as a software and it is run. That is to say, the database tree unfolds every time it is opened, and in this unfolding it emerges the possibility of dynamically adapting to different soils. This is what is known in the programming world as defining conditions or macros. With these definitions, their programs can compile with different compilers, across a variety of hardwares and operating systems. Therefore, these database trees have as their main feature the capacity to unfold their roots in different directions upon demand. 

\paragraph{Trunk}
The trunk of the tree is composed of data types and structures that provide flow between stored (underground) data and the above-ground components. Programming languages handle data types differently, but in essence, data types and structures are usually built in layers going from the lowest (close to roots) to highest levels.

\paragraph{Branches}
These language layers, after they reach a certain level of complexity, begin to form boughs or limbs that, while being separated from each other, are linked to the same trunk and roots. I consider branches to be programs with text-based interfaces such as Bash, C, C++, python, Java, etc. Their feature is their generic functionality.

\paragraph{Twigs}
More complex programs built on top of branches, such as Pure Data, Supercollider, R, octave, Processing, OpenFrameworks etc., are dedicated for a narrower scope of tasks. Their feature is their level of specialization for the task at hand: sound synthesis, statistics, visuals, etc. They might be more application-specific. In general, these programs are commonly considered layers on top of other languages, libraries, or software frameworks.

\paragraph{Leaves}
User interfaces (or GUIs) are the leaves of the tree. I relate the photosynthetic quality of leaves with user input/output interaction. Despite their simple, user-friendly appearance, software leaves are highly complex systems such as multimedia editors (Adobe Creative Suite or Microsoft Office), Internet browsers, mobile apps, etc. A particular kind of leave is the \gls{dbms}, generally used in businesses for data processing and editing, for example: \gls{mysql}, \gls{postgresql}, \gls{nosql}, \gls{couchdb} and \gls{mongodb}.

\paragraph{Networks}
An important feature of database trees is their network capabilities. Networks can be established by connecting leaves, branches, or roots with each other, both within the same tree and with other trees. For example, software can establish a network between its graphical interface and its core program ---as is the case with Pure Data, for example. Another example would be the way in which \gls{dbms}s interact with data: the \gls{mysql} database model allows the user to load a data set in working memory, and establishes a connection between the opened memory and the input/output mechanisms. Networks of trees are data streams running by way of an \gls{ip} and a client-server type of relation. Cloud storage services such as Google Drive, ICloud, OneDrive, and Dropbox are used as a networked way to store and share data. One tree can serve as data storage and processing repository, and other client trees can connect to the server tree and request data or processing of data from it. This is the essence of the internet and all the communication services that it enables, such as email services, social networking sites, and multi-user collaboration platforms like Github. This allows software like Pure Data and MySQL to have their respective core program and data sets in one computer, and their interfaces on a different one.

\paragraph{Clouds}
Combining networked databases with computer clusters forms what is known as cloud computing. For example, most universities provide clusters for data processing ---e.g., NYU's Prince cluster--- that can be accessed from remote locations. These clusters are massive server architectures made out of multiple processing and memory units joined together. These architectures began developing in the 1990s, coining terms like data mining \parencite{DBLP:journals/corr/abs-1109-1145}, data warehouses, data repositories \parencite{ilprints81}.
