
% EH:
% 
% Why does Irineo's isolation preclude a replaying of any infinitesimal
% moment? Your story is poetic and powerfully metaphoric such that I get
% your point without fully understanding how the story demonstrates it
% alas. Thanks!

\begin{quote}
	I suspect, nevertheless, that he was not very capable of thought. To think is to forget differences\dots \parencite[2]{Bor42:Fun}
\end{quote} % 4/7/2018 10:31:06

The importance of memory ---and forgetfulness--- can be represented by Jorge Luis Borges's famous 1942 short story, \textit{Funes, the memorious} \parencite{Bor42:Fun}. Due to an unfortunate accident, the young Irineo Funes was ---``blessed or cursed'' as Hayles points out \parencite[156]{Hay93:The}--- with an ability to ``remember every sensation and thought in all its particularity and uniqueness '' \parencite{Hay93:The}. A blessing, since a capacity to remember with great detail is certainly a virtue and a useful resource for life in general; a curse, because he was unable to forget and, as a consequence, he was unable to think, to dream, to imagine. Throughout the years, he became condemned to absolute memory, and so to its consequence, insomnia:\footnote{In the prologue to \textit{Ficciones}, Borges writes that this story is a long metaphor of insomnia: ``Una larga metáfora del insomnio'' \parencite{Ovi19:Mem}.} he was secluded in a dark and enclosed space so as not to perceive the world. Hayles focuses on one aspect of the story, namely, the fact that Funes invented ---and begun performing--- the infinite task of naming all integers, that is, of giving a unique name ---and sometimes, last name--- to each number without any sequential reference. According to how Hayles describes it, by carrying out his number scheme, Funes epitomizes the impossibilities that disembodiment brings forth. As Hayles writes, ``if embodiment could be articulated separately from the body \dots it would be like Funes's numbers, \textit{a froth of discrete utterances registering the continuous and infinite play of difference}'' \im \parencite[156-159]{Hay93:The}. The point that Hayles touches upon can be seen as the limits and fragility of embodied memory, as well as the need to forget, in opposition to an embodiment `outside' the body (disembodiment), that would require no need to forget. We will see how the difference between forgetting and erasing relates to the database. In that Manovichian world which ``appears to us as an endless and unstructured collection of images, texts, and other data records'' \parencite[219]{Man01:The}, this idea would be perfectly viable. Indeed, data banks have already been growing exponentially much in the same way as Funes' memory. This capability of accumulation without the need for erasure is enabled by the database structure inherent in computers. However, the distinction that Hayles presents is crucial: data is not information because information needs to be embodied, and with that embodiment comes the need to forget. In sum, on one hand, a disembodied data bank can have all the uniqueness and difference that is available by the sum of all cloud computing and storage to date; however, on the other hand, embodied memory is available by the human capacity to forget.

Matías Borg Oviedo \parencite{Ovi19:Mem} relates this incapacity for thought to the negation of narrativity itself, thus finding in the image of Funes a hyperbole for contemporary subjectivity \parencite[5]{Ovi19:Mem}, where there is no room for narration, only data accumulation. In this sense, narrativity can be seen as that which resides in the threshold between knowledge and storage. I believe this distinction stems precisely from the difference between information and data. The process of information, of giving form, requires a certain temporality that is not that of the perceptually immediate and extremely operative zero-time of the \gls{cpu}. Within the zero-time of computer operations (within a millisecond) there simply is no \textit{time} for narration, only for addition or increment. Narrative is temporal ---happening as a historical process--- and algorithms are atemporal ---operating in an constant now. Since neither data structures nor algorithms operate outside the confines of the millisecond, they can't spare time to think and, likewise, they can't forget to count. Counting is all they do, so they cannot tell stories: the difference in the same spanish word \textit{contar números} [to count numbers] and \textit{contar historias} [to tell stories]; or, ``if a German pun may be allowed: \textit{zählen} (counting) instead of \textit{erzählen} (narrating)'' \textcite[128]{Ern13:Dig}. Therefore, Funes' accumulative memory represents the overflow of the now, the totally blinding transparency of the world, and an absolute memory that precludes narration. Because he accumulates data of the world in its totality, he does not have time to think: ``to think is to forget differences'' reads the quote above. The only one in Borges' story who actually thinks is the narrator, to which we can add: Funes could not have told this story himself in first person narrative.\footnote{It is worth noting how Oviedo finds in Funes a premonitory `antithesis of the writer' himself: the latter (blind) Borges could find himself immersed in a constant flow of narrativity \parencite{Ovi19:Mem}.} Thus, we can ask ourselves if this hyperbolic `light' of the Funesian `absolute memory' is not a premonitory figure of the database.\footnote{In this sense, Funes is a bit like an Oracle ---pun intended with the other \gls{oracle}---, with absolute knowledge of the past and the future, but with no time to think\dots} Because of this antithetical condition between databases and narrative, Manovich proposed that the database became a form on its own, in opposition to narration. To the extent that `database form' as a category that identifies art using databases, this can be considered accurate. However, considering this distinction between embodied and disembodied memory that resides in the ability to narrate, databases are inherently deprived of narration. Making art with databases is making them dance (see below). Therefore, to what extent is `database music' in itself a contradiction if we consider `music' to be a form of writing? I will leave this discussion for a later chapter \see{section-4}.

This Funesian database can also be understood in relation to what Gayatri Chakravorty Spivak \parencite{Der76:Of} writes about forgetfulness. She notes in Nietzsche the `joyful' and `affirmative' activity that constitutes forgetfulness as being twofold. On the one hand, this activity is a ``limitation that protects the human being from the blinding light of an absolute historical memory,'' and on the other, it is ``to avoid falling into the trap of `historical knowledge''' \parencite[xxxi]{Der76:Of}. The `historical' here is an ``unquestioned villain,'' which takes two forms: one ``academic and preservative,'' the other ``philosophical and destructive'' \parencite[xxxi]{Der76:Of}. For Nietzsche, as Spivak notes, forgetfulness is a choice that comes as a solution: an ``antidote'' to the ``historical fever,'' or the ``unhistorical,'' that is, ``the power, the art of forgetting\dots'' \parencite[xxxi]{Der76:Of}. I propose an imaginary experiment that would add some noise to Borges' short story, however utterly fantastic his writing was. One thing that can be read from the story is that, in order to seclude himself from perceiving the world, or better, in order to forget the world altogether, Irineo stayed in the darkness of his room. This is how he cancelled light, a quite powerful stimulus if memory-space is to be optimized for the purpose of, say, getting some sleep. However, there is little to no mention of the sonic environment in which Funes was embedded ---somewhere in the outskirts of the quiet Uruguayan city of Fray Bentos. In fact, the only sonic references are focused on the narrator's perspective, referring to Funes' high-pitched and ---due to his being in the darkness--- acousmatic voice. To a certain extent, we might think of Funes' high-pitched (at least this is how the narrator heard it) voice as a hint to the highlighted overtones that links Borges' ``long metaphor of insomnia'' with ``the `laughter' of [Nietzsche's] Over-man [that] will not be a `memorial or\dots guard of the\dots form of the house and the truth\dots He will dance, outside of the house, this\dots active forgetfulness'' \parencite[xxxii]{Der76:Of}. Nonetheless, Funes is deprived of this forgetfulness, and thus cannot go outside, let alone laugh or dance.\footnote{The acousmatic quality of Funes' voice will not be touched here, but it is indeed a good point of departure for another time.} By locking himself inside a room he would have managed to attenuate sound waves coming in from outside. Notwithstanding his isolation ---a house arrest---, sound waves are actually very difficult to cancel.

\sout{(} It is interesting to compare Funes' attempt to filter out the world with John Cage's quest for silence. An interesting experiment would have been to have John Cage take Irineo to an anechoic chamber and ask him what he can remember then. From Cage's own experience, we can guess that Funes would effectively remember his own sounding body. Kim \textcite{Cas00:The} writes that ``[Cage's] experience in an anechoic chamber at Harvard University prior to composing 4'33'' shattered the belief that silence was obtainable and revealed that the state of `nothing' was a condition filled with everything we filtered out'' \parencite[14]{Cas00:The}. It is interesting to place an 80 year-old Irineo in David Tudor's premiere at Maverick Concert Hall in Woodstock, NY, infinitely listening to 4'33'' \sout{)}

It is very unlikely ---but nonetheless possible--- that Borges was aware of American acoustician Leo Beranek's research for the US Army during World War II, that is, when the first anechoic chamber was built.\footnote{\url{https://en.wikipedia.org/wiki/Leo_Beranek}} Furthermore, even if he managed to isolate himself completely from the world by cancelling perception altogether, Funes would have been with his memories (he was not deprived of \textit{anamnesis}, the ability to remember), which were not discrete, but continuous iterations of the world he had accumulated over the years. What this means is that all the sounds he had listened to would be available to his imagination. As far as we can learn from the narrator, while \textit{smell} is referenced to in the story, \textit{sound} was nonetheless out of Funes' concerns. Therefore, one thing we can ask ourselves is how the world would sound for Irineo Funes? The task is not difficult to imagine: the world would be inscribed in poor Irineo's memory in such an infinitely continuous way that each fraction of wave oscillation would be different, unique, leaving no space for repetition of any kind. For example, one of Irineo's concerns was to reduce the amount of memories on a single day, which he downsized to about seventy thousand\dots. What would be Funes' sample rate? What frequencies could he be able to synthesize? All sounds (and all that can be registered) would be listened to completely, with every infinitesimal oscillation of a wave pointing to the most utterly complete scope of imaginable references. A complete state of listening. In fact, we might not be able to call it listening any more. Not even signal processing. An infinitesimal incorporation of sound is unthinkable. Within such total listening there would be no possibility for thought, no processing of any kind, and no synthesis: only infinite accumulation and storage. On the one hand, no matter how accurate our embodied listening might be, we are bound to miss some motion, some waves would pass through us and we would be busy forgetting to register them. On the other hand, if such a recording were humanly possible, thinking would cease to be so. This can be thought of as the intersection of the finite with the infinite: while Funes' nonhuman memory corresponds to a dynamics of the infinite, his human body is quite human. Funes is not deprived of this finitude, and existed (fantastically) on a ghostly liminality. This liminality grows more and more evidently throughout the story, hand in hand with the cumulative growth of the Funesian database, all the way until the end, in a sort of Moore's Law of data congestion, saturating completely in an utterly human pulmonary congestion \parencite{Ovi19:Mem}.



% Studies in cognitive psychology \parencite{Wes08:How} consider forgetting to be an adaptive and functional activity. Nonetheless, it is considered as a ``failure'' of certain search processes on account of an inability to recall information from memory. This failure is a mystery when it comes to human memory ``it is unclear what really happens to old, disused or deliberately ignored memory traces ---they might be retrievable, they might be lost, but no-one can tell'' \parencite[292]{Wes08:How}. Nevertheless, the mystery vanishes the moment memory becomes externalized in writing or in archives, because it becomes extinguishable.



