\img{one-to-many}{0.3}{
	Top: syntagm, paradigm, and their relation.
	Bottom: narrative, database, and their reversed relation.
}{Syntagm and Paradigm reversal}

In order to reveal the extent to which the presence of the database has a radical effect on narrative, however, Manovich reverses the semiotic theory of syntagm and paradigm that governed much of the 20th century \parencite[231]{Man01:The}. Manovich describes the paradigm as a relation subjected to substitution, and the syntagm as a relation subjected to combination. For example, from the entire set of words in a language (the paradigm) a speaker constructs a speech (the syntagm): the paradigm is implicit (absent) and the syntagm is explicit (present). The relation between these two planes (of the paradigmatic and the syntagmatic) is established by the dependence of the latter on the former: ``the two planes are linked in such a way that the syntagm cannot `progress' except by calling successively on new units taken from the associative plane [i.e., the paradigm]'' \parencite[59]{Bar68:Ele}. Barthes gave several examples with different `systems,' one of which was the food system. Put simply, the dish (as a set of choices with which to make a dish) is the paradigm and what you are eating is the syntagm \parencite[63]{Bar68:Ele}. However, when one looks at the restaurant's `menu', one can glance at both planes simultaneously: ``[the menu] actualizes both planes: the horizontal reading of the entrées, for instance, corresponds to the [paradigm], the vertical reading of the menu corresponds to the syntagm'' \parencite[63]{Bar68:Ele}. 

A software menu would come to represent both planes as well: the paradigm is the set of all possible actions the user might make within the specific context of the menu; the syntagm is the actual sequence of clicks that the user makes. Manovich points to a reversal of these planes \fsee{one-to-many}. Given the concrete presence of the database (the options on the software menu), and given the hyperlinked quality of the user interface (those options are clickable links), the database becomes explicit (present) and the sequence of clicks becomes implicit (absent, `dematerialised'). Since Barthes' project was focused mostly on the distinction between speech (as syntagm) and language (as paradigm), this is how the database was first understood in art. On the one hand, narrative (speech) is the syntagm: it is the trajectory through the navigational space of a database, and since this narrative is abstracted, its concreteness comes achieved by the interface, and interface and narrative depend on each other. The result is an interlocking of narrative and interface that results on the conception of the interface-as-artwork. On the other hand, the database is the paradigm (language), since it represents the set of elements to be selected by the user.

For example, consider the case of the typical timeline-view of a video editor.\footnote{This example was used by Manovich in the late 1990s, and it is still valid today with most multimedia editing software.} Normally, the user creates a session and \textit{imports} files to working memory, creating a database of files ---video files, in this case. Once this database is in working memory, the user places on a timeline the videos, cutting, and processing them at will, until an \textit{export} or a \textit{render} is made.\footnote{`Import,' `export,' and `render,' refer to processes that read from or write to the computer's disk.} The timeline where the user places the videos is a visualization of the ``set of links'' to those files on disk, and not their actual data. Such a timeline is an editable graph that allows the user to place in time the pointers to the elements on the database. This is what Manovich means by ``a set of links,'' because the user is not handling the files themselves ---as would be the case with an analog video editor, where the user cuts and pastes the magnetic tape---, but the extremely abstract concept of memory pointers. 

I consider this reversal to be valid as a shift from one-to-many to many-to-one \fsee{one-to-many}. The question of the materiality of the database and of the pointers depends on the materiality of data. Links or pointers have, for Manovich, a different (absent-like) status in relation to stored memory itself. This is because of a distinction between pointers and data on the basis of their use: pointers are of a different nature since they do not store data directly. Instead, they refer to the address in memory where a specific stored data begins. However, the mutual binary condition of pointers and data, and the fact that they are both stored in the same memory, reveal Manovich's reversal to be somewhat misleading. Pointers are just another data type, however functionally different they may be. If one understands them as moving bodies, it follows that pointers are `lighter' and travel much faster than other data types, which are `heavier' and slower to move. However, data types are not moving bodies at all, and thinking of them as such interlocks us in a semiotic trap: accepting this reversal means accepting a certain materiality of data, which is different from a certain materiality of information.