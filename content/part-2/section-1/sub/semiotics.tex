\img{one-to-many}{0.3}{
	Top: syntagm, paradigm, and their relation.
	Bottom: narrative, database, and their reversed relation.
}{Syntagm and Paradigm reversal}

In order to reveal the extent to which the presence of the database has a radical effect on narrative, however, Manovich reverses the semiotic theory of syntagm and paradigm that governed the first half of the 20th century \parencite[231]{Man01:The}. Quoting Roland Barthes' reading of Ferdinand de Saussure in \citetitle{Bar68:Ele}, Manovich describes the paradigm as a relation subjected to substitution ---because it depends on associations---, and the syntagm as a relation subjected to combination ---because it is an instantiation of concrete elements. For example, from the entire set of words in a language (the paradigm) the speaker constructs sentences (the syntagm): the paradigm is implicit (absent) and the syntagm is explicit (present). The relation between these two planes (of the paradigmatic and the syntagmatic) is established by the dependence of the latter on the former: ``the two planes are linked in such a way that the syntagm cannot `progress' except by calling successively on new units taken from the associative plane [i.e., the paradigm]'' \parencite[59]{Bar68:Ele}. Barthes gave several examples with different ``systems,'' one of which was the ``food system,'' which I will borrow in what follows. All the elements that compose a dish, for example, the ``set of foodstuffs which have affinities or differences within which one chooses a dish in view of a certain meaning'' comes to delimit the paradigm. However, the ``sequence of dishes chosen during a meal,'' or simply, what you are eating as you are eating it in a restaurant, comes to represent the syntagm \parencite[63]{Bar68:Ele}. However, when one looks at the restaurant's `menu', one can glance at both planes simultaneously: ``[the menu] actualizes both planes: the horizontal reading of the entrées, for instance, corresponds to the system [i.e., paradigm], the vertical reading of the menu corresponds to the syntagm'' \parencite[63]{Bar68:Ele}. A software menu, for instance, would come to represent both planes as well: the paradigm is the set of all possible actions the user might make within the specific context of the menu; the syntagm is the actual sequence of clicks that the user makes. 

Barthes' reading of Saussure is maintained in Manovich's description of the database. On the one hand, narrative is the syntagm since, at least in Manovich's rendition of narrative in the visual art world and the gaming world, it is the trajectory through the navigational space of a database. Furthermore, since this narrative is achieved by the interface, interface and narrative depend on each other: narrative thusly interlocks with the interface itself, and results on the conception of the interface-as-artwork. On the other hand, the database is the paradigm, since it represents the set of elements to be selected by the user. Materially, however, Manovich points to a reversal of these planes. Given the material presence of the database (i.e., the stored data), and given the hyperlinked quality of the user interface, the database becomes explicit (present) and narrative becomes implicit (absent, dematerialised).

For example, consider the case of the typical timeline-view of a video editor.\footnote{This example was used by Manovich in the late 1990s, and it is still valid today with most multimedia editing software.} Normally, the user creates a session and \textit{imports} files to working memory, creating a database of files ---video files, in this case. Once this database is in working memory, the user places on a timeline the videos, cutting, and processing them at will, until a result is desired, and an \textit{export} or a \textit{render} is made.\footnote{`Import,' `export,' and `render,' refer to processes that read from or write to the computer's disk.} The timeline where the user places the videos is a visualization of the set of links to the files; an editable graph that allows the user to locate in time the pointers to the elements on the database. This is what Manovich means by ``a set of links,'' because the user is not handling the files themselves ---as would be the case with an analog video editor, where the user cuts and pastes the magnetic tape---, but the extremely abstract concept of memory pointers. 

I consider this reversal to be valid, only on a certain quality of the relation itself, that is, as a shift from one-to-many to many-to-one \fsee{one-to-many}. The question of the materiality of the database and of the pointers depends on the materiality of data. Links or pointers have, for Manovich, a different (absent-like) status in relation to stored memory itself. This is because of a distinction between pointers and data on the basis of their use: pointers are of a different nature since they do not store data directly. Instead, they refer to the address in memory where a specific stored data begins. However, the mutual binary condition of pointers and data, and the fact that they are both stored in the same memory, reveal Manovich's reversal to be grounded on an equivocation. Pointers are, however functionally different, another data type. This fact comes from the Von Neumann architecture on which computers are constructed \see{programming}. If one understands them as moving bodies, it follows that pointers are `lighter' and travel much faster than other data types, which are `heavier' and slower to move. However, data types are not moving bodies at all, and thinking of them as such interlocks us in a semiotic trap: accepting this reversal means accepting the materiality of data.