\begin{quote}
	The activity in the receiver's internal structure generates symbolic structures that serve to frame stimuli and thus to \textit{in-form} information: this activity converts regularities in the flux of stimuli into \textit{patterns} of information. \parencite[76]{Han02:Cin}
\end{quote}

The activity of framing, according to Hansen, must be differentiated from that of observation. In this way, ``information remains meaningless in the absence of a (human) framer,'' \parencite[77]{Han02:Cin} and framing becomes a resonance of the (bodily) singularity of the receiver. Quoting MacKay's \textit{Information, Mechanism, Meaning} (1969), the meaning of a message

\begin{quote}
	\dots can be fully represented only in terms of the full basic-symbol complex defined by all the elementary responses evoked. These may include visceral responses and hormonal secretions and what have you\dots an organism probably includes in its elementary conceptual alphabet (its catalogue of basic symbols) all the elementary internal acts of response to the environment which have acquired a sufficiently high probabilistic status, and not merely those for which verbal projections have been found. \parencite[78]{Han02:Cin}
\end{quote}

It is with this conception of framing that Hansen describes precisely that information always requires a frame:

\begin{quote}
	\dots this framing function is ultimately correlated with the meaning-constituting and actualizing capacity of (human) embodiment\dots the digital image, precisely because it explodes the (cinematic) frame, can be said to expose the dependence of this frame (and all other media-supported or technically embodied frames) on the framing activity of the human organism. \parencite[89-90]{Han02:Cin}
\end{quote}

Therefore, in the context of Kittler's digital convergence, framing prevents the human from being rendered a dependent variable. On the contrary, the framing function of the human body is the possibility condition for the digital to become information. The frame, as Hansen describes, is the human body filtering images from the world, and creating a virtual image that gives form to data. The frame needs to happen as a relation, and thus, it is the temporal instantiation of a process. What would a human body without this framing and filtering capability look like? How would this temporality of the process of information be understood? 

In the following interlude I take the concept of an absolute (human) memory to be at an intersection between disembodied theories of information and, precisely, the concept of an embodied memory. The aim is to introduce and differentiate between human and nonhuman in terms of memory and databases. The wonder, admiration, but also the fear and mystery that a notion of embodied memory awakens can speak for the uncanny feeling that occurs whenever databases are involved, and thus can speak for a certain agency of the database. I understand this feeling as what accounts for the aesthetic experience of database music. 





