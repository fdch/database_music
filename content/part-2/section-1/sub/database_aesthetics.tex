
The Internet is a place of unlimited access, it is a database in continuous and exponential growth, that reconfigures the grounds on which art has traditionally been built on. One of these grounds is the role of the author. The collaborative approach in the work of media artist Sharon Daniel \parencite{Dan07:The}, is an example of a different kind of authorial reconfiguration. Daniel raises questions about authority and politics in collaborative art by means of a social model based on the concepts of emergence and \textit{cellular automata}. Cellular automata are systems that reveal emergent (global) behavior from (local) rules. That is to say, each automaton changes states according to its surrounding neighborhood, resulting on an emergent behavior on the global level that precludes top-down behavior. Brought to the social plane, cellular automata result in an inverted hierarchic system: instead of a top-down authority, there exists an emergent, bottom-up behavior. Daniel thus removes her authorial role as artist granting participants a shared authority with the work itself at every stage. Differing from Klein's blurring of the authorial roles, Sharon's participants engage in performative actions that shape the outcome of the artwork in ways she could not anticipate. Therefore, authority is decentered, that is, it does not sediment itself in a single unity. 

% While Daniel's approach resonates with Bourriaud's \textit{Relational Aesthetics}, it differs in one particular way. Bouriaud's aeshtetics was criticized namely on the grounds of the reification of the curator. That is to say, the displacement of authority in the modern artwork, from its artist to the contingencies of context and participants, resulted in a reification of the curator himself.\footnote{I refer the reader to Claire Bishops criticism of Bourriaud's text in \parencite{Bis12:Ant}} Daniel's approach, however, 

% What the interface means in this context, however, is a way for the human to `request' some data out of a database. The interface itself may be of any shape, and nowadays there are conferences such as \gls{nime} dedicated to the creation of interfaces for music. A simple example of the relationship between framing and interface can be pictured as follows. By typing the command \lstinline|ls| or \lstinline|DIR| on a computer terminal, the user is returned a list of all items residing in the terminal's current directory. How can the framing function take place here? The returned list of files and directory names needs to be `figured out.' However, the simplicity of the input (and of the output) is matched by the difficulty of its interpretation. In order for the command to be meaningful, the user has to incorporate the image returned by the computer, so as to actively `get' the list as information, or to `make something out of' it.  Therefore, with this creative act of framing begins a relationship between the human and the computer, consisting on a translation of the computer's directory tree that is referenced to as a list into the user's memory. In this sense, the way files are stored inside the computer is given form in the human memory, that is, the database is subjected to databasing.

