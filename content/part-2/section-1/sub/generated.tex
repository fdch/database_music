\begin{quote}
	Data creators have to collect data and organize it, or create it from scratch. Texts need to be written, photographs need to be taken, video and audio need to be recorded. Or they need to be digitized from already existing media\dots Once digitized, the data has to be cleaned up, organized, and indexed. \parencite[224]{Man01:The}
\end{quote}

Despite Manovich's technologically determined considerations of the database as form, he notes a fundamental aspect of the use of the database when he expresses that data need to be generated \parencite[224]{Man01:The}. In this sense, he begins to describe the actions that need to be performed around data, or what I call databasing \see{databasing}, which connotes the use of databases in terms of their performativity. He even goes further and proposes that this activity has become a ``new cultural algorithm,'' \parencite[225]{Man01:The} which I reinterpret here as a single subroutine with one argument for input: the world \lsee{manovich}. Following this line of thought, artist Victoria Vesna \parencite{Ves07:See} argues that creating a memory bank is a means of testifying to our existence \parencite[25]{Ves07:See}. 

\begin{flushleft}
\small
\begin{lstlisting}[caption={Manovich's cultural algorithm as a pseudocode routine with the world as argument (i.e, as input). The world is then mediatized, stored in some media (film, tape), then digitized into data, then structured as a database. This routine returns the database, which is, henceforth, the world re-presented as database.},captionpos=b,label={lst:manovich}]
function new_cultural_algorithm(world) 
{
	database = data = media = world
	return database
}
\end{lstlisting}
\end{flushleft}

While Manovich calls for an ``info-aesthetics'' \parencite[217]{Man01:The}, as well as a poetics, and ethics of the database, neither Manovich nor the following generation of media artists and theorists could carry out an exhaustive account of an aesthetics of the database. Several authors continue to abide by Manovich's claim that the aesthetics of the database, or the database as form, is a symptom of the uncritical use of database logic throughout the visual art world of the 1990s. It is in hindsight that his argument can be understood as grounded on the same disembodied constructions that prevent him from including human agency in his account.

% I agree that Manovich's language of new media is still valid, only to the extent that it is taken in hindsight as the outcome of the first wave of new media theorists. Galloway claims that Manovich's ``abdication of the political'' in favor of a semiological incursion into software is rooted on a post-communist era intellectual danger on Manovich's end when it comes to form, poetics and aesthetics. 

