Despite Manovich's technologically determined considerations of the database as form, he notes a fundamental aspect of the use of the database when he expresses that data need to be collected, generated, organized, created, etc: ``Texts need to be written, photographs need to be taken, video and audio need to be recorded. Or they need to be digitized [and then] cleaned up, organized, and indexed'' \parencite[224]{Man01:The}. In this sense, he begins to describe the actions that need to be performed around data, what I call databasing \see{databasing}, which connotes the use of databases in terms of their performativity \see{chapter:section-6}. He even goes further and proposes that this activity has become a ``new cultural algorithm,'' \parencite[225]{Man01:The} \fsee{manworld}.
% Following this line of thought, artist Victoria Vesna \parencite{Ves07:See} argues that creating a memory bank is a means of testifying to our existence \parencite[25]{Ves07:See}. 

\img{manworld}{0.6}{
	The world is mediatized, stored in some media (film, tape), then digitized into data, then structured into a database. The result is the world represented by the database.
}{Manovich's cultural algorithm}

While Manovich calls for an ``info-aesthetics'' \parencite[217]{Man01:The}, as well as a poetics, and ethics of the database, neither Manovich nor the following generation of media artists and theorists could carry out an exhaustive account of an aesthetics of the database. Several authors continue to abide by Manovich's claim that the aesthetics of the database, or the database as form, is a symptom of the uncritical use of database logic throughout the visual art world of the 1990s. It is in hindsight that his argument can be understood as grounded on the same disembodied constructions that prevent him from including human agency in his account. In any case, his contribution to the literature on the role of databases in new media have led us to this point of inflexion, in which we can consider different points of view regarding the topic of databases. My revision of this algorithm will come later in this dissertation \fsee{unwork}. In what follows, I will explore the more technical aspects of databasing, in order to trace a connection between the literature on sound-based computer practices with that of the development of databases over the years. In this way, I bring the discussion of databases into the sonic sphere.
