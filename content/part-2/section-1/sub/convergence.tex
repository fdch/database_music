\begin{quote}
	A mere `byproduct' of pleasure, entertainment is a hangover from the media epoch: a function that caters to our (\textit{soon to become obsolescent}) need for imaginary materialization through technology [which, in turn,] serves as a diversion to keep us ignorant of the operative level at which information, and hence reality, is programmed. \im \parencite[59]{Han02:Cin}
\end{quote}

I find in Manovich a silent allegiance to german media theorist Friedrich Kittler's concept of digital convergence. Digital convergence entails that the bodily resonance of media becomes obsolete in the face of absolute digital information storage. Thusly, it turns the human into a ``dependent variable'' \parencite[59]{Han02:Cin}. In the case of physical media, the human body was, for Kittler, directly shaped by media, and the limit of this `shaping' was set by the bodily limits of perception. The body became a by-product of media. However, in the age of digital convergence, of an ``absolute system of information'' \parencite[63]{Han02:Cin}, media remove this bodily limit of perception, making the human body a residual product. The body, then, becomes a residue of digital industries.

For example, the extents of this residual aspect of the human can be seen in writer Norman Klein's considerations of the author \parencite{Kle07:Wai}. Following Manovich's interface-as-artwork, Klein argues that since the reader gets immersed in data, ``[she] evolves pleasantly into the author'' \parencite[93]{Kle07:Wai}. Because the reader participates in the narrative, the result is a reconfigured concept of shared authorship. However, Klein continutes ``instead of an ending, the reader imagines herself about to start writing'' \parencite[93]{Kle07:Wai}. This surprising twist in Klein's consideration adds another layer of complexity, namely, the categorical difference between `writing' and `not-yet-writing.' In Klein's sense, narrative constitutes a promise of authority that equally blurs the roles of the writer and of the reader. Most importantly, this blurred authority is seen as a reflection of control and subordination of the human. In this view, the potentiality of authority arising from the trajectory through the database belongs neither to the reader nor to the writer: it is appropriated by the database. The roles of the reader and the writer fade into each other and vanish, allowing the database to be a dominant middle term. In other words, human agency is absorbed into a shadow, making the database the sole agent to which the human is subjected. In Klein's own words, the human is a `slave' to data, and as a consequence the human is economically `colonized' and `psychologically invaded' by the evolving force of computers, information, or technology in general \parencite[86-8]{Kle07:Wai}. Authority converges, too, in the age of digital convergence.

Media theorist Mark Poster defines technological determinism as the ``anxiety at the possibility of [the human mind's] diminution should these external [technological] objects rise up and threaten it'' \parencite[X]{Pos11:Int}. In other words, the fear or anxiety that the human is ultimately subjected to the power of technology. Understanding new media as digital convergence leads to reading the `new' in new media as the `digital.' In reaction to the anxieties that this convergence brings, and from an embodied approach where databases have an aesthetic agency in resonance with the human, in what follows I propose to shift the focus from narrative (interface) to performance (databasing), and to reconfigure the shadow of the database as a hybrid skin exposing the human and the non. 

