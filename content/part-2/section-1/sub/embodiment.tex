As I describe above, Manovich arrives at this notion of the interface-as-artwork by opposing database and narrative on the semiotic grounds of the reversal of the paradigm and syntagm. In turn, media theorist Mark B. N. Hansen \parencite{Han04:New} notes that the interface-as-artwork constitutes a disembodied ``image-interface'' to information in which the process of information itself (in-formation; giving form) is overlooked. Hansen locates the source of this disembodied conception in Manovich's implicit ---but nonetheless evident--- premise of the overarching dominance of cinema in contemporary culture, which results in a ``disturbing linearity [with] hints of technical determinism'' \parencite[36]{Han04:New}.

For example, Manovich argues that standardization processes originating from the Industrial Revolution have shaped how cinema is produced and received. Attuned to the perceptual limits of the body, the standardization of resolution can be seen (image dimensions, frames per second, and aspect ratio) and heard (audio bit depth, sampling rate, and number of channels). In this sense, the moviegoer and by extension, the listener became industrial by-products, determined by the massively produced electronic devices used for recording and playing. As I have described with Kittler's technological determinism, the devices driven by industrial forces, therefore shaped the body, and as an extension, the aesthetics of cinema. 

For Manovich, due to the internal role of the database, the logic of new media is no longer that of the factory but that of the interface. Through the interface to a database, the user is given access to multiplicities of narrative, and thusly, to endless information. The user is granted the power of the database, making in Manovich's eyes the database an icon of postmodern art. In other words, on an aesthetic level, while mass-standardization and reproducibility of media ---the ``logic of the factory'' \parencite[30]{Man01:The}--- shaped the form of cinema, post-industrial society and its logic of individual customization, shaped the database form. At the bodily level, cinema standardized perception of the passive body, and database individualizes experience. However, this individualized experience still constitutes a technological `shaping' of the body, a shaping that is exploded into every user quietly sitting behind the screen.

In opposition to this passivity of the body, Hansen describes images as something that emerges out of the complex relationship between the body and some sort of sensory stimulus. In radical disagreement with Manovich, Hansen considers that the image has become a process which gives form to information, and that this process needs to be understood in terms of the body as a filtering and creative agent in its construction. Drawing from Henri Bergson's theory of perception, and in resonance with cognitive science, Hansen defines the function of the body as a filtering apparatus. Under this conception, the body acts on and creates images by subtracting ``from the universe of images'' \parencite[3]{Han04:New}. Image creation is world creation, and it is not necessarily in contact with the reality that surrounds the body (or the reality of the body), but it is a result of the embodiment of a virtuality that is inherent to our senses. In other words, through this filtering activity, the body is empowered with ``strongly creative capacities'' \parencite[4]{Han04:New}. The world is a virtuality that is constructed with our senses and our body. The world can only appear if it appears to the body. Therefore, instead of being a passive node, the body actively \textit{in-forms} data as information (Hansen's word play). The databaser (database user) makes information out of data by precisely embodying the performative act that I call databasing. 

