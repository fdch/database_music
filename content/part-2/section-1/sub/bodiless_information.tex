\begin{quote}
	The disembodiment of information was not inevitable, any more than it is inevitable we continue to accept the idea that we are essentially informational patterns. \parencite[22]{Hay99:How}
\end{quote}

Media theorist N. Katherine Hayles \parencite{Hay99:How} unearths the theoretical context of cybernetics, upon which the posthuman has been constructed throughout the 20th century. She identifies three waves of cybernetics, each governed by different concepts which helped build the undergirding structures of the technologically determined and disembodied literature in vogue in the 1990s. 

The foundational wave cybernetics (from 1945 to 1960) was built, among other concepts, on two main theories: Jon von Neumann's architecture of the digital computer \see{databasing} and Claude Shannon's theory of information. As ``a probability function with no dimensions, no materiality, and no necessary connection with meaning'' \parencite[18]{Hay99:How}, Shannon's formal definition of information within communication systems highlighted pattern over randomness \parencite[33]{Hay99:How}. Therefore, disembodied information became a signal to be encoded, decoded, and isolated from noise.

The word `cybernetics' [steersman] thus synthesized three central aspects: information, communication, and control. Since the human was seen as an information processing entity, it was ``essentially similar to intelligent machines'' \parencite[7]{Hay99:How}. Therefore, the conceptualization of the feedback loop as a flow of information came to put at ease notions of human subordination, thus arriving at the governing concept of first wave cybernetics: \textit{homeostasis}. In this sense, the ``ability of living organisms to maintain steady states when they are buffeted by fickle environments'' \parencite[8]{Hay99:How}, became a patch that simultaneously fixed computers as less-than-human, but also pointed to the anxiety of disembodied information that was growing underneath.

However, since the observer of the `feedback loop' became part of the flow of the system, in the second wave (from 1960 to 1980), cybernetitians reconfigured homeostasis into \textit{reflexivity}, that is, ``the movement whereby that which has been used to generate a system [becomes] part of the system it generates'' \parencite[8]{Hay99:How}. This became also known as autopoiesis (i.e., self-generation), based on writings by Humberto Maturana and Francisco Varela. This second wave leaves the feedback loop behind, since it considers that ``systems are informationally closed'' \parencite[10]{Hay99:How}. This means that elements in the system do not see beyond their limits, and the only relation to the `outside' environment is by the concept of a \textit{trigger}. In this sense, disembodied information was buried deeply into the organization of the system, and the system itself appeared in the form of a cyborg.

Shifting from triggers to artificial intelligence signaled the third wave of cybernetics (from 1980 onwards), whose central concept was \textit{virtuality}. Development of cellular automata, genetic algorithms, and principally, emergence, led to the formation of the posthuman, or an embodied virtuality. However, in Hayles view, the underlying premise of this `posthuman' is that the human can be articulated by means of intelligent machines \parencite[17-8]{Hay99:How}. In turn, reconfiguring the concepts of body, consciousness, and technology as inherent to (post-) human life, Hayles argues for the impossibility of artificial intelligence to serve as a proxy for the human. Hayles objective is, then, to dismantle cybernetics from its (relative) assumptions, questioning its major achievements over the years and thereby opening the field for new considerations of the body and its material environment within cybernetics, and by extension, of the body in new media:

\begin{quote}
	My dream is a version of the posthuman that embraces the possibilities of information technologies without being seduced by fantasies of unlimited power and disembodied immortality, that recognizes and celebrates finitude as a condition of human being, and that understands human life is embedded in a material world of great complexity, one on which we depend for our continued survival. \parencite[5]{Hay99:How}
\end{quote}

While her work is focused on the literary narratives that were built in parallel with cybernetics, she leaves incursions in new media theory for other media theorists. This is where Mark B. N. Hansen comes in.
