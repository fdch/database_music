Computer music software is computer music's playground. Composing and programming blend into different forms of play that can be understood by a closer look at the playground's design. A key aspect of software design is delimiting constraints to data structures. The first choice is generally the programming language, after which the database tree unfolds its way up to the leaves. Among these leaves is where computer music programs reside. At this level of `leaves' software users are certainly aware that there is a `tree' in front of them. However, their awareness does not necessarily extend to the branches, trunk, or roots of the tree. There is endless music that can be made with leaves just as it can with paper. However, neither music quantity nor music quality are the point here. My argument is that working with data structures changes how we think and perform music making. I claim that composers using these leaves of computer music software are working indirectly with data structures, and unless they engage with programming, they remain unaware of data structures and their constraints. `Indirectly,' because the twigs and branches connect the leaf to the trunk, but these connections become invisible to the non-programmer composer \textit{by design}. Like a phantom limb of the tree, the database remains invisibly \textit{behind}. In this section, I present different approaches from composers and programmers that show how music concepts change with the presence and performance of the database. By database performance I mean neither the quality of musical output, nor the dexterity of the programming activity. Database performance in music composition is the activity of the databaser: databasing to make music.

\img{comp}{0.2}{
	The database is invisibly behind the computer, within the softwares used to create musical works. 
}{Diagram of database performance in computer music practices.}

\subsubsection{Hierarchical environments}
\label{computer:sssp}

\begin{quote}
	One of the most important aspects in the design of any computer system is determining the basic data types and structures to be used\dots we have been guided by our projection of the interaction between the tool which we are developing, and the composer. \parencite[119]{icmc/bbp2372.1978.012}
\end{quote}

\paragraph{Reducing cognitive burden}
In William Buxton's survey of computer music practices \parencites{Bux77:Aco}{icmc/bbp2372.1978.012}{DBLP:conf/icmc/BuxtonPRB80}, he distinguished between \textit{composing programs} and \textit{computer aided composition}, arguing that they had both failed as software, the former on account of their personalization and formalization, and the latter on their lack of interactivity. On his later interdisciplinary venture called \gls{sssp}, he focused on \gls{hci} ---a field in its very early stages in 1978---\footnote{William Buxton is now considered a pioneer in \gls{hci}, and he is now a major figure in the Microsoft Research department.}. Buxton's concern throughout his work on \gls{sssp} was to address the``problems and benefits arising from the use of computers in musical composition'' \parencite[472]{DBLP:conf/icmc/BuxtonFBRSCM78}. His solution to the problems was to reduce the cognitive burden of the composer, who ``should simply not have to memorize a large number of commands, the sequence in which they may be called, or the order in which their arguments must be specified'' \parencite[474]{DBLP:conf/icmc/BuxtonFBRSCM78}. He argued that reducing the amount of information given to composers helped them focus on music making. Therefore, in \gls{sssp}, the composer's action was reduced to four main selection tasks: timbres, pitch-time structure, orchestration, and playback. Timbres were assigned by defining waveforms for the table lookup oscillators, and pitch-time structure consisted on pitches and rhythms on a score-like \gls{gui} program called \gls{scriva} \parencite{youtube/buxton10}. Orchestration consisted in placing the previously chosen timbres on the score, and playback meant running the score or parts of it. With this simple but very concise structure, stemming from a somewhat dated programming philosophy in relation to audio software, Buxton delimited the scope of action of the composer

\paragraph{A Hierarchical Representation}
\textcite{DBLP:conf/icmc/BuxtonFBRSCM78} based their research on differing approaches to composition: Iannis Xenakis's score-as-entity approach in his 1971 \textit{Formalized Music}, an unpublished 1975 manuscript by Barry Vercoe at \gls{mit} studio for Experimental Music, where Buxton found a note-by-note approach, and Barry Truax's computer music systems \parencite{Tru73:The} which was, for Buxton, located somewhere in between the first two but did not provide a solution for ``the problem of dealing with the different structural levels of composition ---from note to score'' \parencite[120]{icmc/bbp2372.1978.012} \see{computer:balance}. Buxton, however, condensed these different approaches into what he called a ``chunk-by-chunk'' composition, where a `chunk' represented anything from a single note to an entire score, and thus reframed the question of a compositional approach as one of scale. For Buxton, ``the key to allowing this `chunk-by-chunk' addressing lies in our second observation: that the discussion of structural `levels' immediately suggests a hierarchical internal representation of scores'' \parencite[120]{icmc/bbp2372.1978.012}. That is to say, his solution for the scalability problem relied on a hierarchical representation of scores. 

In Buxton's \gls{sssp}, the hierarchical design depended on a data structure called \textit{symbol table}, which he subsequently divided into two objects called \texttt{score} and \texttt{Mevent} (musical events). The \texttt{score} structure had a series of global fields (variables) together with pointers to the first (head) and last (tail) \texttt{Mevent}s. In turn, \texttt{Mevent}s had local fields for each event together with pointers to the next and previous \texttt{Mevents}, so as to keep an ordered sequence \see{computer:linked} and enable temporal traversing of the tree. In turn, \texttt{Mevents} could have two different types: \texttt{MUSICAL\_NOTE} and \texttt{Mscore}, the former relating to terminal nodes editable by the user ---what he referred to as `leaves' of the tree structure---, and the latter consisting of nested \texttt{score} objects that added recursivity to the structure. Buxton's model was thus hierarchic (a tree structure) implemented in nested and doubly-linked symbol tables.

\textcite{icmc/bbp2372.1978.012} gave a detailed exposition of the data structures and their functionality. Buxton's general purpose in his \gls{hci} philosophy was to make the software work in such a way that it became invisible or transparent to the user. This is also known as a black-box approach. His innovations in this and other projects have had enormous resonances in computer science, and the concept of reducing cognitive burden of the user has developed as a standard of \gls{hci} \parencite{youtube/buxton16}.

\paragraph{Black-boxing}

Media theorist Vílem \textcite{Flu11:Int} proposed the term `envision' to describe a person's power to visualize beyond the surface of the image, and to bring the technical image into a concrete state of experience. The `image,' in Flusser's case is the television screen in its abstract state of ``electrons in a cathode ray tube.'' Therefore, he argues, ``if we are asking about the power to envision, we must let the black box remain ---cybernetically--- black'' \parencite[35]{Flu11:Int}. By seeing past the abstract quality of media we bring an image into experience. The black box allows envisioning to take place. In a similar way, by seeing past the hidden complexities of the software, composers are able to create music with unrestrained imagination. However, as I have shown before, Hansen makes a divergent point claiming that the virtuality inherent in the body is the creative potential of image \textit{in-formation} \see{embodiment}. 

Understanding the process of information as the experience of technical images, it follows that virtuality and envisioning can be considered complementary. On one hand, there is the technical device, whose multidimensionality is as complex as it is hidden from the envisioner. On the other, the human body with its capacity to create and embody. Flusser's point is, however, paradoxical: ``The envisioner's superficiality, to which the apparatus has \textit{condemned} him and for which the apparatus has \textit{freed} him, unleashes a wholly unanticipated power of invention'' \im \parencite[37]{Flu11:Int}. Therefore, the black-box is what condemns and frees the envisioner to a state of superficiality. However, Flusser continues, ``envisioners press buttons to inform, in the strictest sense of that word, namely, to make something improbable out of possibilities'' \parencite[37]{Flu11:Int}. In other words, Flusser justifies the invisibility of the technological device in favor of its most useful consequence, that is, its ability to make the user create something ``out of possibilities.'' Composers, therefore, are often given these possibilities to create, at the cost of a restricted creation space.

\paragraph{Generality and Portability}
\label{computer:free}

\begin{quote}
	Music data structures must be general enough so that as many styles of music as possible may be represented. This implies that the data structures (or the application's interface to them) should not enforce a musical model (such as equal temperament) that is inappropriate for the musical task at hand. \parencite[318]{icmc/bbp2372.1987.046}
\end{quote}

The \gls{sssp} lasted until 1982 due to lack of funding, and in the mid-1980s its research re-emerged with the work of \textcites{DBLP:conf/icmc/FreeV86}{icmc/bbp2372.1987.046}{DBLP:conf/icmc/FreeV88}, under Helicon Systems' \gls{camp}. Free's programming philosophy called for generality, portability, and simplicity. Due to \gls{sssp}'s many hardware dependencies, the code had to be completely re-written \parencite{DBLP:conf/icmc/FreeV86}. A crucial aspect of Free's programming concerns was portability \see{portability}, which moved him to create higher levels of software abstractions, so that software continued to live on in newer hardware. Free also developed \gls{scriva}, \gls{sssp}'s \gls{gui} program into extensible data structures for music notation arguing that software had to be general enough so that composers could work in multiple styles. The larger implication in Free's argument is that enforcing musical concepts in data structures limits the style that the program can achieve. Therefore, if the program fails to provide a certain level of generic functionality, the composer's output will be modelled by the data structure. On the one hand, it can be argued that this implication is simultaneously overestimating the agency of the database and underestimating that of the composer. In any case, the database works for the composer by taking care of the more tedious task. The cost of this, nonetheless, is that by working for the composer, the database guides the composer through certain paths while hiding other paths.

\paragraph{Simplification}
\label{computer:vanilla}

Hardware-independence led Free to imagine a general purpose, or \textit{vanilla} synthesizer, with which students in ``a music lab with multiple users on a networked computer system'' \parencite[127]{DBLP:conf/icmc/FreeV88} could seamlessly use the timbre world offered by various synthesizers made by different manufacturers. Free created a database that enabled simultaneous interaction among different types of hardware. The \textit{Music Configuration Database} consisted of an intermediate program between the physical \gls{midi} input devices (such as the Yamaha DX7 or Casio CZ101), and the computers in the network, so that ``rather than have the user tediously specify the \gls{midi} device properties for each synthesizer'' \parencite[133]{DBLP:conf/icmc/FreeV88} (channel management, control mapping, etc), these processes were handled by an intermediary database. Free's approach, in comparison to Buxton's, was not entirely black-boxed, since the database was open to modification by a specific set of commands provided to the user. The user could edit the database with a library of database access subroutines such as open/close, create/delete items, querying fields/keys, and loading/storing property items. With this library, Free simultaneously simplified user's interaction and reduced the ``chance of corrupting the database'' \parencite[137]{DBLP:conf/icmc/FreeV88}. 

\paragraph{Balance}
\label{computer:balance}

\img{truax_generality_b}{0.7}{
	Barry Truax' ``Inverse Relation Between Generality and Strength'' \parencite[51]{Tru80:The}. Another version of this graph can be found in \parencite[38]{laske_otto_1999}.
}{Generality vs. Strength}

\begin{quote}
	\dots all computer music systems both \textit{explicitly and implicitly embody a model of the musical processes that may be inferred from the program and data structure of the system}, and from the behavior of user working with the system. The inference of this model is independent of whether the system designer(s) claim that the system reflects such a model, or is simply a tool. \im \parencite[230-231]{Tru76:ACo}
\end{quote}

\textcites{Tru73:The}{Tru76:ACo}{Tru80:The}[Chapter~8]{Emm86:The} often compared grammatical structures of natural language to the structures of computer music systems, claiming that in both cases one can find certain constraints and facilitations for thought \parencite[156]{Emm86:The}. Arguing for balance between generality of applicability and strength of embedded knowledge within models for computer music systems \fsee{truax_generality_b}, he writes:

\begin{quote}
	In a computer music system, the grouping of data into larger units such as a sound-object, event, gesture, distribution, texture, or layer may have a profound effect on the composer's process of organization. The challenge for the software designer is how to provide powerful controls for such interrelated sets of data, how to make intelligent correlations between parameters, and how to make such data groupings \textit{flexible according to context}. \im \parencite[157]{Emm86:The}
\end{quote}

Truax's notion of balance speaks of a `meeting halfway' between the system and the user regarding the programmer's capability to embed a more complex conception of hierarchy in the system. What provides this balance is a certain flexibility among data structures which would enable them to adapt to the different hierarchical contexts with which music is understood. That is to say, since data structures can embody models of musical processes, they have an effect on the composer's overall performance of the database, and by extension, on the resulting music. 

\subsubsection{Music Notation Software}
\label{applications:notation}

Music representation has occupied an important area of research within the programming community. Formats and specifications such as \gls{midi}, \gls{musicxml}, the Humdrum \texttt{**kern} data format \parencite{DBLP:conf/ismir/Sapp05}, \gls{guido}, to name a few, have appeared over the years in conjunction with music engraving software. An extensive guide on musical representations can be seen in \textcite{Selfridge-Field:1997:BMH:275928}. In this section, I point to certain aspects of music notation software development that reveal different approaches towards data structures, and the possibilities that arise henceforth.

\paragraph{DARMS and SCORE}
Two major programs were developed during the 1960s and 1970s: Stefan Bauer-Mengelberg's \gls{darms} project for music engraving which started in 1963 \parencites{icmc/bbp2372.1983.002}{10.2307/30204239}, and Leland Smith's  \gls{score} \parencite{smith1971}. Both of these programs worked first in mainframe computers and were used for music printing and publishing. At first, \gls{score}'s character scanner was designed to interpret complex musical input into \gls{music-v} output, thus acting as an link between music notation and computer music synthesis. However, with the appearance of vector graphics in the 1970s it shifted solely to music printing.  With the appearance of the PostScript format in the 1980s, it became commercially available thus becoming one of the earliest music engraving softwares still in use today by major publishing houses \parencite{scoremus}. 

\paragraph{From Staves to Speakers}
Other programming approaches stemming from \gls{darms} and \gls{score} were developed during the 1980s. \textcite{icmc/bbp2372.1980.020} joined together the \gls{darms} data structures with those used in \gls{music-v} in a first attempt to obtain sonic feedback out of a notation system. Clements' attempt was nonetheless overshadowed by \gls{score}'s success. Later, \textcite{icmc/bbp2372.1987.045} worked on an interface to the \gls{darms} language called the \textit{Note Processor}, which became one of the earliest commercially available music notation systems. Dydo's data structures, however, were not publicly released when he presented his software at the \gls{icmc} in 1987. He later released it commercially in the early 1990s at a significantly lower price than other notation software such as \textit{Finale} which is still available today by MakeMusic, Inc. \parencites{10.2307/941442}{10.2307/940555}. \textcite{icmc/bbp2372.1981.018} modeled the \gls{score} input format into \textit{Score-11}, adapting it to Barry Vercoe's \gls{music-11}. Written in Pascal, \textit{Score-11} used circular linked lists traversed by an interpreter to produce \gls{music-11}-formatted output. The user creates a text file with blocks dedicated to individual instruments and specifies parameters such as rhythm, pitch, movement (glissandi, crescendo), amplitude, etc. These parameters are then re-formatted to fit the less musically-oriented notation of the \gls{music-n} programs. Brinkman argued that such a software would result in faster and less arduous performance on the composer's end: ``a crescendo over several hundred very short notes requires several hundred different amplitude values representing the increasing volume. \textit{Typing in several hundred note statements each with a slightly larger amplitude number would take forever}. If the computer could be instructed to gradually increase the amplitude value over twenty seconds then \textit{life would be much simpler}'' \im \parencite{score11manual}. Brinkman emphasized on the program's extensibility by users, inspiring Mikel Kuehn's recent \textit{nGen} program \parencite{csoundMethods}, a version of Brinkman's program for the current Csound. \textcite{icmc/bbp2372.1983.002} later designed an interpreter for the \gls{darms} language, which became useful for obtaining computable data structures for automated music analysis \parencite{icmc/bbp2372.1984.033}. Another approach to music notation was carried out at \gls{ccrma}, when \textcites{icmc/bbp2372.1988.020}{10.2307/3680043} devised a ``pure structure'' devoted to the ``hierarchical organization of musical objects into musical scores:'' the \textit{TTree} \parencite[184]{icmc/bbp2372.1988.020}. Stemming from his PhD research on formal languages in music theory \parencite{diener1985}, this data structure was based in the hierarchic structures of the \gls{sssp} project. The change Diener introduced to these structures was their capability of sustaining links between not only the previous and the next data records, but to the `parent' or `child' data records to which it was related. This is known as `inheritance,' and it enabled ``any event in the [structure] to communicate with any other event'' \parencite[188]{icmc/bbp2372.1988.020}. While Diener implemented this data structure in the object-oriented programming language \gls{smalltalk}, he later developed it into \textit{Nutation} \parencite{DBLP:conf/icmc/Diener92}, a visual programming environment for music notation. \textit{Nutation} was written in \gls{objective-c}, and it combined the previously developed \textit{TTree} structure with glyphs and a music synthesis toolkit called \textit{Music Kit} that the \gls{next} computer provided. This resulted in an extremely malleable \gls{cac} environment, which enabled fast manipulation and sonic feedback at the cost of limiting timbre to a predefined, hardware-specific set of digital instruments. 

\paragraph{Theoretical Performance}
What notation software is most often criticised for is the way in which sonic feedback often comes to be equiparated to (human) music performance. When Leeland Smith presented \gls{score} as ``not a `performer's' instrument, but rather a `musician's' instrument,'' for example, he claimed that ``theoretically, any performance, clearly conceived in the mind, can be realized on [the computer]'' \parencite[14]{smith1971}. It is indeed a fact that computers can offer automated tasks to an unimaginable extent. However, to translate this type of automation into music composition and performance, results in a disembodied music conception. In other words, an algorithmically generated stream of notes may result in physically impossible tasks for a performer, or for the listener. This is the point of inflexion when envisioning goes beyond the threshold of embodiment. It can be argued, however, that further developments in musical performance techniques can be achieved by pushing the limits of bodily skills. Nonetheless, what I am stressing here is the extent to which music composition can be reconfigured by the possibilities data structures have brought to the field. Furthermore, what is at stake with notation-based music software is yet another musical concern that governed most of music software development during the 1980s: style.

\subsubsection{Enter Objects}

\img{realtime}{0.9}{
A bodiless abstract published at the \gls{icmc} (1981) stating that a real-time version of \gls{music-11} was ``near completion'' by a group at MIT \parencite{DBLP:conf/icmc/PucketteVS81}.
}{A real-time version of \gls{music-11}.}

\paragraph{Max}
\label{computer:real-time}

Faster, cheaper, and portable microcomputers with real-time capabilities for audio processing began to appear onstage within institutions such as \gls{mit} and \gls{ircam}, and a growing interest among composers and programmers circled around real-time computer music software \fsee{realtime}. Towards the end of the 1980s, after the proliferation of \gls{midi} \parencite{Loy85:Mus}, composers were already incorporating real-time techniques within musical instruments and software \parencites{Ver84:The}{Puc91:Som}. This is the context for Miller Puckette's development of \gls{max} for the 4X real-time audio processor at \gls{ircam} \parencite{DBLP:conf/icmc/Puckette86}. With an emphasis on time and scheduling, Puckette devised a new approach towards complexity in computer music software:

\begin{quote}
	\dots complexity must never appear in the dealings between objects, only within them. Three other features currently in vogue seem unnecessary. First, there is no point in having a built-in notion of hierarchy; it is usually a hindrance. Second, I would drop the idea of continuously-running processes; they create overhead and anything they do can be done better through [input, output] related timing. Third, there should be few defaults. Rather than hide complexity I would keep it visible as an incentive to avoid it altogether. \parencite[43]{DBLP:conf/icmc/Puckette86}
\end{quote}

Puckette keeps complexity ``visible'' within the concept of the programming \textit{object}. Furthermore, he removes the notion of hierarchical programming proposing a light-weight, on-the-spot programming practice based on discontinuous processes: ``the scheduler keeps the runnable-message pool in the form of a separate queue for each latency'' \parencite[46]{DBLP:conf/icmc/Puckette86}. In other words, the structure of the database was placed \textit{horizontally} along the time axis, and Puckette's efforts were dedicated to optimizing the internal timing of audio processes. Specifically, linked lists are used to keep track of the order of processes that are run, and each process is scheduled according to its own temporality (latency). Thus, the entire network of processes that can be run is maintained in a dynamic list (stack) that can be changed at any time by adding or removing elements (push/pop). The way in which these processes (methods) are called is by messages that can be sent (input/output) by the user or objects themselves.\footnote{``The scheduler always sends the first message in the lowest-latency non empty queue. When the associated method returns the scheduler sends another message and so on. The only situation in which we need to interrupt a method before it is done is when I/O (including the clock) causes a lower-latency message to appear\dots In this case the scheduler causes a software interrupt to occur by pushing a new stack frame onto the stack and executing the lower-latency method. When this method returns \dots we pop the stack back to the prior frame at latency \(d_2\) and resume the associated method'' \parencite[46]{DBLP:conf/icmc/Puckette86}.} In sum, the object-oriented paradigm was thus applied to the scheduling system, resulting in a ground-breaking implementation that changed the real-time computer music performance scene: ``\dots rather than a programming environment, \gls{max} is fundamentally a system for scheduling real-time tasks and managing intercommunication between them'' \parencite{DBLP:journals/comj/Puckette02}.

\paragraph{Kyma}
\label{computer:kyma}

Another powerful example of an object-oriented language for non-real-time music composition is Carla Scaletti' Kyma, developed at the University of Illionis' \gls{cerl} \parencite{DBLP:conf/icmc/Scaletti87}. It is designed as an interactive composition environment for the Platypus digital signal processor. Scaletti's language was hierarchical in its structure, enabling data records to be linked vertically and horizontally. Together, these data structures formed objects, enabling the composer to treat any set of sounds within the composition, and even starting from the composition itself as an object. In such a way: ``\dots the composer could create a `sound universe,' endow the sound objects in this universe with certain properties and relationships, and explore this universe in a logically consistent way'' \parencite[50]{DBLP:conf/icmc/Scaletti87}. Given the ``vast amounts of data required for sound synthesis'' \parencite[50]{DBLP:conf/icmc/Scaletti87}, Kyma's objective was to fit timbre creation and temporal event lists into the same traversable database underlying the program. Like Puckette and Free, Scaletti's design was aimed at a language that ``itself would not impose notational or stylistic preconceptions'' \parencite[50]{DBLP:conf/icmc/Scaletti87}.

On one hand, Scaletti based her research on Larry Polansky's \gls{hmsl}, another ``non-stylistically based'' music composition environment that was ``not fundamentally motivated by a desire to imitate certain historical compositional procedures'' \parencite[224]{DBLP:conf/icmc/RosenboomP85}. Polansky's focus was on a language that would ``reflect as little as possible musical styles and procedures that have already been implemented ---like conventional music notation---'' \parencite[224]{DBLP:conf/icmc/RosenboomP85}. On the other hand, the ``notational bias'' \parencite[49]{DBLP:conf/icmc/Scaletti87} that Scaletti recognized in languages such as \gls{music-v} and FORMES \parencites{DBLP:conf/icmc/RodetBCP82}{DBLP:conf/icmc/BoyntonDPR86}, prescribed a very clear division between composition and synthesis which, in turn, was a very difficult and time-consuming ``wall'' she had to ``circumvent'' \parencite[49]{DBLP:conf/icmc/Scaletti87}. Therefore, she imagined a language in which the composer could ``choose to think in terms of notes and keyboards and staves but in which this structuring would be no easier and no harder to implement than any of countless, as yet uninvented, alternatives'' \parencite[49]{DBLP:conf/icmc/Scaletti87}. Both \gls{hmsl} and Kyma are still in used today, the former with a further Java version by Didkovsky and Burk \parencite{DBLP:conf/icmc/DidkovskyB01}, and the latter embedded into a commercially available workstation.\footnote{\url{https://kyma.symbolicsound.com/}}

\paragraph{Pure Data}
\label{computer:puredata}

Ten years after \gls{max}, \textcite{icmc/bbp2372.1997.060} moved on to Pure Data. The commercially available \gls{max/msp} \parencite{DBLP:conf/icmc/Zicarelli98} presents, like Pure Data, the \gls{max} programming paradigm \parencite{DBLP:journals/comj/Puckette02}. In resonance with the neutrality of the 1980s, Puckette introduced more data structure flexibility as a means to provide a musical instrument without stylistic constraints. Data structures became a more accessible feature for the user to define and edit:

\begin{quote}
	The design of \gls{max} goes to great lengths to avoid imposing a stylistic bias on the musician's output. To return to the piano analogy, although pianos might impose constraints on the composer or pianist, a wide variety of styles can be expressed through it. To the musician, the piano is a vehicle of empowerment, not constraint. \parencite{DBLP:journals/comj/Puckette02}
\end{quote}

Puckette, therefore, aims to a certain stylistic neutrality, which he represents by the way in which the user opens the program: a blank page: ``no staves, time or key signatures, not even a notion of `note,' and certainly none of instrumental `voice' or `sequence''' \parencite{DBLP:journals/comj/Puckette02}. While acknowledging that even the `blank page' is a culturally loaded symbol referring to the use of paper in Western Art Music (much in the same way that it is favoring complexity altogether), Puckette reconfigured computer music design, composition, and performance by considering the way in which the structure of the program resonates aesthetically.

\paragraph{Graphic Scores}
\label{graphic_scores}
In order to include graphic scores for electronic music within the Pure Data, Puckette implemented a data structure deriving from those of the C programming language, which can be used in relation to any type of data: ``the underlying idea is to allow the user to display any kind of data he or she wants to, associating it in any way with the display'' \parencite[184]{DBLP:conf/icmc/Puckette02}. Puckette's philosophy, as I have mentioned earlier, was aimed at detaching music software from music concepts, leaving these aesthetic decisions to the user. To this end, anything within the canvas can be customizable, and there is no notion of time assigned to canvas coordinates. However, Puckette provided the user with a sorting function, ``on the assumption that users might often want to use Pd data collections as $x$-ordered sequences'' \parencite[185]{DBLP:conf/icmc/Puckette02}. In fact, this is the only sorting function within Pure Data, and it is the same function that sorts the patch `graph,' only now made accessible to the user. A common and elementary database routine (\texttt{sort}) that emerged to the program's surface because of traditional music notation practices.

Puckette contextualized his research with the \textit{Animal} project by Lindemann and de Cecco which allowed users to ``graphically draw pictures which define complex data objects'' \parencite{DBLP:conf/icmc/Lindemann90a}, three cases of graphic scores used to model electroacoustic music: Stockhausen's \textit{Kontakte} and \textit{Studio II}, Yuasa's \textit{Towards the Midnight Sun}, and Xenakis' \textit{Mycenae Alpha}, and, most interestingly, the \gls{sssp}'s user-defined features for graphical representations. Although in the \gls{max} papers Puckette does not quote Buxton's research, the latter's numerous publications at \gls{icmc} towards the end of the 1970s suggests that they reached the scope of \gls{mit}'s Experimental Studio where Puckette studied with Barry Vercoe. Furthermore, in Puckette's later introduction of graphic scores to Pure data \parencite{DBLP:conf/icmc/Puckette02} \see{graphic_scores}, he references the \gls{sssp} quoted by Curtis Roads (1985) as one source of inspiration, indicating that at least in 2002 Puckette was aware of Buxton's research. In any case, both Buxton's and Puckette's approaches can be considered musical resonances that go beyond geographical limits, reaching the level of data structures in computer music software.

An interesting point in common, however, between much of the interactive composition programs that emerged during the 1980s is that stylistic neutrality became a leitmotif. Computer music software designers were interested in providing stylistic freedom by user-definability. This became a programming need that stemmed from earlier computer music software implementations, and their experimentation. This shift in the course of computer music programs can be understood from two perspectives. On the one hand, by experiencing first-hand the extent to which data structures can indeed structure musical output, the composer-programmers of the 1980s took charge on data structure design and devised new approaches to music-making software. On the other hand, the novel flexibility allowed by the object-oriented model within the programming world made its way to the community by the younger generation of composer-programmers. In any case, the database was moving, expanding through computer music networks, institutions, and softwares.

\paragraph{OpenMusic}
In the same \gls{icmc} 1997 where Pure Data was presented, two object-oriented languages appeared: \gls{rtcmix} \parencite{DBLP:conf/icmc/GartonT97} and OpenMusic \parencite{DBLP:conf/icmc/AssayagAFH97}. While neither real-time nor a synthesis engine, the strength of OpenMusic resides in its ability to provide the composer access to a variety of sound analysis tools for composition \parencites{icmc/bbp2372.2004.004}{icmc/bbp2372.2010.129}, as well as the possibility to generate algorithmic streams that output directly into a traditionally notated score. For example, OpenMusic introduced the concept of a \textit{maquette}, which is a graphic canvas upon which a heterogenous set of elements as varied as audio waveforms, scores, or piano-roll type notation can be displayed. The \gls{lisp}-based graphic language developed as a collaboration at \gls{ircam} held music notation as a focal point, distinguishing it from other stylistically neutral software. 

\paragraph{Heaps and Nodes}
\textcite{DBLP:conf/icmc/GartonT97} presented \gls{rtcmix}, a real-time version of Paul Lansky's \gls{cmix} \parencite{DBLP:conf/icmc/Lansky90}. What they described as innovative in this project was, in a similar way to the data structures for time management that Puckette presented, the scheduling capabilities of the program. In contrast to the \gls{cmix} language, which assumes a non-real-time access of objects, ``event scheduling is accomplished through a binary tree, priority-queue dynamic heap\dots '' \parencite{DBLP:conf/icmc/GartonT97}. A heap is a tree-based data structure where both keys and parent-child relationships follow a hierarchical logic. Garton and Topper thus introduced hierarchy into the scheduler. What this allowed, in turn, was ``scheduling-on-the-fly,'' that is, ``allowing notes to be scheduled at run-time (usually triggered by an external event, such as a \gls{midi} note being depressed)'' \parencite{DBLP:conf/icmc/GartonT97}. The real-time problem became once again a scheduling problem of computational tasks, and it was solved differently with yet another element: instruments instantiated ``on-the-fly'' could also establish their own \gls{tcp/ip} connection sockets in order to allow for networked access to the individual synthesizers \parencite{DBLP:conf/icmc/GartonT97}. That is to say, whenever a new instrument appears, it has the potential to enter into networked communication with earlier and future nodes. This means that synthesizer nodes could enter and leave the scheduler at any time, always in communication with each other. A musical equivalent would be for a violin player to enter in an out of the orchestra at will, while being able to lend the violin to any other player, and also play any other instrument except the conductor. In a similar networked way, SuperCollider \parencites{DBLP:conf/icmc/McCartney96}{DBLP:conf/icmc/McCartney98} is a high-level language that provides the user with a different paradigm to handle audio processes scheduling. The innovation that this language implemented, however, is the ``garbage collection'' of each process. McCartney took the hierarchic structure of the object-oriented paradigm and defined `nodes' in a tree-like structure, each with its own capability of nesting groups of other nodes, but most importantly, with its own initiation and expiration times. In other words, in contrast to Pure Data and \gls{max/msp}'s constantly running audio processes, SuperCollider only consumes \gls{cpu} resources whenever it needs to.

Both \gls{rtcmix} and SuperCollider meant a step forward towards networked musical environments that have resulted in recent forms of music making such as laptop orchestras and live coding, along with new music software such as ChucK \parencite{DBLP:conf/icmc/WangC03}. The literature on computer music software for composition alone would extend beyond the scope of this dissertation. For further reference in other sound synthesis data structures, see: the Diphone synthesis program \parencites{DBLP:conf/icmc/RodetDP88}{Rodet1989}{DBLP:conf/icmc/DepalleRGE93}{DBLP:conf/icmc/RodetL96}{DBLP:conf/icmc/RodetL97}; the Otkinshi system \parencite{icmc/bbp2372.2002.039}. For an overview of existing audio software up to 2004, see Xamat's PhD Dissertation \parencite[Chapter~2]{Amatriain/2004/phdthesis}. See also the Integra project \parencites{Bullock2009}{Bullock2011}, and Ariza's work on python's data structures \parencite{Ari05:Ano}. %, and Rowe's work on interactive music systems \parencite{Row92:Int, Row01:Mac}





