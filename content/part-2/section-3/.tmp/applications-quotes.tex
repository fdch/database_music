

% It is constituted by two \gls{midi}-based interconnected agents, a \textit{listener} and a \textit{player}. The player outputs musical material that results from generative or transforming processes on the (machine) listened input. The listener interprets incoming streams of \gls{midi} data with \gls{mir} techniques (central pitch detection, beat tracking, among others) grouping \texttt{Note}s into larger \texttt{Event}s, and then into larger musical phrases. Both \texttt{Note} and \texttt{Event} are objects with specific data structure: the former closely linked to the \gls{midi} message, the latter forming a circular linked list that enables traversing between neighbouring \texttt{Event}s. Attached to these \texttt{Event}s, however, reside intermediary agents that each store musical information that is locally relevant to the parent \texttt{Event}. The attached information, therefore, is fundamental to the system: ``The feature space representation \dots tends to treat parameters of the sound-pressure waves carrying musical percepts as central. Register, loudness, and density are all primary components of the representation'' \parencite[Chapter~7]{Row92:Int}. This constitutes Rowe's implementation of the object-oriented model. Upon deciding which analytical model to apply to the \textit{listener} agent, however, Rowe finds a coexistence of two virtually opposing models for music analysis: a hierarchical one (Shenker, Forte, Lerdahl) and a networked one (Narmour). While extending an analysis of database models to music theory is beyond the scope of this study, it is interesting to note how the versatility of the object model can integrate two very different ways of thinking musical relationships: ``Cypher's listener and player are organized hierarchically, though these hierarchies tend toward Narmour's network ideas rather than the more strictly structured trees of Lerdahl and Jackendoff'' \parencite[Chapter~4.3]{Row92:Int}. 

% \citeauthor{icmc/bbp2372.1999.411} used sensor data such as breath input, head movement, and finger positioning to enhance the sound analysis engine of the \gls{straight} system, thus achieving more degrees of freedom when carrying out the sound synthesis \parencite{icmc/bbp2372.1999.411}.\footnote{Later, \citeauthor{Kawahara:2004} demonstrated the morphing techniques of his software using the \gls{rwc} dataset \parencite{Kawahara:2004}.} 


% Christopher Ariza \parencite{icmc/bbp2372.2003.030} was able to implement a model for heterophonic texture by pitch-tracking the highly ornamented music of the Csángó\footnote{``The Csángó, in some cases a Szekler ethnic group, are found in eastern Transylvania (Kalotaszeg), the Gyimes valley, and Moldavia'' \parencite{icmc/bbp2372.2003.030}.} music into a database that enabled him to present a data structure of the ornament. While not strictly used as computer synthesis, his implementation of analysis and subsequent algorithmic rule extraction can be thought of as a form of analysis-based sound generation.