In order to define and contextualize database practices, I engage with the existing literature on data-driven art. Drawing mostly from media theory, I provide a sample of a variety of authors who have studied the use of databases in art. Specifically, I emphasize certain aspects of affect theory which relate to the intersection between the database and the body, in order to link database practice with sound and performance practices. 

Therefore, in `databasing music,' I describe different approaches to music practices ---computer music, sonification, music information retrieval--- and their interrelation with software design, to show how some of the major breakthroughs of these practices are related to changes in data structures. In the last section of the chapter, I describe all layers of the concept of the database, from lower ---data structures--- to higher ---databases--- levels, and describe the basic algorithmic designs in between. Specifically, I argue that all of these layers constitute what I call the performativity of the database, which is what is incorporated in the practice of database music.