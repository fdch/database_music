El objetivo de esta tesis es delinear un marco para discutir la agencia estética de la base de datos en la composición musical. Pongo mi tesis en relación con investigaciones existentes de artistas y desarrolladores que trabajan en los campos de la composición musical, la informática, el afecto y la ontología, haciendo énfasis en la ubicuidad de las bases de datos y en la necesidad de reflexionar sobre su práctica, particularmente en relación con lo que llamo 'databasing' (la performance de la base de datos) y la composición musical. Hay una base de datos en todas partes, en cualquier momento, siempre ya afectando nuestras vidas; son agentes en nuestras vidas estéticas y políticas tanto como nosotros lo somos en su composición y performance. La música de la base de datos vive entre las computadoras y el sonido. Mi argumento es que para conceptualizar la agencia de la base de datos en la composición musical, es necesario trazar la historia de la práctica, tanto en su uso técnico como artístico, para encontrar nodos de acción que inciden en la estética que resulta. Por lo tanto, esta tesis se compone de dos secciones principales.
 
En la primera parte, trazo una historia de prácticas de bases de datos desde tres puntos de vista. La primera es de la teoría de los nuevos medios, enfatizando la intersección entre la base de datos y el cuerpo. El segundo es de la historia de la base de datos en la informática, dando una visión panorámica de las herramientas y conceptos detrás de los sistemas, modelos y estructuras de la base de datos. La tercera es de su uso en prácticas de sonido, describiendo diferentes enfoques de la creación de bases de datos desde los campos de la recuperación de información musical (MIR), la sonificación y la música por computadora.
 
En la segunda parte, discuto la agencia de la base de datos bajo los conceptos más amplios de sonido, el ser y comunidad. Estos tres ejes se abordan en cuatro secciones, cada una con una perspectiva diferente. Primero me enfoco en la escucha, presentando la base de datos como un sujeto resonante en una relación de red y comunidad con lo humano. En segundo lugar, me concentro en la memoria, comparando la memoria humana y la escritura con el almacenamiento de información digital, relacionando así la base de datos y la composición con la memoria, los archivos y su espectralidad. Tercero, analizo la performatividad de la base de datos, entendiendo a la misma bajo una concepción de género, en su temporalidad, repetición, y en su apariencia contingente como estilo, piel y timbre. En la última sección, reviso la noción de obra musical, reflexionando sobre las consecuencias de lo anárquico y lo inoperante en la comunidad de la música de base de datos.
 
Como apéndice, presento una biblioteca de código abierto para la creación y performance de una base de datos multimodal que combina algoritmos de análisis de visión por computadora y de timbre para generar una base de datos de descriptores de imágenes y sonido adecuados para la navegación automatizada, la búsqueda y la performance audiovisual en vivo.
