The aim of this dissertation is to understand the aesthetic agency of the database in music composition. I place my dissertation in relation to existing scholarship, artists, and developers working in the fields of music composition, computer science, affect, and ontology, with emphasis on the ubiquity of databases and on the need to reflect on their practice, particularly in relation to databasing and music composition. There is a database everywhere, anytime, always already affecting our lives; it is an agent in our aesthetic and political lives just as much as we are agents in its composition and performance. Database music lives in between computers and sound. My argument is that in order to conceptualize the agency of the database in music composition, we need to trace the history of the practice, in both its technical and its artistic use, so as to find nodes of action that have an effect on the resulting aesthetics. Therefore, this dissertation is composed of two main sections.

In the first section, I trace a history of database practices from three points of view. The first is from new media theory, emphasizing the intersection between the database and the body. The second point of view is from the history of the database in computer science, giving a panoramic view of the tools and concepts behind database systems, models, structures. The third is from their use in sound practices, describing different approaches to databasing from the fields of music information retrieval, sonification, and computer music. 

In the second section, I discuss this agency under the broader concepts of sound, self, and community. These three axes are addressed in four sections, each with a different perspective. First I focus on listening, delineating Jean-Luc Nancy's ontology of sound in order to present the database as a resonant subject in a networked relation and community with the human. Second, I focus on memory, comparing human memory and writing with digital information storing, thus relating databasing and composition with memory, archives and their spectrality. Third, I analyze the performativity of databasing, understanding the database as gendered, in its temporality, repetition, and in its contingent appearance as style, skin, and timbre. In the last section, I revise the notion of music work, reflecting on the consequences of the anarchic and the inoperative in the community of database music.

% As an appendix, I develop an open-source library for multimedia composition that combines computer vision and timbre analysis algorithms to generate a database of descriptors, interpreting them as nodes in a network suitable for automated navigation.
