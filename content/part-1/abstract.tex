I begin this dissertation with a word (`database') and a proposition: Is there something we can call database music? This sudden jump from noun to adjective comes not without its audible retaliation, and I make no attempt to muffle it. The reader would perhaps forgive the clumsiness of my condition of composer in the midst of writting his dissertation, which made me jump to quickly at the opportunity to make some sound in this initial gesture. Nevertheless, there is a sound and we can listen to it.

In the literature that I have mined (for this dissertation is not only a text, but the sedimented layers of text that I initially traversed with keywords in database queries, such as ``\texttt{database AND music}''), the database has many histories and many names attached to it. I make no attempt to cover all of these, but I admit that I have (foolishly) tried: the subtitle does begin ``a history\dots.'' However, it is the `a' that would account for the gaps and missing parts to which this text is inevitably bound. It is also the `a' that accounts for the path that I begin delineating accross the two nodes that compose the focus of this text: \textit{database and music}. Both nodes have their historical, technical, and aesthetical idiosyncracies. The primary goal of this dissertation (if I dare to say that it has one) is thus to find where these intersect. The \texttt{AND} of the query had indeed much less results than the \texttt{OR}, which meant my quest was already promising some reduction. For instance, I indeed decided that the search would only pertain to situations in which computers where involved. Needless to say, not only was my search dramatically expanded through the plethora of database applications, the history of their systematization, and the ongoing struggle between models, it also opened up the programming world, and with it the history of programming languages, and of the computer itself. 

Upon this abysmal enterprise, I did as any musician would and started listening for the sound of databases. I realized that this is not just the sound of your computer reading its hard-drive: it is the sound made with its software. At this point, a network (and this is one of the key terms throughout this text) of sonic software had begun to appear. What this network pointed to, besides the key to open door number one, is a certain silence that much of the literature relating to databases in art continues to abide to: a sonic silence. This relates to the ``history, technology\dots'' part of the subtitle. In addressing this silence, I do not attempt to invalidate previous approaches \parencites{Man01:The}{Ves07:Dat}. On the contrary, these texts have shed light of the key concepts with which I have traversed the sonic software I present, the types of programming decisions I discuss, the various disciplines I place at the intersection of computers and music, and the plurality of shapes that have appeared in relation to databases: door number two. That is to say, through these authors, I introduce notions of embodiment, virtuality, and framing coming from posthumanism \parencite{Hay99:How} and new media theory \parencite{Han04:New}, in order to contextualize the role of the database within the practices of \gls{mir}, Sonification, and Computer Music. Each one of these disciplines has its own history and it is evidenced by the many conference proceedings and journals that I have (again) mined, as well as the various authors (most of them composers, and most of them programmers) that I refer to. In between these two, that is, in between my exploration of the database in new media theroy, and the its corresponding exploration witihin sound practices, I introduce the more technical evolution of the database, in order to develop a secondary concept that speaks of the performativity of databases: \textit{databasing}. I use this non-existing gerund to refer to all the actions that need to take place around databases, whether these are made by humans or not, refering mostly to these as \textit{databasers}.

After having reached this point, in which the intersection of the database and music was covered in terms of its facticity, as the evidence of a motion, the trace of the database, I could not help noticing door number three. This door refers to the next part of the subtitle ``\dots aesthetics of the database\dots'' and opens up to the complex world of sound, flipping this text upside down: \textit{music and database}. (Bang!) The meaninglessness of this reversal, at least semantically or even in terms of a database query, is precisely the point. The sound of databases that I begin to delineate through the first two doors now reaches a point where it faces a difference. I enter first with \poscite{Nan07:Lis} ontology of sound, with which I understand the networks of music software as resonating networks. With this move, the database begins to appear as a resonant nonhuman agent \parencite{Lat90:On} that reconfigures the way we think communities \parencite{Nan91:The}. I then distinguish databases from memory and archives, and find within it a spectral authority \parencites{Der78:Wri}{Der95:Arc}. The final move brings the activity of the body \parencite{But88:Per}, and the relationship between database, gender, and performance. These three stages (resonance, spectrality, and performance) address an aesthetics of the database that I situate around sound. 

Up to this point, my argument might seem to have arrived to an end: I contextualize and define database practices (chapter 1); I develop a technical overview of the performatic activity of the database into what I call databasing (chapter 2); I review the existing literature with emphasis on sound practice (chapter 3); I conceptualize sound in terms of resonance, networks, and community (chapter 4); I delineate the differences between database, memory, and archive, in order to present the spectrality of databases (chapter 5); I develop databasing in terms of the performance, gender, and style in relation ot databases (chapter 6). However, there is yet one more leap that the more adventurous reader might take with the final chapter of this dissertation. In chapter 7, I bring the discussion of database and music and database and music\dots to the latter part of the subtitle ``in music composition.'' With this chapter (a fourth door) I engage with the work of music composition. That is to say, with the history of the database in mind, I rethink the activity of composing \parencite{Vag01:Som}, the role of the composer \parencite{Lew99:Int}, and the operativity of the music work \parencite{Cas00:The}. The reader will be warned that this last chapter has no conclusions, let alone answers. Neither it has proper questions. I consider it an attempt to incite, if anything, a provocation before the question. I have already warned the reader of the clumsiness of a composer in the midst of writing, but that should not discourage neither academic rigor, nor literary thirst. I have thus included an exhaustive bibliography, two interludes and a poslude (work in progress), together with graphs, tables, and some snippets of pseudocode and sometimes working code. In the hope that you will continue reading these pages, I will {draft} an end to these remarks
