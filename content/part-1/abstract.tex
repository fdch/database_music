%-----------------------------------------------------------------------------
%-----------------------------------------------------------------------------
%-----------------------------------------------------------------------------
%-----------------------------------------------------------------------------
%-----------------------------------------------------------------------------
%-----------------------------------------------------------------------------
%-----------------------------------------------------------------------------

\begin{quote}
\footnotesize
\flushright
\raggedleft
\textit {
	I am sitting in a room different from the one you are in now. I am recording the sound of my speaking voice and I am going to play it back into the room again and again until the resonant frequencies of the room reinforce themselves so that any semblance of my speech, with perhaps the exception of rhythm, is destroyed. What you will hear, then, are the natural resonant frequencies of the room articulated by speech. I regard this activity not so much as a demonstration of a physical fact, but more as a way to smooth out any irregularities my speech might have.\footnote{Alvin Lucier. I Am Sitting In A Room. See: \url{https://en.wikipedia.org/wiki/I_Am_Sitting_in_a_Room}}
}
\end{quote}


In going below the `note' level, you can also go below the `screen' of the interface, \textit{through} the program, into its data structures.







% ./content/art/section-1/sub/abstract.tex

Since Manovich's \textit{The Language of New Media} (2001) the database became a term related to internet and digital art, and as such it was conceptualized in relation to interface design and interactivity. However, I reconsider these assumptions from later theorizations in new media, namely Hayles' posthumanist critique and Hansen's embodiment approach. in Manovich a silent allegiance to Kittler's posthumanism, I analyze this allegiance as a consequence of a confusion between data and information.  


% ./content/art/section-2/sub/abstract.tex

I describe all layers of the concept of the database, from lower ---data structures--- to higher ---databases--- levels, and describe the basic algorithmic designs in between. Specifically, I argue that all of these layers constitute what I call the performativity of the database, which is what is incorporated in the practice of database music.




% ./content/art/section-3/sub/abstract.tex






The database has been present in the music literature as the silent partner since the first computers were used to make music. For a figure representing the position of the database in relation to music practices involving computers \see{img:mir_comp_sonif_interaction}. Through the 1990s the use of computers ---and databases--- can be found in diverse music fields such as computer assisted composition (CAC), electroacoustic music, computer music, sonification, music information retrieval (MIR). 

describe different approaches to music practices ---computer music, sonification, music information retrieval--- and their interrelation with software design, to show how some of the major breakthroughs of these practices are related to changes in data structures. 

For now, I use the words `database' and `computer' somewhat interchangeably. I will provide a more acute definition of the database at the end of this chapter. However, the decision is not random, since the database is itself the condition of possibility of the computer. What does this lead to when we can speak of database music? Is computer music ---or, better, all music made with computers--- database music? 



















% ./content/aesthetics/abstract.tex



Delineating the agency of the Database in the practice of music composition, I discuss the aesthetics of Database Music, developing the concepts of listening, memory, and performance. 

First, I analyze the extent to which the Database can be a listening subject which promotes illusions of style and authority. I consider style and authority as central aspects of the sphere of aesthetic agency of the Database. I then focus on a form of collective `listening' and I arrive at my conception of the Database as an inherently deterministic system. This system is shaped as a network of nonhuman agents, whose `resonance' is fundamental to its definition. I use this resonant network to further analyze the agency of the Database, in terms of how authorial qualities percolate through the network. I use Jean-Luc Nancy's ontology of sound to understand how the database can be a listening subject. In Brian Kane's reading \parencite[143-144]{Gra15:The} of Nancy's work \parencite{Nan07:Lis}, he presents the this ontology ---i.e., what Nancy calls \textit{resonance}---, considering it as a process constitutive of a phenomenology of the self. 


% ./content/aesthetics/section-1/sub/abstract.tex



In this chapter, I analyze resonant networks in order to assess the extent to which databasing can be reconfigured by listening, and vice versa. The following questions will be revised: To what extent is the listening subject present within database music? How is the notion of listening subject reconfigured by way of the database? To what extent can the database be thought of as a listening subject, and, if so, to what extent does the agency of the database as listening subject resonate aesthetically?

In section one, I delineate Jean-Luc Nancy's ontology of sound in order to present the database as a resonant subject in itself. 

In section two and three, given the multiplicity of factors that are in play when databases enter into the process of music listening, I focus on the links that exist between Nancy's resonance and Bruno Latour's actor-network theory \parencite{Lat90:On, Lat93:We}, arriving at the concept of a resonant network. Then, a distinction between sound and networks is made, and the concept of the work of actors is introduced.

Finally, I understand resonant networks in terms of community. Since the notion of inoperativity is closely related to that of community ---and the exposure of selves---, this relation of selves can also be understood as the resonating force that unfolds hand in hand with the performativity of the network. Thus, the expansion of the network, the propagation of sound, and the exposure of selves, can be connected to each other with a force of inoperativity.





% ./content/aesthetics/section-2/sub/abstract.tex

In order to narrow the gap between human and nonhuman agency, I assess the extent to which computer memory resembles human memory. On the one hand, I compare memory and writing with digital information storing, and thus arrive at databasing as a form of memory. On the other hand, I consider archives as collective memory, which serves to to explain how the Database can also be a form of collective memory. 

There is yet another substitution that can be amended to Latour's definition of the network. If the network is `recorded movement,' that is, a trace, a trajectory, this means that its existence is evidenced by way of not only motion in itself, in the sense that the very same oscillatory motion of reference in between the nodes creates its defining gesture. It is also the case that the recollection of movement constitutes an structurally inseparable part of the definition of the network. Therefore, in this chapter, I understand the database in its relation with memory, understanding memory from three points of view: the human, the nonhuman, and the spectral.

In section one, I analyze memory as process of embodiment and relate it to resonant networks.

In section two, I analyze the concept of the archive and its relation to the database, specifically, to databasing as a collective form of memory. In this sense, the nonhuman comes as an instantiation of memory outside the human. I assess the extent to which resonant networks can be considered under the scope of the concept of the archive. 

In section three, I understand the dynamics of resonant networks spectrally, that is, as an expression of a force, or a power, that comes out of the spectrality that results out of the resonance of the human and the nonhuman. By considering the spectrality of the database, I assess the extent of its aesthetic agency in terms of a \textit{haunting} force. Therefore, if database music is indeed haunted by the specter of the database, how does this affect the aesthetic result? If the ghost of the database can be understood as an image of the nonhuman, that is, an image emerging out of the plurality of memory, then, to what extent can it be considered a singularity, a self in itself? Finally, if, as databasers, we are engaged with this force in creative action, how does the music made with databases sound, and what is making it?



% ./content/aesthetics/section-3/sub/abstract.tex




I focus on the performativity of the database. On the one hand, I claim that the database is gendered. I argue that the notion of `style' is what promotes the illusion of a gendered subject in the Database. I argue that since both the performance and the directionality of the `styling process' remain strictly on the virtual skin of the database, the database's authorial subject, like the gendered self, remains in the spectrum of the illusory. On the other hand, I claim that the limit of the Database resides on its performativity. I consider the technical aspects of databases and define computer systems as networks of interconnected-but-independent databases. This definition serves to extend the performatic limit of databases to computers, and therefore to link the performance of the database to the performance of the computer. My goal in this final section is to lead the way to the connection between Database performance and Music Composition: the performatic limit of the Database is also the limit of Music Composition.


In this section, I draw from performance and gender studies to analyze database practice as a performative activity. I use Judith Butler's concept of gender \parencite{But88:Per} to analyze the extent to which authority in database practices can be understood in terms of style.  The databaser, which is the human subject in this case, begins resonating with the database, the nonhuman subject, in a form of feedback loop. This resonant loop is only possible through the performance of the database. I thus locate the origin of the database as listening subject the moment its performance begins. In this moment of performance, both human and nonhuman listening subjects are resounded upon their limit. I consider this limit to be a surface in between the human and the nonhuman, and I like to think of it as the `skin' of the database. I argue that this skin of the database is the possibility condition for illusions of style ---as in the style that is visible in one's clothes, or in one's decoration of the skin--- and authority ---as in the appearance of a subject who has some sort of power: the subject of the listening database. I thus analyze these illusions as belonging to the sphere of Agency that the Database presents, and assess the extent to which they affect the aesthetics of Database practice.



The databaser, which is the human subject in this case, begins resonating with the database, the nonhuman subject, in a form of feedback loop. This resonant loop is only possible through the performance of the database. 


% ./content/politics/abstract.tex




In search of understanding the political in Database and Composition practices, I question the established concept of music composition and arrive to new definitions of the music work, practice, and authorship. 

First, I consider the concepts developed in the previous chapter to understand Music Composition as Database Performance. I propose that the ontology of Composition needs to be redefined in terms of the agency of the Database. My goal in this section is to reveal that the Database agency, when contextualized within Music Composition, has the form and the politics of a music listening to itself. 

Second, I use Nancy's concept of inoperativity to redefine the music object. I argue that the inoperativity of the listening experience, which resides on the delay between sense and sensuality, provides insight on the type of unworking that affects music composition. I thus redefine the outcome of music composition as the \textit{severed music object}, emphasizing its inoperative status of suspension, withdrawal, and its inherent state non-completeness. I then consider how this state of suspension of the severed music object can be analyzed in terms of a Community of artists, database performers, composers, etc., mutually exposed to each other (Nancy 1991). Therefore, in order to understand the dynamics of this transversal community of Database and Composition, I analyze the paradox of anarchy and reflect on the consequences of both the anarchic and the inoperative in Database and Composition practices. 

Finally, I present my view on collaboration, and propose a redefinition of the term uprooting it from the traditional union of forces forming a whole. I claim that the new form of collaboration can be understood as a form of collective, or \textit{trans-inoperation}, consisting in the mutual exposure of the limits of singular, performing beings. As a consequence of this form of collective inoperance, I claim that a new politics of authorship needs to be analyzed, particularly in terms of the spectral in the Database. I question the power of this illusory figure in terms of the effectiveness of the archontic principle that is present in \textit{trans-inoperant} works of art. I believe the specter of the author loses the sensuality and the sense of the listening subjects in state of trans-inoperance, and thus the power of the author ceases to take place.



% ./content/politics/section-1/sub/abstract.tex

The concepts exposed in \nameref{chapter:Database_Aesthetics} affect and reconfigure transversally the practices of composition and databasing. Traditionally, music composition was considered a single author practice, in which the composer's technique or aesthetic intuition is the sole agent, romanticizing the artist as an ``involuntary vessel through which inspiration flows'' \parencite{Bor95:Rat}. As I have outlined in , this is no longer the case, since understood in terms of its resonance and of its performativity, composition explodes the name of the composer, leaving as many spectral remains of its trace as can be imagined. Conversely, databasing is already embedded in a networked structure that only allows partial and temporary allocation of authors (databaser), since in the structural database tree exist multi-authored branches that renew themselves, outgrowing themselves in perpetual difference and instability. The notions of stability and authority can only be related to snapshots in the history of a software. However, the institutional quality of both databasing and composition is still at play, namely in the many cases of proprietary software and in the composer's name, that is, in the commercial release and the objectification of music work. Less than focusing on general criticism, in this section I argue that, since the agency of the database reveals itself as aesthetic experience, then it is the dynamics of this agency need to be addressed. I claim that this agency, when contextualized within music composition, specifically composing with computers, it has the form and the politics of a music listening to itself. 


% ./content/politics/section-2/sub/abstract.tex


How does the concepts of inoperativity and anarchy, in their relation to database community, resonate politically in the works of database music?

In this section, I analyze the anarchic element in database practice and bring it to music composition practice. 

Defining anarchy as a paradoxically productive force ---a form of destruction which ``produces the very thing it reduces'' \parencite{Der95:Arc}---, Derrida locates it at the core of the concept of the archive \see{archontic}. As I have outlined before, databasing brings together with its relation to the archive, the archontic principle that is bound to the origin and the rule. That is to say, since the database has the potential of becoming a source, databasing becomes an activity of this source, and thus embeds the databaser with a specter of authority. Therefore, given the circumstances of this authority of databasing, claiming that composition can be identified with databasing means translating the `archic' not only to the performativity of composition, also to the product of composing, to the composer and the composed. I have mentioned above the presence of the skin of the database, now I shall refer to the skin of the music object. 


I argue that the link between the archive, the database, and the music object is this capacity to prescribe its own origin ---the commencement--- and rules ---the command. Finally, I analyze the extent to which this anarchic element is present in the inoperative object of music, and how this presence affects the unwork of art. 

My goal in this reflection on the consequences of the anarchic and the inoperative in database and composition practices is to understand the dynamics of community within both database and composition fields. 

My argument is that in order to understand what is in common between database and composition, from the points of view of art, aesthetics, and politics, we need to define the transversality of the underlying structures of anarchy and inoperativity.



In an intersection between music and computers, I situate the database around the broader question of agency in art and technology. 


Finally, as an instantiation of the propositions above, I will develop an open-source library for multimedia composition that combines computer vision and timbre analysis algorithms to generate a database of descriptors, interpreting them as nodes in a network suitable for automated navigation. 


I use William Brent's \obj{timbreID} ---timbre description algorithms--- and Antoine Villeret's \obj{pix\_opencv} ---image descriptors using Computer Vision algorithms---, to develop a new software library for Pure Data. My model consists of a joint database structure for image and audio descriptors suitable for real-time navigation. At its core, the Database is generated by calculating derivatives between both data sets, and it is performed by applying random probabilities, markov chains, or chaotic generators to this navigation. This allows for multiple paths to be traced on each navigation.

In order to write this dissertation, I have developed ``Abby'' an online Text Database tool namely to build an annotated bibliography. The program is mostly written in Javascript, with the data navigation and programming hosted in Github, and the datasets stored in the Google account that New York University has provided me. The annotated bibliography is available at \url{https://fdch.github.io/abby}, and the code can be accessed or cloned from \url{https://github.com/fdch/litrev}.






% ./content/introduction/section-1/sub/abstract.tex



% ./appendix/abstract.tex
% abstract of appendices------------------------------


% ./appendix/section-1/sub/abstract.tex



% ./appendix/section-2/sub/abstract.tex
% 
% 
% 





This dissertation goes through the state of database art, the use of computers in music and art, the collaborative aspect surrounding computers, the immateriality that percolates through the arts as data, the terminological struggles in the definition of data-based media, the possibilities of new linkages between different media through data, the arbitrary world of the composer in the midst of an emergent, autopoietic, bottom-up art-world, the ubiquitous architecture enabling all of it, the resonating self in between, the non-human agency, the software communities, and the topology of the networked world.







Chapter 1 serves to contextualize the database historically and technically. First I engage with the database from the point of view of media studies, as it as commented in the arts since the beginning of the 21st century. I then trace a history of the events which lead to the current database panorama, and refer to it as a database tree. In the end of this chapter, I trace the use of the database in relation to music, particularly in three fields based on computer-based sound: MIR, sonification, and composition. In Chapter 2, I dedicated to locate the aesthetic agency of the database from three points of departure: listening, memory, and performance. These three aspects relate sound, networks, memory and archives, in order to delineate the performativity of databasing. Thus, the agency of the database is seen at the intersection of the human and the nonhuman. The final chapter deals with the dynamics of databasing and composition. I engage with the political in database practices and question the established concepts behind music composition. Thus, I present a different conceptualization of the music work.








%-----------------------------------------------------------------------------
%-----------------------------------------------------------------------------
%-----------------------------------------------------------------------------
%-----------------------------------------------------------------------------
%-----------------------------------------------------------------------------
%-----------------------------------------------------------------------------
