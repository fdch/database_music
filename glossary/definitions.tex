\newdualentry{aflowlib}{AFLOWLIB}
	{Automatic Flow For Materials Discovery}
	{A globally available database of 2,118,033 material compounds with over 281,698,389 calculated properties See also: \url{http://aflowlib.org/}}
\newdualentry{api}{API}
	{Application Programming Interface}
	{In computer programming, an application programming interface is a set of subroutine definitions, communication protocols, and tools for building software.  See also: \url{https://en.wikipedia.org/wiki/Application_programming_interface}}
\newdualentry{arangodb}{ArangoDB}
	{}
	{ArangoDB is a native multi-model database system developed by ArangoDB Inc. The database system supports three data models with one database core and a unified query language AQL. The query language is declarative and allows the combination of different data access patterns in a single query. ArangoDB is a NoSQL database system but AQL is similar in many ways to SQL. See also: \url{https://en.wikipedia.org/wiki/ArangoDB}}
\newdualentry{ase}{ASE}
	{Adaptive Server Enterprise}
	{SAP ASE, originally known as Sybase SQL Server, and also commonly known as Sybase DB or Sybase ASE, is a relational model database server developed by Sybase Corporation, which later became part of SAP AG. ASE is predominantly used on the Unix platform, but is also available for Microsoft Windows. See also: \url{https://en.wikipedia.org/wiki/Adaptive_Server_Enterprise}}
\newdualentry{asf}{ASF}
	{Apache Software Foundation}
	{an American non-profit corporation to support Apache software projects, including the Apache HTTP Server. The ASF was formed from the Apache Group and incorporated on March 25, 1999. See also: \url{https://en.wikipedia.org/wiki/The_Apache_Software_Foundation}}
\newdualentry{autocad}{AutoCAD}
	{Automatic Computer Assisted Design}
	{AutoCAD is a commercial computer-aided design and drafting software application. Developed and marketed by Autodesk, AutoCAD was first released in December 1982 as a desktop app running on microcomputers with internal graphics controllers. See also: \url{https://en.wikipedia.org/wiki/AutoCAD}}
\newdualentry{bfcc}{BFCC}
	{Bark Frequency Cepstrum Coefficient}
	{ See also: \url{}}
\newdualentry{cac}{CAC}
	{Computer Aided Composition}
	{ See also: \url{}}
\newdualentry{cad}{CAD}
	{Computer Aided Design}
	{the use of computers (or workstations) to aid in the creation, modification, analysis, or optimization of a design. See also: \url{https://en.wikipedia.org/wiki/Computer-aided_design}}
\newdualentry{caddc}{CADDC}
	{Computer Aided Data Driven Composition}
	{ See also: \url{} \cite{icmc/bbp2372.2015.072}}
\newdualentry{camp}{CAMP}
	{Computer Assisted Music Project}
	{A general purpose composition and performance software environment originally based upon William Buxton's SSSP. See also: \url{} \cite{DBLP:conf/icmc/FreeV88}}
\newdualentry{case}{CASE}
	{Computer Aided Software Engineering}
	{the domain of software tools used to design and implement applications. CASE tools are similar to and were partly inspired by computer-aided design (CAD) tools used for designing hardware products. See also: \url{https://en.wikipedia.org/wiki/Computer-aided_software_engineering}}
\newdualentry{cc}{CC}
	{Creative Commons}
	{an American non-profit organization devoted to expanding the range of creative works available for others to build upon legally and to share. See also: \url{https://en.wikipedia.org/wiki/Creative_Commons}}
\newdualentry{ccrma}{CCRMA}
	{Center For Computer Research In Music And Acoustics}
	{a multi-disciplinary facility where composers and researchers work together using computer-based technology both as an artistic medium and as a research tool. See also: \url{https://ccrma.stanford.edu/}}
\newdualentry{cdip}{CDIP}
	{Coastal Data Information Program}
	{an extensive network for monitoring waves and beaches along the coastlines of the United States. Since its inception in 1975, the program has produced a vast database of publicly-accessible environmental data for use by coastal engineers and planners, scientists, mariners, and marine enthusiasts.  See also: \url{https://cdip.ucsd.edu/}}
\newdualentry{cerl}{CERL}
	{Computer Based Education Research Laboratory}
	{A research center based in the University of Illinois See also: \url{https://en.wikipedia.org/wiki/PLATO_(computer_system)}}
\newdualentry{cmix}{Cmix}
	{}
	{A computer music software designed and developed by Paul Lansky. Belongs to the Music N family, although it was designed for a more specific context of concrete music. See also: \url{http://www.musicainformatica.org/topics/cmix.php}}
\newdualentry{cnmat}{CNMAT}
	{Center For New Music And Audio Technologies}
	{a multidisciplinary research center within University of California, Berkeley Department of Music. The Center's goal is to provide a common ground where music, cognitive science, computer science, and other disciplines meet to investigate, invent, and implement creative tools for composition, performance, and research. It was founded in the 1980s by composer Richard Felciano. See also: \url{https://en.wikipedia.org/wiki/Center_for_New_Music_and_Audio_Technologies}}
\newdualentry{cobol}{COBOL}
	{Common Business Oriented Language}
	{A compiled English-like computer programming language designed for business use. See also: \url{https://en.wikipedia.org/wiki/COBOL}}
\newdualentry{codasyl}{CODASYL}
	{Conference/Committee On Data Systems Languages}
	{a consortium formed in 1959 to guide the development of a standard programming language that could be used on many computers. This effort led to the development of the programming language COBOL and other technical standards. See also: \url{https://en.wikipedia.org/wiki/CODASYL}}
\newdualentry{compath}{COMPath}
	{Composition Path}
	{A music making method with interactive map interface See also: \url{https://sihwapark.com/COMPath}}
\newdualentry{cpu}{CPU}
	{Central Processing Unit}
	{the electronic circuitry within a computer that carries out the instructions of a computer program by performing the basic arithmetic, logic, controlling, and input/output (I/O) operations specified by the instructions.  See also: \url{https://en.wikipedia.org/wiki/Central_processing_unit}}
\newdualentry{csv}{CSV}
	{Comma Separated Values}
	{ a delimited text file that uses a comma to separate values. See also: \url{https://en.wikipedia.org/wiki/Comma-separated_values}}
\newdualentry{cuidado}{CUIDADO}
	{Content Based Unified Interfaces And Descriptors For Audio/Music Databases Available Online}
	{a new chain of applications through the use of audio/music content descriptors, in the spirit of the MPEG-7 standard See also: \url{} \cite{DBLP:conf/ismir/VinetHP02, DBLP:conf/icmc/VinetHP02}}
\newdualentry{cwm}{CWM}
	{Cluster Weighted Modeling}
	{an algorithm-based approach to non-linear prediction of outputs (dependent variables) from inputs (independent variables) based on density estimation using a set of models (clusters) that are each notionally appropriate in a sub-region of the input space. See also: \url{https://en.wikipedia.org/wiki/Cluster-weighted_modeling}}
\newdualentry{darms}{DARMS}
	{Digital Alternate Representation Of Musical Scores}
	{The DARMS project started in 1963 by Stefan Bauer-Mengelberg and it is one of the first programming languages for music engraving See also: \url{} \cite{icmc/bbp2372.1983.002, 10.2307/30204239}}
\newdualentry{dbms}{DBMS}
	{Database Managemet System}
	{a computer program (or more typically, a suite of them) designed to manage a database, a large set of structured data, and run operations on the data requested by numerous users. Typical examples of DBMS use include accounting, human resources and customer support systems. See also: \url{https://en.wikipedia.org/wiki/Category:Database_management_systems}}
\newdualentry{ddl}{DDL}
	{Data Definition Language}
	{A data definition or data description language (DDL) is a syntax similar to a computer programming language for defining data structures, especially database schemas. See also: \url{https://en.wikipedia.org/wiki/Data_definition_language}}
\newdualentry{dj}{DJ}
	{Disk Jockey}
	{a person who plays existing recorded music for a live audience. Most common types of DJs include radio DJ, club DJ who performs at a nightclub or music festival and turntablist who uses record players, usually turntables, to manipulate sounds on phonograph records.  See also: \url{https://en.wikipedia.org/wiki/Disc_jockey}}
\newdualentry{dml}{DML}
	{Data Manipulation Language}
	{A data manipulation language is a computer programming language used for adding, deleting, and modifying data in a database. A DML is often a sublanguage of a broader database language such as SQL, with the DML comprising some of the operators in the language See also: \url{https://en.wikipedia.org/wiki/Data_manipulation_language}}
\newdualentry{dom}{DOM}
	{Document Object Model}
	{a cross-platform and language-independent application programming interface that treats an HTML, XHTML, or XML document as a tree structure wherein each node is an object representing a part of the document See also: \url{https://en.wikipedia.org/wiki/Document_Object_Model}}
\newdualentry{dsl}{DSL}
	{Domain Specific Language}
	{a computer language specialized to a particular application domain. This is in contrast to a general-purpose language, which is broadly applicable across domains. See also: \url{https://en.wikipedia.org/wiki/Domain-specific_language}}
\newdualentry{edm}{EDM}
	{Electronic Dance Music}
	{a broad range of percussive electronic music genres made largely for nightclubs, raves and festivals See also: \url{https://en.wikipedia.org/wiki/Electronic_dance_music}}
\newdualentry{er}{ER}
	{Entity Relationship}
	{A database model that describes interrelated things of interest in a specific domain of knowledge. A basic ER model is composed of entity types and specifies relationships that can exist between entities. See also: \url{https://en.wikipedia.org/wiki/Entity-relationship_model}}
\newdualentry{exasol_ag}{Exasol AG}
	{}
	{Exasol is an analytic database management software company. Its product is called Exasol, an in-memory, column-oriented, relational database management system. See also: \url{https://en.wikipedia.org/wiki/Exasol}}
\newdualentry{fm}{FM}
	{Frequency Modulation}
	{In telecommunications and signal processing, frequency modulation is the encoding of information in a carrier wave by varying the instantaneous frequency of the wave.  See also: \url{https://en.wikipedia.org/wiki/Frequency_modulation}}
\newdualentry{formes}{FORMES}
	{}
	{an object-oriented programming environment for music composition and synthesis See also: \url{} \cite{DBLP:conf/icmc/BoyntonDPR86, 10.2307/3679811}}
\newdualentry{fortran}{FORTRAN}
	{Formula Translation}
	{a general-purpose, compiled imperative programming language that is especially suited to numeric computation and scientific computing. See also: \url{https://en.wikipedia.org/wiki/Fortran}}
\newdualentry{gb}{GB}
	{Gigabyte}
	{a unit of information equal to one thousand million (109) or, strictly, 230 bytes. See also: \url{https://en.wikipedia.org/wiki/Gigabyte}}
\newdualentry{gdif}{GDIF}
	{Gesture Description Interchange Format}
	{a format for storing, retrieving and sharing information about music-related gestures. See also: \url{} \cite{Jensenius2006a}}
\newdualentry{gpu}{GPU}
	{Graphics Processing Unit}
	{a specialized electronic circuit designed to rapidly manipulate and alter memory to accelerate the creation of images in a frame buffer intended for output to a display device.  See also: \url{https://en.wikipedia.org/wiki/Graphics_processing_unit}}
\newdualentry{gsr}{GSR}
	{Galvanic Skin Response}
	{GSR is another name for Electrodermal activity (EDA) is the property of the human body that causes continuous variation in the electrical characteristics of the skin. See also: \url{https://en.wikipedia.org/wiki/Electrodermal_activity}}
\newdualentry{gttm}{GTTM}
	{Generative Theory Of Tonal Music}
	{a theory of music conceived by American composer and music theorist Fred Lerdahl and American linguist Ray Jackendoff and presented in the 1983 book of the same title.  See also: \url{https://en.wikipedia.org/wiki/Generative_theory_of_tonal_music} \cite{DBLP:conf/ismir/HamanakaHT14}}
\newdualentry{hci}{HCI}
	{Human Computer Interaction}
	{A field that researches the design and use of computer technology, focused on the interfaces between people (users) and computers. See also: \url{https://en.wikipedia.org/wiki/Human-computer_interaction}}
\newdualentry{hmdb}{HMDB}
	{Human Metabolome Database}
	{a comprehensive, high-quality, freely accessible, online database of small molecule metabolites found in the human body. Created by the Human Metabolome Project funded by Genome Canada. See also: \url{https://en.wikipedia.org/wiki/Human_Metabolome_Database}}
\newdualentry{hp}{HP}
	{Hewlett Packard}
	{an American multinational information technology company headquartered in Palo Alto, California. See also: \url{https://en.wikipedia.org/wiki/Hewlett-Packard}}
\newdualentry{html}{HTML}
	{Hypertext Markup Language}
	{the standard markup language for creating web pages and web applications. With Cascading Style Sheets and JavaScript, it forms a triad of cornerstone technologies for the World Wide Web See also: \url{https://en.wikipedia.org/wiki/HTML}}
\newdualentry{http}{HTTP}
	{Hypertext Transfer Protocol }
	{an application protocol for distributed, collaborative, hypermedia information systems. See also: \url{https://en.wikipedia.org/wiki/Hypertext_Transfer_Protocol}}
\newdualentry{ibm}{IBM}
	{International Business Machines Corporation}
	{an American multinational information technology company headquartered in Armonk, New York, with operations in over 170 countries. See also: \url{https://en.wikipedia.org/wiki/IBM}}
\newdualentry{icloud}{ICloud}
	{}
	{cloud storage and cloud computing service from Apple Inc. launched on October 12, 2011. As of February 2016, the service had 782 million users. See also: \url{https://en.wikipedia.org/wiki/ICloud}}
\newdualentry{icmc}{ICMC}
	{International Computer Music Conference}
	{ a yearly international conference for computer music researchers and composers. It is the annual conference of the International Computer Music Association (ICMA). See also: \url{https://en.wikipedia.org/wiki/International_Computer_Music_Conference}}
\newdualentry{idms}{IDMS}
	{Integrated Database Management System}
	{a network model database management system for mainframes See also: \url{https://en.wikipedia.org/wiki/IDMS}}
\newdualentry{ids}{IDS}
	{Integrated Data Store}
	{an early network database management system largely used by industry, known for its high performance. IDS became the basis for the CODASYL Data Base Task Group standards. See also: \url{https://en.wikipedia.org/wiki/Integrated_Data_Store}}
\newdualentry{iem}{IEM}
	{Institute Of Electronic Music And Acoustics}
	{ a multidisciplinary research center within the University of Music and Performing Arts, Graz, (Austria). See also: \url{https://en.wikipedia.org/wiki/Institute_of_Electronic_Music_and_Acoustics}}
\newdualentry{ift}{IFT}
	{Inverse Fourier Transform}
	{ See also: \url{https://en.wikipedia.org/wiki/Fast_Fourier_transform}}
\newdualentry{imdb}{IMDB}
	{Internet Movie Database}
	{ See also: \url{https://www.imdb.com/}}
\newdualentry{imdbs}{IMDBs}
	{In Memory Data Bases}
	{a database management system that primarily relies on main memory for computer data storage. It is contrasted with database management systems that employ a disk storage mechanism. See also: \url{https://en.wikipedia.org/wiki/In-memory_database}}
\newdualentry{ims}{IMS}
	{Information Management System}
	{the first database model ever created. It was created by IBM during the early 1960s, in conjunction with two other American manufacturing conglomerates (Rockwell and Caterpillar) for NASA's Project Apollo See also: \url{https://en.wikipedia.org/wiki/IBM_Information_Management_System}}
\newdualentry{ios}{iOS}
	{I Operating System}
	{ a mobile operating system created and developed by Apple Inc. exclusively for its hardware.  See also: \url{https://en.wikipedia.org/wiki/IOS}}
\newdualentry{ip}{IP}
	{Internet Protocol}
	{the principal communications protocol in the Internet protocol suite for relaying datagrams across network boundaries. Its routing function enables internetworking, and essentially establishes the Internet. See also: \url{https://en.wikipedia.org/wiki/Internet_Protocol}}
\newdualentry{ir}{IR}
	{Information Retrieval}
	{the activity of obtaining information system resources relevant to an information need from a collection of information resources. Searches can be based on full-text or other content-based indexing. See also: \url{https://en.wikipedia.org/wiki/Information_retrieval} https://nlp.stanford.edu/IR-book/pdf/irbookonlinereading.pdf}
\newdualentry{ircam}{IRCAM}
	{Institut De Recherche Et Coordination Acoustique/Musique}
	{a French institute for science about music and sound and avant garde electro-acoustical art music. See also: \url{https://www.ircam.fr/}}
\newdualentry{iris}{IRIS}
	{Incorporated Research Institutions For Seismology}
	{IRIS is a consortium of over 120 US universities dedicated to the operation of science facilities for the acquisition, management, and distribution of seismological data. See also: \url{https://www.iris.edu}}
\newdualentry{ismir}{ISMIR}
	{International Society For Music Information Retrieval}
	{an international forum for research on the organization of music-related data. See also: \url{http://ismir.net}}
\newdualentry{ixd}{IXD}
	{Videoalive Indexer Extracted Closed Captions And Metadata}
	{This is an index file in text form that shows closed captions and file offset information from a MPEG or WM video file. The .IXD file is used by VBrick Systems, Videoalive, Discovervideo.com, and other vendors. See also: \url{https://filext.com/file-extension/IXD}}
\newdualentry{json}{JSON}
	{Javascript Object Notation}
	{an open-standard file format that uses human-readable text to transmit data objects consisting of attribute-value pairs and array data types (or any other serializable value) See also: \url{https://www.json.org/}}
\newdualentry{lbdm}{LBDM}
	{Local Boundaries Detection Model}
	{The Local Boundary Detection Model (LBDM) calculates boundary strength values for each interval of a melodic surface according to the strength of local discontinuities; peaks in the resulting sequence of boundary strengths are taken to be potential local boundaries. See also: \url{} \cite{DBLP:conf/icmc/Cambouropoulos01}}
\newdualentry{lfcc}{LFCC}
	{Log Frequency Cepstral Coefficients}
	{ See also: \url{https://en.wikipedia.org/wiki/Mel-frequency_cepstrum}}
\newdualentry{lisp}{LISP}
	{List Processor}
	{a family of computer programming languages with a long history and a distinctive, fully parenthesized prefix notation. Originally specified in 1958, Lisp is the second-oldest high-level programming language in widespread use today. Linked lists are one of Lisp's major data structures, and Lisp source code is made of lists. See also: \url{https://en.wikipedia.org/wiki/Lisp_(programming_language)}}
\newdualentry{lpc}{LPC}
	{Linear Predictive Coding}
	{ a tool used mostly in audio signal processing and speech processing for representing the spectral envelope of a digital signal of speech in compressed form, using the information of a linear predictive model See also: \url{https://en.wikipedia.org/wiki/Linear_predictive_coding}}
\newdualentry{madbpm}{madBPM}
	{}
	{A modular C++ software platform serving as a data-ingestion engine suitable for database perceptualization (i.e. sonification and visualization) See also: \url{https://ccrma.stanford.edu/~rob/rpi/madbpm/} \cite{icmc/bbp2372.2017.087}}
\newdualentry{max}{Max}
	{Max Programming Language}
	{ See also: \url{}}
\newdualentry{max/msp}{MAX/MSP}
	{Max (Mathews) + ("Max Signal Processing", Or The Initials Miller Smith Puckette)}
	{also known as Max/MSP/Jitter, is a visual programming language for music and multimedia developed and maintained by San Francisco-based software company Cycling '74 See also: \url{https://en.wikipedia.org/wiki/Max_(software)}}
\newdualentry{metrixml}{MetriXML}
	{Metrix In Extensible Markup Language}
	{A computer music synthesis programming language based in the Music-N type. See also: \url{https://xamat.github.io/Thesis/html-thesis/} \cite{Amatriain/2004/phdthesis}}
\newdualentry{midi}{MIDI}
	{Musical Instrument Digital Interface}
	{ a technical standard that describes a communications protocol, digital interface, and electrical connectors that connect a wide variety of electronic musical instruments, computers, and related audio devices. See also: \url{https://en.wikipedia.org/wiki/MIDI}}
\newdualentry{mir}{MIR}
	{Music Information Retrieval }
	{the interdisciplinary science of retrieving information from music. MIR is a small but growing field of research with many real-world applications. Those involved in MIR may have a background in musicology, psychoacoustics, psychology, academic music study, signal processing, informatics, machine learning, optical music recognition, computational intelligence or some combination of these. See also: \url{https://en.wikipedia.org/wiki/Music_information_retrieval}}
\newdualentry{mit}{MIT}
	{Massachusetts Institute Of Technology }
	{ private research university in Cambridge, Massachusetts. See also: \url{http://www.mit.edu/}}
\newdualentry{mongodb}{MongoDB}
	{}
	{a cross-platform document-oriented database program. Classified as a NoSQL database program, MongoDB uses JSON-like documents with schemata. See also: \url{https://en.wikipedia.org/wiki/MongoDB}}
\newdualentry{music-11}{MUSIC-11}
	{}
	{Barry Vercoe's development of \gls{music-iv} which later grew into the still widely used Csound. See also: \url{}}
\newdualentry{music-v}{MUSIC-V}
	{}
	{See \gls{music-n} See also: \url{}}
\newdualentry{music-iv}{MUSIC-IV}
	{}
	{See \gls{music-n} See also: \url{}}
\newdualentry{music-n}{MUSIC-N}
	{}
	{MUSIC-N refers to a family of computer music programs and programming languages descended from or influenced by MUSIC, a program written by Max Mathews in 1957 at Bell Labs. See also: \url{https://en.wikipedia.org/wiki/MUSIC-N}}
\newdualentry{mysql}{MySQL}
	{}
	{an open source relational database management system (RDBMS). Its name is a combination of "My", the name of co-founder Michael Widenius's daughter, and "SQL", the abbreviation for Structured Query Language. See also: \url{https://en.wikipedia.org/wiki/MySQL}}
\newdualentry{nasa}{NASA}
	{National Aeronautics And Space Administration}
	{an independent agency of the United States Federal Government responsible for the civilian space program, as well as aeronautics and aerospace research.NASA was established in 1958, succeeding the National Advisory Committee for Aeronautics (NACA). See also: \url{https://www.nasa.gov/}}
\newdualentry{neo4j}{Neo4j}
	{}
	{ a graph database management system developed by Neo4j, Inc.  See also: \url{https://github.com/neo4j/neo4j}}
\newdualentry{net.loadbang-sql}{net.loadbang-SQL}
	{}
	{A Java library for communicating with SQL databases from MXJ. We currently support MySQL and HSQLDB. The HSQLDB system includes an embedded database instance, so it runs automatically from text files in Max\'s search path; no external database server configuration is necessary. See also: \url{http://www.maxobjects.com/?v=libraries&id_library=99}}
\newdualentry{next}{NeXT}
	{}
	{an American computer and software company founded in 1985 by Apple Computer co-founder Steve Jobs. See also: \url{https://en.wikipedia.org/wiki/NeXT}}
\newdualentry{nime}{NIME}
	{New Interfaces For Musical Expression}
	{an international conference dedicated to scientific research on the development of new technologies and their role in musical expression and artistic performance. Researchers and musicians from all over the world gather to share their knowledge and late-breaking work on new musical interface design. See also: \url{https://en.wikipedia.org/wiki/New_Interfaces_for_Musical_Expression}}
\newdualentry{nmr}{NMR}
	{Nuclear Magnetic Resonance}
	{a physical phenomenon in which nuclei in a strong static magnetic field are perturbed by a weak oscillating magnetic field (in the near field and therefore not involving electromagnetic waves) and respond by producing an electromagnetic signal with a frequency characteristic of the magnetic field at the nucleus. See also: \url{https://en.wikipedia.org/wiki/Nuclear_magnetic_resonance}}
\newdualentry{nosql}{NoSQL}
	{Non Or Not Only Sql}
	{a mechanism for storage and retrieval of data that is modeled in means other than the tabular relations used in relational databases. See also: \url{https://en.wikipedia.org/wiki/NoSQL}}
\newdualentry{nyu}{NYU}
	{New York University}
	{a private research university spread throughout the world. See also: \url{}}
\newdualentry{oracle}{Oracle}
	{Oracle Corporation}
	{an American multinational computer technology corporation headquartered in Redwood Shores, California.  See also: \url{https://en.wikipedia.org/wiki/Oracle_Corporation}}
\newdualentry{os_2200}{OS 2200}
	{Unisys Os 2200 Databases}
	{The OS 2200 database managers are all part of the Universal Data System (UDS). UDS provides a common control structure for multiple different data models. Flat files (sequential, multi-keyed indexed sequential - MSAM, and fixed-block), network (DMS), and relational (RDMS) data models all share a common locking, recovery, and clustering mechanism. OS 2200 applications can use any mixtures of these data models along with the high-volume transaction file system within the same program while retaining a single common recovery mechanism. See also: \url{https://en.wikipedia.org/wiki/Unisys_OS_2200_databases}}
\newdualentry{osc}{OSC}
	{Open Sound Control }
	{a protocol for networking sound synthesizers, computers, and other multimedia devices for purposes such as musical performance or show control. OSC's advantages include interoperability, accuracy, flexibility and enhanced organization and documentation. See also: \url{https://en.wikipedia.org/wiki/Open_Sound_Control}}
\newdualentry{postgresql}{PostgreSQL}
	{}
	{an open source object-relational database management system with an emphasis on extensibility and standards compliance. It can handle workloads ranging from small single-machine applications to large Internet-facing applications with many concurrent users. See also: \url{https://en.wikipedia.org/wiki/PostgreSQL}}
\newdualentry{pvc}{PVC}
	{Phase Vocoder}
	{a collection of phase vocoder signal processing routines and accompanying shell scripts for use in the transformation and manipulation of sounds. It is written in C and designed to be used in a UNIX environment. See also: \url{http://www.cs.princeton.edu/courses/archive/spr99/cs325/koonce.html}}
\newdualentry{qcd-audio}{QCD-Audio}
	{}
	{QCD-audio works on data of computer physics, stemming from the Institute for Physics. Our interdisciplinary proposal QCD-Audio will develop sonification techniques for data of numerical models in physics. See also: \url{https://qcd-audio.at}}
\newdualentry{qed}{QED}
	{Quantuum Electrodynamics}
	{In particle physics, quantum electrodynamics is the relativistic quantum field theory of electrodynamics See also: \url{https://en.wikipedia.org/wiki/Quantum_electrodynamics}}
\newdualentry{ram}{RAM}
	{Random Access Memory }
	{a form of computer data storage that stores data and machine code currently being used. A random-access memory device allows data items to be read or written in almost the same amount of time irrespective of the physical location of data inside the memory. See also: \url{https://en.wikipedia.org/wiki/Random-access_memory}}
\newdualentry{rdm}{RDM}
	{Raima Database Manager }
	{an ACID-compliant embedded database management system designed for use in embedded systems applications. RDM has been designed to utilize multi-core computers, networking (local or wide area), and on-disk or in-memory storage management. See also: \url{https://en.wikipedia.org/wiki/Raima_Database_Manager}}
\newdualentry{rm/t}{RM/T}
	{Relational Model/Tasmania}
	{Extensions to the Relational Model See also: \url{https://en.wikipedia.org/wiki/Relational_Model/Tasmania}}
\newdualentry{roi}{ROI}
	{Regions Of Interest}
	{samples within a data set identified for a particular purpose. For example, in medical imaging, the boundaries of a tumor may be defined on an image or in a volume, for the purpose of measuring its size. See also: \url{https://en.wikipedia.org/wiki/Region_of_interest}}
\newdualentry{rtcmix}{RTcmix}
	{Realtime Cmix}
	{a real-time software "language" for doing digital sound synthesis and signal-processing. It is written in C/C++, and is distributed open-source, free of charge.  See also: \url{http://rtcmix.org/}}
\newdualentry{rwc}{RWC}
	{Real World Computing Music Database}
	{a copyright-cleared music database (DB) that is available to researchers as a common foundation for research. It was built by the RWC Music Database Sub-Working Group of the Real World Computing Partnership (RWCP) of Japan See also: \url{https://staff.aist.go.jp/m.goto/RWC-MDB/}}
\newdualentry{sap_hana}{SAP HANA}
	{High Performance Analytics Appliance}
	{an in-memory, column-oriented, relational database management system developed and marketed by SAP SE See also: \url{https://en.wikipedia.org/wiki/SAP_HANA}}
\newdualentry{score}{SCORE}
	{}
	{ a scorewriter program, written in FORTRAN for DOS by Stanford Professor Leland Smith (1925-2013) with a reputation for producing very high-quality results. See also: \url{https://en.wikipedia.org/wiki/SCORE_(software)} \cite{smith1971}}
\newdualentry{scriva}{SCRIVA}
	{}
	{notation software for the SSSP See also: \url{https://www.billbuxton.com/SSSP.html}}
\newdualentry{sdif}{SDIF}
	{Sound Description Interchange Format}
	{a standard for the well-defined and extensible interchange of a variety of sound descriptions. See also: \url{https://en.wikipedia.org/wiki/SDIF}}
\newdualentry{smil}{SMIL}
	{Synchronized Multimedia Integration Language}
	{a World Wide Web Consortium recommended Extensible Markup Language markup language to describe multimedia presentations. It defines markup for timing, layout, animations, visual transitions, and media embedding, among other things.  See also: \url{https://en.wikipedia.org/wiki/Synchronized_Multimedia_Integration_Language}}
\newdualentry{sonart}{SonART}
	{}
	{a flexible, multi-purpose multimedia environment that allows for networked collaborative interaction with applications for art, science and industry. It provides an open ended framework for integration of powerful image and audio processing methods with a flexible network features. See also: \url{https://ccrma.stanford.edu/~woony/software/sonart/index.html} \cite{icmc/bbp2372.2004.128}}
\newdualentry{spear}{SPEAR}
	{Sinusoidal Partial Editing Analysis And Resynthesis}
	{an application for audio analysis, editing and synthesis. The analysis procedure (which is based on the traditional McAulay-Quatieri technique) attempts to represent a sound with many individual sinusoidal tracks (partials), each corresponding to a single sinusoidal wave with time varying frequency and amplitude. See also: \url{http://www.klingbeil.com/spear/}}
\newdualentry{sql}{SQL}
	{Structured Query Language}
	{a domain-specific language used in programming and designed for managing data held in a relational database management system, or for stream processing in a relational data stream management system. See also: \url{https://en.wikipedia.org/wiki/SQL}}
\newdualentry{sqlite}{SQLite}
	{Structured Query Language Lite}
	{a relational database management system contained in a C programming library. In contrast to many other database management systems, SQLite is not a client-server database engine. Rather, it is embedded into the end program.  See also: \url{https://en.wikipedia.org/wiki/SQLite}}
\newdualentry{sssp}{SSSP}
	{Structured Sound Synthesis Project}
	{an interdisciplinary project whose aim is to conduct research into problems and benefits arising from the use of computers in musical composition See also: \url{https://www.billbuxton.com/SSSP.html}}
\newdualentry{stft}{STFT}
	{Short Time Fourier Transform}
	{ a Fourier-related transform used to determine the sinusoidal frequency and phase content of local sections of a signal as it changes over time. See also: \url{https://en.wikipedia.org/wiki/Short-time_Fourier_transform}}
\newdualentry{straight}{STRAIGHT}
	{Speech Transformation And Representation Using Adaptive Interpolation Of Weighted Spectrogram}
	{a high-quality vocoder designed for speech analysis, modification, and synthesis See also: \url{} \cite{icmc/bbp2372.1999.411}}
\newdualentry{sybase}{Sybase}
	{}
	{ an enterprise software and services company that produced software to manage and analyze information in relational databases.  See also: \url{https://en.wikipedia.org/wiki/Sybase}}
\newdualentry{tcp/ip}{TCP/IP}
	{Transmission Control Protocol / Internet Protocol}
	{The Internet protocol suite is the conceptual model and set of communications protocols used in the Internet and similar computer networks. It is commonly known as TCP/IP because the foundational protocols in the suite are the Transmission Control Protocol (TCP) and the Internet Protocol (IP). See also: \url{https://en.wikipedia.org/wiki/Internet_protocol_suite}}
\newdualentry{timbreid}{timbreID}
	{Timbre Identification}
	{a collection of audio feature analysis externals for [Pd]. The classification extern (timbreID) accepts arbitrary lists of features and attempts to find the best match between an input feature and previously stored instances of training data. Besides doing identification, timbreID is also designed to facilitate real time concatenative synthesis and timbre-based orderings of sound sets. Its usage is fully explained in the accompanying helpfile. See also: \url{http://williambrent.conflations.com/pages/research.html} \cite{icmc/bbp2372.2010.044}}
\newdualentry{timesten}{TimesTen}
	{}
	{an in-memory, relational database management system with persistence and recoverability. See also: \url{https://en.wikipedia.org/wiki/TimesTen}}
\newdualentry{turboimage}{TurboIMAGE}
	{}
	{a database developed by Hewlett Packard and included with the HP3000 minicomputer. See also: \url{https://en.wikipedia.org/wiki/TurboIMAGE}}
\newdualentry{uri}{URI}
	{Uniform Resource Identifier}
	{a string of characters that unambiguously identifies a particular resource. To guarantee uniformity, all URIs follow a predefined set of syntax rules, but also maintain extensibility through a separately defined hierarchical naming scheme. See also: \url{https://en.wikipedia.org/wiki/Uniform_Resource_Identifier}}
\newdualentry{url}{URL}
	{Uniform Resource Locator }
	{colloquially termed a web address, is a reference to a web resource that specifies its location on a computer network and a mechanism for retrieving it. A URL is a specific type of \gls{uri} although many people use the two terms interchangeably. See also: \url{https://en.wikipedia.org/wiki/URL}}
\newdualentry{vmware}{VMWare}
	{}
	{a subsidiary of Dell Technologies that provides cloud computing and platform virtualization software and services. See also: \url{https://en.wikipedia.org/wiki/VMware}}
\newdualentry{webdna}{WebDNA}
	{}
	{a server-side scripting, interpreted language with an embedded database system, specifically designed for the World Wide Web. Its primary use is in creating database-driven dynamic web page applications. See also: \url{https://en.wikipedia.org/wiki/WebDNA}}
\newdualentry{xml}{XML}
	{Extensible Markup Language}
	{a markup language that defines a set of rules for encoding documents in a format that is both human-readable and machine-readable. The W3C's XML 1.0 Specification and several other related specifications-all of them free open standards-define XML See also: \url{https://en.wikipedia.org/wiki/XML}}
\newdualentry{xplain}{XPlain}
	{}
	{A semantic database model developed by J.H. ter Bekke See also: \url{http://www.jhterbekke.net/XplainDBMS.html}}
\newdualentry{yaml}{YAML}
	{Yaml Ain'T Markup Language}
	{a human-readable data serialization language. It is commonly used for configuration files, but could be used in many applications where data is being stored or transmitted.  See also: \url{https://en.wikipedia.org/wiki/YAML}}
\newdualentry{access}{Access}
	{Microsoft Access}
	{Microsoft Access is a database management system from Microsoft that combines the relational Microsoft Jet Database Engine with a graphical user interface and software-development tools. See also: \url{https://products.office.com/en/access}}
\newdualentry{apache}{Apache}
	{}
	{Lauded among the most successful influencers in Open Source, The Apache Software Foundation's commitment to collaborative development has long served as a model for producing consistently high quality software that advances the future of open development. See also: \url{https://www.apache.org/}}
\newdualentry{boost}{BOOST}
	{Boost Software Library}
	{Boost is a set of libraries for the C++ programming language that provide support for tasks and structures such as linear algebra, pseudorandom number generation, multithreading, image processing, regular expressions, and unit testing. See also: \url{https://www.boost.org}}
\newdualentry{caac}{CAAC}
	{Computer-Aided Algorithmic Composition}
	{Unlike software for sequencing, mixing, or notation, these systems are often diverse and innovative, breaking with traditional musical paradigms of meter, part, or score. These systems expand compositional resources, and offer diverse models of compositional design. See also: \url{} \parencite[1]{Ari05:Ano}}
\newdualentry{cassandra}{Cassandra}
	{}
	{The Apache Cassandra database is the right choice when you need scalability and high availability without compromising performance. Linear scalability and proven fault-tolerance on commodity hardware or cloud infrastructure make it the perfect platform for mission-critical data. Cassandra's support for replicating across multiple datacenters is best-in-class, providing lower latency for your users and the peace of mind of knowing that you can survive regional outages. See also: \url{http://cassandra.apache.org/}}
\newdualentry{catart}{CataRT}
	{Real-Time Corpus-Based Concatenative Synthesis}
	{The concatenative real-time sound synthesis system CataRT plays grains from a large corpus of segmented and descriptor-analysed sounds according to proximity to a target position in the descriptor space. This can be seen as a content-based extension to granular synthesis providing direct access to specific sound characteristics. See also: \url{http://imtr.ircam.fr/imtr/CataRT}}
\newdualentry{cmam}{CMAM}
	{Centre Des Musiques Arabes Et Mediterraneennes}
	{The Centre for Arab and Mediterranean Music --- Centre des Musiques Arabes et Mediterraneennes (CMAM) --- is an institution operating under the authority of the Ministry of Cultural Affairs. It was established on 20 December 1991and its statutes were enacted in October 1994. See also: \url{http://cmam.tn/}}
\newdualentry{couchdb}{CouchDB}
	{Appache Couch Database}
	{Apache CouchDB is open-source database software that focuses on ease of use and having a scalable architecture. See also: \url{http://couchdb.apache.org/}}
\newdualentry{emi}{EMI}
	{Experiments In Music Intelligence}
	{``Experiments in Music Intelligence (1984) was developed in order to create an interactive tool for composing\dots. By applying an augmented transition network parser and an object oriented approach, intervals, through inheritance and message passing, have both local and global impact (non-linear composition)'' See also: \url{http://artsites.ucsc.edu/faculty/Cope/experiments.htm}  \parencite{DBLP:conf/icmc/Cope87}}
\newdualentry{esac}{EsAC}
	{Essen Associative Code}
	{Essen Associative Code (EsAC) was developed for one-part music, especially for European folksong databases. The code itself was inspired by the Chinese notation JIANPU and consists of cyphers (meaning pitches related to the declared tonic of the mode) and underlines and dots (standing for rhythmic durations). In addition to the code, several programs have been written for PC to analyze, listen, transpose and represent melodies.
Conversions are possible into TEX, PCX, MIDI and other formats. See also: \url{http://www.esac-data.org/}}
\newdualentry{essentia}{Essentia}
	{}
	{Open-source library and tools for audio and music analysis, description and synthesis See also: \url{https://essentia.upf.edu}}
\newdualentry{freesound}{Freesound}
	{}
	{Freesound is a collaborative database of Creative Commons Licensed sounds. Browse, download and share sounds. See also: \url{https://freesound.org/}}
\newdualentry{gui}{GUI}
	{Graphical User Interface}
	{The graphical user interface is a form of user interface that allows users to interact with electronic devices through graphical icons and visual indicators such as secondary notation, instead of text-based user interfaces, typed command labels or text navigation. See also: \url{https://en.wikipedia.org/wiki/Graphical_user_interface}}
\newdualentry{guido}{GUIDO}
	{Guido Music Notation Format}
	{The GUIDO Music Notation Format is a formal language for score level music representation. It is a plain-text, i.e. readable and platform independent format capable of representing all information contained in conventional musical scores. See also: \url{http://guidolib.sourceforge.net/GUIDO/}}
\newdualentry{ibm-7090}{IBM-7090}
	{}
	{The IBM 7090 is a second-generation transistorized version of the earlier IBM 709 vacuum tube mainframe computer that was designed for `large-scale scientific and technological applications.' The first 7090 installation was in November 1959. See also: \url{https://en.wikipedia.org/wiki/IBM_7090}}
\newdualentry{marsyas}{Marsyas}
	{Music Analysis, Retrieval And Synthesis For Audio Signals}
	{Marsyas is an open source software framework for audio processing with specific emphasis on Music Information Retrieval applications. It has been designed and written by George Tzanetakis with help from students and researchers from around the world. Marsyas has been used for a variety of projects in both academia and industry. See also: \url{http://marsyas.info/}}
\newdualentry{mimo}{MIMO}
	{Musical Instrument Museums Online}
	{The world's largest freely accessible database for information on musical instruments held in public collections. See also: \url{http://www.mimo-international.com}}
\newdualentry{musedata}{MuseData}
	{}
	{The MuseData database is a project of the Center for Computer Assisted Research in the Humanities (CCARH). The database was created by Walter Hewlett. Data entry has been primarily done by Frances Bennion, Edmund Correia, Walter Hewlett, and Steve Rasmussen. See also: \url{http://www.musedata.org/}}
\newdualentry{music21}{Music21}
	{}
	{Music21 is a set of tools for helping scholars and other active listeners answer questions about music quickly and simply. It is a python module. See also: \url{http://web.mit.edu/music21/}}
\newdualentry{musicomp}{MUSICOMP}
	{Music Simulator-Interpreter For Compositional Procedures}
	{``Development of MUSICOMP began in the late 1950s, and is considered by Loy as `the granddaddy of all programming systems for automatic music generation' (1989, p. 368) and by Roads as the `first composition language' (1996, p. 815)\dots. From 1967 to 1969 these tools were used [with John Cage] in the production of HPSCHD'' See also: \url{} \parencite[44]{Ari05:Ano}}
\newdualentry{musicxml}{MusicXML}
	{Musicxml}
	{MusicXML is an \gls{xml}-based file format for representing Western musical notation. The format is open, fully documented, and can be freely used. See also: \url{https://en.wikipedia.org/wiki/MusicXML}}
\newdualentry{nc2if}{NC2IF}
	{Center For Network-Centric Cognition And Information Fusion}
	{ The Center for Network-Centric Cognition and Information Fusion (NC2IF) explores the information chain from energy detection via sensors and human observation to physical modeling, signal and image processing, pattern recognition, knowledge creation, information infrastructure, and human decision-making-all in the context of organizations and the nation. See also: \url{https://ist.psu.edu/research/centers_labs/nc2if}}
\newdualentry{networkx}{NetworkX}
	{Networkx}
	{NetworkX is a Python package for the creation, manipulation, and study of the structure, dynamics, and functions of complex networks. See also: \url{http://networkx.github.io}}
\newdualentry{objective-c}{Objective-C}
	{}
	{Objective-C is a programming language combining the Smalltalk messaging system and the C programming language, which enables an object-oriented approach to the latter. See also: \url{https://en.wikipedia.org/wiki/Objective-C}}
\newdualentry{ofx}{OFX}
	{Openframeworks}
	{openFrameworks is an open source toolkit designed for creative coding. It is written in C++ and built on top of OpenGL. It runs on Microsoft Windows, macOS, Linux, iOS, Android and Emscripten. See also: \url{https://openframeworks.cc}}
\newdualentry{omax}{OMax}
	{}
	{OMax is a software environment (Creative Agent) which learns in real-time typical features of a musician's style and plays along with him interactively, giving the flavor of a machine co-improvisation.  See also: \url{http://repmus.ircam.fr/omax/home}}
\newdualentry{redis}{Redis}
	{}
	{Redis is an open source (BSD licensed), in-memory data structure store, used as a database, cache and message broker See also: \url{https://redis.io}}
\newdualentry{rism}{RISM}
	{Repertoire International Des Sources Musicales}
	{The International Inventory of Musical Sources ---Repertoire International des Sources Musicales (RISM)--- is an international, non-profit organization which aims for comprehensive documentation of extant musical sources worldwide\dots Nearly all of the records may be downloaded as open data and linked open data in MARCXML and RDF format under a Creative Commons CC-BY license. See also: \url{https://opac.rism.info}}
\newdualentry{sedna}{Sedna}
	{}
	{Sedna is a free native XML database which provides a full range of core database services - persistent storage, ACID transactions, security, indices, hot backup. Flexible XML processing facilities include W3C XQuery implementation, tight integration of XQuery with full-text search facilities and a node-level update language. See also: \url{https://www.sedna.org/}}
\newdualentry{smalltalk}{Smalltalk}
	{}
	{Smalltalk is an object-oriented, dynamically typed reflective programming language. See also: \url{https://en.wikipedia.org/wiki/Smalltalk}}
\newdualentry{smc}{SMC}
	{Sound And Music Computing Conference}
	{ See also: \url{http://www.smcnetwork.org/}}
\newdualentry{sondata}{SonData}
	{Sonifying Data}
	{SonData is an Interactive Data Sonification toolkit, targeted at all practitioners interested in sonifying data. However, it also provides a set of tools that are useful to the academic and scientific Data Sonification community. See also: \url{https://github.com/JoaoMenezes/SonData}}
\newdualentry{sparksee}{Sparksee}
	{}
	{Sparksee is a high-performance and scalable graph database management system written in C++. Its development started in 2006 and its first version was available on Q3 - 2008. The fourth version is available since Q3-2010. See also: \url{http://www.sparsity-technologies.com}}
\newdualentry{sparsity}{Sparsity}
	{}
	{High-performance human solutions for Extreme Data See also: \url{http://www.sparsity-technologies.com/}}
\newdualentry{sqlobject}{SQLObject}
	{}
	{SQLObject is a popular Object Relational Manager for providing an object interface to your database, with tables as classes, rows as instances, and columns as attributes. SQLObject includes a Python-object-based query language that makes SQL more abstract, and provides substantial database independence for applications. See also: \url{http://www.sqlobject.org/}}
\newdualentry{telemeta}{Telemeta}
	{}
	{Telemeta is a free and open source collaborative multimedia asset management system (MAM) which introduces fast and secure methods to archive, backup, transcode, analyse, annotate and publish any digitalized video or audio file with extensive metadata. It is dedicated to collaborative media archiving projects, research laboratories and digital humanities --- especially in ethno-musicological use cases --- who need to easily organize and publish documented sound collections of audio files, CDs, digitalized vinyls and magnetic tapes over a strong database, through a smart and secure platform, in accordance with open web standards. Telemeta stands for Tele for ``remote access'' and meta for ``metadata.'' See also: \url{http://phonotheque.cmam.tn/}}
\newdualentry{unisys}{Unisys}
	{United Information Systems}
	{Unisys Corporation is an American global information technology company based in Blue Bell, Pennsylvania, that provides a portfolio of IT services, software, and technology. It is the legacy proprietor of the Burroughs and UNIVAC line of computers, formed when the former bought the latter. See also: \url{https://www.unisys.com}}
