\paragraph{Dataloquy (you don't need to read this)}
(The initial title that explains how databases are everywhere) The name of the database. (Now think of the data, and the base, and how these relate) These words must point to two things placed one inside the other. (Is it base in data or is it data in base?) The base (of the data). A basement, a basis, a basic foundation for data. The house where data resides. (Is it the base or the data that is economical? Or both?) The addresses in which they are located. (Clearly, you are talking about pointers) The discretized space that guards data. (Guardians of space? This starts to look like a bad sci-fi thing\dots) Data as in the plurality of datum, and database as the plain (\textit{planicie} in spanish) or the lattice upon which the address space is laid for data. (Data under house arrest) Datum as in bit, as the zero or the one, and nothing in between (Are you sure there is nothing in the middle?) Data as in bytes, and the eight bits that follow it around (like ducklings without mom [\textit{pata} in spanish]) Data as in data types, the many names of the binary words representing the values of almost all numbers (and this `almost' is still more than enough (Some would disagree)). Data as in data structures and their unions, symbols, lists, tables, arrays, sequences, dictionaries, simultaneously pointing to their interfaces and their implementations and assemblage (The assembly is in order) Data as in files \textit{fichiers} in french, \textit{archivos} in spanish, and their kilobytes and megabytes (inside directories and folders, etc.) Data as in data flow and data streams (are there data fountains?) Data you translate from slot to slot, transmit from client to server to client, transduce to and fro with \obj{adc} and \obj{dac}, transcode from format to format (transgress from torrent to torrent) Data as in dataset for your algorithms to test, to improve, to fit, to make them more efficient, to teach them the right tendencies, to drive your models data-driven (You are driving me crazy) Data as in data banks (also its transactions and currency) Data as in data corpus (Oh, so it has body?) Data as in database (finally), basing gigabytes with models meant for system management and wearhouses, repositories, their mining, and their subsequent data clouds, clusters, spacing out into the (in)famous big data leap from bit to big.



% \begin{quote}
% \footnotesize
% \flushright
% \raggedleft
% \textit {
% 	I am sitting in a room different from the one you are in now. I am recording the sound of my speaking voice and I am going to play it back into the room again and again until the resonant frequencies of the room reinforce themselves so that any semblance of my speech, with perhaps the exception of rhythm, is destroyed. What you will hear, then, are the natural resonant frequencies of the room articulated by speech. I regard this activity not so much as a demonstration of a physical fact, but more as a way to smooth out any irregularities my speech might have.\footnote{Alvin Lucier. I Am Sitting In A Room. See: \url{https://en.wikipedia.org/wiki/I_Am_Sitting_in_a_Room}}
% }
% \end{quote}

